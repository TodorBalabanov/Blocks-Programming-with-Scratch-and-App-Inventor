\documentclass[14pt, a4paper, openany]{book}

% Използване на български език.
\usepackage[T2A, T1]{fontenc}
\usepackage[utf8]{inputenc}
\usepackage[english]{babel}

% Използва се за групиране на изображения.
\usepackage{subcaption}

% Използване на графика.
\usepackage[pdftex]{graphicx}

% Използване на по прецизни позиции за изображенията.
\usepackage{float}

% Използване на PDF-и за кориците.
\usepackage{pdfpages}

% Използване на хедър и футър.
\usepackage{fancyhdr}

% Използване на кавички при цитиране.
%\usepackage{dirtytalk}
\usepackage{csquotes}

% Използва се за създаване на азбучен указател.
\usepackage{imakeidx}

% Добавя възможност за сензитивни хипер-връзки в самия документ.
\usepackage[pdftex, bookmarks, linktocpage]{hyperref}

% Команда с множество опции за настройка на поведението на пакета hyperref, с най-полезната опция - кирилизация на заглавията от Bookmarks в Acrobat.
\hypersetup{unicode=true, colorlinks=true, linkcolor=black, citecolor=black, urlcolor=black}

% Използва се за листинги с програмен код.
\usepackage{listings}

% Използва се за многоредови коментари.
\usepackage{verbatim}

% Използвасе за междуредово разстояние.
\usepackage{lipsum}

% Използва се за таблици, които да са на повече от една страница.
\usepackage{longtable}

% Заглавие.
\title{Block Programming with Scratch and App Inventor}

% Автори.
\author{Todor Balabanov, Galya Petrova, Antonia Guzgunova}

% Директория с изображения.
\graphicspath{{images/}}

% Избор на активен език.
\selectlanguage{english}

% Текстове за декорация на страницата в горната и долната част.
\pagestyle{fancy}
\fancyhf{}
\fancyhead[LE,RO]{\thepage}
\fancyhead[RE]{Block Programming with Scratch and App Inventor}
\fancyhead[LO]{Todor Balabanov, Galya Petrova, Antonia Guzgunova}
\fancyfoot[LE,RO]{Publishing House ''Education and Knowledge'', 2023, Sofia}

% Дебелина на разделителните линии.
\renewcommand{\headrulewidth}{2pt}
\renewcommand{\footrulewidth}{1pt}

% Генериране на азбучен указател.
\onecolumn
\makeindex[columns=2, title=Index, intoc]

% Подменя думата използван а за номерация на фрагментите програмен код.
\renewcommand{\lstlistingname}{Listing}

% Смяна на названието за списъка от листингите.
%\renewcommand{\lstlistlistingname}{Списък на листингите}

% Определя характеристиките на листигните за програмния код.
\lstset{backgroundcolor=\color{gray!30}, breaklines=true, language=r, frame=single}

% Разстояние от ред и половина.
\linespread{1.5}

% Начало на документа.
\begin{document}

% Предна корица.
\includepdf[pages={1}]{covers-en/front}
\thispagestyle{empty}

% Страница с авторски права.
~\vfill
\thispagestyle{empty}

\noindent Copyright \copyright\ 2023 \\

\noindent Todor Balabanov, Galya Petrova, Antonia Guzgunova \\ 

\noindent \textsc{Publishing House ''Education and Knowledge''} \\
\noindent \textsc{https://www.obrazovaniebg.net/} \\

\noindent {\footnotesize This book is supported by the Ministry of Education and Science of the Republic of Bulgaria under the National Science Program ''INTELLIGENT ANIMAL HUSBANDRY, grant agreement No. D01-62/18.03.2021''}. \\

\noindent \textit{1st Editon, 2023}



% Номериране на страниците със служебна информация.
\pagenumbering{roman}
\setcounter{page}{1}

% Таблица на съдържанието.
\addcontentsline{toc}{chapter}{Table of Contents}
\tableofcontents\newpage

% Списък с фигурите.
\addcontentsline{toc}{chapter}{List of Figures}
\listoffigures\newpage

% Списък с таблиците.
%\addcontentsline{toc}{chapter}{List of Tables}
%\listoftables\newpage

% Списък с листингите.
%\addcontentsline{toc}{chapter}{List of Source Codes}
%\lstlistoflistings\newpage

% Номериране на страниците с основното изложение.
\pagenumbering{arabic}
\setcounter{page}{1}

% Отделните глави са в отделни файлове.
%\addcontentsline{toc}{chapter}{Preface}
\chapter*{Preface}
\thispagestyle{empty}

This book is intended for all people who are excited about topics in programming and especially about teaching children in this field. We hope anyone interested in the area finds something valuable in the material presented. Our experience primarily concerns the academic world and pedagogy dedicated to children's education. The material is presented in such a way as to reveal the basic mechanisms of learning by doing. Generally speaking, the book guides the reader with minimal computer knowledge to a good understanding of programming concepts.

From our point of view, the presentation of program constructions with the help of visualization, as close as possible to classic puzzles, gives ample opportunities for learning valuable knowledge and skills. Such a visualization approach allows for an effective lowering of the programming learning age. While classical programming languages are appropriate in high school classrooms, block languages effectively find their application in the junior high school course of study.

Part of the presented material explains fundamental concepts in programming, such as - sequence of instructions, conditional and unconditional transitions, cyclicity of actions, events, modular organization, and others. Another part emphasizes practical implementation and ideas that have the potential to become independent software solutions. The chosen approach to presenting the information is through examples of the principle - do, step by step.

The book assumes no prerequisites for an advanced level of computer literacy but relies on the reader having a basic knowledge of what a computer system is, what an operating system is, what the Internet is, and how it handles loading and browsing web pages. The considered block programming systems are web-based, and work with them takes place in a cloud space. A deep knowledge of mathematics is not required, but a basic knowledge of algebra and geometry helps understand some examples. Artistic skills, such as musicality or visual arts, are unnecessary, but their presence would give additional flavor to the results achieved.

The material is organized into chapters related to each other, and for complete absorption, it is desirable to read them in the given sequence. Part of the exposition in the book is based on subjects taught in the junior high school course of study.

\vspace{0.5cm}

\large{\textbf{Acknowledgments}}

\vspace{0.5cm}

The authors would like to thank their families for their patience and understanding while writing this book. They would also like to thank their colleagues and friends who helped to achieve higher quality.

\newpage
%\chapter{Work Environments}

Block programming languages are a specific category within visual programming languages. Their key characteristic is that program instructions are represented as colored blocks rather than the traditional text-based commands in classical programming languages. The primary objective of block languages is to make programming more accessible to beginners. This goal is accomplished through three main approaches.

Firstly, from a syntactic perspective, instructions in block languages are presented as colored icons. This significantly reduces the likelihood of errors in program instruction entry. Secondly, block languages focus on improving semantics by providing thorough documentation for each available program instruction. This ensures a better understanding of the intended functionality. Lastly, block programming emphasizes pragmatism, allowing learners to explore different program states and their effects.

Over the past decade, block programming environments have gained substantial popularity. Among the most widely used are Scratch, Blockly, App Inventor for Android, and Ardublock. This book concentrates on two block programming environments developed at MIT: Scratch and App Inventor for Android. The rationale behind this choice is that Scratch targets younger learners, particularly children in primary school, and integrates well with the visual representation of block programs on mobile devices using App Inventor for Android. Both environments can be accessed via a modern computer with an internet connection and a contemporary web browser, eliminating the need for specialized software installations.

\section{Getting Started in Scratch}

To begin working in the Scratch environment, you can start by loading the main web page, which can be found at: \\ \href{https://scratch.mit.edu/}{https://scratch .mit.edu/}

You can access this page by entering the URL into your web browser's address bar. (See Fig. \ref{fig010001} for reference, which unfortunately cannot be displayed in a text-based conversation.)

Once the page is loaded, you can explore and utilize Scratch's features and resources for programming and creating interactive projects.

\begin{figure}[H]
   \centering
   \includegraphics[width=1.0\linewidth,height=0.5\linewidth]{fig010001.png}
   \caption{Sratch Homepage}
\label{fig010001}
\end{figure}

The Scratch program environment operates based on cloud services. Therefore, individuals must register by creating an account to use the service. Registration entails providing a username and password (See Fig. \ref{fig010002} for a visual representation, which unfortunately cannot be displayed in a text-based conversation).

By registering, users gain access to the full functionality of Scratch, including the ability to save and share their projects, collaborate with others, and participate in the Scratch community.

\begin{figure}[H]
   \centering
   \includegraphics[width=1.0\linewidth,height=0.5\linewidth]{fig010002.png}
   \caption{Sratch User Registration}
\label{fig010002}
\end{figure}

After selecting a username and password during the registration process in Scratch (Fig. \ref{fig010002}), the next step typically involves specifying the geographical region where the user is located (Fig. \ref{fig010003}, which cannot be displayed in a text-based format). This information helps Scratch provide localized content, relevant events, and community engagement opportunities that may be specific to the user's region.

Users can further personalize their Scratch experience by selecting the geographical region and connecting with other Scratch users from their area.

\begin{figure}[H]
   \centering
   \includegraphics[width=1.0\linewidth,height=0.5\linewidth]{fig010003.png}
   \caption{Geographic location}
\label{fig010003}
\end{figure}

The Scratch platform is primarily designed for children who are interested in programming, as well as for parents and teachers. The system collects information about their age to cater to its user's specific needs and safety considerations (Fig. \ref{fig010004}, which cannot be displayed in a text-based format). By gathering age information, Scratch can provide age-appropriate content, features, and interactions suitable for the specific user group.

This age information helps tailor the experience within Scratch, ensuring that the platform offers appropriate educational resources and maintains a safe and supportive environment for young users. It also enables parents and teachers to guide and supervise children's activities on the platform effectively.

\begin{figure}[H]
   \centering
   \includegraphics[width=1.0\linewidth,height=0.5\linewidth]{fig010004.png}
   \caption{Age of User}
\label{fig010004}
\end{figure}

In addition to collecting information about the user's age, the Scratch system also offers the option to provide information regarding gender classification (Fig. \ref{fig010005}, which unfortunately cannot be displayed in a text-based conversation). However, it's important to note that providing gender information is optional, and users can choose not to disclose it if they prefer.

The inclusion of gender classification is primarily aimed at fostering an inclusive and diverse community within Scratch. It allows for tailored content and experiences that address various interests and perspectives. By making this information optional, Scratch aims to prevent any form of discrimination based on gender while respecting the privacy and preferences of its users.

\begin{figure}[H]
   \centering
   \includegraphics[width=1.0\linewidth,height=0.5\linewidth]{fig010005.png}
   \caption{User gender}
\label{fig010005}
\end{figure}

To create a user profile on Scratch, along with selecting a username and password, it is required to associate the profile with an email address (Fig. \ref{fig010006}, which cannot be displayed in a text-based format). The email address serves as a means of communication and verification for the user account.

Users can receive important notifications, account-related updates, and password reset instructions if necessary by providing an email address. It also helps maintain the security and integrity of the user's profile. The email address is treated with confidentiality and is not publicly visible to other Scratch users.

\begin{figure}[H]
   \centering
   \includegraphics[width=1.0\linewidth,height=0.5\linewidth]{fig010006.png}
   \caption{User email address}
\label{fig010006}
\end{figure}

Once the user registration process in the system is nearly complete (Fig. \ref{fig010007}, which cannot be displayed in a text-based format), the final step is to confirm the selected email address. This step ensures the provided email's accuracy and helps verify the user's identity.

After entering the email address during registration, Scratch sends a confirmation email to the provided address. The user must then access their email inbox, find the verification email from Scratch, and follow the instructions to confirm the email address. This step helps prevent unauthorized access and ensures the user has control over the associated email account.

By confirming the email address, users can fully activate their Scratch account and gain access to all the features and benefits offered by the platform.

\begin{figure}[H]
   \centering
   \includegraphics[width=1.0\linewidth,height=0.5\linewidth]{fig010007.png}
   \caption{Completing the user information entry process}
\label{fig010007}
\end{figure}

When a user registers in Scratch and reaches the stage of email confirmation, they will receive an email that includes an electronic link to the Scratch website (Fig. \ref{fig010008}, which unfortunately cannot be displayed in a text-based format). This link must be clicked or followed to finalize the new user registration process.

The user will be directed to a specific page on the Scratch website by clicking on the link provided in the email. This page typically confirms that the user's email address has been successfully verified, completing the registration process.

Following the link is crucial to validate the email address and ensure the user's account activation. It is an essential step to provide a secure and legitimate registration process within the Scratch platform.

\begin{figure}[H]
   \centering
   \includegraphics[width=1.0\linewidth,height=0.5\linewidth]{fig010008.png}
   \caption{Email confirmation email}
\label{fig010008}
\end{figure}

The registration process for a new user concludes with loading the initial working screen in Scratch (Fig. \ref{fig010009}, which cannot be displayed in a text-based format). At the top right corner of the screen, you will find the username chosen during the first step of the registration process.

Upon reaching the working screen, users can start exploring and utilizing Scratch's various features and tools to create their interactive projects. The username displayed at the top right corner is a quick reference to identify the logged-in user and allows easy access to their account settings and project management options.

\begin{figure}[H]
   \centering
   \includegraphics[width=1.0\linewidth,height=0.5\linewidth]{fig010009.png}
   \caption{Home screen}
\label{fig010009}
\end{figure}

To confirm a successful registration in Scratch, users can create a small project to experience the development environment in action. This can be done by selecting the "Create" option from the menu (Fig. \ref{fig010010}, which cannot be displayed in a text-based conversation).

By choosing the "Create" option, users will be directed to the project editor to build their own interactive projects using blocks and scripts. This demonstrates that the registration process was completed successfully and that users now have access to the full functionality of Scratch.

Creating a small project not only allows users to familiarize themselves with the development environment but also serves as a practical demonstration of their ability to utilize Scratch's features and begin exploring the possibilities of programming and coding within the platform.

\begin{figure}[H]
   \centering
   \includegraphics[width=1.0\linewidth,height=0.5\linewidth]{fig010010.png}
   \caption{Selecting a New Project Menu Option}
\label{fig010010}
\end{figure}

When creating a new project in Scratch, the process involves several steps related to allocating the initially required resources (Fig. \ref{fig010011}, which cannot be displayed in a text-based format). These steps ensure that users have the necessary components to build their projects.

Typically, the resource allocation steps include selecting a backdrop (the background image or scene for the project), choosing sprites (the characters or objects that will interact in the project), and possibly importing additional media files such as sounds or images.

By going through these steps, users can set up the initial elements of their project, providing a foundation for further development and programming. It allows users to define their project's visual and auditory aspects before moving on to coding and creating interactions between the sprites and backdrop.

\begin{figure}[H]
   \centering
   \includegraphics[width=1.0\linewidth,height=0.5\linewidth]{fig010011.png}
   \caption{Loading Resources}
\label{fig010011}
\end{figure}

Once the new project is loaded, users will be presented with the workspace in Scratch (Fig. \ref{fig010012}, which cannot be displayed in a text-based format). The workspace is divided into three main sections:

1. the list of available program instructions is on the far left, represented as puzzle pieces. These blocks contain different commands and functions that users can drag and drop into the workspace to build their program.

2. In the central part of the workspace is where users arrange and connect the program instructions. They can snap together the blocks to create a sequence of actions and define the logic of their project.

3. On the far right is the active stage, where the visual representation of the project is displayed. It shows the results of the actions embedded in the series of instructions as users interact with and run their program.

This workspace layout allows users to visually construct their program by assembling the puzzle-like blocks, creating a clear and intuitive interface for programming and coding within Scratch.

\begin{figure}[H]
   \centering
   \includegraphics[width=1.0\linewidth,height=0.5\linewidth]{fig010012.png}
   \caption{Workspace Organization}
\label{fig010012}
\end{figure}

In Scratch, every computer program has designated starting and ending points. To signify the beginning of a program, Scratch provides a particular block known as the program launch block (Fig. \ref{fig010013}, which cannot be displayed in a text-based format).

The program launch block is tied explicitly to the green flag click event. When the green flag is clicked in the Scratch environment, the program execution begins. This block serves as a trigger, indicating that the program should start running and that the actions and instructions defined in the blocks connected to it will be executed.

By associating the program launch block with the green flag click event, Scratch ensures a clear and consistent way to initiate the execution of programs, allowing users to interact with their projects and observe the desired behavior or animations when the green flag is clicked.

\begin{figure}[H]
   \centering
   \includegraphics[width=1.0\linewidth,height=0.5\linewidth]{fig010013.png}
   \caption{Start of program}
\label{fig010013}
\end{figure}

One of Scratch's most uncomplicated and most intuitive instructions is the ability to move a sprite by a specified number of steps (Fig. \ref{fig010014}, which cannot be displayed in a text-based format). In the default opening scene of Scratch, the leading actor is the orange cat sprite. If the scene remains unchanged, the instructions to perform various actions will be applied directly to that cat sprite.

By using the "move" block and specifying the number of steps, users can control the sprite's movement within the Scratch stage. This instruction allows the sprite to move horizontally, vertically, or diagonally across the screen.

When creating a project, users can apply various actions and instructions to the orange cat sprite, such as moving, changing costumes, playing sounds, or interacting with other sprites or user inputs. This enables them to bring their ideas to life and create interactive stories, animations, and games using the familiar orange cat as the protagonist or main character.

\begin{figure}[H]
   \centering
   \includegraphics[width=1.0\linewidth,height=0.5\linewidth]{fig010014.png}
   \caption{Step by step instructions}
\label{fig010014}
\end{figure}

After moving the cat sprite, it is crucial to incorporate a waiting pause to make the movement visually noticeable. In Scratch, this can be achieved using the "wait" instruction, which introduces a specified delay before the following action or instruction is executed (Fig. \ref{fig010015}, which cannot be displayed in a text-based format).

By using the "wait" block and specifying the number of seconds or fractions of a second, users can introduce a pause in the program execution. This allows for better control over the timing and synchronization of actions within a project. During the waiting period, the program remains idle, allowing users to observe the effects of previous actions before proceeding to the next step.

Combined with the "move" block, the "wait" block helps create a smoother and more comprehensible visual experience in Scratch projects, providing time for movements or changes to be observed and appreciated.

\begin{figure}[H]
   \centering
   \includegraphics[width=1.0\linewidth,height=0.5\linewidth]{fig010015.png}
   \caption{Wait instruction}
\label{fig010015}
\end{figure}

After the waiting period, if you want the cat sprite to return to its original position, you can achieve this by executing a move instruction with a negative number of steps (Fig. \ref{fig010016}, which cannot be displayed in a text-based format).

In Scratch, when a negative value is provided in the "move" block, it instructs the sprite to move in the opposite direction of the positive value. The cat sprite will move back to its original position by specifying a negative number of steps, effectively undoing the previous movement.

This technique allows you to create animations or interactions where sprites move forward and return to their starting point. Combining positive and negative values with appropriate wait periods enables you to achieve various visual effects and create engaging experiences within your Scratch projects.

\begin{figure}[H]
   \centering
   \includegraphics[width=1.0\linewidth,height=0.5\linewidth]{fig010016.png}
   \caption{Instructions for moving back}
\label{fig010016}
\end{figure}

Once all the desired instructions have been executed in a Scratch program, it is necessary to end the program. For this purpose, Scratch provides a separate block in the list of instructions known as the "end" block (Fig. \ref{fig010017}, which cannot be displayed in a text-based format).

The "end" block serves as the final instruction in a Scratch program, indicating that the program execution should cease at that point. When the program reaches the "end" block, all actions and instructions will halt and no longer continue running.

Including the "end" block in your program ensures proper termination and prevents unintended behavior or infinite loops. It allows you to control the execution flow and explicitly define the end point of your program in Scratch.

\begin{figure}[H]
   \centering
   \includegraphics[width=1.0\linewidth,height=0.5\linewidth]{fig010017.png}
   \caption{End instruction}
\label{fig010017}
\end{figure}

The program you have written in Scratch can be executed by pressing the green flag button (Fig. \ref{fig010018}, which cannot be displayed in a text-based format). When the green flag is clicked, the program will start running, and your written instructions will be executed in sequence.

If you need to stop the program immediately during its execution, Scratch provides an emergency stop button with a red circle to the right of the green flag button. Clicking the red circle will halt the program's execution and bring it to a stop.

These buttons give you control over the start and stop of your program, allowing you to test and observe the behavior of your project. The green flag initiates the program, while the red circle provides a means to interrupt the program's execution if necessary.

\begin{figure}[H]
   \centering
   \includegraphics[width=1.0\linewidth,height=0.5\linewidth]{fig010018.png}
   \caption{Program Execution}
\label{fig010018}
\end{figure}

When using Scratch, each program you write is organized as a separate project. To access and manage your projects, you can use the "My Stuff" menu, which is part of the options available for registered users (Fig. \ref{fig010019}, which cannot be displayed in a text-based format).

The "My Stuff" menu provides a centralized location to view and manage your projects. From this menu, you can access your saved projects, create new projects, delete or rename existing projects, and perform other project-related actions.

By navigating to the "My Stuff" menu, you can easily keep track of your Scratch projects and conveniently work on multiple projects or revisit previously created ones. It provides a user-friendly interface to organize and handle your projects within the Scratch environment.

\begin{figure}[H]
   \centering
   \includegraphics[width=1.0\linewidth,height=0.5\linewidth]{fig010019.png}
   \caption{Project Organization Menu}
\label{fig010019}
\end{figure}

In Scratch, each project is assigned a working name by default (Fig. \ref{fig010020}, which cannot be displayed in a text-based format). However, users can change the project name as desired, providing a more descriptive or meaningful title that reflects the project's content or purpose.

One of the most appealing advantages of the Scratch programming environment is the ability to share projects with a vast audience. This sharing feature lets users quickly disseminate their knowledge, skills, and creations to others. Users can showcase their work, receive feedback, collaborate with peers, and inspire others to learn and explore programming by sharing projects.

Sharing projects facilitates the transfer of knowledge and skills and enables the assessment and evaluation of the work done. Users can receive comments, suggestions, and appreciation for their projects, fostering community and engagement within the Scratch platform.

The sharing feature in Scratch promotes a collaborative and interactive learning environment where users can learn from one another, exchange ideas, and be recognized for their creativity and accomplishments.

\begin{figure}[H]
   \centering
   \includegraphics[width=1.0\linewidth,height=0.5\linewidth]{fig010020.png}
   \caption{List of projects}
\label{fig010020}
\end{figure}

One of the most fascinating aspects of block languages is their resemblance to puzzle-solving. Writing instructions and forming a complete program in a block language is akin to assembling puzzle pieces. This characteristic holds particular appeal for children, as many of them enjoy putting together puzzles.

Using colorful blocks and visually appealing elements in block languages adds to their attractiveness. Children are naturally drawn to bright colors and beautiful pictures, and incorporating these elements into programming enhances their engagement and interest. Combining the charm of classic puzzles with the allure of programming can lead to astonishing results.

By presenting programming concepts as puzzles, block languages make the learning experience more enjoyable and accessible for children. It allows them to approach programming as a creative and problem-solving activity, where they can explore and experiment with different combinations of blocks to achieve desired outcomes. This puzzle-like approach promotes critical thinking, logical reasoning, and problem-solving skills engagingly and playfully.

The fusion of puzzle-solving and programming in block languages captivates children's imaginations and motivates them to explore the programming world with enthusiasm and curiosity.

\section{Getting Started in App Inventor}

Working in the App Inventor environment starts by accessing the main web page (Fig. \ref{fig010021}). You can find the App Inventor web page at: \\ \href{https://appinventor.mit.edu/}{https:// appinventor.mit.edu/}

To begin using App Inventor, you must navigate to this web page and load it in your browser. The web page is the entry point to the App Inventor environment, where you can create, design, and develop your mobile applications.

Accessing the App Inventor web page gives you access to a powerful visual programming platform that allows you to build mobile apps using a block-based interface. Whether you're a beginner or have some programming experience, App Inventor provides a user-friendly environment to create functional and interactive mobile applications.

Once you've loaded the main web page, you can explore the various features, resources, and tutorials available on the App Inventor platform to get started with your app development journey.

\begin{figure}[H]
   \centering
   \includegraphics[width=1.0\linewidth,height=0.5\linewidth]{fig010021.png}
   \caption{App Inventor Home Web Page}
\label{fig010021}
\end{figure}

While the Massachusetts Institute of Technology also develops App Inventor, it differs from Scratch in certain aspects. When using App Inventor, you can work with it as a cloud service, and registration is necessary to access its features. However, unlike Scratch, App Inventor provides an alternative method for user registration.

To begin the registration process in App Inventor, you can click on the orange "Create Apps!" button (Fig. \ref{fig010022}), which cannot be displayed in a text-based format. This button initiates the registration procedure and allows you to create your App Inventor account.

Unlike the traditional creation of a user profile, App Inventor allows logging in using a Google account. By choosing this method, you can utilize your existing Gmail credentials for authentication, making the registration process more streamlined and convenient.

Following the prompts after clicking the "Create Apps!" button will guide you through the necessary steps to create your App Inventor account. This process allows you to leverage the cloud-based features of App Inventor and start developing your mobile applications.

\begin{figure}[H]
   \centering
   \includegraphics[width=1.0\linewidth,height=0.5\linewidth]{fig010022.png}
   \caption{Choosing a GMail user to log in}
\label{fig010022}
\end{figure}

Once you have selected a user to work within the App Inventor environment, you must authenticate yourself by entering a password (Fig. \ref{fig010023}, which cannot be displayed in a text-based format). This step ensures the security and privacy of your App Inventor account.

After selecting the user, you will be prompted to enter the associated password. This password should correspond to the account you use for App Inventor, whether it is a password specific to App Inventor or your Gmail password if you log in through your Google account.

You will successfully authenticate and gain access to your App Inventor account by providing the correct password. This authentication process verifies your identity and allows you to continue working within the App Inventor environment to create and develop mobile applications.

Keeping your password secure and confidential is essential to protect your account from unauthorized access.

\begin{figure}[H]
   \centering
   \includegraphics[width=1.0\linewidth,height=0.5\linewidth]{fig010023.png}
   \caption{User Authentication}
\label{fig010023}
\end{figure}

To work in the App Inventor programming environment, users must agree to the general terms and conditions of the platform (Fig. \ref{fig010024}, which cannot be displayed in a text-based format). This step ensures that users are aware of and agree to comply with the rules and guidelines set forth by App Inventor.

Upon accessing the App Inventor platform, users will be presented with the general terms and conditions that outline the rights and responsibilities of both the users and the platform. These terms typically cover aspects such as acceptable use, intellectual property rights, data privacy, and any specific policies or guidelines that must be followed.

To proceed with using App Inventor, users are required to read and accept these terms and conditions. This confirms their agreement to abide by the platform's rules and regulations. It is essential to carefully review the terms and seek clarification before accepting them.

By agreeing to the general terms of the platform, users can continue with their App Inventor experience and leverage the tools and resources provided to create and develop their mobile applications.

\begin{figure}[H]
   \centering
   \includegraphics[width=1.0\linewidth,height=0.5\linewidth]{fig010024.png}
   \caption{General Terms of Use of the Program Environment}
\label{fig010024}
\end{figure}

After completing the necessary steps to enter the App Inventor programming environment, the process culminates with a welcome web page (Fig. \ref{fig010025}), which provides more detailed information about the program environment, the type of instance launched, and the version.

You will find valuable information regarding the App Inventor programming environment on the welcome web page. This includes details about the features, functionalities, and tools available within the platform. The page may also provide information about any updates or improvements in the current version of App Inventor.

Additionally, the welcome web page may display the type of instance that has been launched, which could be specific to your account or project settings. This information helps identify the configuration you are currently working with within App Inventor.

By presenting this detailed information, the welcome web page offers an overview of the program environment and sets the stage for your App Inventor experience. It serves as a helpful reference for understanding the capabilities and specifications of the programming environment you are about to engage with.

\begin{figure}[H]
   \centering
   \includegraphics[width=1.0\linewidth,height=0.5\linewidth]{fig010025.png}
   \caption{Welcome Page}
\label{fig010025}
\end{figure}

Upon entering the App Inventor programming environment, users are presented with several options, including the opportunity to view various learning projects that serve as an initial introduction to working with the programming environment (Fig. \ref{fig010026}, which cannot be displayed in a text-based format).

These learning projects are designed to provide users, especially beginners, with hands-on examples and demonstrations of how the programming environment functions. By exploring these projects, users can better understand the App Inventor interface, the components available for building mobile applications, and the logic behind creating functional apps.

These learning projects often cover various topics, from basic app development concepts to more advanced functionalities. They serve as a starting point for users to familiarize themselves with the App Inventor environment, its features, and the overall app creation process.

By selecting this option, users can browse through the collection of learning projects and explore them at their own pace. It is an excellent way to gain practical experience and learn by example, providing a foundation for further exploration and development within the App Inventor programming environment.

\begin{figure}[H]
   \centering
   \includegraphics[width=1.0\linewidth,height=0.5\linewidth]{fig010026.png}
   \caption{Ability to choose learning projects}
\label{fig010026}
\end{figure}

If a user does not select a learning project or choose the option to create an empty project, the system will redirect them to a page displaying a list of their projects (Fig. \ref{fig010027}, which cannot be displayed in a text-based format). This page provides an overview of the projects the user has created and saved within their App Inventor account.

The list of projects serves as a central hub for users to manage and access their ongoing app development work. Each project is typically represented by a title or name, allowing users to quickly identify and select the project they wish to work on.

From this page, users can perform various actions related to their projects, such as opening a project for further editing, making copies of projects, or deleting no longer-needed projects. This centralized view helps users keep track of their progress, organize their work, and seamlessly navigate between different projects.

By displaying the list of their projects, App Inventor ensures that users have easy and convenient access to their previously created projects, enabling them to continue their app development journey from where they left off.

\begin{figure}[H]
   \centering
   \includegraphics[width=1.0\linewidth,height=0.5\linewidth]{fig010027.png}
   \caption{Own Projects List Page}
\label{fig010027}
\end{figure}

To start a new project in the App Inventor programming environment, you can click on the "New Project" button located at the top left corner of the main work screen (Fig. \ref{fig010028}, which cannot be displayed in a text-based format). This button is a starting point for creating a new project and initiating your app development process.

By clicking on the "New Project" button, you will be prompted to provide a name for your new project. This name can be descriptive and should reflect the purpose or theme of your app. After entering the lovely name, you can create your project and work on your app's design and functionality.

Starting a new project gives you a blank canvas to unleash your creativity and develop your app from scratch. You can utilize various components, blocks, and features App Inventor offers to bring your app idea to life.

By providing an easily accessible "New Project" button, App Inventor ensures that users can swiftly start new projects and embark on their app development journey without any unnecessary hurdles or complications.

\begin{figure}[H]
   \centering
   \includegraphics[width=1.0\linewidth,height=0.5\linewidth]{fig010028.png}
   \caption{Start New Project Button}
\label{fig010028}
\end{figure}

In the App Inventor programming environment, work is organized through projects, and each project is assigned an appropriate name when created (Fig. \ref{fig010029}, which cannot be displayed in a text-based format).

When creating a new project, one of the initial steps is to provide a name that accurately represents the purpose or theme of the project. This name helps you identify and distinguish your projects within the App Inventor environment, especially when multiple projects are in progress.

Choosing an appropriate name for your project is essential for effective organization and easy retrieval. It allows you to quickly locate and work on the specific project you desire, especially when you have an extensive collection of projects over time.

App Inventor facilitates a structured and organized approach to managing your app development endeavors by assigning a name to each project. It ensures that you can easily navigate your projects, stay focused on specific tasks, and maintain clarity and efficiency in your coding and development process.

\begin{figure}[H]
   \centering
   \includegraphics[width=1.0\linewidth,height=0.5\linewidth]{fig010029.png}
   \caption{Project Name}
\label{fig010029}
\end{figure}

After selecting a name for your project in the App Inventor programming environment, the interface visualizes the first working screen, providing you with the opportunity to design the visual user interface (Fig. \ref{fig010030}, which cannot be displayed in a text-based format). Creating the graphical user interface involves dragging and dropping various visual controls into the main work area.

The visual controls represent different elements that users interact with in your app, such as buttons, labels, text boxes, images, and more. These controls are available in the palette or toolbox, typically on the screen's left side. Selecting a control and dragging it onto the main work area allows you to position and arrange it as desired.

This drag-and-drop approach allows you to design the layout and appearance of your app's user interface. You can visually organize the controls, resize and customize them, and create an intuitive and engaging user experience.

App Inventor simplifies designing the visual user interface by eliminating the need for complex coding. Instead, it emphasizes a visual and intuitive approach, allowing you to focus on the design aspects of your app without getting overwhelmed by technical details.

By providing a user-friendly interface and a drag-and-drop feature for visual control placement, App Inventor empowers users to create appealing and functional user interfaces for their mobile applications.

\begin{figure}[H]
   \centering
   \includegraphics[width=1.0\linewidth,height=0.5\linewidth]{fig010030.png}
   \caption{Design view of the development environment}
\label{fig010030}
\end{figure}

In the App Inventor programming environment, the button is one of the primary visual controls available. The button represents a designated area in the visual field that typically has a text label or icon to visually represent an action performed when the button is pressed (Fig. \ref{fig010031}, which cannot be displayed in a text-based format).

The button control is an interactive element in your app's user interface, allowing users to trigger specific actions or events. Adding a button to your app's design provides a clear and intuitive way for users to initiate functionality or navigate through different sections of your app.

You can use a button control to demonstrate the process of working with the compiled program code. You can customize the button's appearance by setting its properties, such as the text label, color, size, and position on the screen. Additionally, you can define the specific behavior of the button by assigning event handlers or programming logic to execute when the button is pressed.

Utilizing the button control in your app's design allows you to create interactive and responsive user interfaces that enhance user engagement and provide a seamless user experience. The button serves as a bridge between the visual design of your app and the functional logic implemented through programming code, allowing users to interact with your app and easily trigger desired actions.

\begin{figure}[H]
   \centering
   \includegraphics[width=1.0\linewidth,height=0.5\linewidth]{fig010031.png}
   \caption{Place button}
\label{fig010031}
\end{figure}

When designing software applications, using expressive and meaningful names for buttons and other interactive elements is crucial. These names should convey the intended action or functionality to the users. In the case of the button in question (Fig. \ref{fig010031}, which cannot be displayed in a text-based format), the name "Push" is used.

The name "Push" suggests a straightforward and intuitive action to the users, indicating that pressing the button will initiate a specific action or behavior. By choosing a concise and descriptive name like "Push," users can quickly understand the button's purpose and anticipate the outcome of pressing it.

Using expressive names for buttons enhances your software application's usability and user experience. It helps users navigate and interact with the interface more effectively, reducing confusion and improving the overall accessibility of your app. By providing clear and meaningful button names, you enable users to intuitively engage with your application's functionality, enhancing their satisfaction and engagement.

\begin{figure}[H]
   \centering
   \includegraphics[width=1.0\linewidth,height=0.5\linewidth]{fig010032.png}
   \caption{Select text on button}
\label{fig010032}
\end{figure}

Unlike the Scratch programming environment where program instructions are represented as blocks directly in the main workspace, in App Inventor, the program instructions are organized in a separate work screen known as the "Blocks Editor" (Fig. \ref{fig010033}, which cannot be displayed in a text-based format).

The Blocks Editor in App Inventor provides a dedicated space for assembling the program instructions in a puzzle-like format. It allows users to connect visually and arrange blocks representing different programming logic and actions. Each block corresponds to a specific function or operation that can be performed within the app.

By separating the program instructions into the Blocks Editor, App Inventor aims to provide a more organized and structured approach to programming. Users can quickly locate and manipulate the desired blocks, connecting them to define the behavior and functionality of their app. The puzzle-like arrangement of the blocks fosters a logical and intuitive understanding of the code flow and structure.

Users can explore blocks corresponding to different programming concepts in the Blocks Editor, such as control structures, event handlers, data manipulation, and more. Users can build complex program logic without writing traditional text-based code by dragging and connecting the blocks.

The separation of program instructions into the Blocks Editor allows users to focus on the visual representation and arrangement of the code, making it accessible and engaging for beginners and those who prefer a more visual approach to programming. It provides a unique way to construct programs, promoting creativity and problem-solving skills.

\begin{figure}[H]
   \centering
   \includegraphics[width=1.0\linewidth,height=0.5\linewidth]{fig010033.png}
   \caption{Program view of the development environment}
\label{fig010033}
\end{figure}

In App Inventor, the execution of program instructions is based on event-driven programming. Instead of having a predefined starting point and endpoint like in Scratch, App Inventor programs respond to specific events triggered by user interactions or other system events.

When the program is run in App Inventor, it first visualizes the graphical user interface (GUI) components designed in the workspace. The program then waits for specific events, such as a button click, screen touch, sensor input, or timer expiration.

In the example shown in Fig. \ref{fig010034} (which cannot be displayed in a text-based format), the program is waiting for the user to interact with the "Push" button. Once the user clicks the button, the associated event is triggered, and the program executes the instructions connected to that event.

App Inventor uses event handlers to define the actions that should occur when a specific event is triggered. These event handlers are represented as blocks in the Blocks Editor and are associated with the corresponding graphical components on the GUI.

App Inventor allows users to create interactive and responsive apps by leveraging events. The program instructions associated with each event can include various actions, such as displaying messages, performing calculations, updating the interface, or even interacting with external services and APIs.

The event-driven approach in App Inventor provides a flexible and intuitive way to design apps that respond to user input and external stimuli. It allows for creating dynamic and interactive experiences, empowering users to build functional and engaging mobile applications without the need for extensive coding knowledge.

\begin{figure}[H]
   \centering
   \includegraphics[width=1.0\linewidth,height=0.5\linewidth]{fig010034.png}
   \caption{List of instructions}
\label{fig010034}
\end{figure}

In App Inventor, the button component generates an event called "Click" when pressed. This event can be captured in the Blocks Editor, where specific actions can be defined to execute when the button clicks.

In Fig. \ref{fig010035}, the Blocks Editor shows the event handler for the "Click" event of the button component. When the button clicks, the program will execute the instructions connected to this event handler.

In this example, the event handler consists of a single instruction represented by the "Say Hello World" block. This block is a text-to-speech component that will speak the phrase "Hello World" when the button is clicked.

The event handler in the Blocks Editor allows you to add more instructions or blocks to perform various actions in response to the button click event. These actions can include displaying messages, changing the appearance of other components, performing calculations, or even connecting to external services and APIs.

Users can create interactive behavior in their apps by associating specific actions with the button click event. The event-driven nature of App Inventor allows for creating dynamic and responsive applications without the need for complex programming syntax.

Overall, the event-handling feature in App Inventor provides a straightforward and intuitive way to define actions that should occur when specific events, such as button clicks, happen in the app.

\begin{figure}[H]
   \centering
   \includegraphics[width=1.0\linewidth,height=0.5\linewidth]{fig010035.png}
   \caption{Select event for button press}
\label{fig010035}
\end{figure}

The button press event in App Inventor is represented as a puzzle piece, commonly known as an event handler block. This block features a slot where you can insert the instructions to execute when the button is pressed (Fig. \ref{fig010036}).

The event handler block has a distinctive rounded shape and is labeled with the specific event it represents, such as "Button1.Click" indicating the click event of Button1 component. Within the event handler block, you can place additional blocks to define the actions or behaviors triggered by the button press.

By arranging the blocks inside the event handler, you have complete control over the flow and logic of your program. This visual representation simplifies creating event-driven programs and allows users, including beginners, to connect user interface components with desired actions easily.

The puzzle-like nature of App Inventor's event handler blocks makes it intuitive and engaging in building interactive behaviors in your applications without the need for complex coding syntax. It encourages creativity and empowers users to craft dynamic and responsive apps.

\begin{figure}[H]
   \centering
   \includegraphics[width=1.0\linewidth,height=0.5\linewidth]{fig010036.png}
   \caption{Select action when button is pressed}
\label{fig010036}
\end{figure}

One of the most straightforward actions triggered by pressing the button is to stop the program and close the application window (Fig. \ref{fig010037}).

In App Inventor, this action is achieved using the "CloseScreen" block, a built-in instruction specifically designed to terminate the program and close the current window. When placed inside the event handler block for the button press event, the "CloseScreen" block will be executed as soon as the button is pressed.

The "CloseScreen" block ensures a smooth and controlled application termination, allowing users to exit the program effortlessly. This functionality is handy when creating simple applications or prototypes where a single button press can signify the end of the program's execution.

By utilizing the visual programming interface of App Inventor, even beginners can easily incorporate this action into their applications without the need for extensive coding knowledge. The simplicity and intuitiveness of App Inventor's block-based system empower users to quickly implement desired functionalities, enhancing the overall user experience of their applications.

\begin{figure}[H]
   \centering
   \includegraphics[width=1.0\linewidth,height=0.5\linewidth]{fig010037.png}
   \caption{Close application on button click}
\label{fig010037}
\end{figure}

Once the graphical programming interface is designed and the desired instructions are arranged, the next step is compiling the code and generating the complete program installation package (Fig. \ref{fig010038}).

In App Inventor, the compilation process takes the visual blocks and translates them into the corresponding code that the target device can execute. This includes converting the event handlers, such as button presses, into the necessary programming constructs and handling other program logic.

The compilation process ensures that the program is transformed from its visual representation into a format that the device can understand and execute. This allows users to create fully functional applications without extensive manual coding.

Once the compilation is complete, the installation package is generated, which typically consists of an executable file or an app file, depending on the target platform. This package can then be installed and run on compatible devices, allowing users to interact with and experience the program they have created.

The ability to compile and build installation packages in App Inventor provides users with a convenient way to share and distribute their applications. It empowers them to turn their ideas into tangible software solutions that others can install and use, whether on their own devices or through app marketplaces.

By simplifying the process of code compilation and packaging, App Inventor enables users to focus on their creativity and problem-solving skills, making it an accessible platform for both beginners and experienced developers alike.

\begin{figure}[H]
   \centering
   \includegraphics[width=1.0\linewidth,height=0.5\linewidth]{fig010038.png}
   \caption{Mobile App Build Menu}
\label{fig010038}
\end{figure}

One key distinction between Scratch and App Inventor is how the execution of programming instructions is visualized. The result is immediately visible in Scratch within the programming environment's workspace. On the other hand, in App Inventor, the program undergoes a compilation process in the cloud infrastructure, resulting in an installation package that needs to be downloaded and installed on a mobile device. 

Compiling and building the installation package in App Inventor requires a certain amount of computational time, represented by a progress bar (Fig. \ref{fig010039}). This progress bar indicates the ongoing process of converting the visual blocks and program logic into a format that can be installed and executed on a mobile device. The progress bar's length varies depending on the complexity of the program and the computational resources available.

Various optimizations and transformations are applied during the compilation and building process to ensure the program is correctly packaged and ready for deployment. This includes converting the visual components and program logic into the appropriate code the target mobile device can understand and execute.

The visualization of the progress bar serves as a feedback mechanism for users, giving them an estimate of the time required to compile and build the installation package. Once the progress bar is completed, the user can download the installation package and transfer it to their mobile device for installation.

By separating the compilation and installation steps, App Inventor enables users to develop and test their programs within the web-based environment while allowing them to deploy the final application on their mobile devices. This approach provides flexibility and portability, allowing users to create mobile apps that can be shared and used on a broader scale.

\begin{figure}[H]
   \centering
   \includegraphics[width=1.0\linewidth,height=0.5\linewidth]{fig010039.png}
   \caption{App build progress}
\label{fig010039}
\end{figure}

Once the project is compiled and the installation package is generated, the programming environment provides a convenient option to download the package onto a mobile device (such as a phone or tablet) using a QR code (Fig. \ref{fig010040}).

A QR code, short for Quick Response code, is a barcode that can be scanned using the camera or a mobile device's dedicated QR code reader app. In the context of App Inventor, the generated QR code acts as a direct link to the installation package, simplifying the process of transferring the app to a mobile device.

To download the app, the user can launch a QR code scanner app or use the built-in camera app on their mobile device to scan the QR code displayed in the programming environment. Upon scanning, the device recognizes the code and prompts the user to download and install the app. This eliminates the need for manual file transfers or complex installation procedures.

The use of QR codes streamlines the distribution and installation process, making it easier for users to access and test their created apps on their mobile devices. It provides a seamless way to bridge the gap between the development environment and the target device, allowing users to experience their apps in a real-world setting.

By leveraging QR codes, App Inventor enhances the convenience and accessibility of deploying apps created within the platform, enabling users to share and showcase their projects with others or use them personally on their mobile devices.

\begin{figure}[H]
   \centering
   \includegraphics[width=1.0\linewidth,height=0.5\linewidth]{fig010040.png}
   \caption{Code for installing the application on a mobile device}
\label{fig010040}
\end{figure}

The creators of the App Inventor programming environment have developed a dedicated application called MIT AI2 Companion to streamline the installation process of the written programs (Fig. \ref{fig010041}). This application acts as a bridge between the user's mobile device and the cloud infrastructure of App Inventor, facilitating faster and easier testing and deployment of the created apps.

The MIT AI2 Companion app is available for download on Google Play Store and is designed to work seamlessly with App Inventor projects. It eliminates the need for complex manual installations or file transfers by establishing a direct communication channel with the App Inventor cloud infrastructure.

Users can search for and install the MIT AI2 Companion app from Google Play Store onto their mobile devices to utilize this feature. Once installed, they can connect the companion app and their App Inventor projects by scanning a QR code or entering a unique pairing code.

By establishing this connection, the MIT AI2 Companion app allows users to instantly test and run their projects on their mobile devices without complex setup procedures. The app communicates with the cloud infrastructure, enabling real-time synchronization between the development environment and the mobile device.

This approach offers a more streamlined and user-friendly experience, eliminating the need for manual installations or additional technical skills. Users can easily install the MIT AI2 Companion app and leverage its functionality to deploy their projects quickly and efficiently on their mobile devices.

\begin{figure}[H]
   \centering
   \includegraphics[width=1.0\linewidth,height=0.5\linewidth]{fig010041.png}
   \caption{Mobile application for managing compiled projects}
\label{fig010041}
\end{figure}

When launching the MIT AI2 Companion application, users have two options to download their written programs (Fig. \ref{fig010042}). The first option is to enter a code manually, while the second option is to conveniently scan a QR code (Fig. \ref{fig010043}).

Scanning the QR code provides a faster and more convenient method to initiate the download process. Users can easily capture the QR code associated with their project using the device's camera. This eliminates the need for manual entry and reduces the chance of errors.

However, it's important to note that when downloading the installation package, which is an executable file, the Android operating system may issue a warning (Fig. \ref{fig010044}). This is a standard security precaution, as the system alerts users that such files can be harmful.

To proceed with the download, users can follow the on-screen instructions and confirm their intention to download the file despite the warning. It's essential to ensure that the source of the file is trusted and reliable, such as the App Inventor platform, to minimize any potential risks.

By acknowledging and proceeding past the warning, users can safely download and install the application onto their Android devices, allowing them to run and test their projects seamlessly using the MIT AI2 Companion.

\begin{figure}[H]
   \begin{subfigure}{0.31\textwidth}
   \includegraphics[width=\linewidth]{fig010042.png}
   \subcaption{\tiny Installation Selection}
   \label{fig010042}
   \end{subfigure}
   \begin{subfigure}{0.31\textwidth}
   \includegraphics[width=\linewidth]{fig010043.png}
   \subcaption{\tiny Code Scan}
   \label{fig010043}
   \end{subfigure}
   \begin{subfigure}{0.31\textwidth}
   \includegraphics[width=\linewidth]{fig010044.png}
   \subcaption{\tiny Download File}
   \label{fig010044}
   \end{subfigure}
   \caption{Install via QR code}
\end{figure}

After successfully downloading the installation package, the operating system displays a message confirming the successful recording of the installer (Fig. \ref{fig010045}). At this point, the user needs to locate the downloaded installation package and initiate the installation process. Upon selecting the installation package, a prompt appears asking for confirmation to proceed with the program installation contained in the file (Fig. \ref{fig010046}).

It's important to note that even though we know how the installation file was created, the operating system perceives it as a program developed by an unverified developer. As a security measure, the system again prompts the user to confirm their intention to proceed with the installation (Fig. \ref{fig010047}).

This additional confirmation ensures that the user is fully aware of the potential risks of installing applications from unverified sources. By requiring this confirmation, the operating system aims to protect users from potentially harmful or malicious software.

To continue with the installation, the user can confirm their intent by selecting the appropriate option in the prompt. It's recommended to only proceed with the installation if the package's source is trusted and known to be reliable.

Once the user confirms the installation, the operating system will install the program on the device, allowing it to be launched and used as intended.

\begin{figure}[H]
   \begin{subfigure}{0.31\textwidth}
   \includegraphics[width=\linewidth]{fig010045.png}
   \subcaption{\tiny File Downloaded}
   \label{fig010045}
   \end{subfigure}
   \begin{subfigure}{0.31\textwidth}
   \includegraphics[width=\linewidth]{fig010046.png}
   \subcaption{\tiny Choose to install}
   \label{fig010046}
   \end{subfigure}
   \begin{subfigure}{0.31\textwidth}
   \includegraphics[width=\linewidth]{fig010047.png}
   \subcaption{\tiny Installation Confirmation}
   \label{fig010047}
   \end{subfigure}
   \caption{Installation on the mobile device}
\end{figure}

After selecting "Install Anyway," the written program is successfully installed on the mobile device. The installation process concludes with a window prompting the user to start the newly installed program (Fig. \ref{fig010048}).

To verify the functionality of the written code, press the button displayed in the upper left corner of the application's interface (Fig. \ref{fig010049}). Upon pressing the button, the window will close, and the user will be presented with the virtual wallpaper, indicating that the program is functioning as intended (Fig. \ref{fig010049}).

At this point, the user can interact with the program and experience its intended features and functionalities. It's worth noting that the visualized button triggers the programmed actions, and pressing it initiates the desired behavior or operations within the application.

By following this process, the user can successfully install and execute their written program on their mobile device, allowing them to explore and enjoy the functionality they have created.

\begin{figure}[H]
   \begin{subfigure}{0.31\textwidth}
   \includegraphics[width=\linewidth]{fig010048.png}
   \subcaption{\tiny Launch}
   \label{fig010048}
   \end{subfigure}
   \begin{subfigure}{0.31\textwidth}
   \includegraphics[width=\linewidth]{fig010049.png}
   \subcaption{\tiny Button Selection}
   \label{fig010049}
   \end{subfigure}
   \begin{subfigure}{0.31\textwidth}
   \includegraphics[width=\linewidth]{fig010050.png}
   \subcaption{\tiny Closed App}
   \label{fig010050}
   \end{subfigure}
   \caption{Working with the application}
\end{figure}

App Inventor offers a fascinating aspect - the ability to deploy and showcase developed programs directly on mobile devices. This means that creators can share their work and demonstrate their applications even when away from a computer or without internet connectivity.

This unique feature of App Inventor empowers users to take their creations on the go, allowing them to showcase their programs anytime, anywhere. It opens up possibilities for sharing projects with a broader audience, such as friends, family, or even during presentations or events.

By enabling programs to run on mobile devices, App Inventor brings mobility and accessibility to the forefront, extending the reach and impact of the developed applications beyond traditional computing environments. It adds an exciting dimension to the programming experience, allowing creators to demonstrate their work and engage with users in real-world scenarios.
\newpage
%\chapter{Programming Constructs}

Computer programs are made up of a strict sequence of instructions. Such sequences are called an algorithm. We, humans, run various algorithms every day as part of our daily lives. Getting ready and going to school is an algorithm. We wake up, get dressed, do our morning outfit, have breakfast, leave the house, and move to school. A very vivid example of an algorithm is cooking recipes. There are starting products in a recipe, then precise instructions on how to process and mix the products, with a clear idea of the end result. In computer programs, a basic set of instructions make up the means of expression of the corresponding programming language. The charm of block languages is that this basic set of instructions is represented visually in the form of colored blocks. The arrangement of the colored blocks in a strictly defined sequence leads to the creation of small computer programs.

In the case of Scratch, the program has well-defined start and end points. With App Inventor, the approach is slightly different. There, the sequence of instructions that make up the written program is entered in small fragments called events. Events are triggered by various user or operating system actions. In Scratch, we talk about sequential programming; in App Inventor, we talk about event programming. The basic programming constructs in the two programming environments are identical, but there are also some significant differences. To write efficient and reliable programs, knowing the means of expression of our programming environments is essential. 

\section{Programming Constructs in Scratch}

The basic building blocks in Scratch are organized into colored groups (Fig. \ref{fig020001}). This organization helps to navigate faster and use the different blocks more efficiently.

\begin{figure}[H]
   \centering
   \includegraphics[width=1.0\linewidth,height=0.5\linewidth]{fig020001.png}
   \caption{Grouping instructions}
\label{fig020001}
\end{figure}

The most significant block in the program is the block that initiates the execution of the instructions arranged below it. This block has a green flag (Fig. \ref{fig020002}) and defines what will happen after the program is started.

\begin{figure}[H]
   \centering
   \includegraphics[width=1.0\linewidth,height=0.5\linewidth]{fig020002.png}
   \caption{Starting point of the program}
\label{fig020002}
\end{figure}

The program launch block is in the light orange group, designed to react to user events. The exact moment the user wants the program to start its execution is undefined in time, so Scratch must catch an event triggered by the user himself.

The second most significant block ends the program (Fig. \ref{fig020003}). It is located in the dark orange group and has the task of stopping all processes taking place during the execution of the program itself.

\begin{figure}[H]
   \centering
   \includegraphics[width=1.0\linewidth,height=0.5\linewidth]{fig020003.png}
   \caption{Program endpoint}
\label{fig020003}
\end{figure}

The dark orange group contains performance control blocks. These blocks allow the program to take different paths and a group of actions to be repeated many times.

In Scratch, instruction blocks basically control pictures called sprites. Unlike an ordinary computer image, a sprite is a graphic object containing multiple frames showing the character's image in different configurations. Every new Scratch program starts with a single sprite of the orange cat, located at coordinates (x=0,y=0). The workspace is a two-dimensional coordinate system centered at (0,0).

\begin{figure}[H]
   \centering
   \includegraphics[width=1.0\linewidth,height=0.5\linewidth]{fig020004.png}
   \caption{Finish immediately after starting}
\label{fig020004}
\end{figure}

The program does nothing if the start and end blocks are joined (Fig. \ref{fig020004}). Practically, this program ends as soon as it starts. A program that does nothing is entirely pointless. To start something happening, use the blocks in the blue group. The first block instructs the kitten to move 10 steps, and the number of steps can be changed by writing another number inside the block (Fig. \ref{fig020005}).

\begin{figure}[H]
   \centering
   \includegraphics[width=1.0\linewidth,height=0.5\linewidth]{fig020005.png}
   \caption{Moving character}
\label{fig020005}
\end{figure}

The next block in the group instructs the character to rotate a specified number of degrees, clockwise, relative to its own center (Fig. \ref{fig020006}).

\begin{figure}[H]
   \centering
   \includegraphics[width=1.0\linewidth,height=0.5\linewidth]{fig020006.png}
   \caption{Clockwise Rotation}
\label{fig020006}
\end{figure}

Similarly, with the next block in the group, the rotation can be performed counterclockwise (Fig. \ref{fig020007}).

\begin{figure}[H]
   \centering
   \includegraphics[width=1.0\linewidth,height=0.5\linewidth]{fig020007.png}
   \caption{Counterclockwise Rotation}
\label{fig020007}
\end{figure}

The next block in the group allows the character to move to random coordinates or coordinates specified with the mouse (Fig. \ref{fig020008}).

\begin{figure}[H]
   \centering
   \includegraphics[width=1.0\linewidth,height=0.5\linewidth]{fig020008.png}
   \caption{Move to random position}
\label{fig020008}
\end{figure}

The character's movement can also be set by absolute coordinates with a block allowing entering numbers for the abscissa and ordinate axis (Fig. \ref{fig020009}).

\begin{figure}[H]
   \centering
   \includegraphics[width=1.0\linewidth,height=0.5\linewidth]{fig020009.png}
   \caption{Move by absolute coordinates}
\label{fig020009}
\end{figure}

Smooth movement at a predetermined time interval is possible at random coordinates or coordinates specified with the mouse, thanks to the next block in the group (Fig. \ref{fig020010}).

\begin{figure}[H]
   \centering
   \includegraphics[width=1.0\linewidth,height=0.5\linewidth]{fig020010.png}
   \caption{Slide to random position}
\label{fig020010}
\end{figure}

Smooth sliding to predetermined coordinates for a predetermined time interval is possible with the block designed for this purpose (Fig. \ref{fig020011}).

\begin{figure}[H]
   \centering
   \includegraphics[width=1.0\linewidth,height=0.5\linewidth]{fig020011.png}
   \caption{Slide to set coordinates}
\label{fig020011}
\end{figure}

The animated character has an orientation property in the form of an angle. At 90 degrees, the orange cat is looking to the right. To change the character's orientation, a block with the possibility of entering a specific angle is used (Fig. \ref{fig020012}).

\begin{figure}[H]
   \centering
   \includegraphics[width=1.0\linewidth,height=0.5\linewidth]{fig020012.png}
   \caption{Corner Orientation}
\label{fig020012}
\end{figure}

In more complex character control scenarios, sometimes the character needs to follow the mouse pointer. For this purpose, a specific block executes this instruction (Fig. \ref{fig020013}).

\begin{figure}[H]
   \centering
   \includegraphics[width=1.0\linewidth,height=0.5\linewidth]{fig020013.png}
   \caption{Mouse Pointer Orientation}
\label{fig020013}
\end{figure}

Blocks can be placed one after the other, and for sequential change of the relative x and y coordinates (relative to the current position) of the character, there are specially defined blocks (Fig. \ref{fig020014}).

\begin{figure}[H]
   \centering
   \includegraphics[width=1.0\linewidth,height=0.5\linewidth]{fig020014.png}
   \caption{Sequential change of relative coordinates}
\label{fig020014}
\end{figure}

In addition to a relative change of the coordinates, an absolute change of the coordinates is also possible, with the absolute change relative to the center of the coordinate system (Fig. \ref{fig020015}).

\begin{figure}[H]
   \centering
   \includegraphics[width=1.0\linewidth,height=0.5\linewidth]{fig020015.png}
   \caption{Sequential change of absolute coordinates}
\label{fig020015}
\end{figure}

In its movement, when the animated character reaches the boundaries of the workspace, one option is to continue the movement outside the visible area. The other option is to take action and have the character bounce off the edges of the workspace. There is a specific block for this bounce (Fig. \ref{fig020016}). A slightly more complicated sequence of instructions is needed to illustrate its operation. Each time the program is run. First, the relative coordinates are changed, and then an edge bounce is performed if necessary. To make the verification scenario a bit more interesting, instead of fixed relative offset values, an embedding of one of the green blocks are used, allowing for the generation of a random number within a predetermined range. It is important to note that the green block has an oval shape, which suggests it is intended to fit into one of the other blocks with an oval slot.

\begin{figure}[H]
   \centering
   \includegraphics[width=1.0\linewidth,height=0.5\linewidth]{fig020016.png}
   \caption{Bouncing off the edges}
\label{fig020016}
\end{figure}

Next, a handy block from the group of dark orange is the block for waiting a period (Fig. \ref{fig020017}). When this block is placed between the start and end blocks, the program waits the specified number of seconds before stopping execution. During execution, it can be clearly observed that a yellow frame appears around the sequence of instructions, which symbolizes the mode of executing instructions.

\begin{figure}[H]
   \centering
   \includegraphics[width=1.0\linewidth,height=0.5\linewidth]{fig020017.png}
   \caption{Wait instruction}
\label{fig020017}
\end{figure}

The group of purple blocks contains instructions for the outer layout of the animated character. The first two blocks are intended for lines (Fig. \ref{fig020018}) that the character says (spelled as in a comic book). The first block sets the text on the screen until the next instruction. This is precisely why there needs to be a few seconds of waiting so the text remains visible to the user. The second block also has a parameter to determine how many seconds the text should be visible to the user.

\begin{figure}[H]
   \centering
   \includegraphics[width=1.0\linewidth,height=0.5\linewidth]{fig020018.png}
   \caption{Writing cues to speak}
\label{fig020018}
\end{figure}

The second two blocks are intended for lines the animated character thinks but does not say. The difference is in how the text is visualized (Fig. \ref{fig020019}).

\begin{figure}[H]
   \centering
   \includegraphics[width=1.0\linewidth,height=0.5\linewidth]{fig020019.png}
   \caption{Writing lines, as a thought}
\label{fig020019}
\end{figure}

Animated characters in Scratch are in the form of sprites. A sprite is a set of images of the character in different poses. Two blocks are used to change these different poses (Fig. \ref{fig020020}). The first set is a specific frame in the sprite, and the second is the next frame in the sequence.

\begin{figure}[H]
   \centering
   \includegraphics[width=1.0\linewidth,height=0.5\linewidth]{fig020020.png}
   \caption{Changing Poses}
\label{fig020020}
\end{figure}

In addition to the animated characters (sprites), the work scene has a background image. This background image is also subject to change, for which two separate blocks are provided (Fig. \ref{fig020021}). With the first one, background images can be selected forward, backward, randomly, or with a specific name, and with the second block, the next image in the sequence.

\begin{figure}[H]
   \centering
   \includegraphics[width=1.0\linewidth,height=0.5\linewidth]{fig020021.png}
   \caption{Change background}
\label{fig020021}
\end{figure}

There are two specific blocks for resizing the animated character, the first resizing in absolute values and the second resizing in percentages relative to the original size (Fig. \ref{fig020022}).

\begin{figure}[H]
   \centering
   \includegraphics[width=1.0\linewidth,height=0.5\linewidth]{fig020022.png}
   \caption{Resize}
\label{fig020022}
\end{figure}

Three blocks are provided for changing the visual layout of the animated character (Fig. \ref{fig020023}). The first two set a change, which can be in color, various distortions, pixelation, mosaic, transparency, or brightness, and the third block cancels any decorations made. The first block causes a relative change to the character's current state, and the second block sets a fundamental change. Again, giving it a few seconds is essential so that the changes are clearly discernible.

\begin{figure}[H]
   \centering
   \includegraphics[width=1.0\linewidth,height=0.5\linewidth]{fig020023.png}
   \caption{Change appearance}
\label{fig020023}
\end{figure}

Working with sprites is primarily for achieving animated effects. The different animated characters in the scene have specific interactions with each other. The script of the developed project determines at what moment each of the characters appears on the scene and at what moment they disappear. Two blocks performing these actions are provided to carry out the appearance and disappearance (Fig. \ref{fig020024}).

\begin{figure}[H]
   \centering
   \includegraphics[width=1.0\linewidth,height=0.5\linewidth]{fig020024.png}
   \caption{Hide and show}
\label{fig020024}
\end{figure}

Many bitmap software products organize different images into layers. Examples are Adobe Photoshop, GIMP, Microsoft Word, and LibreOffice Draw. The layered organization is logical, as different sprites can overlap at specific points in time. In some of the graphics software packages, layers are perceived as a Z-buffer. In Scratch, the ability to work with layers is also available, with two specific blocks allowing the sprite to move forward and backward through the layers (Fig. \ref{fig020025}).

\begin{figure}[H]
   \centering
   \includegraphics[width=1.0\linewidth,height=0.5\linewidth]{fig020025.png}
   \caption{Navigation through layers}
\label{fig020025}
\end{figure}

The group of blocks in magenta is for sound layout. The performance of sounds is achieved with the first two blocks in the group (Fig. \ref{fig020026}). The first block plays the sound until it is finished, and the second block starts it and passes the playback to the next block. With the third block, all playing sounds are stopped. The software environment also allows sounds to be recorded from the user's computer.

\begin{figure}[H]
   \centering
   \includegraphics[width=1.0\linewidth,height=0.5\linewidth]{fig020026.png}
   \caption{Playing Sounds}
\label{fig020026}
\end{figure}

The pitch (frequency) and stereo (left/right) blocks can change two of the sounds' characteristics. Both blocks have numerical values for the specified characteristics (Fig. \ref{fig020027}).

\begin{figure}[H]
   \centering
   \includegraphics[width=1.0\linewidth,height=0.5\linewidth]{fig020027.png}
   \caption{Sound Characteristics}
\label{fig020027}
\end{figure}

To achieve a richer sound picture, the strength of the different sounds can be controlled with two blocks (Fig. \ref{fig020028}). The former controls volume in absolute value, the latter as percentages.

\begin{figure}[H]
   \centering
   \includegraphics[width=1.0\linewidth,height=0.5\linewidth]{fig020028.png}
   \caption{Volume}
\label{fig020028}
\end{figure}

The orange group of blocks is for events to occur. Events are a tool for executing instructions when there is no explicit cutoff for when program instructions must be performed. Such an event is pressing a button on the keyboard by the user (Fig. \ref{fig020029}).

\begin{figure}[H]
   \centering
   \includegraphics[width=1.0\linewidth,height=0.5\linewidth]{fig020029.png}
   \caption{Key Press Event}
\label{fig020029}
\end{figure}

Mouse clicking on a specific sprite can also be handled using a suitable block (Fig. \ref{fig020030}).

\begin{figure}[H]
   \centering
   \includegraphics[width=1.0\linewidth,height=0.5\linewidth]{fig020030.png}
   \caption{Mouse Click Event}
\label{fig020030}
\end{figure}

Changing the background can also trigger an event to be handled. A block is provided for this purpose (Fig. \ref{fig020031}).

\begin{figure}[H]
   \centering
   \includegraphics[width=1.0\linewidth,height=0.5\linewidth]{fig020031.png}
   \caption{BackgroundChange Event}
\label{fig020031}
\end{figure}

An event can be caught after a particular time has elapsed to a timer or a certain sound level has been reached (Fig. \ref{fig020032}).

\begin{figure}[H]
   \centering
   \includegraphics[width=1.0\linewidth,height=0.5\linewidth]{fig020032.png}
   \caption{Timer or sound event}
\label{fig020032}
\end{figure}

Event handling is also associated with a mechanism for sending/receiving messages. One block of instructions can broadcast a predefined message, and another can subscribe to receive precisely that kind of message (Fig. \ref{fig020033}).

\begin{figure}[H]
   \centering
   \includegraphics[width=1.0\linewidth,height=0.5\linewidth]{fig020033.png}
   \caption{Broadcasting and receiving messages}
\label{fig020033}
\end{figure}

Since working with the messaging engine may require synchronization, it has a separate block that propagates the message and waits for actions to be taken on its interception (Fig. \ref{fig020034}). The programmer can create different messages to be sent in different situations.

\begin{figure}[H]
   \centering
   \includegraphics[width=1.0\linewidth,height=0.5\linewidth]{fig020034.png}
   \caption{Propagating a pending message}
\label{fig020034}
\end{figure}

The most essential and valuable blocks are organized in the dark orange group. These blocks define which execution path to take, given the possible choices for executing instructions. When the desire is to perform a specific action repeatedly, with a set number of repetitions, there is a particular block for this purpose (Fig. \ref{fig020035}). Multiple iterations are accomplished in programming using loop constructs, as with this iteration block.

\begin{figure}[H]
   \centering
   \includegraphics[width=1.0\linewidth,height=0.5\linewidth]{fig020035.png}
   \caption{Fixed number of iterations}
\label{fig020035}
\end{figure}

The word repeat in English means repeat. The number in the pad determines how many repetitions to execute, and the slot in the pad is where the instructions to be repeated are placed. In this example, the kitten is moved to randomly selected coordinates, followed by a predetermined number of seconds to wait. In infrequent situations, there is a need for an infinitely repeating loop, for which a separate block is provided (Fig. \ref{fig020036}).

\begin{figure}[H]
   \centering
   \includegraphics[width=1.0\linewidth,height=0.5\linewidth]{fig020036.png}
   \caption{Infinite Replays}
\label{fig020036}
\end{figure}

The next block is one of the most essential blocks in programming. It is called a conditional execution block (Fig. \ref{fig020037}) or a conditional transition. The block's content is executed only if the condition in its header is fulfilled.

\begin{figure}[H]
   \centering
   \includegraphics[width=1.0\linewidth,height=0.5\linewidth]{fig020037.png}
   \caption{Execution on condition}
\label{fig020037}
\end{figure}

This dark orange block cannot be used alone. It is always paired with at least one green block and sometimes with two, as in the current example. Some of the green blocks are irregular hexagons made to fit into the header of some dark orange blocks. The hexagonal block has an oval slot into which some green oval blocks fit. In the example, a green block is selected, which requires equality to a specific number, and a random number generator is used for the oval block according to a predetermined interval. If the condition in the header of the conditional transition block is not met, then the body is skipped, and the following instructions are after the block is passed. The block for conditional transition also has a variant in which slots are provided for the execution of both possibilities – a true or a false condition (Fig. \ref{fig020038}). If the condition is met, the first instruction block is executed. The second block of instructions is executed if the condition is not met.

\begin{figure}[H]
   \centering
   \includegraphics[width=1.0\linewidth,height=0.5\linewidth]{fig020038.png}
   \caption{Execution on condition with alternative}
\label{fig020038}
\end{figure}

The next exciting block is waiting until a particular event happens. In this case, the event is the sprite being touched with the mouse (Fig. \ref{fig020039}). If this touch happens, the execution of the program continues to the next block. What event is expected is defined by an additional block (light blue) with the shape of an irregular hexagon.

\begin{figure}[H]
   \centering
   \includegraphics[width=1.0\linewidth,height=0.5\linewidth]{fig020039.png}
   \caption{Waiting condition}
\label{fig020039}
\end{figure}

The last three blocks in the dark orange group must be demonstrated together (Fig. \ref{fig020040}). The first block sets a new chain of instructions when a particular sprite is cloned (a copy of the original sprite). The second block serves to clone the current sprite. And the third block serves to delete the current sprite.

\begin{figure}[H]
   \centering
   \includegraphics[width=1.0\linewidth,height=0.5\linewidth]{fig020040.png}
   \caption{Clone Sprites}
\label{fig020040}
\end{figure}

The group of light blue blocks is dedicated to interactions related to the sprite. The second block in the group is intended to fulfill a condition when the sprite touches a particular color. The block has a hexagonal shape, suggesting it is designed for embedding. To demonstrate the operation of this block, a loop will be run that will move the kitten to random coordinates and wait a small time interval before the next move (Fig. \ref{fig020041}).

\begin{figure}[H]
   \centering
   \includegraphics[width=1.0\linewidth,height=0.5\linewidth]{fig020041.png}
   \caption{Cyclical jump of random coordinates}
\label{fig020041}
\end{figure}

A loop made like this will loop endlessly since no end condition is set. In the end condition, I want to place the block defining the touch of color. We will add a new sprite to the scene (Fig. \ref{fig020042}) of a red apple (Fig. \ref{fig020043}), which the kitten must catch. Once he catches her, he will stop prompting and meow (Fig. \ref{fig020044}).

\begin{figure}[H]
   \centering
   \includegraphics[width=1.0\linewidth,height=0.5\linewidth]{fig020042.png}
   \caption{Add sprite}
\label{fig020042}
\end{figure}

\begin{figure}[H]
   \centering
   \includegraphics[width=1.0\linewidth,height=0.5\linewidth]{fig020043.png}
   \caption{Choose sprite from gallery}
\label{fig020043}
\end{figure}

\begin{figure}[H]
   \centering
   \includegraphics[width=1.0\linewidth,height=0.5\linewidth]{fig020044.png}
   \caption{Positioning the apple}
\label{fig020044}
\end{figure}

One of the most common problems when working with sprites is whether two sprites touch or overlap. There are various techniques for detecting collisions between sprites, but one of the most effective is touching a particular color. Modern computers work with just over 16 million different colors. A judicious selection of the characters' colors can give limitless possibilities for detecting collisions. Since the apple is red, the choice to end the cycle is when the kitten touches the red color (Fig. \ref{fig020045}).

\begin{figure}[H]
   \centering
   \includegraphics[width=1.0\linewidth,height=0.5\linewidth]{fig020045.png}
   \caption{Touch by Color}
\label{fig020045}
\end{figure}

With the previous block, no matter which part of the kitten touches the apple, the loop stops spinning, and the meow is heard. A much finer definition of collision between sprites can be obtained if only the black outline of the kitten is checked for touching the red color of the apple, which is what the next block is for (Fig. \ref{fig020046}).

\begin{figure}[H]
   \centering
   \includegraphics[width=1.0\linewidth,height=0.5\linewidth]{fig020046.png}
   \caption{Collision on two preset colors}
\label{fig020046}
\end{figure}

The next block is oval-shaped and supplies the program with the distance between the sprite and the mouse pointer. The oval shape suggests that this block should be embedded in one of the arithmetic expression blocks (Fig. \ref{fig020047}).

\begin{figure}[H]
   \centering
   \includegraphics[width=1.0\linewidth,height=0.5\linewidth]{fig020047.png}
   \caption{Mouse Pointer Distance}
\label{fig020047}
\end{figure}

Sometimes the user needs to type something. To give this possibility is the next block in the light blue group (Fig. \ref{fig020048}). The animated character prompts the user by suggesting in specific text what is expected to be written.

\begin{figure}[H]
   \centering
   \includegraphics[width=1.0\linewidth,height=0.5\linewidth]{fig020048.png}
   \caption{Enter text}
\label{fig020048}
\end{figure}

The next block is one of the hexagonal blocks intended for embedding. This block returns a result of "true" when a particular key is pressed (Fig. \ref{fig020049}).

\begin{figure}[H]
   \centering
   \includegraphics[width=1.0\linewidth,height=0.5\linewidth]{fig020049.png}
   \caption{Defining key pressed}
\label{fig020049}
\end{figure}

Similar behavior can be achieved with the next block, but a mouse key is expected instead of pressing a key on the keyboard (Fig. \ref{fig020050}).

\begin{figure}[H]
   \centering
   \includegraphics[width=1.0\linewidth,height=0.5\linewidth]{fig020050.png}
   \caption{Detect mouse button pressed}
\label{fig020050}
\end{figure}

The following two blocks are oval and are also for embedding. The first gives the coordinates of the animated character along the abscissa axis, and the second provides the coordinates of the animated character along the ordinate axis (Fig. \ref{fig020051}).

\begin{figure}[H]
   \centering
   \includegraphics[width=1.0\linewidth,height=0.5\linewidth]{fig020051.png}
   \caption{Coordinates of the animated character}
\label{fig020051}
\end{figure}

During the program operation, a functioning timer measures the time from the start of execution. With the next block, this timer can be reset (Fig. \ref{fig020052}).

\begin{figure}[H]
   \centering
   \includegraphics[width=1.0\linewidth,height=0.5\linewidth]{fig020052.png}
   \caption{Reset Timer}
\label{fig020052}
\end{figure}

The next block is one of the ovals, providing background information, variables, or sound level (Fig. \ref{fig020053}).

\begin{figure}[H]
   \centering
   \includegraphics[width=1.0\linewidth,height=0.5\linewidth]{fig020053.png}
   \caption{Scene Component Information}
\label{fig020053}
\end{figure}

The last block in the group is also intended for embedding and returns the number of days from the year 2000 (Fig. \ref{fig020054}).

\begin{figure}[H]
   \centering
   \includegraphics[width=1.0\linewidth,height=0.5\linewidth]{fig020054.png}
   \caption{Number of days since the beginning of the century}
\label{fig020054}
\end{figure}

The group of green blocks is intended for embedding. The first four blocks are oval and intended for arithmetic operations – addition, subtraction, multiplication, and division (Fig. \ref{fig020055}).

\begin{figure}[H]
   \centering
   \includegraphics[width=1.0\linewidth,height=0.5\linewidth]{fig020055.png}
   \caption{Arithmetic operations}
\label{fig020055}
\end{figure}

The random number block has already been demonstrated, but it fits perfectly into the three following blocks. These comparison blocks are intended to be embedded in the execution control blocks (Fig. \ref{fig020056}).

\begin{figure}[H]
   \centering
   \includegraphics[width=1.0\linewidth,height=0.5\linewidth]{fig020056.png}
   \caption{Comparison Operations}
\label{fig020056}
\end{figure}

Next are three hexagonal-shaped blocks (Fig. \ref{fig020057}), which serve to embed in control blocks. The three blocks perform the three basic logical operations ("and", "or", "not"). Both conditions must be met in the first block to enter the conditional transition construct. Precisely for this reason, the logical operation is called "and". One condition must be met in the second block to enter the conditional transition construct. For this reason, the logical operation is called "or". In the third block, the result is reversed, so the conditional transition construct is entered under a false condition. For this reason, this operation is called "negation".

\begin{figure}[H]
   \centering
   \includegraphics[width=1.0\linewidth,height=0.5\linewidth]{fig020057.png}
   \caption{Boolean operations}
\label{fig020057}
\end{figure}

The following four blocks are for working with character strings (Fig. \ref{fig020058}). The first three are oval in shape, and the last is hexagonal in form. The first block concatenates two character strings. The second block specifies a letter at a particular position in the character string. The third block specifies the length of the character string. The fourth block searches for a specific letter in the character string.

\begin{figure}[H]
   \centering
   \includegraphics[width=1.0\linewidth,height=0.5\linewidth]{fig020058.png}
   \caption{Working with character strings}
\label{fig020058}
\end{figure}

The last three blocks in the green group are intended for working with functions (Fig. \ref{fig020059}). The first block calculates the remainder of the integer division. The second block rounds a fractional number to its whole part. The third block offers the calculation of an entire list of mathematical functions.

\begin{figure}[H]
   \centering
   \includegraphics[width=1.0\linewidth,height=0.5\linewidth]{fig020059.png}
   \caption{Mathematical Functions}
\label{fig020059}
\end{figure}

The last group of blocks is the dark orange group (Fig. \ref{fig020060}). They are designed to work with variables. When writing programs, it is often necessary to save intermediate calculated results temporarily and use them for subsequent calculations. This is achieved through the variables. Variables are temporary containers that store their assigned values. The first block in the group establishes the variable's value. The second block in the group changes the variable's value. The third block in the group serves for program visualization of the variable. The last box in the group helps to hide the preview.

\begin{figure}[H]
   \centering
   \includegraphics[width=1.0\linewidth,height=0.5\linewidth]{fig020060.png}
   \caption{Working with variables}
\label{fig020060}
\end{figure}

Now that all the most important constructs in the Scratch programming environment have been introduced, one can move on to writing more complex programs, appropriately combining the basic building blocks.

\section{Programming Constructs in App Inventor}

A significant difference between App Inventor and Scratch is that App Inventor does not use sprites but builds a graphical user interface. This is because App Inventor takes a classic approach to writing Android apps. This difference makes it necessary to consider two types of expressions in App Inventor: the GUI components and the programming blocks for building a series of instructions.

Building an application in App Inventor starts on a new, blank screen (Fig. \ref{fig020061}). Screens are scenes, and the program's work moves from scene to scene. When the program is something straightforward, it can be realized as only one scene.

\begin{figure}[H]
   \centering
   \includegraphics[width=1.0\linewidth,height=0.5\linewidth]{fig020061.png}
   \caption{Opening Scene}
\label{fig020061}
\end{figure}

\subsection{Graphical Interface}

GUI components are organized into groups, just as instruction blocks are arranged. Most visual components have a graphical layout directly on the screen, but some are not visualized. An example of non-renderable components is layout management managers. These managers are represented in the second group, and their function is to serve as grouping components that arrange the visually presented components.

A hierarchical structure of the positioned graphic components is presented on the right of the working scene. Components can be deleted or renamed in this panel. On the far right is a panel with the characteristics of the currently selected graphic component. Components have different features, which can be established while designing the interface.

The first group includes the main components for building a graphical user interface. The first component in this group is the button (Fig. \ref{fig020062}). Placing it in the workspace of the scene is done by selecting with the mouse and dragging it to the workspace. The button has characteristics related to the text on the component itself, the ability to place an image, dimensions, shape, font size, background, and foreground colors, and others.

\begin{figure}[H]
   \centering
   \includegraphics[width=1.0\linewidth,height=0.5\linewidth]{fig020062.png}
   \caption{Button Graphical Component}
\label{fig020062}
\end{figure}

The button is followed by a marking component (Fig. \ref{fig020063}), which has similar functionality to the button, but the on or off state is marked. It is often used to denote properties. The most important characteristic of this component is whether it is in the established state or in the disabled state.

\begin{figure}[H]
   \centering
   \includegraphics[width=1.0\linewidth,height=0.5\linewidth]{fig020063.png}
   \caption{Graphical ticker component}
\label{fig020063}
\end{figure}

Entering dates by the user is a process that can lead to many errors. This is because different months have different lengths, and the month of February is determined by leap years and whether the corresponding leap year is a multiple of four hundred. To avoid date entry errors, Android offers a visual component for controlled date entry (Fig. \ref{fig020064}).

\begin{figure}[H]
   \centering
   \includegraphics[width=1.0\linewidth,height=0.5\linewidth]{fig020064.png}
   \caption{Graphic component for entering dates}
\label{fig020064}
\end{figure}

The date input component may have a different presentation in different versions or proprietary modifications of the Android operating system. One possibility is a counter with three segments for day, month, and year (Fig. \ref{fig020065}).

\begin{figure}[H]
   \centering
   \includegraphics[width=1.0\linewidth,height=0.5\linewidth]{fig020065.png}
   \caption{Enter date}
\label{fig020065}
\end{figure}

The following visual component has the sole task of displaying an image (Fig. \ref{fig020066}). This is also the most essential characteristic in the characteristics panel for the component.

\begin{figure}[H]
   \centering
   \includegraphics[width=1.0\linewidth,height=0.5\linewidth]{fig020066.png}
   \caption{Graphic component for images}
\label{fig020066}
\end{figure}

Next is the label, a text field with no possibility for the user to change the text content (Fig. \ref{fig020067}).

\begin{figure}[H]
   \centering
   \includegraphics[width=1.0\linewidth,height=0.5\linewidth]{fig020067.png}
   \caption{Label Graphical Component}
\label{fig020067}
\end{figure}

In the next component, selecting from a list of character strings is possible. A comma is used as a separator between strings (Fig. \ref{fig020068}).

\begin{figure}[H]
   \centering
   \includegraphics[width=1.0\linewidth,height=0.5\linewidth]{fig020068.png}
   \caption{Selectable Graphical Component}
\label{fig020068}
\end{figure}

Each option is visualized on a separate line (Fig. \ref{fig020069}).

\begin{figure}[H]
   \centering
   \includegraphics[width=1.0\linewidth,height=0.5\linewidth]{fig020069.png}
   \caption{List options}
\label{fig020069}
\end{figure}

In the list view component, a separate cell is provided for each option (Fig. \ref{fig020070}).

\begin{figure}[H]
   \centering
   \includegraphics[width=1.0\linewidth,height=0.5\linewidth]{fig020070.png}
   \caption{List widget}
\label{fig020070}
\end{figure}

The next component is one of the components that need to be visualized at design time. Used to display notifications (Fig. \ref{fig020071}).

\begin{figure}[H]
   \centering
   \includegraphics[width=1.0\linewidth,height=0.5\linewidth]{fig020071.png}
   \caption{Notification widget}
\label{fig020071}
\end{figure}

To be visualized, it is necessary to add several instructions to the intercepted event so that during execution, the written texts are displayed (Fig. \ref{fig020072}). The interception is for the back button pressed event when the app shows the first scene.

\begin{figure}[H]
   \centering
   \includegraphics[width=1.0\linewidth,height=0.5\linewidth]{fig020072.png}
   \caption{A series of instructions to display a notification}
\label{fig020072}
\end{figure}

During the preview, the popup dialog can be dismissed (Fig. \ref{fig020073}) because the cancel option is enabled.

\begin{figure}[H]
   \centering
   \includegraphics[width=1.0\linewidth,height=0.5\linewidth]{fig020073.png}
   \caption{Notification window}
\label{fig020073}
\end{figure}

Password input fields look like regular text input fields, but the difference is that when typing, the characters are not visible but are replaced by asterisks (Fig. \ref{fig020074}).

\begin{figure}[H]
   \centering
   \includegraphics[width=1.0\linewidth,height=0.5\linewidth]{fig020074.png}
   \caption{Graphical component for entering passwords}
\label{fig020074}
\end{figure}

With a slider component, the two most important characteristics are the minimum and maximum values that the component can take. The slider serves to visualize a position on a linear scale (Fig. \ref{fig020075}).

\begin{figure}[H]
   \centering
   \includegraphics[width=1.0\linewidth,height=0.5\linewidth]{fig020075.png}
   \caption{Position Graphical Component}
\label{fig020075}
\end{figure}

At the next component, choices are given, again as an enumerated list of character strings (Fig. \ref{fig020076}).

\begin{figure}[H]
   \centering
   \includegraphics[width=1.0\linewidth,height=0.5\linewidth]{fig020076.png}
   \caption{Selectable Graphical Component}
\label{fig020076}
\end{figure}

The options' visual presentation differs from those presented in the previous components (Fig. \ref{fig020077}).

\begin{figure}[H]
   \centering
   \includegraphics[width=1.0\linewidth,height=0.5\linewidth]{fig020077.png}
   \caption{Selection via radio buttons}
\label{fig020077}
\end{figure}

The key type component is an alternative to the check box component (Fig. \ref{fig020078}). The most important characteristic of this component is the state it is in - on or off.

\begin{figure}[H]
   \centering
   \includegraphics[width=1.0\linewidth,height=0.5\linewidth]{fig020078.png}
   \caption{Switch widget}
\label{fig020078}
\end{figure}

The text field is a component that enters text from the user (Fig. \ref{fig020079}).

\begin{figure}[H]
   \centering
   \includegraphics[width=1.0\linewidth,height=0.5\linewidth]{fig020079.png}
   \caption{Text input widget}
\label{fig020079}
\end{figure}

By analogy with the date input component, a time input component is also available (Fig. \ref{fig020080}).

\begin{figure}[H]
   \centering
   \includegraphics[width=1.0\linewidth,height=0.5\linewidth]{fig020080.png}
   \caption{Time input widget}
\label{fig020080}
\end{figure}

One of its possible implementations takes the form of three fields, two for scrolling up/down and one for specifying morning or afternoon (Fig. \ref{fig020081}).

\begin{figure}[H]
   \centering
   \includegraphics[width=1.0\linewidth,height=0.5\linewidth]{fig020081.png}
   \caption{Choose a time}
\label{fig020081}
\end{figure}

The most feature-rich component is the last in the group and is an entire web browser (Fig. \ref{fig020082}).

\begin{figure}[H]
   \centering
   \includegraphics[width=1.0\linewidth,height=0.5\linewidth]{fig020082.png}
   \caption{Web Browser Graphical Component}
\label{fig020082}
\end{figure}

Entire web pages (Fig. \ref{fig020083}) can be loaded into this component, including those that require JavaScript interactivity.

\begin{figure}[H]
   \centering
   \includegraphics[width=1.0\linewidth,height=0.5\linewidth]{fig020083.png}
   \caption{Loading Web Page}
\label{fig020083}
\end{figure}

The second group of visual components organizes the graphical user interface and are containers for the components with a visual representation. This mechanism for managing the graphical user interface was proposed with the first graphic user interface libraries offered with the Java programming language. This organization aims to make the graphical user interface suitable for devices with different screen sizes. Visual components are arranged according to the available area and the containers' rules.

With the first component in the group, the visual components are arranged horizontally, hence its name (Fig. \ref{fig020084}).

\begin{figure}[H]
   \centering
   \includegraphics[width=1.0\linewidth,height=0.5\linewidth]{fig020084.png}
   \caption{Horizontal stacking container}
\label{fig020084}
\end{figure}

In the first container, if the visual components go outside the user's visible field of operation, they cannot be reached. For this reason, the second container provides scrolling capabilities (horizontally) so that visual components that go outside the work area can be reached (Fig. \ref{fig020085}).

\begin{figure}[H]
   \centering
   \includegraphics[width=1.0\linewidth,height=0.5\linewidth]{fig020085.png}
   \caption{Horizontal stacking container with slider}
\label{fig020085}
\end{figure}

The third container in the group allows the visual components to be arranged as a table with rows and columns (Fig. \ref{fig020086}).

\begin{figure}[H]
   \centering
   \includegraphics[width=1.0\linewidth,height=0.5\linewidth]{fig020086.png}
   \caption{Table arrangement container}
\label{fig020086}
\end{figure}

By analogy with the container for horizontal stacking, a container for vertical stacking is also provided (Fig. \ref{fig020087}). In it, visual components are stacked on top of each other.

\begin{figure}[H]
   \centering
   \includegraphics[width=1.0\linewidth,height=0.5\linewidth]{fig020087.png}
   \caption{Vertical stack container}
\label{fig020087}
\end{figure}

In case of insufficient working space along the vertical axis, it is also possible to use a container with the possibility of sliding (Fig. \ref{fig020088}).

\begin{figure}[H]
   \centering
   \includegraphics[width=1.0\linewidth,height=0.5\linewidth]{fig020088.png}
   \caption{Vertical stacking container with slider}
\label{fig020088}
\end{figure}

The slider appears on the container's borders but disappears when there is no sliding, so it only takes up a little visual space (Fig. ef {fig020089}).

\begin{figure}[H]
   \centering
   \includegraphics[width=1.0\linewidth,height=0.5\linewidth]{fig020089.png}
   \caption{Slide content into container}
\label{fig020089}
\end{figure}

A significant advantage of containers is that they can be nested within other containers (Fig. \ref{fig020090}). By appropriately arranging the different embeddings, a graphical user interface layout that looks good on devices with different screen sizes can be achieved.

\begin{figure}[H]
   \centering
   \includegraphics[width=1.0\linewidth,height=0.5\linewidth]{fig020090.png}
   \caption{Inserting containers}
\label{fig020090}
\end{figure}

There is a multimedia group after the component group is used to arrange the visible components. In this group, components have no graphical representation at design time but no visual representation at runtime. Two components are an exception. The first is an image selection component, and the second is a video display component (Fig. \ref{fig020091}).

\begin{figure}[H]
   \centering
   \includegraphics[width=1.0\linewidth,height=0.5\linewidth]{fig020091.png}
   \caption{Multimedia Group}
\label{fig020091}
\end{figure}

The group of multimedia components provides programming capabilities to perform specific tasks: video recording, photo recording, sound file playback, sound management, sound file recording, speech recognition, speech synthesis, and machine translation between spoken languages. The complexity of the components in this group prevents their easy demonstration, but some of them will be used in the following examples.

After the multimedia group comes the animation group (Fig. \ref{fig020092}). When writing games, the concept of the canvas (Canvas) and moving animated characters (Sprites) are often used. The familiar Scratch sprites appear here, too, but in an exceptional case.

\begin{figure}[H]
   \centering
   \includegraphics[width=1.0\linewidth,height=0.5\linewidth]{fig020092.png}
   \caption{Animation Group}
\label{fig020092}
\end{figure}

Generally, a canvas is a two-dimensional matrix of colored dots (pixels) on which various two-dimensional primitives or bitmaps with a transparency channel are drawn. It is important to note that sprites cannot be placed independently but must be below the canvas hierarchy.

Since the Android operating system is primarily implemented on mobile devices, and they very often have GPS sensors, the next group of components provides opportunities for working with geographic maps and geolocation (Fig. \ref{fig020093}).

\begin{figure}[H]
   \centering
   \includegraphics[width=1.0\linewidth,height=0.5\linewidth]{fig020093.png}
   \caption{Geolocation Group}
\label{fig020093}
\end{figure}

Analogous to the drawing canvas, this group also has a primary map visualization component, which can contain graphic primitives such as circles, feature selection, lines, markers, polygons, and rectangles. There is also a component that does not have a preview but serves to enable map navigation functionality. Map rendering is done in layers, which allows graphics primitives to be added above the map rendering layer itself.

Different mobile devices have different set of hardware sensors (Fig. \ref{fig020094}). Sensors are parts of the device that collect information from the external environment. In the next group of components, it is possible to program work with different types of sensors, such as an accelerometer, barcode reader, pressure sensor, clock, spatial orientation sensor, humidity sensor, illumination sensor, a location sensor, magnetic field strength, proximity sensor, spatial orientation sensor, pedometer, object proximity sensor, and thermometer.

\begin{figure}[H]
   \centering
   \includegraphics[width=1.0\linewidth,height=0.5\linewidth]{fig020094.png}
   \caption{Sensor Working Group}
\label{fig020094}
\end{figure}

All components in the group have no visual representation and are used through program constructs. Working with the hardware and its sensors requires considerable skill and is beyond the scope of this presentation.

The next group presents components that are related to social contacts. The first component allows the selection of a person from the contact list (Fig. \ref{fig020095}). The contact list saves information about various people with whom the user communicates.

\begin{figure}[H]
   \centering
   \includegraphics[width=1.0\linewidth,height=0.5\linewidth]{fig020095.png}
   \caption{Contact Selector Graphical Component}
\label{fig020095}
\end{figure}

Next is a component for entering an e-mail address (Fig. \ref{fig020096}). E-mail addresses have a strictly fixed format, which must be followed when the user enters.

\begin{figure}[H]
   \centering
   \includegraphics[width=1.0\linewidth,height=0.5\linewidth]{fig020096.png}
   \caption{Email input widget}
\label{fig020096}
\end{figure}

The phone call initiation component is for programmatic use and has no visual representation (Fig. \ref{fig020097}).

\begin{figure}[H]
   \centering
   \includegraphics[width=1.0\linewidth,height=0.5\linewidth]{fig020097.png}
   \caption{Phone Call Component}
\label{fig020097}
\end{figure}

Next is a component for selecting a phone number from the contact list (Fig. \ref{fig020098}).

\begin{figure}[H]
   \centering
   \includegraphics[width=1.0\linewidth,height=0.5\linewidth]{fig020098.png}
   \caption{Graphic component for selecting a phone number}
\label{fig020098}
\end{figure}

The last three components have no visual representation and serve to share information, send text messages, and post to Twitter (Fig. \ref{fig020099}). These components are intended for programmatic use only and enable applications within the operating system.

\begin{figure}[H]
   \centering
   \includegraphics[width=1.0\linewidth,height=0.5\linewidth]{fig020099.png}
   \caption{Information Sharing Component}
\label{fig020099}
\end{figure}

Next is a group of components without visual representation. This group has the task of storing the information between separate program starts (Fig. \ref{fig020100}). The first component stores the information on a remote cloud service. An address to the remote server is provided for this purpose. The second component serves to work with files on the local drive. The third component serves to store structured information between separate program launches. The storage is on the local drive and can be likened to variables saved after the program is stopped. Using the web services mechanism, the latter component stores information on a remote server.

\begin{figure}[H]
   \centering
   \includegraphics[width=1.0\linewidth,height=0.5\linewidth]{fig020100.png}
   \caption{Information storage component}
\label{fig020100}
\end{figure}

The next group of components is responsible for communication connectivity (Fig. \ref{fig020101}). All components have no visual representation and are intended for programmatic use. The first component is used to open the next screen, the way it happens in Android programs. The second component adds client-side Bluetooth functionality. The third component adds server-side Bluetooth functionality. The fourth component enables serial communication with devices such as Arduino. The last component in the group allows web-based communication without rendering, as is the case with the web browser component.

\begin{figure}[H]
   \centering
   \includegraphics[width=1.0\linewidth,height=0.5\linewidth]{fig020101.png}
   \caption{Communication Connectivity Component}
\label{fig020101}
\end{figure}

One of the largest groups of components is for working with Lego Mindstorms (Fig. \ref{fig020102}). This series from the Lego company is designed for children interested in robotics. Ince the topic of robotics falls outside the scope of this presentation, these components will not be discussed.

\begin{figure}[H]
   \centering
   \includegraphics[width=1.0\linewidth,height=0.5\linewidth]{fig020102.png}
   \caption{Lego Mindstorms Component}
\label{fig020102}
\end{figure}

The group of experimental components includes only a component for working with a Firebase database (Fig. \ref{fig020103}).

\begin{figure}[H]
   \centering
   \includegraphics[width=1.0\linewidth,height=0.5\linewidth]{fig020103.png}
   \caption{Experimental Components}
\label{fig020103}
\end{figure}

The graphical user interface in the Android operating system is designed so that third-party manufacturers of visual components can add them in the form of libraries. This option is also available in App Inventor as the last group in the component groups panel.

\subsection{Program Constructs}

Unlike Scratch, App Inventor has many more blocks, as each GUI component has multiple event-handling capabilities and accordingly offers slots for nesting block constructs. For this reason, only the main blocks will be considered, and the rest will be partially demonstrated in the subsequent exposition.

Basic blocks in App Inventor have identical functionality to blocks in Scratch. Visually, they are shaped differently, but the idea is the same – the blocks follow or are built into each other. For the demonstration of most blocks, one button and one instance of the notification component will be used (Fig. \ref{fig020104}). The button press event is the ideal slot to place the demonstrated constructs.

\begin{figure}[H]
   \centering
   \includegraphics[width=1.0\linewidth,height=0.5\linewidth]{fig020104.png}
   \caption{A minimal interface for demonstrating block constructions}
\label{fig020104}
\end{figure}

The block constructions are also arranged in a workspace specially set aside for this purpose (Fig. \ref{fig020105}).

\begin{figure}[H]
   \centering
   \includegraphics[width=1.0\linewidth,height=0.5\linewidth]{fig020105.png}
   \caption{Workspace for block structures}
\label{fig020105}
\end{figure}

The blocks are again organized into colored groups, which will be arranged in the button-pressed event slot (Fig. \ref{fig020106}).

\begin{figure}[H]
   \centering
   \includegraphics[width=1.0\linewidth,height=0.5\linewidth]{fig020106.png}
   \caption{Groups of colored blocks}
\label{fig020106}
\end{figure}

First is the colored group of brown blocks, which controls the performance. It starts with the familiar conditional transition block (Fig. \ref{fig020107}).

\begin{figure}[H]
   \centering
   \includegraphics[width=1.0\linewidth,height=0.5\linewidth]{fig020107.png}
   \caption{Conditional transition block}
\label{fig020107}
\end{figure}

If the condition in the transition construct evaluates to true, then a notification display is called in the block's body by embedding a purple block (Fig. \ref{fig020108}) from the list of blocks in the notifications component.

\begin{figure}[H]
   \centering
   \includegraphics[width=1.0\linewidth,height=0.5\linewidth]{fig020108.png}
   \caption{View Notification}
\label{fig020108}
\end{figure}

The bolded text of the notification is written in a magenta block and embedded in the notification preview block (Fig. \ref{fig020109}).

\begin{figure}[H]
   \centering
   \includegraphics[width=1.0\linewidth,height=0.5\linewidth]{fig020109.png}
   \caption{Notification text}
\label{fig020109}
\end{figure}

Next is the formation of the header part of the conditional transition construction. A blue block is placed next to the title slot, where the transition condition will be entered (Fig. \ref{fig020110}).

\begin{figure}[H]
   \centering
   \includegraphics[width=1.0\linewidth,height=0.5\linewidth]{fig020110.png}
   \caption{Transition block header}
\label{fig020110}
\end{figure}

A blue box on the left side of the condition expression generates a random number in a set interval (Fig. \ref{fig020111}).

\begin{figure}[H]
   \centering
   \includegraphics[width=1.0\linewidth,height=0.5\linewidth]{fig020111.png}
   \caption{Left side of condition expression}
\label{fig020111}
\end{figure}

On the right side in the condition, there is a blue block with an exact predefined value (Fig. \ref{fig020112}). That way, the caption will be displayed on some button presses, and on others, it won't.

\begin{figure}[H]
   \centering
   \includegraphics[width=1.0\linewidth,height=0.5\linewidth]{fig020112.png}
   \caption{Right side of condition expression}
\label{fig020112}
\end{figure}

When the random number is below the set threshold, the notification is displayed for a short interval and then disappears (Fig. \ref{fig020113}).

\begin{figure}[H]
   \centering
   \includegraphics[width=1.0\linewidth,height=0.5\linewidth]{fig020113.png}
   \caption{Notification Preview}
\label{fig020113}
\end{figure}

The second block in the brown group is for a conditional transition, executing a block construction when the condition is met but another construction when the condition is not (Fig. \ref{fig020114}).

\begin{figure}[H]
   \centering
   \includegraphics[width=1.0\linewidth,height=0.5\linewidth]{fig020114.png}
   \caption{Conditional transition block and alternative}
\label{fig020114}
\end{figure}

The third block in the brown group represents a cascade for conditional transitions (Fig. \ref{fig020115}). More than one condition is checked.

\begin{figure}[H]
   \centering
   \includegraphics[width=1.0\linewidth,height=0.5\linewidth]{fig020115.png}
   \caption{Conditional transition cascade block}
\label{fig020115}
\end{figure}

The next block in the brown group is a step loop block (Fig. \ref{fig020116}). The variable's value is taken via an orange block at each loop turn and displayed as a notification.

\begin{figure}[H]
   \centering
   \includegraphics[width=1.0\linewidth,height=0.5\linewidth]{fig020116.png}
   \caption{Step loop block}
\label{fig020116}
\end{figure}

The next block in the brown group is for looping over the elements of a list structure (Fig. \ref{fig020117}). The purple block is handy for forming a list, which divides a character string into substrings according to a predefined delimiter.

\begin{figure}[H]
   \centering
   \includegraphics[width=1.0\linewidth,height=0.5\linewidth]{fig020117.png}
   \caption{List loop block}
\label{fig020117}
\end{figure}

Next is a block in the brown group to loop over the elements of a "dictionary" type structure (Fig. \ref{fig020118}). This type of structure is also known as an "associative array". To access the elements, the key value does not have to be a number but can be a character string, for example. The key is used to access the items. A dark blue block is used to create the dictionary. Some blocks have a small gear in the upper left corner. This wheel is an icon that expands to a block setting menu. In this case, the number of slots for key-value pairs is determined through the setting. The orange blocks take the contents of the two variables local to the loop. One variable contains the key, and the other contains the value corresponding to that key. A purple string concatenation block forms the text message displayed in the notification component.

\begin{figure}[H]
   \centering
   \includegraphics[width=1.0\linewidth,height=0.5\linewidth]{fig020118.png}
   \caption{Vocabulary loop block}
\label{fig020118}
\end{figure}

The next block implements a loop with a precondition of type "while" (Fig. \ref{fig020119}). In this loop, the iteration termination condition precedes the loop body. For execution control, creating an external variable via an orange variable initialization block is necessary. In this case, the variable is initialized with a random value. In the loop header, a check is made for the variable's value, and a decision is made on whether the loop should continue running. The variable's value is visualized in the notifications component, and then, with an appropriate orange block, a new random value is selected. The loop stops spinning when the value in the variable drops below the preset threshold.

\begin{figure}[H]
   \centering
   \includegraphics[width=1.0\linewidth,height=0.5\linewidth]{fig020119.png}
   \caption{For loop block with precondition}
\label{fig020119}
\end{figure}

The next block has the meaning of a ternary operation in the Java programming language and resembles the conditional transition construction with an alternative (Fig. \ref{fig020120}). If the condition evaluates to true, the result returned is the first possibility. If it evaluates to "false", the result returned is the second possibility.

\begin{figure}[H]
   \centering
   \includegraphics[width=1.0\linewidth,height=0.5\linewidth]{fig020120.png}
   \caption{Ternary operation block}
\label{fig020120}
\end{figure}

The next block executes a series of other blocks and returns a result (Fig. \ref{fig020121}). This case uses a blue block that generates a random fractional number in the range of zero to one without including the unit.

\begin{figure}[H]
   \centering
   \includegraphics[width=1.0\linewidth,height=0.5\linewidth]{fig020121.png}
   \caption{Instruction grouping block}
\label{fig020121}
\end{figure}

The next block executes the instructions attached to it but ignores the resulting result (Fig. \ref{fig020122}). This block is useful when calling a function that returns a result, but the result is unnecessary.

\begin{figure}[H]
   \centering
   \includegraphics[width=1.0\linewidth,height=0.5\linewidth]{fig020122.png}
   \caption{Block to execute instructions with no result}
\label{fig020122}
\end{figure}

The next block opens a new screen (Fig. \ref{fig020123}); for this purpose, a second screen must be added to the project.

\begin{figure}[H]
   \centering
   \includegraphics[width=1.0\linewidth,height=0.5\linewidth]{fig020123.png}
   \caption{Open new screen block}
\label{fig020123}
\end{figure}

The new screen should be given a service name (Fig. \ref{fig020124}). Each screen has its own set of visual components and its own set of program constructs.

\begin{figure}[H]
   \centering
   \includegraphics[width=1.0\linewidth,height=0.5\linewidth]{fig020124.png}
   \caption{Screen Naming}
\label{fig020124}
\end{figure}

The second screen uses the same concept: one button and one notification component (Fig. \ref{fig020125}).

\begin{figure}[H]
   \centering
   \includegraphics[width=1.0\linewidth,height=0.5\linewidth]{fig020125.png}
   \caption{Second screen user interface}
\label{fig020125}
\end{figure}

In the initialization event of the second screen, the text is displayed by using the notification component in the second screen (Fig. \ref{fig020126}).

\begin{figure}[H]
   \centering
   \includegraphics[width=1.0\linewidth,height=0.5\linewidth]{fig020126.png}
   \caption{Notification when opening the second screen}
\label{fig020126}
\end{figure}

The next block opens a new screen, passing a value to the newly opened screen (Fig. \ref{fig020127}).

\begin{figure}[H]
   \centering
   \includegraphics[width=1.0\linewidth,height=0.5\linewidth]{fig020127.png}
   \caption{Open parameter passing screen}
\label{fig020127}
\end{figure}

The next block takes the value with which the screen was started (Fig. \ref{fig020128}).

\begin{figure}[H]
   \centering
   \includegraphics[width=1.0\linewidth,height=0.5\linewidth]{fig020128.png}
   \caption{Value the screen is started with}
\label{fig020128}
\end{figure}

The next block closes the screen (Fig. \ref{fig020129}). In this case, the closing is performed when the button is pressed on the second screen.

\begin{figure}[H]
   \centering
   \includegraphics[width=1.0\linewidth,height=0.5\linewidth]{fig020129.png}
   \caption{Screen close block}
\label{fig020129}
\end{figure}

The next block closes the screen, returning a result (Fig. \ref{fig020130}). The different screens can exchange information using the startup value and the returned slenderness.

\begin{figure}[H]
   \centering
   \includegraphics[width=1.0\linewidth,height=0.5\linewidth]{fig020130.png}
   \caption{Screen close block with return value}
\label{fig020130}
\end{figure}

The next block closes the entire program (Fig. \ref{fig020131}).

\begin{figure}[H]
   \centering
   \includegraphics[width=1.0\linewidth,height=0.5\linewidth]{fig020131.png}
   \caption{Program close block}
\label{fig020131}
\end{figure}

The next block gives the starting value as text (Fig. \ref{fig020132}).

\begin{figure}[H]
   \centering
   \includegraphics[width=1.0\linewidth,height=0.5\linewidth]{fig020132.png}
   \caption{Text of the value with which the screen is started}
\label{fig020132}
\end{figure}

With the next block, the screen is closed, and text is sent as the return value (Fig. \ref{fig020133}).

\begin{figure}[H]
   \centering
   \includegraphics[width=1.0\linewidth,height=0.5\linewidth]{fig020133.png}
   \caption{Screen close block with text value return}
\label{fig020133}
\end{figure}

The last block in the group of browns serves for emergency interruption of rotating cycles (Fig. \ref{fig020134}). If the termination condition of a loop is always false, then the loop becomes infinite, and then the break block is the only way to stop the loop.

\begin{figure}[H]
   \centering
   \includegraphics[width=1.0\linewidth,height=0.5\linewidth]{fig020134.png}
   \caption{Emergency Loop Break Block}
\label{fig020134}
\end{figure}

The group of brown blocks is followed by the group of green blocks. All blocks in this group are intended for embedding and represent the set of basic logic operations. The first two boxes set the "true" and "false" constants. Logical operations are the basis of Boolean algebra, where everything boils down to "true" or "false" expressions. The third block in the group is the negation operation. If the argument of this operation is "true", then its result is "false". If the argument is "false", the result is "true". The fourth block in the group performs the compare/difference operation of two boolean values. For comparison, if they are equal, then the result is "true" and vice versa, and for difference, if they are different, then the result is "true" and vice versa. The fifth block in the group is the "and" operation. In this operation, both operands must be true for the result of the operation to be true. The last block in the group is the "or" operation. In this operation, at least one of the two operands must be true for the result of the operation to be true. The "and" and "or" operations allow setting blocks, where the setting is the addition of more operands.

\begin{figure}[H]
   \centering
   \includegraphics[width=1.0\linewidth,height=0.5\linewidth]{fig020135.png}
   \caption{Boolean operation blocks}
\label{fig020135}
\end{figure}

The green block group is followed by the blue block group, which contains mathematical operations and functions. Some of the blocks have already been used, so the presentation will be for those that have yet to come into use. Blocks in this group are designed primarily for embedding. At the beginning are the blocks for arithmetic operations (Fig. \ref{fig020136}). Integers can be represented in several different number systems, such as decimal, binary, octal, and hexadecimal. The first of the presented blocks allows this representation in the other number systems. Then come the addition, subtraction, multiplication, division, and exponentiation operations.

\begin{figure}[H]
   \centering
   \includegraphics[width=1.0\linewidth,height=0.5\linewidth]{fig020136.png}
   \caption{Blocks for arithmetic operations}
\label{fig020136}
\end{figure}

The following sub-group of blue blocks performs some more special mathematical operations (Fig. \ref{fig020137}). The first block enables logical operations to be performed, but bit by bit. This means that the corresponding logical operation is applied in pairs of bits, according to the binary representation of the two numbers that are operands of the operation. The second block feeds the random number generator with an initial value. The most commonly used random number generators in computers are essentially mathematical formulas. This formula starts its calculation from an explicitly set, preset value. Through the power supply block of the random generator, the exact execution of the range of random numbers can be achieved at different starts of the program. In actual practice, the initial value is taken from the system clock. The third of the blocks presented defines a minimum/maximum value. This block can also be parameterized, allowing comparison of more than two values. The fourth block determines how many decimal places to present if a fractional number is present. The fifth of the blocks presented checks for a number or number system of the number. The last of the blocks transform an integer into a specified number system.

\begin{figure}[H]
   \centering
   \includegraphics[width=1.0\linewidth,height=0.5\linewidth]{fig020137.png}
   \caption{Number Operations Blocks}
\label{fig020137}
\end{figure}

The last subgroup of blue blocks represents a set of mathematical functions (Fig. \ref{fig020138}). The first of these is for the square root function. The second block gives the absolute value of the number. The third box provides a negative value of the number. The fourth block gives mathematical rounding of a fractional number to a whole number. The fifth block gives an upper integer value. The sixth block gives a lower integer value. The seventh block is intended for the remainder of a division. The eighth block is a sine function. The ninth block is a cosine function. The ninth block is a tangent function. The last block calculates an angle in degrees at given coordinates using the arctangent function.

\begin{figure}[H]
   \centering
   \includegraphics[width=1.0\linewidth,height=0.5\linewidth]{fig020138.png}
   \caption{Math Function Blocks}
\label{fig020138}
\end{figure}

Instructions for working with character strings are organized in the purple group of blocks. Some blocks have already been used to illustrate previous examples, so they will not be presented again. The first subset of blocks (Fig. \ref{fig020139}) do the following: determine length, determine if a string is empty, lexicographic comparison of strings, trim strings (remove leading and trailing blank characters), transform to lowercase /uppercase string, index of a substring, search for a substring, determine if an object is a string, and reverse letters in a string.

\begin{figure}[H]
   \centering
   \includegraphics[width=1.0\linewidth,height=0.5\linewidth]{fig020139.png}
   \caption{Blocks for basic string operations}
\label{fig020139}
\end{figure}

The second subset of blocks (Fig. \ref{fig020140}) do the following: split a string by a specified delimiter, split a string by a space character, cut a substring, replace a substring with a string, recode a string, and replace substrings by a list of strings.

\begin{figure}[H]
   \centering
   \includegraphics[width=1.0\linewidth,height=0.5\linewidth]{fig020140.png}
   \caption{Blocks for more complex string operations}
\label{fig020140}
\end{figure}

The group of light blue blocks is for working with list data structures. Lists are data containers that are ordered and of variable length. Lists can contain heterogeneous elements, unlike most arrays. Some of the blocks in the group are considered together with other groups of blocks and are therefore not represented in the following examples. The first subset of light blue blocks (Fig. \ref{fig020141}) do the following: create an empty list (saved as a variable), add values to the list, check for an item in the list, length of the list, check for an empty list, selecting a random element from the list, index of an element in the list and selecting an element by index.

\begin{figure}[H]
   \centering
   \includegraphics[width=1.0\linewidth,height=0.5\linewidth]{fig020141.png}
   \caption{Blocks for basic list operations}
\label{fig020141}
\end{figure}

The second subset of the blue blocks is for manipulations with the elements of the lists (Fig. \ref{fig020142}). The actions that can be performed with these blocks are as follows: insert an element at a given index, replace an element at a given index, and remove an element at a given index.

\begin{figure}[H]
   \centering
   \includegraphics[width=1.0\linewidth,height=0.5\linewidth]{fig020142.png}
   \caption{List element manipulation blocks}
\label{fig020142}
\end{figure}

The following subset of blue blocks is for working with more than one list (Fig. \ref{fig020143}). The actions that can be performed with them are as follows: copy a list, add one list to another list, reverse the elements of the list, convert the list to text from a row in CSV format, convert the list to text from a table in CSV format, loading list from a row in CSV format and loading list from a table in CSV format.

\begin{figure}[H]
   \centering
   \includegraphics[width=1.0\linewidth,height=0.5\linewidth]{fig020143.png}
   \caption{Blocks for working with more than one list}
\label{fig020143}
\end{figure}

The last two blocks in the light blue group are for searching for a pair (pairs from the dictionary group) by key and concatenating the list elements using a predefined separator (Fig. \ref{fig020144}).

\begin{figure}[H]
   \centering
   \includegraphics[width=1.0\linewidth,height=0.5\linewidth]{fig020144.png}
   \caption{Blocks for searching pairs by key and concatenation of elements}
\label{fig020144}
\end{figure}

The group of dark blue blocks is a group for working with dictionary-type structures. Dictionary-type structures are analogous to associative arrays. Their characteristic is that the information is organized in key-value pairs. The key serves as the address to access the value. Keys can be different data types but don't have to be numbers. The first subset of dark blue blocks represents the basic actions (Fig. \ref{fig020145}) that can be performed with dictionaries as follows: create a dictionary, represent a key-value pair, create an empty dictionary, access a value by specified key, set value by specified key, remove value by specified key, retrieve all keys, retrieve all values, and check for key in a dictionary.

\begin{figure}[H]
   \centering
   \includegraphics[width=1.0\linewidth,height=0.5\linewidth]{fig020145.png}
   \caption{Blocks for basic dictionary operations}
\label{fig020145}
\end{figure}

The second subset of dark blue blocks serves for slightly more complex dictionary operations (Fig. \ref{fig020146}), as follows: dictionary size, transform a list of pairs to a dictionary, transform a dictionary to a list of pairs, copy dictionary, merging dictionaries and checking if an object is a dictionary.

\begin{figure}[H]
   \centering
   \includegraphics[width=1.0\linewidth,height=0.5\linewidth]{fig020146.png}
   \caption{Blocks for more complex dictionary operations}
\label{fig020146}
\end{figure}

Next in order is the group of gray blocks intended for working with colors. In the first subgroup, the predefined values of some of the base colors are presented (Fig. \ref{fig020147}). The example selects a random item from a list structure with the colors and sets that color as the button's background.

\begin{figure}[H]
   \centering
   \includegraphics[width=1.0\linewidth,height=0.5\linewidth]{fig020147.png}
   \caption{Work blocks with predefined colors}
\label{fig020147}
\end{figure}

The last two blocks in the gray group are for color composition and decomposition (Fig. \ref{fig020148}). The colors on the computer screen are formed from three basic components - red, green, and blue. Each component has 256 values, generating just over 16 million colors. In addition to the three color components, it is possible to use another fourth value that sets the level of transparency and can be specified in the range from 0 to 255.

\begin{figure}[H]
   \centering
   \includegraphics[width=1.0\linewidth,height=0.5\linewidth]{fig020148.png}
   \caption{Blocks for composing and decomposing color}
\label{fig020148}
\end{figure}

The group of orange blocks is for working with variables. Of this group, only the block for a variable on a return value from a procedure is not represented (Fig. \ref{fig020149}). This block will be presented along with the purple blocks used to work with procedures.

\begin{figure}[H]
   \centering
   \includegraphics[width=1.0\linewidth,height=0.5\linewidth]{fig020149.png}
   \caption{Blocks for working with patterns}
\label{fig020149}
\end{figure}

Procedures are small pieces of code (an assembly of blocks) that are called and, in some cases, return a result. Procedures can receive input parameters and can have an output parameter as a return value. The first block in the purple group creates a procedure with no return value. The second block creates a procedure that returns a value. A block appears for each procedure the user develops, allowing it to be called.

\section{Program Code Design}

With programming languages, it is of great importance that the programmer knows the capabilities of the language well. This knowledge includes the expressive means of the language as well as the available libraries provided by the manufacturer. With this knowledge and a great deal of creative effort, any programmer can effectively create programs. It is not without reason that software engineering is classified as a type of engineering activity. This is because well-written programs are an arrangement of the small building blocks that development environments offer. After familiarizing yourself with Scratch and App Inventor's expressions, you can create programs that perform the tasks set by the programmer.


\newpage
%\chapter{Click and Win}

In this project, the two characters move forward when the player clicks the blue or red button. The faster the player clicks the button, the faster their character moves forward. The first player to reach the green finish line wins.

\begin{figure}[H]
   \centering
   \includegraphics[width=1.0\linewidth,height=0.5\linewidth]{fig030001.png}
   \caption{Click and win}
\label{fig030001}
\end{figure}

\section{Adding Background and Characters}

Building the game starts with choosing a background. From the Backdrops->Choose a Backdrop section. A suitable one can be selected from the available ones that Scratch provides.

\begin{figure}[H]
   \centering
   \includegraphics[width=1.0\linewidth,height=0.5\linewidth]{fig030002.png}
   \caption{Choosing a suitable background for the game}
\label{fig030002}
\end{figure}

In this dumbbell, the background needs to be enhanced to make the race course along with the green finish line. For this purpose, the Backdrops option is first selected.

\begin{figure}[H]
   \centering
   \includegraphics[width=1.0\linewidth,height=0.5\linewidth]{fig030003.png}
   \caption{Drawing additional elements on the background}
\label{fig030003}
\end{figure}

Using the line tool, the race track is added. If the thickness and color of the line are changed, the finish line can also be added.

\begin{figure}[H]
   \centering
   \includegraphics[width=1.0\linewidth,height=0.5\linewidth]{fig030004.png}
   \caption{Final Game Background}
\label{fig030004}
\end{figure}

If the game does not need the main character in the Scratch cat, he can be deleted.

\begin{figure}[H]
   \centering
   \includegraphics[width=1.0\linewidth,height=0.5\linewidth]{fig030005.png}
   \caption{Delete main character}
\label{fig030005}
\end{figure}

The characters should also be added to the game. There are many sprites available in Scratch. For this game, two are needed - one positioned on the left and one on the right (Fig. \ref{fig030006}). The Size and Direction properties change the size and direction of the character.

\begin{figure}[H]
   \centering
   \includegraphics[width=1.0\linewidth,height=0.5\linewidth]{fig030006.png}
   \caption{In-Game Characters}
\label{fig030006}
\end{figure}

In addition to these two sprites, two more are needed: the buttons the players must click on. They are found again in the Sprite section. In this game, the buttons must be a different color to differentiate them. To change the color of a button, it must change its suit.

\begin{figure}[H]
   \centering
   \includegraphics[width=1.0\linewidth,height=0.5\linewidth]{fig030007.png}
   \caption{Blue Button}
\label{fig030007}
\end{figure}

Using the Fill tool, change the color of the button (Fig. \ref{fig030008}). The same can be done for the red button. Again, the Size property can change the size of this sprite.

\begin{figure}[H]
   \centering
   \includegraphics[width=1.0\linewidth,height=0.5\linewidth]{fig030008.png}
   \caption{Red Button}
\label{fig030008}
\end{figure}

\section{Blue Button Programming}
When the player clicks the blue button, the message "blue" should be sent. The first starting block to be placed is when this sprite is clicked.

\begin{figure}[H]
   \centering
   \includegraphics[width=1.0\linewidth,height=0.5\linewidth]{fig030009.png}
   \caption{When the character is clicked}
\label{fig030009}
\end{figure}

Next, the character sends a "blue" message. From the dark orange group, the message propagation instruction. The message to spread is "blue".

\begin{figure}[H]
   \centering
   \includegraphics[width=1.0\linewidth,height=0.5\linewidth]{fig030010.png}
   \caption{Send Message}
\label{fig030010}
\end{figure}

The code for this character looks like this:

\begin{figure}[H]
   \centering
   \includegraphics[width=1.0\linewidth,height=0.5\linewidth]{fig030011.png}
   \caption{The entire blue button code}
\label{fig030011}
\end{figure}

\section{Programming the Red Button}

When the player clicks on the red button, similarly to the blue one, it should send a "red" message. The code for the red button is as follows:

\begin{figure}[H]
   \centering
   \includegraphics[width=1.0\linewidth,height=0.5\linewidth]{fig030012.png}
   \caption{All red button code}
\label{fig030012}
\end{figure}

Up to this point, the program consists of when the player presses the blue or red button, they send the corresponding messages.

\section{Programming characters to move}
The instructions on the blue button are that it sends a message. The left character must subscribe to receive the message.

\begin{figure}[H]
   \centering
   \includegraphics[width=1.0\linewidth,height=0.5\linewidth]{fig030013.png}
   \caption{Subscribe to the message from the blue button}
\label{fig030013}
\end{figure}

The character must move right to the green finish line, which means the x coordinate must be changed, increasing it by 3 steps.

\begin{figure}[H]
   \centering
   \includegraphics[width=1.0\linewidth,height=0.5\linewidth]{fig030014.png}
   \caption{Move Character Right}
\label{fig030014}
\end{figure}

The instructions for the other character are similar. The main differences are two:
- the message this character subscribes to was sent by the red button
- the character must move left towards the green finish line, which means the x coordinate must be changed, decreasing it by 3 steps

\begin{figure}[H]
   \centering
   \includegraphics[width=1.0\linewidth,height=0.5\linewidth]{fig030015.png}
   \caption{Move Character Left}
\label{fig030015}
\end{figure}

\section{Programming the winner}
To complete the game, check which characters have reached the green finish line.

In the right character, the first instruction to add is the initial to start the game. At each moment of the game, it must be checked whether the character has reached the finish line. For this reason, a forever loop instruction must be added.

\begin{figure}[H]
   \centering
   \includegraphics[width=1.0\linewidth,height=0.5\linewidth]{fig030016.png}
   \caption{Cycle Forever}
\label{fig030016}
\end{figure}

The check should be inside the loop's body to see if the character has touched the finish line.

\begin{figure}[H]
   \centering
   \includegraphics[width=1.0\linewidth,height=0.5\linewidth]{fig030017.png}
   \caption{Checking if the character has reached finish}
\label{fig030017}
\end{figure}

The eyedropper tool should be used to select the same green color as the finish line.

\begin{figure}[H]
   \centering
   \includegraphics[width=1.0\linewidth,height=0.5\linewidth]{fig030018.png}
   \caption{Choose a color}
\label{fig030018}
\end{figure}

The instructions inside the condition will be executed when the character wins. Then he should increase his size and write a message "I won!".

\begin{figure}[H]
   \centering
   \includegraphics[width=1.0\linewidth,height=0.5\linewidth]{fig030019.png}
   \caption{Instructions for victory}
\label{fig030019}
\end{figure}

The last improvement needed to complete this character is to be placed in the starting position every time the game starts. In the blue section is the instruction that tells where to position the character.

\begin{figure}[H]
   \centering
   \includegraphics[width=1.0\linewidth,height=0.5\linewidth]{fig030020.png}
   \caption{Hero Positioning Instructions}
\label{fig030020}
\end{figure}

The final code of this character is:

\begin{figure}[H]
   \centering
   \includegraphics[width=1.0\linewidth,height=0.5\linewidth]{fig030021.png}
   \caption{Final code of left character}
\label{fig030021}
\end{figure}

Once that character is ready, it remains to add the instructions for defeating the other character. The code is similar. The only difference is in the starting coordinates.

\begin{figure}[H]
   \centering
   \includegraphics[width=1.0\linewidth,height=0.5\linewidth]{fig030022.png}
   \caption{Final code of the right character}
\label{fig030022}
\end{figure}

It's time to have fun with friends and compete to see which character will reach the finish line faster.
\newpage
%\chapter{Bouquet for Mom}

The ultimate goal of this project is for the children to create a bouquet for their mothers. When the player clicks on this spot, a beautiful flower will appear. To get the most beautiful bouquet, the player can change the type of flowers using the up arrow. Use the left and right arrows to adjust the size of the flower.

\begin{figure}[H]
   \centering
   \includegraphics[width=1.0\linewidth,height=0.5\linewidth]{fig040001.png}
   \caption{Bouquet for Mom}
\label{fig040001}
\end{figure}

\section{Adding Background and Characters}
The first step in creating the game is adding a suitable background. From the Backdrops->Choose a Backdrop section. A suitable one can be selected from the available ones that Scratch provides.

In this game, there will be no need for the main character in Scratch the cat. For this, he must be deleted.

\begin{figure}[H]
   \centering
   \includegraphics[width=1.0\linewidth,height=0.5\linewidth]{fig040002.png}
   \caption{Game Background}
\label{fig040002}
\end{figure}

The character representing a flower petal created when the player clicks on the screen should be added. This sprite must be drawn using the tools. More than one costume can be added for this character to make the game more interesting. Each of the suits will represent a different petal.

\begin{figure}[H]
   \centering
   \includegraphics[width=1.0\linewidth,height=0.5\linewidth]{fig040003.png}
   \caption{Drawing Petal Character}
\label{fig040003}
\end{figure}

\section{Drawing Flowers}

When the player clicks on the background, a flower will be drawn. This means that instructions should be placed on the background when clicked to send a "draw" message.

\begin{figure}[H]
   \centering
   \includegraphics[width=1.0\linewidth,height=0.5\linewidth]{fig040004.png}
   \caption{Background Instructions}
\label{fig040004}
\end{figure}

A new instruction section should be added to draw the flowers and stems. This is the Pen section.

\begin{figure}[H]
   \centering
   \includegraphics[width=1.0\linewidth,height=0.5\linewidth]{fig040005.png}
   \caption{Add Pen Section}
\label{fig040005}
\end{figure}

When the game starts, the petal character must be hidden, and everything drawn up to that point must be erased. The instruction that erases everything on the screen is located in the new Pen section and is "erase all".

\begin{figure}[H]
   \centering
   \includegraphics[width=1.0\linewidth,height=0.5\linewidth]{fig040006.png}
   \caption{Start of the game}
\label{fig040006}
\end{figure}

Before drawing the flower itself, its companion will be drawn first. Drawing the handle will start when the "draw" message is received. First, the thickness of the pencil that will be drawn and the color will be set. First, the pencil must be positioned. The initial x coordinate is the same as the mouse's. The instruction in the light blue group gives the mouse position for x. The initial y coordinate should be -150. Once the character is positioned, the pencil should be instructed to come down to start drawing. To complete the flower's stem, the character must go to where the mouse is and pick up the pencil.

The result of the following code (Fig. \ref{fig040007}) is that the flower will be drawn when the player clicks on different places on the screen.

\begin{figure}[H]
   \centering
   \includegraphics[width=1.0\linewidth,height=0.5\linewidth]{fig040007.png}
   \caption{Drawing the flower stems}
\label{fig040007}
\end{figure}

The drawn character must leave traces, like a stamp, to get the flower. Each time it leaves a trail, it must rotate and decrease its size by one. This algorithm must be repeated. In this way, the effect of the flowers is achieved. To obtain different flowers, the iterations of the drawing algorithm will be a random number between 40 and 60.

\begin{figure}[H]
   \centering
   \includegraphics[width=1.0\linewidth,height=0.5\linewidth]{fig040008.png}
   \caption{Drawing the flowers}
\label{fig040008}
\end{figure}

The result so far is that when the player clicks on different places on the screen, the flowers for mom will appear.

\section{Changing the type and size of flowers}

To make the game more interesting, using the up arrow will change the character's costumes. So the more suits there are, i.e., the more petals there are, the more diverse the bouquet will be.

\begin{figure}[H]
   \centering
   \includegraphics[width=1.0\linewidth,height=0.5\linewidth]{fig040009.png}
   \caption{Character Costume Changes}
\label{fig040009}
\end{figure}

Another improvement that can be made is to change the size of the flowers. For this purpose, the first thing to be done is a variable to hold the character's size. A characteristic of variables is that they have an initial value, and that value can be changed. In this case, the initial value of the variable will be 100. Pressing the left arrow will decrease the variable by 10, and pressing the right arrow will increase the variable by 10. This is what the final program code looks like:

\begin{figure}[H]
   \centering
   \includegraphics[width=1.0\linewidth,height=0.5\linewidth]{fig040010.png}
   \caption{Full program code}
\label{fig040010}
\end{figure}

The children must present the most beautiful bouquet to their mothers!
\newpage
%\chapter{Hungarian Eight}

\begin{figure}[H]
   \centering
   \includegraphics[width=1.0\linewidth,height=0.5\linewidth]{fig050001.png}
   \caption{"Hungarian Eight" \\ https://www.sfu.ca/~jtmulhol/math302/images/pic-puzzle-hr.png}
\label{fig050001}
\end{figure}

The game "Hungarian eight" (Fig. \ref{fig050001}) belongs to the group of logical puzzles, as does the Rubik's cube. It consists of two intersecting circles, chutes filled with colored balls. The circles intersect at two points, so two of the balls belong to both chutes. The balls in the chutes can be rotated so that the puzzle is scrambled. After the shuffle, the object of the game is to arrange the marbles in the original state.

\section{Designing the interface}

Although the game looks relatively simple, its arrangement involves relative complexity and quite a bit of mathematics. The organization of the puzzle itself is not particularly complex, which makes it an ideal candidate for implementation in a programming environment such as Scratch. With only thirty-eight dots and four arrows, the entire game interface can be built. Work begins with the initiation of a new project (Fig. \ref{fig050002}).

\begin{figure}[H]
   \centering
   \includegraphics[width=1.0\linewidth,height=0.5\linewidth]{fig050002.png}
   \caption{Creating a "Hungarian Eight" project}
\label{fig050002}
\end{figure}

The first step is to choose a project name and accordingly clear the workspace of the cat sprite (Fig. \ref{fig050003}).

\begin{figure}[H]
   \centering
   \includegraphics[width=1.0\linewidth,height=0.5\linewidth]{fig050003.png}
   \caption{Choose a name for the project}
\label{fig050003}
\end{figure}

Placing the colored balls (in this case polka dots) is a relatively labor-intensive task, as two intersecting circles must be described. The work can be significantly facilitated if you work according to a previously prepared scheme (Fig. \ref{fig050004}). The polka dot pattern will only be visible until the rest of the sprites are lined up. In game mode, the scheme will be made invisible.

\begin{figure}[H]
   \centering
   \includegraphics[width=1.0\linewidth,height=0.5\linewidth]{fig050004.png}
   \caption{Game diagram \\ https://www.sfu.ca/~jtmulhol/math302/images/hungarianrings-labeled-nocolor.png}
\label{fig050004}
\end{figure}

The graphics file for the game layout is added to the sprite group, using the add icon (Fig. \ref{fig050005}).

\begin{figure}[H]
   \centering
   \includegraphics[width=1.0\linewidth,height=0.5\linewidth]{fig050005.png}
   \caption{Add the image for the game outline}
\label{fig050005}
\end{figure}

The newly added sprite is centered at coordinates x=0 and y=0 (Fig. \ref{fig050006}) to achieve visual symmetry and the images of the balls to be added are in the middle of the visual space.

\begin{figure}[H]
   \centering
   \includegraphics[width=1.0\linewidth,height=0.5\linewidth]{fig050006.png}
   \caption{Center the diagram}
\label{fig050006}
\end{figure}

From the gallery of pre-available sprites (Fig. \ref{fig050007}) you can choose a suitable sprite for the arrows that will cause the rotation of the two circles.

\begin{figure}[H]
   \centering
   \includegraphics[width=1.0\linewidth,height=0.5\linewidth]{fig050007.png}
   \caption{Gallery of pre-available sprites}
\label{fig050007}
\end{figure}

The arrow sprite has four states (Fig. \ref{fig050008}) that allow this sprite to be used for rotation directions.

\begin{figure}[H]
   \centering
   \includegraphics[width=1.0\linewidth,height=0.5\linewidth]{fig050008.png}
   \caption{Arrow Sprite}
\label{fig050008}
\end{figure}

The first arrow is located at the top right (Fig. \ref{fig050009}), and it will serve to rotate the right ring clockwise.

\begin{figure}[H]
   \centering
   \includegraphics[width=1.0\linewidth,height=0.5\linewidth]{fig050009.png}
   \caption{Arrow to rotate right ring clockwise}
\label{fig050009}
\end{figure}

The second arrow is located at the bottom right (Fig. \ref{fig050010}), and it will serve to rotate the right ring in a counter-clockwise direction.

\begin{figure}[H]
   \centering
   \includegraphics[width=1.0\linewidth,height=0.5\linewidth]{fig050010.png}
   \caption{Arrow to rotate right ring counterclockwise}
\label{fig050010}
\end{figure}

The third arrow is positioned at the bottom-left by selecting the second frame in the sprite as active so that the arrow points to the left (Fig. \ref{fig050011}).

\begin{figure}[H]
   \centering
   \includegraphics[width=1.0\linewidth,height=0.5\linewidth]{fig050011.png}
   \caption{Left ring clockwise rotation arrow}
\label{fig050011}
\end{figure}

The fourth arrow is located at the top-left (Fig. \ref{fig050012}), and its task is to cause the left ring to rotate in the counterclockwise direction.

\begin{figure}[H]
   \centering
\includegraphics[width=1.0\linewidth,height=0.5\linewidth]{fig050012.png}
   \caption{Left ring counter-clockwise rotation arrow}
\label{fig050012}
\end{figure}

One way to visualize the balls from the original game is through a ball sprite (Fig. \ref{fig050013}). This sprite allows the ball to be rendered in several different colors, which perfectly fits the need to render four different colored balls.

\begin{figure}[H]
   \centering
   \includegraphics[width=1.0\linewidth,height=0.5\linewidth]{fig050013.png}
   \caption{Choosing a sprite for the dots}
\label{fig050013}
\end{figure}

The first checker placed goes to the position marked with number one on the already included pattern. In order to successfully overlay the different sprites, it is necessary to send the schematic sprite to the back of the Z-buffer so that all other sprites are rendered in front of it.

The ball is shrunk (in this case to 55) and then adjusted with the mouse to fit exactly on the space marked with a unit (Fig. \ref{fig050014}).

\begin{figure}[H]
   \centering
   \includegraphics[width=1.0\linewidth,height=0.5\linewidth]{fig050014.png}
   \caption{Sizing and positioning the first bead}
\label{fig050014}
\end{figure}

\section{Data Structures}

When the yoke starts (Fig. \ref{fig050015}), the state of the playing field will be reflected in a list structure. The numbers one through four will reflect what color ball should be visualized in the corresponding numbered position.

\begin{figure}[H]
   \centering
   \includegraphics[width=1.0\linewidth,height=0.5\linewidth]{fig050015.png}
   \caption{Start of game in scene program field}
\label{fig050015}
\end{figure}

For this purpose, a "state" list is created and the numbers defining the colors of the polka dots will be written into it.

\begin{figure}[H]
   \centering
   \includegraphics[width=1.0\linewidth,height=0.5\linewidth]{fig050016.png}
   \caption{List of game board status}
\label{fig050016}
\end{figure}

The contents of the list are always completely deleted first so that no values from a previous run remain. The first five positions are of the first color (Fig. \ref{fig050017}).

\begin{figure}[H]
   \centering
   \includegraphics[width=1.0\linewidth,height=0.5\linewidth]{fig050017.png}
   \caption{Color of first five positions}
\label{fig050017}
\end{figure}

The fourth color appears in the sixth position, then from the seventh to the sixteenth position is the second color. Four positions of the first suit follow, and from twenty one to thirty are positions of the third suit. All other checkers are of the fourth color (Fig. \ref{fig050018}).

\begin{figure}[H]
   \centering
   \includegraphics[width=1.0\linewidth,height=0.5\linewidth]{fig050018.png}
   \caption{Overall status by color}
\label{fig050018}
\end{figure}

After a change in the internal state of the list, it is important to send an update message to all sprites that render the polka dots (Fig. \ref{fig050019}).

\begin{figure}[H]
   \centering
   \includegraphics[width=1.0\linewidth,height=0.5\linewidth]{fig050019.png}
   \caption{Preview Message}
\label{fig050019}
\end{figure}

The message is sent with a command that waits for its execution (Fig. \ref{fig050020}). Each of the 38 beads will subscribe to receive this message.

\begin{figure}[H]
   \centering
   \includegraphics[width=1.0\linewidth,height=0.5\linewidth]{fig050020.png}
   \caption{Sending on hold}
\label{fig050020}
\end{figure}

Before starting to copy the first pool so that it is multiplied another 37 times, the code that will listen for a draw message should be compiled. First this code is written so that it is multiplied 37 times when duplicating the sprite. A pool with number one makes four checks on list item number one. According to the number in the list, one of the four possible colors is selected (Fig. \ref{fig050021}).

\begin{figure}[H]
   \centering
   \includegraphics[width=1.0\linewidth,height=0.5\linewidth]{fig050021.png}
   \caption{Instructions for redrawing the pool}
\label{fig050021}
\end{figure}

The first pool thus prepared can be duplicated and spread over the entire scheme, taking into account the colors of the individual positions (Fig. \ref{fig050022}).

\begin{figure}[H]
   \centering
   \includegraphics[width=1.0\linewidth,height=0.5\linewidth]{fig050022.png}
   \caption{Duplicate pool}
\label{fig050022}
\end{figure}

When arranging all 38 checkers, on the game diagram, the two rings are clearly formed (Fig. \ref{fig050023}).

\begin{figure}[H]
   \centering
   \includegraphics[width=1.0\linewidth,height=0.5\linewidth]{fig050023.png}
   \caption{Duplicate pool}
\label{fig050023}
\end{figure}

The game plan, at this stage, is only an aid (Fig. \ref{fig050024}), when the numbering is removed, the same image can be used for the background of both checker rings.

\begin{figure}[H]
   \centering
   \includegraphics[width=1.0\linewidth,height=0.5\linewidth]{fig050024.png}
   \caption{Remove Scheme}
\label{fig050024}
\end{figure}

\section{Algorithms for Manipulation of the Playing Field}

Pressing the first arrow (top-right) first propagates a message to perform a rotation in the right ring, clockwise (Fig. \ref{fig050025}). A message is then propagated to refresh the entire visual space.

\begin{figure}[H]
   \centering
   \includegraphics[width=1.0\linewidth,height=0.5\linewidth]{fig050025.png}
   \caption{Rotation and drawing message}
\label{fig050025}
\end{figure}

In an absolutely analogous way, messages are sent from the other three hands, with the messages indicating the ring and the direction of rotation. After the initial initialization of the color list, a preview follows, then a small interval is given so that the user can see the initial state, and a message is sent to shuffle the puzzle (Fig. \ref{fig050026}).

\begin{figure}[H]
   \centering
   \includegraphics[width=1.0\linewidth,height=0.5\linewidth]{fig050026.png}
   \caption{Send shuffle message}
\label{fig050026}
\end{figure}

Shuffling the puzzle can happen in many ways, but the most successful is by randomly calling the four rotation possibilities. This algorithm will also be assembled in the main scene space, not as code for some of the sprites. The algorithm starts upon receiving the shuffle message. Since the checkers are 38 in number, on average statistically each one can be given a chance to move 10 times. This suggests that the total number of random moves can be determined to be 380, which is 38 times 10 (Fig. \ref{fig050027}).

\begin{figure}[H]
   \centering
   \includegraphics[width=1.0\linewidth,height=0.5\linewidth]{fig050027.png}
   \caption{Shuffling Algorithm}
\label{fig050027}
\end{figure}

The ring rotation instructions will also be placed in the stage space. Each of the four arrows sends an appropriate message. The intercepted rotation message should change the contents of the list to reflect the desired rotation. To get the permutations right, the game board is very helpful because it tells you which number should go where.

The rotation of the left ring clockwise is performed with the following sequence of actions (Fig. \ref{fig050028}).

\begin{figure}[H]
   \centering
   \includegraphics[width=1.0\linewidth,height=0.5\linewidth]{fig050028.png}
   \caption{Left ring clockwise rotation instructions}
\label{fig050028}
\end{figure}

The rotation of the left ring, counter-clockwise is performed with the following sequence of actions (Fig. \ref{fig050029}).

\begin{figure}[H]
   \centering
   \includegraphics[width=1.0\linewidth,height=0.5\linewidth]{fig050029.png}
   \caption{Instructions for rotating the left ring counter-clockwise}
\label{fig050029}
\end{figure}

The rotation of the right ring, clockwise is performed with the following sequence of actions (Fig. \ref{fig050030}).

\begin{figure}[H]
   \centering
   \includegraphics[width=1.0\linewidth,height=0.5\linewidth]{fig050030.png}
   \caption{Instructions for rotating the right ring clockwise}
\label{fig050030}
\end{figure}

The rotation of the right ring, counterclockwise, is performed with the following sequence of actions (Fig. \ref{fig050031}).

\begin{figure}[H]
   \centering
   \includegraphics[width=1.0\linewidth,height=0.5\linewidth]{fig050031.png}
   \caption{Instructions for rotating the right ring counterclockwise}
\label{fig050031}
\end{figure}

\section{Publish the project}

After the execution of the last instructions, the game takes on its finished form. Of course, it is possible to continue development in the direction of ordering algorithms, but this task is far beyond the scope of the present exposition. Using the "SHARE" button, the game is published to the general audience (Fig. \ref{fig050032}).

\begin{figure}[H]
   \centering
   \includegraphics[width=1.0\linewidth,height=0.5\linewidth]{fig050032.png}
   \caption{Share the completed project}
\label{fig050032}
\end{figure}

The game has its finished look, but it is not yet shaped as a finished, final product. It would be nice to add functionality to arrange the puzzle. Help information is missing. It is possible to add counting time for stacking. All of the above brings additional complexity, which is beyond the scope of this presentation, but is an opportunity for additional exercise on the part of the readers.
\newpage
%\chapter{Играта 15}

\begin{figure}[H]
  \centering
  \includegraphics[width=1.0\linewidth,height=0.5\linewidth]{fig060001.png}
  \caption{„Играта 15“ \\ http://www.murderousmaths.co.uk/games/loyd/15 puzzle wood.gif}
\label{fig060001}
\end{figure}

„Играта 15“ (Фиг. \ref{fig060001}) е детски пъзел от групата игри „пътешествие по граф“. Игралното поле е оформено в 4x4 клетки, като в него са поместени 15 плочки. Плочките са номерирани и шестнадесетата позиция е празна. Празната позиция служи като буфер в който могат да се местят съседните плочки. С помощта на буферната клетка, играта се разбърква и целта е плочките да бъдат подредени според началната номерация. 

\section{Структуриране на графичния интерфейс}

Това е детска игра, подреждането на която не е особено трудно и сложност създава единствено последния ред в пъзела. Относително простичката организация на играта я прави идеална за реализация като App Inventor приложение. С помощта само на 16 бутона може да се изгради целият нужен интерфейс. Изработката започва със създаването на нов проект (Фиг. \ref{fig060002}).

\begin{figure}[H]
  \centering
  \includegraphics[width=1.0\linewidth,height=0.5\linewidth]{fig060002.png}
  \caption{Стартиране на нов проект за „Играта 15“}
\label{fig060002}
\end{figure}

Тъй като ще се използват 15 номерирани бутона и един без номерация, то най-удачно е те да бъдат организирани с мениджър на разположението от тип таблица, с 4 реда и 4 колони (Фиг. \ref{fig060003}).

\begin{figure}[H]
  \centering
  \includegraphics[width=1.0\linewidth,height=0.5\linewidth]{fig060003.png}
  \caption{Мениджър на съдържанието с 4x4 клетки}
\label{fig060003}
\end{figure}

Следва поставяне на 16 бутона (Фиг. \ref{fig060004}), като нечетните числа се оцветяват в червено, а четните числа в синьо. Оцветяването подсилва визуалния ефект на играта. Шестнадесетият бутон вместо надпис има два празни интервала, така че ширината му да съвпада с ширината на другите бутони. Без допълнителни настройки, ширината на бутоните се определя от броя символи в надписите им. 

\begin{figure}[H]
  \centering
  \includegraphics[width=1.0\linewidth,height=0.5\linewidth]{fig060004.png}
  \caption{Поставяне на 16 бутона}
\label{fig060004}
\end{figure}

Възможно е да се разменят самите бутони, когато се натисне бутон в съседство на празната клетка, но значително по-лесно е да се разменят надписите и цветовете на бутоните, а самите бутони да остават винаги на първоначалните си места. Натискането на бутона ще се прихваща в общо събитие за всички бутони, но след натискането трябва да се определи дали в съседство е празната клетка. Най-елегантно съседството може да бъде установено, ако структура от тип речник съдържа всички бутони като ключове (Фиг. \ref{fig060005}), а като стойности списъчни структури със съседите.

\begin{figure}[H]
  \centering
  \includegraphics[width=1.0\linewidth,height=0.5\linewidth]{fig060005.png}
  \caption{Бутоните като ключови стойности}
\label{fig060005}
\end{figure}

\section{Структури от данни}

Този речник на съответствията става достъпен като глобална променлива, така че да се ползва в различните събития от визуалния интерфейс. Бутон едно за съседи има бутон две и бутон пет  (Фиг. \ref{fig060006}). 

\begin{figure}[H]
  \centering
  \includegraphics[width=1.0\linewidth,height=0.5\linewidth]{fig060006.png}
  \caption{Бутони като списък от съседи}
\label{fig060006}
\end{figure}

Бутон две има за съседи бутон едно, шест и три. Съседи на бутон три са две, седем и четири. Съседи на бутон четири са три и осем. Съседи на бутон пет са едно, шест и девет. Съседи на бутон шест са две, пет, седем и десет. Съседи на бутон седем са три, шест, осем и единадесет. Съседи на бутон осем са четири, седем и дванадесет. Съседи на бутон девет са пет, десет и тринадесет. Съседи на бутон десет са шест, девет, единадесет и четиринадесет. Съседи на бутон единадесет са седем, десет, дванадесет и петнадесет. Съседи на бутон дванадесет са осем, единадесет и шестнадесет. Съседи на бутон тринадесет са девет и четиринадесет. Съседи на бутон четиринадесет са десет, тринадесет и петнадесет. Съседи на бутон петнадесет са единадесет, четиринадесет и шестнадесет. Съседи на бутон шестнадесет са дванадесет и петнадесет. Списъците за съседство се попълват по идентичен начин (Фиг. \ref{fig060007}).

\begin{figure}[H]
  \centering
  \includegraphics[width=1.0\linewidth,height=0.5\linewidth]{fig060007.png}
  \caption{Идентично попълване на съседните бутони}
\label{fig060007}
\end{figure}

Натискането на който и да е от бутоните предизвика събитие, което е общо за всички бутони (Фиг. \ref{fig060008}). Тъй като не е определено кой бутон ще натисне потребителя, то се прави проверка за натиснатия бутон (идва като параметър на събитието). 

\begin{figure}[H]
  \centering
  \includegraphics[width=1.0\linewidth,height=0.5\linewidth]{fig060008.png}
  \caption{Събитие за натиснат бутон}
\label{fig060008}
\end{figure}

\section{Алгоритми за обработка на състоянието на играта}

Проверката дали празната клетка е съседство и евентуалната размяна с празната клетка се поверява на допълнителна процедура (Фиг. \ref{fig060009}). Процедурата получава като входен параметър компонента, предизвикал събитието. 

\begin{figure}[H]
  \centering
  \includegraphics[width=1.0\linewidth,height=0.5\linewidth]{fig060009.png}
  \caption{Процедура за размяна}
\label{fig060009}
\end{figure}

За да се определи дали празната клетка е в съседство на натиснатия бутон се проверява целият списък, който се намира в речника на ключова позиция, посочена от компонента предизвикал събитието. Тази проверка е възможна с цикъл за обхождане на елементите в списъчна структура (Фиг. \ref{fig060010}). Ключовата стойност би трябвало винаги да връща списък със съседните бутони, но ако ключът не е намерен, за безопасност, се връща празен списък.

\begin{figure}[H]
  \centering
  \includegraphics[width=1.0\linewidth,height=0.5\linewidth]{fig060010.png}
  \caption{Цикъл за обхождане на съседите}
\label{fig060010}
\end{figure}

Размяна с прави само, ако в списъка бъде открита празната клетка. Направата на размяната спира и цикъла за търсене на празната клетка. Условието съседен бутон да обозначава праната клетка е неговият надпис да бъде два интервала (Фиг. \ref{fig060011}). 

\begin{figure}[H]
  \centering
  \includegraphics[width=1.0\linewidth,height=0.5\linewidth]{fig060011.png}
  \caption{Проверка за празната клетка}
\label{fig060011}
\end{figure}

За улесняване на размяната се залагат четири локални, помощни променливи. Две променливи за текстовете на двата бутона и две променливи за цветовете на текстовете (Фиг. \ref{fig060012}).

\begin{figure}[H]
  \centering
  \includegraphics[width=1.0\linewidth,height=0.5\linewidth]{fig060012.png}
  \caption{Локални помощни променливи}
\label{fig060012}
\end{figure}

Размяната се осъществява, чрез записване на помощните променливи, като нови стойности за двата бутона (Фиг. \ref{fig060013}).

\begin{figure}[H]
  \centering
  \includegraphics[width=1.0\linewidth,height=0.5\linewidth]{fig060013.png}
  \caption{Размяна на стойностите}
\label{fig060013}
\end{figure}

На този етап играта притежава абсолютно цялата функционалност, която притежава и механичната играчка (Фиг. \ref{fig060014}).

\begin{figure}[H]
  \centering
  \includegraphics[width=1.0\linewidth,height=0.5\linewidth]{fig060014.png}
  \caption{Завършен вид на играта}
\label{fig060014}
\end{figure}

Въпреки, че всичко необходимо е налично, използването на компютър позволява да се добави още една полезна функция, а именно – автоматично разбъркване на пъзела. Операционната система позволява да се прихване събитие за дълго натискане на бутон (Фиг. \ref{fig060015}).

\begin{figure}[H]
  \centering
  \includegraphics[width=1.0\linewidth,height=0.5\linewidth]{fig060015.png}
  \caption{Събитие за дълго натискане на бутон}
\label{fig060015}
\end{figure}

Бутонът за празната клетка не е натоварен с много действия и поради тази причина може да се използва точно като бутон за разбъркване, стига да бъде натиснат за продължително време (Фиг. \ref{fig060016}).

\begin{figure}[H]
  \centering
  \includegraphics[width=1.0\linewidth,height=0.5\linewidth]{fig060016.png}
  \caption{Активиране на разбъркването}
\label{fig060016}
\end{figure}

Тъй като при разбъркването празната клетка ще се мести, то е добър вариант позицията й да се съхранява в локална, помощна променлива (Фиг. \ref{fig060017}).

\begin{figure}[H]
  \centering
  \includegraphics[width=1.0\linewidth,height=0.5\linewidth]{fig060017.png}
  \caption{Помощна променлива за позицията на празната клетка}
\label{fig060017}
\end{figure}

Игралното табло се състои от 16 позиции. За да може всяка позиция да участва, средно-статистически, 10 пъти в разбъркването, случайно избрани размествания могат да се направят 160 пъти, което е 16 клетки по десет пъти. За целта на разбъркването, цикъл с единична стъпка е най-удачната конструкция (Фиг. \ref{fig060018}).

\begin{figure}[H]
  \centering
  \includegraphics[width=1.0\linewidth,height=0.5\linewidth]{fig060018.png}
  \caption{Цикъл за разбъркване}
\label{fig060018}
\end{figure}

Изборът на следваща празна клетка се прави на случаен принцип от съседите на текущата празна клетка  (Фиг. \ref{fig060019}). Следващата празна клетка се записва във временна променлива, докато се извърши размяната.

\begin{figure}[H]
  \centering
  \includegraphics[width=1.0\linewidth,height=0.5\linewidth]{fig060019.png}
  \caption{Избор на случаен съсед на празната клетка}
\label{fig060019}
\end{figure}

Самата размяна се извършва с помощната процедура (Фиг. \ref{fig060020}). След размяната, празната клетка за следващото завъртане на цикъла се сменя.

\begin{figure}[H]
  \centering
  \includegraphics[width=1.0\linewidth,height=0.5\linewidth]{fig060020.png}
  \caption{Размяна на празната клетка}
\label{fig060020}
\end{figure}

\section{Публикуване на проекта}

След като играта е завършена, проектът може да се публикува за широката аудитория с помощта на бутона „Publish to Gallery“ (Фиг. \ref{fig060021}).

\begin{figure}[H]
  \centering
  \includegraphics[width=1.0\linewidth,height=0.5\linewidth]{fig060021.png}
  \caption{Бутон за публикуване в галерия}
\label{fig060021}
\end{figure}

Дори семпло описание (Фиг. \ref{fig060022}) на приложението е важно за потребителите, тъй като това е второто нещо, което привлича вниманието им.

\begin{figure}[H]
  \centering
  \includegraphics[width=1.0\linewidth,height=0.5\linewidth]{fig060022.png}
  \caption{Описание на приложението}
\label{fig060022}
\end{figure}

Първото най-важно нещо в едно софтуерно приложение е картинка, която най-много може да привлече вниманието на потребителите (Фиг. \ref{fig060023}).

\begin{figure}[H]
  \centering
  \includegraphics[width=1.0\linewidth,height=0.5\linewidth]{fig060023.png}
  \caption{Картинка за представяне на приложението}
\label{fig060023}
\end{figure}

В публичната страница на проекта (Фиг. \ref{fig060024}), потребителите могат да стартират програмата или да я заредят в средата на App Inventor.

\begin{figure}[H]
  \centering
  \includegraphics[width=1.0\linewidth,height=0.5\linewidth]{fig060024.png}
  \caption{Публична страница на проекта}
\label{fig060024}
\end{figure}

Играта е напълно използваема, но все още липсват някои функционалности. Хубаво би било да се добави възможност за автоматично нареждане на пъзела. Липсата на помощна информация също е недостатък. Възможно е да се добави отчитане на времето за подреждане, така че в по-напреднал етап да има възможност за организиране на онлайн класация. Изброените функционалности носят определена сложност и излизат извън рамките на настоящото изложение, но пък са идеална възможност за допълнително упражнение за читателите. 

\newpage
%\chapter{Leaping Rainbow}

The aim of the game is to get the ball into the bowl. The player controls the ball using the mouse. When the game starts, the bowl bounces from the left side of the screen to the right side, leaving behind a trail that is a rainbow. The player must control the ball so that it passes only along the arc. If it does not touch the rainbow - the game starts over.

\begin{figure}[H]
   \centering
   \includegraphics[width=1.0\linewidth,height=0.5\linewidth]{fig070001.png}
   \caption{Leaping Rainbow}
\label{fig070001}
\end{figure}

\section{Adding Background and Characters}
The first step of the game is to add a suitable background and characters. Characteristic of the background is that there should be a brown band at the bottom, which will serve as a reference. If the background to be selected is not there, then it can be added additionally using the drawing tools.

The main character is not needed in this game, for that he should be deleted. The characters bowl and ball are among the ready-made characters in Scratch. The rainbow character must be drawn (Fig. \ref{fig070002}).

\begin{figure}[H]
   \centering
   \includegraphics[width=1.0\linewidth,height=0.5\linewidth]{fig070002.png}
   \caption{Adding the character arc}
\label{fig070002}
\end{figure}

One more character needs to be drawn. This is the inscription that will appear when the ball touches the bowl, ie. the player successfully goes through the entire arc.

\begin{figure}[H]
   \centering
   \includegraphics[width=1.0\linewidth,height=0.5\linewidth]{fig070003.png}
   \caption{Adding the character to end the game}
\label{fig070003}
\end{figure}

\section{Programming the Bowl}
The first instructions to be constructed are those for the bowl. Her goal is to go from the left side of the screen to the right by bouncing. The algorithm to be constructed for the bounce effect requires the creation of a variable. This variable will contain the value by which the y-coordinate will be changed. Until the bowl touches the brown border the variable will decrement by 1. When it touches the border then it will take a value of 15.

When the game starts the cup position should be on the left side of the screen. This means that the value of the x-coordinate should be -199 and that of the Y-coordinate 148. The initial value of the variable "velocity" should be equal to 0. The size of the character should be changed to be smaller.

\begin{figure}[H]
   \centering
   \includegraphics[width=1.0\linewidth,height=0.5\linewidth]{fig070004.png}
   \caption{Position of Hero Cup}
\label{fig070004}
\end{figure}

The movement of the bowl must be programmed. This movement must be repeated, which means that a loop must be added with a goal that is "until the character touches an edge". This means that the character will move until it touches the right part of the screen. In addition to moving the character with the 3-step movement instruction, it must also change its y coordinate with the value of the "velocity" variable.

An if/else construct should be added to check if the character touches the brown border. If it is not touching it, then the variable must be decreased to move the character down. If it touches the border - the variable should have a value of 15. To make the check that the brown color is not touched, a negation instruction should be added, which is located in the green group. The color checker is in the light blue group. To find the exact color, use the eyedropper tool by placing it on the brown color. By instructions, the movement algorithm looks like this:

\begin{figure}[H]
   \centering
   \includegraphics[width=1.0\linewidth,height=0.5\linewidth]{fig070005.png}
   \caption{Movement of hero cup}
\label{fig070005}
\end{figure}

After the character reaches his goal, he must send a "start game" message. Only then can the game begin. Instructions for sending messages are located in the yellow instructions group.

\begin{figure}[H]
   \centering
   \includegraphics[width=1.0\linewidth,height=0.5\linewidth]{fig070006.png}
   \caption{The Hero Cup Code}
\label{fig070006}
\end{figure}

When the game starts, it is noticed that the character goes from one part of the screen to the other, jumping.

\section{Programming the Rainbow}
To program this character to leave traces, a new group of instructions must be added - Pen.

\begin{figure}[H]
   \centering
   \includegraphics[width=1.0\linewidth,height=0.5\linewidth]{fig070007.png}
   \caption{New group of Pen instructions}
\label{fig070007}
\end{figure}

Based on how the character is drawn, it should be reduced. This can be done without instructions, but by changing the Size property.

\begin{figure}[H]
   \centering
   \includegraphics[width=1.0\linewidth,height=0.5\linewidth]{fig070008.png}
   \caption{Change the Size property}
\label{fig070008}
\end{figure}

This character's instructions are only to follow the cup character and leave tracks. At the beginning of the game, everything drawn must be erased. This is done using the "erase all" instruction, which is peace in the newly added group of instructions. The rainbow, in addition to following the movement of the bowl character, must also follow its direction. This is done using the instruction from the blue "point in direction" group. To indicate exactly which direction to follow, an instruction from the light blue group, which is "backdrop of Stage", should be used. First, the second value of the "Stage" of the character name is changed. Then the type is also changed, which should be "direction".

\begin{figure}[H]
   \centering
   \includegraphics[width=1.0\linewidth,height=0.5\linewidth]{fig070009.png}
   \caption{Movement of the character arc}
\label{fig070009}
\end{figure}

If you start the program, you will notice that the rainbow character only follows the cup character. To leave a mark, the "stamp" instruction had to be used, which was added to the new group.

\begin{figure}[H]
   \centering
   \includegraphics[width=1.0\linewidth,height=0.5\linewidth]{fig070010.png}
   \caption{All character code arc}
\label{fig070010}
\end{figure}

\section{Programming the Ball}
The ball's edges must be such that it can pass through the entire arc touching the purple (or outermost) color. These can be changed by changing the value of the Size property.

\begin{figure}[H]
   \centering
   \includegraphics[width=1.0\linewidth,height=0.5\linewidth]{fig070011.png}
   \caption{Hero Ball Size}
\label{fig070011}
\end{figure}

The ball should appear when the bowl reaches the end of the screen. By instructions, this means when the character gets the "start game" message, then it appears. When the green flag is pressed, it should be hidden. When displayed, a slight animation can be made to move it to a position with coordinates for x=-231 and for y=114.

\begin{figure}[H]
   \centering
   \includegraphics[width=1.0\linewidth,height=0.5\linewidth]{fig070012.png}
   \caption{Starting position of the ball character}
\label{fig070012}
\end{figure}

The character should start moving when pressed. It moves as it touches the outermost color - in this case purple. Again, a loop with a goal should be used. The goal should be "until you stop touching the color purple". The statements outside the loop will be executed when the condition is false. In this case, it means when the character does not touch the arc, then move to the starting position.

The movement instruction should be - follow the mouse. This instruction is in the blue instruction group and is go to mouse-pointer. If the character needs to move slower, this is done by adding the wait instruction from the orange group.

\begin{figure}[H]
   \centering
   \includegraphics[width=1.0\linewidth,height=0.5\linewidth]{fig070013.png}
   \caption{Movement of character ball}
\label{fig070013}
\end{figure}

The last thing to do is check if the character has reached the end of the arc. If it is, then it will send a message to the character that is captioned to appear and the game will end. At the end of the arc is the arc hero. Then checking whether to end the game is very easy - has the ball touched the rainbow. If it is - sends a message and ends the game. If not, the game continues.

\begin{figure}[H]
   \centering
   \includegraphics[width=1.0\linewidth,height=0.5\linewidth]{fig070014.png}
   \caption{The Orb Hero Code}
\label{fig070014}
\end{figure}

\section{End Game}
The character that is the end of the game must be hidden at the beginning. It appears when the ball sends an end-of-game message.

\begin{figure}[H]
   \centering
   \includegraphics[width=1.0\linewidth,height=0.5\linewidth]{fig070015.png}
   \caption{Hero Code Caption}
\label{fig070015}
\end{figure}
\newpage
%\chapter{Bulls and Cows}

\begin{figure}[H]
   \centering
   \includegraphics[width=1.0\linewidth,height=0.5\linewidth]{fig080001.png}
   \caption{"Bulls and cows" \\ https://i.ytimg.com/vi/r\_dw8iV\_52g/hqdefault.jpg}
\label{fig080001}
\end{figure}

The Bulls and Cows game (Fig. \ref{fig080001}) is from the group of cipher games. It is played by two players, with paper and pencil, each thinking of a four-digit number (secret) that cannot start with zero. Each of the players tries to guess the opponent's number. The guessing process involves mentioning a four-digit number, with the opponent responding with information about how many digits of the guess are in their correct places and how many digits are in other places. A bull is reported for each digit that matches in position with the secret number, and for each digit that does not match in position, a cow is reported. The game ends when one of the players succeeds in guessing a number that results in four bulls.

\section{Designing the GUI}

The game isn't complicated, but it's an ideal way to demonstrate a computer opponent who uses set theory tricks. Game development begins with creating a new project (Fig. \ref{fig080002}).

\begin{figure}[H]
   \centering
   \includegraphics[width=1.0\linewidth,height=0.5\linewidth]{fig080002.png}
   \caption{Creating a new Bulls and Cows project}
\label{fig080002}
\end{figure}

Various ways of organizing the graphical user interface are possible, but it is appropriate to use the most straightforward option for demonstration purposes. In a table-type visual column control manager, visual controls can be arranged in a matrix of 9 columns and 2 rows (Fig. \ref{fig080003}).

\begin{figure}[H]
   \centering
   \includegraphics[width=1.0\linewidth,height=0.5\linewidth]{fig080003.png}
   \caption{9x2 Visual Component Manager}
\label{fig080003}
\end{figure}

Since each player queries four digit numbers, the first row in the table, the first four positions, is filled with four visual components to select from a list (Fig. \ref{fig080004}). These four list controls will visualize the four-digit numbers guessed by the computer opponent. Each component initially displays the star symbol, and the list of possible values is the digits zero through nine. In the first visual component, selecting the zero should not be possible, but for symmetry, the zero is left in the list.

\begin{figure}[H]
   \centering
   \includegraphics[width=1.0\linewidth,height=0.5\linewidth]{fig080004.png}
   \caption{Visual controls for computer assumptions}
\label{fig080004}
\end{figure}

Similarly, four more list controls (Fig. \ref{fig080005}) are arranged, which will be used by the player to guess what the computer opponent's secret number is. A blank cell is left between the two sets of controls, which will be used as a button to start the guessing procedures.

\begin{figure}[H]
   \centering
   \includegraphics[width=1.0\linewidth,height=0.5\linewidth]{fig080005.png}
   \caption{Visual controls for human assumptions}
\label{fig080005}
\end{figure}

On the second row, four more list controls will mark the number of bulls and the number of cows (Fig. \ref{fig080006}). The first two are for the number of bulls and cows known by the computer opponent, and the second two are for the number of bulls and cows known by the player. In pairs, the left control will show the bulls, and the right control the cows. The possible options are from zero to four, as the possible bulls or cows are from zero to four inclusive.

\begin{figure}[H]
   \centering
   \includegraphics[width=1.0\linewidth,height=0.5\linewidth]{fig080006.png}
   \caption{Visual controls for counting bulls and cows}
\label{fig080006}
\end{figure}

List selection components do not display the selected option in their text automatically. For this reason, it is necessary to intercept the selection made event and write the chosen option to the text field of the visual component. For this purpose, a common event generated by all list components on the screen is caught (Fig. \ref{fig080007}).

\begin{figure}[H]
   \centering
   \includegraphics[width=1.0\linewidth,height=0.5\linewidth]{fig080007.png}
   \caption{Preview of selected option}
\label{fig080007}
\end{figure}

It is reasonable to place labels in front of the cells for counting the number of bulls and the number of cows (Fig. \ref{fig080008}). The Latin letter B is used for the bulls, and the Latin letter C for the number of cows.

\begin{figure}[H]
   \centering
   \includegraphics[width=1.0\linewidth,height=0.5\linewidth]{fig080008.png}
   \caption{Labels in front of bull and cow counting controls}
\label{fig080008}
\end{figure}

The user interface ends with two buttons (Fig. \ref{fig080009}). The first prompts the computer opponent to guess the player's number, and the second prompts the computer opponent to report the number of bulls and cows hit by the human.

\begin{figure}[H]
   \centering
   \includegraphics[width=1.0\linewidth,height=0.5\linewidth]{fig080009.png}
   \caption{Buttons to activate the guessing process}
\label{fig080009}
\end{figure}

\section{Using Data Structures}

For the needs of the computer adversary and according to set theory, it is essential to create a list structure (Fig. \ref{fig080010}) that contains all possible combinations of four-digit numbers, the digits of which do not repeat and do not start with zero. For each person's response, all combinations that do not meet the conditions for established bulls and cows are excluded from the list.

\begin{figure}[H]
   \centering
   \includegraphics[width=1.0\linewidth,height=0.5\linewidth]{fig080010.png}
   \caption{List of combinations}
\label{fig080010}
\end{figure}

The numbers are four digits, so the list of combinations can be filled using four loops (Fig. \ref{fig080011}). The first rotates from one to nine since zero cannot participate as the first digit. The other three cycles rotate from zero to nine. The Latin letters a, b, c, and d are chosen as cycle counters. Each of the counters will participate in the formation of a four-digit number.

\begin{figure}[H]
   \centering
   \includegraphics[width=1.0\linewidth,height=0.5\linewidth]{fig080011.png}
   \caption{Cycles to generate the combinations}
\label{fig080011}
\end{figure}

To avoid repeating the numbers, a series of checks must be made (Fig. \ref{fig080012}). No check will be made for the first loop, but the second loop is rotated only if counter b differs from counter a. The third loop is looped only if counter c differs from the counter a and counter c differs from counter b. The fourth loop only rotates if counter d is different from the counter a, different from counter b, and different from counter c.

\begin{figure}[H]
   \centering
   \includegraphics[width=1.0\linewidth,height=0.5\linewidth]{fig080012.png}
   \caption{Checks to avoid duplicate digits}
\label{fig080012}
\end{figure}

Two global variables help store the computer and human opponents' secret numbers (Fig. \ref{fig080013}). Two more global variables help handle the computer opponent and human guesses. All four variables are initially initialized to zeros.

\begin{figure}[H]
   \centering
   \includegraphics[width=1.0\linewidth,height=0.5\linewidth]{fig080013.png}
   \caption{Helper variables for players' secret numbers}
\label{fig080013}
\end{figure}

\section{Game Manipulation Algorithms}

The cycles for initializing the list of combinations and the subsequent selection of one of them as the computer opponent's secret number must occur in an event marking the initialization of the work screen (Fig. \ref{fig080014}).

\begin{figure}[H]
   \centering
   \includegraphics[width=1.0\linewidth,height=0.5\linewidth]{fig080014.png}
   \caption{Initial screen initialization}
\label{fig080014}
\end{figure}

The task of the first button is to take a guess from the computer opponent (calling an additional procedure) and reset all other digits on the visual components (Fig. \ref{fig080015}).

\begin{figure}[H]
   \centering
   \includegraphics[width=1.0\linewidth,height=0.5\linewidth]{fig080015.png}
   \caption{First Button Actions}
\label{fig080015}
\end{figure}

The computer opponent guesses the player's secret number by choosing one item from the list of remaining combinations. If the list is empty (Fig. \ref{fig080016}), the person has given incorrect information on one of the previous moves, and knowing his number is impossible. The notification component sends a message that the game cannot continue.

\begin{figure}[H]
   \centering
   \includegraphics[width=1.0\linewidth,height=0.5\linewidth]{fig080016.png}
   \caption{Checking for remaining available combinations}
\label{fig080016}
\end{figure}

If there are elements in the list of combinations, one element is chosen randomly (Fig. \ref{fig080017}). The selected number is visualized in the first four cells of the interface, taking the individual digits for this purpose.

\begin{figure}[H]
   \centering
   \includegraphics[width=1.0\linewidth,height=0.5\linewidth]{fig080017.png}
   \caption{Choose a number to guess}
\label{fig080017}
\end{figure}

When the second button is pressed, three actions are performed, and for each of them, a helper procedure is called (Fig. \ref{fig080018}). First, the information is taken from the visual controls for the assumption made by the person. The second action is to visualize the number of bulls and cows hit by the human. The third action is to exclude all combinations in the list that do not meet the criteria of the last guess made by the computer opponent.

\begin{figure}[H]
   \centering
   \includegraphics[width=1.0\linewidth,height=0.5\linewidth]{fig080018.png}
   \caption{First Button Actions}
\label{fig080018}
\end{figure}

The human's guess is taken from the visual controls, and each digit is lumped together and written to the auxiliary global variable (Fig. \ref{fig080019}).

\begin{figure}[H]
   \centering
   \includegraphics[width=1.0\linewidth,height=0.5\linewidth]{fig080019.png}
   \caption{Taking the guesswork out of the human}
\label{fig080019}
\end{figure}

The number of bulls and cows known by humans is determined by a helper procedure that returns a list of two elements. The first element contains the number of bulls, and the second is the number of cows. To calculate these numbers, the computer opponent's secret number and the guess made by the player are used (Fig. \ref{fig080020}). The result is displayed in the last two components of the GUI.

\begin{figure}[H]
   \centering
   \includegraphics[width=1.0\linewidth,height=0.5\linewidth]{fig080020.png}
   \caption{Visualization of the number of bulls and cows known to man}
\label{fig080020}
\end{figure}

The result of checking the assumption made, that is, the number of bulls and cows reported by the person, should be taken to reduce the number of combinations. This is accomplished with a helper procedure that also returns a list whose first element contains the number of bulls and the second element the number of cows (Fig. \ref{fig080021}).

\begin{figure}[H]
   \centering
   \includegraphics[width=1.0\linewidth,height=0.5\linewidth]{fig080021.png}
   \caption{Retrieving the number of bulls and cows declared by the person}
\label{fig080021}
\end{figure}

Weeding out unnecessary combinations is done by taking the answer given by the human opponent. After that, a new empty list is created, in which only the numbers that meet the criteria specified in the response will enter the list. Screening is achieved by traversing the list of combinations (Fig. \ref{fig080022}). Each combination is fed to the utility function to determine how many bulls and cows the combination makes based on the guess made by the computer opponent. If the currently checked combination has the same characteristics (number of bulls and number of cows) as the characteristics returned by the human opponent, then the current combination enters the new list. After completely traversing the list of combinations, the old list is replaced with the new one.

\begin{figure}[H]
   \centering
   \includegraphics[width=1.0\linewidth,height=0.5\linewidth]{fig080022.png}
   \caption{Weeding out unnecessary combinations}
\label{fig080022}
\end{figure}

The calculation of the number of bulls and the number of cows for two numbers is done by rotating two loops. One cycle traverses the first number, digit by digit, and the second cycle, the second number, again digit by digit (Fig. \ref{fig080023}).

\begin{figure}[H]
   \centering
   \includegraphics[width=1.0\linewidth,height=0.5\linewidth]{fig080023.png}
   \caption{Treading the digits of two numbers}
\label{fig080023}
\end{figure}

The currently viewed digits of the two numbers are loaded into two auxiliary variables (Fig. \ref{fig080024}). With these two variables and with the cycle counters, the necessary comparisons are made for the presence of a bull or a cow.

\begin{figure}[H]
   \centering
   \includegraphics[width=1.0\linewidth,height=0.5\linewidth]{fig080024.png}
   \caption{Helper variables for the numbers}
\label{fig080024}
\end{figure}

If the two numbers match, it means a bull or a cow (Fig. \ref{fig080025}). Whether the match is, a bull or a cow is determined by the values of the counters for the two cycles.

\begin{figure}[H]
   \centering
   \includegraphics[width=1.0\linewidth,height=0.5\linewidth]{fig080025.png}
   \caption{Number Match}
\label{fig080025}
\end{figure}

If the two counters have the same value, then a bull is present, otherwise, it is a cow (Fig. \ref{fig080026}).

\begin{figure}[H]
   \centering
   \includegraphics[width=1.0\linewidth,height=0.5\linewidth]{fig080026.png}
   \caption{Counter match}
\label{fig080026}
\end{figure}

Encountering a bull increases the value of the first element in the result by one, and encountering a cow increases the value of the second element of the result list by one (Fig. \ref{fig080027}).

\begin{figure}[H]
   \centering
   \includegraphics[width=1.0\linewidth,height=0.5\linewidth]{fig080027.png}
   \caption{Counting the bulls and cows}
\label{fig080027}
\end{figure}

The game is played sequentially (Fig. \ref{fig080028}) by pressing the first button. Pressing the first button allows the computer opponent to guess the human opponent's secret number. Then, the human marks how many bulls and cows the computer opponent can guess. The third step is for the human opponent to make their guess about the computer opponent's secret number. The second button is pressed, in which the computer opponent reports how many bulls and cows the human guessed.

\begin{figure}[H]
   \centering
   \includegraphics[width=1.0\linewidth,height=0.5\linewidth]{fig080028.png}
   \caption{Game Home Screen}
\label{fig080028}
\end{figure}

Following the sequence of steps for working with the graphical interface (Fig. \ref{fig080029}), the game continues until one of the players hits four bulls.

\begin{figure}[H]
   \centering
   \includegraphics[width=1.0\linewidth,height=0.5\linewidth]{fig080029.png}
   \caption{Intermediate game move}
\label{fig080029}
\end{figure}

\section{Publish the project}

After reaching a fully functional version of the game, the project can be published in the gallery to be available to a broad audience (Fig. \ref{fig080030}).

\begin{figure}[H]
   \centering
   \includegraphics[width=1.0\linewidth,height=0.5\linewidth]{fig080030.png}
   \caption{Project Description}
\label{fig080030}
\end{figure}

After publishing, hyperlinks appear on the application page to run the program or view its code in the development environment (Fig. \ref{fig080031}).

\begin{figure}[H]
   \centering
   \includegraphics[width=1.0\linewidth,height=0.5\linewidth]{fig080031.png}
   \caption{Published Project Page}
\label{fig080031}
\end{figure}

Although fully functional, the game still needs to be finished to the level of a final product. A series of checks to ensure that the user is correctly entering numbers (e.g., repeating digits) must be included. There is no functionality to announce the winner, although the appearance of four bulls clearly indicates who wins. It lacks a help screen as well as a sound layout. The missing functionality is beyond the scope of this presentation and could serve as an additional exercise for those wishing to upgrade their knowledge.
\newpage
%\chapter{Pop the Balloons}

The goal of this game is for the player to pop as many balloons as possible in 30 seconds. The player will have a catapult with the help of which the player will pop the balloons.

\begin{figure}[H]
   \centering
   \includegraphics[width=1.0\linewidth,height=0.5\linewidth]{fig090001.png}
   \caption{Pop the balloons}
\label{fig090001}
\end{figure}

\section{Adding Background and Characters}
The first step of the game is to choose a suitable background and characters. The required characters in this game are an arrow representing the catapult, a balloon, and a character announcing the score when the 30 seconds are up.

\begin{figure}[H]
   \centering
   \includegraphics[width=1.0\linewidth,height=0.5\linewidth]{fig090002.png}
   \caption{Adding background and characters}
\label{fig090002}
\end{figure}

\section{Programming the Catapult}
In this game, two variables must be defined - the first will store the time, and the second the score. At the start of the game, the initial value of the time variable should be 30, and the value of the score variable should be 0.

\begin{figure}[H]
   \centering
   \includegraphics[width=1.0\linewidth,height=0.5\linewidth]{fig090003.png}
   \caption{Initialize variables}
\label{fig090003}
\end{figure}

The game continues until the time variable is equal to 0. It must decrease its value every 1 second.

\begin{figure}[H]
   \centering
   \includegraphics[width=1.0\linewidth,height=0.5\linewidth]{fig090004.png}
   \caption{Change time variable}
\label{fig090004}
\end{figure}

When the time becomes equal to 0, the game is over. Then that character should send a "stop game" message (Fig. \ref{fig090005}. When the game starts, the time variable will be decremented by 1 every second.

\begin{figure}[H]
   \centering
   \includegraphics[width=1.0\linewidth,height=0.5\linewidth]{fig090005.png}
   \caption{Send Game End Message}
\label{fig090005}
\end{figure}

During gameplay, the catapult must be positioned at the bottom of the screen and follow the direction of the mouse until the player clicks it. A nested loop must be used, meaning there will be a loop that contains another inside of itself. The outer loop will be infinite. It will end when the game is over. The inner loop will be a loop with a goal, with the goal being until the mouse is clicked. Inside the loop body, the statement should be "point towards mouse-pointer, " meaning "follow the mouse".
 
\begin{figure}[H]
   \centering
   \includegraphics[width=1.0\linewidth,height=0.5\linewidth]{fig090006.png}
   \caption{Mouse Tracking}
\label{fig090006}
\end{figure}

When the game is launched, it is noticed that the catapult follows the direction of the mouse. What remains to be done is when it is clicked to fire, and when it touches any of the edges of the screen, it returns to its original state. Created by instructions, this is done by adding one more inner loop. This time the condition of this loop should be - until the catapult touches some edge of the screen. And in the body of the cycle, the instruction should be placed - it moves by 20 steps. The instructions outside this loop are executed when the character touches the edge. For this, the last instruction is to position the character in a starting position.

\begin{figure}[H]
   \centering
   \includegraphics[width=1.0\linewidth,height=0.5\linewidth]{fig090007.png}
   \caption{Final catapult code}
\label{fig090007}
\end{figure}

\section{Programming the Balloon}
In the next step, the instructions for the balloon will also be constructed. Many balloons appear during the game, and the hero is only one. This is done by adding instructions to clone the balloon character. The instructions needed for cloning are in the orange group and are "create a clone of myself" and the instruction "delete this clone".

When the game starts, the original balloon character must hide. His sole purpose is to create clones. Have the bubble create 10 clones of itself, then wait 5 seconds and delete one clone. This algorithm should be repeated 5 times. Here again, the nested loop construct must be used.

\begin{figure}[H]
   \centering
   \includegraphics[width=1.0\linewidth,height=0.5\linewidth]{fig090008.png}
   \caption{Clone Creation}
\label{fig090008}
\end{figure}

The balloon clone should also be programmed. From the orange group, the instruction "When I start as a clone" should be used, which means "When I start a clone". The first thing to do is position the clone. To make the game more interesting, let its position be random but at the top of the screen. This means that the number for the x-coordinate should be randomly between -200 and 200 (that's the borders of the screen), and the y-coordinate should be between 50 and 150 (the top of the screen).

Once positioned, the clone should be displayed and resized to be smaller. The two instructions are located in the purple instruction group.

\begin{figure}[H]
   \centering
   \includegraphics[width=1.0\linewidth,height=0.5\linewidth]{fig090009.png}
   \caption{Clone Positioning}
\label{fig090009}
\end{figure}

Once the bubble appears, it must move one step. In addition to the balloon's movement, two checks must be made. One check is to see if the arrow has touched the clone. If the condition is true, then the value of the result variable should be incremented by 1, and the bubble should be placed at a random position again.

\begin{figure}[H]
   \centering
   \includegraphics[width=1.0\linewidth,height=0.5\linewidth]{fig090010.png}
   \caption{Increasing score}
\label{fig090010}
\end{figure}

The second check to be made is if the balloon is not hit by the catapult and reaches the end of the screen. It must be checked if the x coordinate is greater than 220. If the condition is true, the clone must be placed on the left side of the screen, and to make the game more attractive, it will also change its costume (color).

\begin{figure}[H]
   \centering
   \includegraphics[width=1.0\linewidth,height=0.5\linewidth]{fig090011.png}
   \caption{The Bubble Code}
\label{fig090011}
\end{figure}

The game is almost ready. When launched, clones of the balloon appear and can be burst using the catapult. Also, the time variable decreases, and the result variable increases.

The final step is to program the character to report the final result. When the game starts, this character must be hidden. He should show up when he gets the "stop game" message from the catapult. The instruction to print the result is from the purple group - "thing Hmm... for 2 seconds". The result should be displayed instead of the "Hmm..." message. The "join apple banana" instruction pastes the two words from the green group of instructions. In this game, again, two things need to stick together. One is the message "Your score is," and the second is the value of the score variable.

\begin{figure}[H]
   \centering
   \includegraphics[width=1.0\linewidth,height=0.5\linewidth]{fig090012.png}
   \caption{The character code that announces the result}
\label{fig090012}
\end{figure}

Game over. Play it with your friends to see who can pop the most balloons in 30 seconds.
\newpage
%\chapter{War}

\begin{figure}[H]
   \centering
   \includegraphics[width=1.0\linewidth,height=0.5\linewidth]{fig100001.png}
   \caption{"War"}
\label{fig100001}
https://images.squarespace-cdn.com/ \\ content/v1/59ea6080a803bb2f70ecbae5/1529350057743-92YH5Y0BN0JUMYB6X7NV/ \\ close-call-slide.jpg
\end{figure}

The game "War" (Fig. \ref{fig100001}) is a children's card game, and in its basic version, it is played by two players. The cards are standard, 52 playing cards. Card suits are equal, and there is no power per suit. Each card has a power with which it participates in the game, starting with the pairs (2 points) and going up to the aces (14 points). The deck of cards is shuffled and dealt equally to both players. The cards are dealt face down, with each player showing the top card on each turn. The player with the stronger card takes both cards. If the cards are of equal strength, it is a "war," and the players show three cards each. The war is won by the player with the stronger third card. If the third cards also match, the war continues until one player loses the war. The winning player collects all face-up cards. Draw cards always go to the bottom of the respective player's deck. The game is lost by that player who runs out of cards.

\section{Designing the GUI}

The game is relatively simple and does not require special skills, which makes it an ideal option for young children. Developing this game as a mobile application begins with creating a new project (Fig. \ref{fig100002}).

\begin{figure}[H]
   \centering
   \includegraphics[width=1.0\linewidth,height=0.5\linewidth]{fig100002.png}
   \caption{Creating a new War game project}
\label{fig100002}
\end{figure}

The user interface will be as simple as possible. Two buttons (Fig. \ref{fig100003}) at the workspace will start a new game and make a move.

\begin{figure}[H]
   \centering
   \includegraphics[width=1.0\linewidth,height=0.5\linewidth]{fig100003.png}
   \caption{Start New Game and Make Move Buttons}
\label{fig100003}
\end{figure}

Immediately below the buttons, two rows of visual components for displaying graphic images are arranged in tabular form (Fig. \ref{fig100004}). The first column will show a card back, which symbolizes the decks of both players, and the other three adjacent components will display one card when there is no war and three cards when there is a war.

\begin{figure}[H]
   \centering
   \includegraphics[width=1.0\linewidth,height=0.5\linewidth]{fig100004.png}
   \caption{Map Visualization Components}
\label{fig100004}
\end{figure}

For the map images, any set of maps distributed under a free license for non-commercial use can be used (Fig. \ref{fig100005}).

\begin{figure}[H]
   \centering
   \includegraphics[width=1.0\linewidth,height=0.5\linewidth]{fig100005.png}
   \caption{Images of playing cards}
\label{fig100005}
\end{figure}

If the card set is in a common image, it is cut into 52 separate images and at least one card back image. The 53 graphic files thus prepared are loaded into the project by uploading file by file (Fig. \ref{fig100006}).

\begin{figure}[H]
   \centering
   \includegraphics[width=1.0\linewidth,height=0.5\linewidth]{fig100006.png}
   \caption{Upload image files}
\label{fig100006}
\end{figure}

From the graphics files thus uploaded, the card back image is loaded into the first column of images (Fig. \ref{fig100007}).

\begin{figure}[H]
   \centering
   \includegraphics[width=1.0\linewidth,height=0.5\linewidth]{fig100007.png}
   \caption{Images for marking the decks}
\label{fig100007}
\end{figure}

\section{Using Data Structures}

Cards will circulate in the game model as integers. For this purpose, five auxiliary global variables are declared - a list of the main deck, two lists for the cards held by the player, and two lists for the cards placed on the table by the players (Fig. \ref{fig100008}).

\begin{figure}[H]
   \centering
   \includegraphics[width=1.0\linewidth,height=0.5\linewidth]{fig100008.png}
   \caption{Basic Auxiliary Variables}
\label{fig100008}
\end{figure}

Two additional lists make it easier to visualize the cards being placed on the table. These lists contain references to the visual components for displaying images (Fig. \ref{fig100009}).

\begin{figure}[H]
   \centering
   \includegraphics[width=1.0\linewidth,height=0.5\linewidth]{fig100009.png}
   \caption{Helper Visualization Lists}
\label{fig100009}
\end{figure}

The last helper variable trims the map images into a list. The order is essential, with deuces in the first four places, threes in the second four places, and so on until the last four places are aces (Fig. \ref{fig100010}). Thanks to this arrangement, the calculation of the winners in the individual rounds will be achieved by a simple arithmetic calculation.

\begin{figure}[H]
   \centering
   \includegraphics[width=1.0\linewidth,height=0.5\linewidth]{fig100010.png}
   \caption{Helper variable for order of cards by strength}
\label{fig100010}
\end{figure}

\section{Algorithms for manipulation of structures in the game}

Starting the game is all about organizing the individual lists correctly. To begin with, all auxiliary lists should be initialized with an empty value so that no parasitic information from previous plays is passed. A helper function can best achieve this (Fig. \ref{fig100011}).

\begin{figure}[H]
   \centering
   \includegraphics[width=1.0\linewidth,height=0.5\linewidth]{fig100011.png}
   \caption{Initialization of empty lists}
\label{fig100011}
\end{figure}

The main deck of cards contains, in a shuffled form, the numbers 1 to 52. Each number is the index of the corresponding card in the auxiliary list of card images. A loop is spun to populate the main deck (Fig. \ref{fig100012}), and the loop index is written to a randomly selected position. A random element can be inserted by generating a random number for the insertion index. This random number must range from 1 to the current list size. When the list is empty, it has a length of 0. In this situation, the random number would be 0 or 1. Indexes in the lists cannot be 0, so an additional block is needed, providing a maximum value between the randomly generated number and the number 1.

\begin{figure}[H]
   \centering
   \includegraphics[width=1.0\linewidth,height=0.5\linewidth]{fig100012.png}
   \caption{Deck Initialization Cycle}
\label{fig100012}
\end{figure}

After the main deck is shuffled, the cards are dealt to both players. Each player is dealt half of the cards from the main deck, which are recorded in two auxiliary lists (Fig. \ref{fig100013}). The step in this cycle is 2 because the even cards go to one player, and the odd cards go to the other.

\begin{figure}[H]
   \centering
   \includegraphics[width=1.0\linewidth,height=0.5\linewidth]{fig100013.png}
   \caption{Dealing the cards}
\label{fig100013}
\end{figure}

After the cards are dealt, the main deck can be assumed to be empty (Fig. \ref{fig100014}).

\begin{figure}[H]
   \centering
   \includegraphics[width=1.0\linewidth,height=0.5\linewidth]{fig100014.png}
   \caption{Reset Deck}
\label{fig100014}
\end{figure}

Internal structures need to be initialized in two situations - when the application is launched and when the user decides to start a new game (Fig. \ref{fig100015}).

\begin{figure}[H]
   \centering
   \includegraphics[width=1.0\linewidth,height=0.5\linewidth]{fig100015.png}
   \caption{Initialization Calls}
\label{fig100015}
\end{figure}

The following helper procedure removes images if loaded in the image viewer components (Fig. \ref{fig100016}). This procedure is helpful on any turn when it has already been determined which player will collect the cards from the table.

\begin{figure}[H]
   \centering
   \includegraphics[width=1.0\linewidth,height=0.5\linewidth]{fig100016.png}
   \caption{Image Clearing Procedure}
\label{fig100016}
\end{figure}

Pressing the second button advances the game one step forward. His first task is to clean up the displayed cards from the previous move (Fig. \ref{fig100017}). Then follows a cascade of possible situations.

\begin{figure}[H]
   \centering
   \includegraphics[width=1.0\linewidth,height=0.5\linewidth]{fig100017.png}
   \caption{Catch event for next game step}
\label{fig100017}
\end{figure}

The first two situations on the game board are when one player has cards in his hand, the other does not, and there are no cards on the table (Fig. \ref{fig100018}). In this situation, the player who has cards in his hand is the winner.

\begin{figure}[H]
   \centering
   \includegraphics[width=1.0\linewidth,height=0.5\linewidth]{fig100018.png}
   \caption{Situations with already determined winner}
\label{fig100018}
\end{figure}

The second two situations (Fig. \ref{fig100019}) are when there are no cards on the table, that is, they must be drawn from the decks, and when there are cards on the table, and it is necessary to determine which player wins the round.

\begin{figure}[H]
   \centering
   \includegraphics[width=1.0\linewidth,height=0.5\linewidth]{fig100019.png}
   \caption{Situations according to the presence of cards on the table}
\label{fig100019}
\end{figure}

Only the winner is announced in a message for the first two conditions. Auxiliary functions are defined for the second two states. Immediately after the possible states follows an auxiliary function for updating the visual space (Fig. \ref{fig100020}).

\begin{figure}[H]
   \centering
   \includegraphics[width=1.0\linewidth,height=0.5\linewidth]{fig100020.png}
   \caption{Helper functions to work out different situations}
\label{fig100020}
\end{figure}

In the most complex case, the auxiliary function for visualizing the cards on the table shows up to three cards per player, a total of six. When there is no state of war, one card is visualized. When there is a state of war, the three cards of the last war are visualized (wars may be several in a row). Visualization occurs in two consecutive cycles (Fig. \ref{fig100021}).

\begin{figure}[H]
   \centering
   \includegraphics[width=1.0\linewidth,height=0.5\linewidth]{fig100021.png}
   \caption{Cycles to view cards on the table}
\label{fig100021}
\end{figure}

If there are not enough items in the card lists on the table, then the rendering loop must be aborted (Fig. \ref{fig100022}). An example is when a state of war appears, but one player only has two last cards left.

\begin{figure}[H]
   \centering
   \includegraphics[width=1.0\linewidth,height=0.5\linewidth]{fig100022.png}
   \caption{Stop preview when cards are missing}
\label{fig100022}
\end{figure}

If there are enough cards on the table, the image viewer component displays that image from the list of images, which is specified as an index in the list of cards on the table (Fig. \ref{fig100023}).

\begin{figure}[H]
   \centering
   \includegraphics[width=1.0\linewidth,height=0.5\linewidth]{fig100023.png}
   \caption{Determining which images to display}
\label{fig100023}
\end{figure}

To draw a card, each player must take it from their hand first and place it in the card row on the table (Fig. \ref{fig100024}). A card enters the table list and leaves the hand list.

\begin{figure}[H]
   \centering
   \includegraphics[width=1.0\linewidth,height=0.5\linewidth]{fig100024.png}
   \caption{Download one card at a time}
\label{fig100024}
\end{figure}

When there are cards on the table, three situations are possible - the move is won by the first player, the move is won by the second player, and the players are tied, resulting in a state of war (Fig. \ref{fig100025}). The cards in the main deck are arranged to form four subgroups according to their strength. For this reason, it is easy to determine which card wins in an integer division over two. One must be subtracted because the numbering in the main deck starts from one and goes up to 52, and integer division can be applied to enumeration from 0 to 51.

\begin{figure}[H]
   \centering
   \includegraphics[width=1.0\linewidth,height=0.5\linewidth]{fig100025.png}
   \caption{Possible situations with cards on the table}
\label{fig100025}
\end{figure}

If the first player wins on the current turn, he draws his cards from the table and the cards from the second player's table into his hand. If the second player wins in the current turn, he takes his and his opponent's cards from the table. These actions are performed by filling a list by hand and emptying the lists for the table (Fig. \ref{fig100026}).

\begin{figure}[H]
   \centering
   \includegraphics[width=1.0\linewidth,height=0.5\linewidth]{fig100026.png}
   \caption{Collecting the cards after winning a separate turn of the game}
\label{fig100026}
\end{figure}

In a state of war, each player draws three cards to participate in the war. This action occurs in a loop rotating three times (Fig. \ref{fig100027}).

\begin{figure}[H]
   \centering
   \includegraphics[width=1.0\linewidth,height=0.5\linewidth]{fig100027.png}
   \caption{War map display cycle}
\label{fig100027}
\end{figure}

In the process of drawing cards for the war, it is possible that one of the players runs out of cards and cannot draw, then the draw cycle must be stopped (Fig. \ref{fig100028}).

\begin{figure}[H]
   \centering
   \includegraphics[width=1.0\linewidth,height=0.5\linewidth]{fig100028.png}
   \caption{Stop withdrawal when cards run out}
\label{fig100028}
\end{figure}

Drawing a card and placing it on the table happens exactly as in the procedure of drawing a single card, but it happens three times (Fig. \ref{fig100029}).

\begin{figure}[H]
   \centering
   \includegraphics[width=1.0\linewidth,height=0.5\linewidth]{fig100029.png}
   \caption{Removing three cards to the table}
\label{fig100029}
\end{figure}

\section{Publish the project}

The game is made publicly available by publishing it in the "gallery" area (Fig. \ref{fig100030}).

\begin{figure}[H]
   \centering
   \includegraphics[width=1.0\linewidth,height=0.5\linewidth]{fig100030.png}
   \caption{Publishing the game to the general audience}
\label{fig100030}
\end{figure}

With this, the game takes on its original completeness. Of course, there is still a lot of work to be done to turn it from a hobbyist project into a mass-use software product. First, the visualization of the cards can be improved and give a better account of which player holds how many cards. Help information needs to be included and layout in the form of animations when the cards move. These improvements are beyond the scope of this paper and are left for further exercise by the readers.
\newpage
%\chapter{Soccer}

One of the favorite games for children is soccer. The goal of this game is for the player to score as many goals as possible. If he misses, it's game over.

\begin{figure}[H]
   \centering
   \includegraphics[width=1.0\linewidth,height=0.5\linewidth]{fig110001.png}
   \caption{Football}
\label{fig110001}
\end{figure}

\section{Creating Game Design}
The first step of creating this game is adding all the components that will be programmed. The color of the field should be green, for this the value of the BackgroundColor property should be changed to green.

\begin{figure}[H]
   \centering
   \includegraphics[width=1.0\linewidth,height=0.5\linewidth]{fig110002.png}
   \caption{Change background color}
\label{fig110002}
\end{figure}

One of the most important tasks in mobile application programming is their interface. To make this game more attractive, you can change the value of the Theme property to Device Default and the name of the Title property to Footbal.

\begin{figure}[H]
   \centering
   \includegraphics[width=1.0\linewidth,height=0.5\linewidth]{fig110003.png}
   \caption{Change topic}
\label{fig110003}
\end{figure}

In order for the player to shoot the ball when he touches the screen of his phone and also for the goalkeeper to move on the screen, the Canvas element must be added. Its width and height should be the same as it is on the phone screen. To do this the Height property values to be Fill parent... The Width property value is the same. This element should blend in with the background color. For this purpose, the value of the BackgroundColor property must be changed.

\begin{figure}[H]
   \centering
   \includegraphics[width=1.0\linewidth,height=0.5\linewidth]{fig110004.png}
   \caption{The Canvas Element}
\label{fig110004}
\end{figure}

Three ImageSprite elements should be added for the net, goalkeeper and ball respectively. They can be renamed to make it clear which element is for what. Images of these elements can be used which are freely licensed for non-commercial use.

\begin{figure}[H]
   \centering
   \includegraphics[width=1.0\linewidth,height=0.5\linewidth]{fig110005.png}
   \caption{Image of a soccer ball}
\label{fig110005}
\end{figure}

Once the images are available they should be placed on the appropriate ImageSprite elements and the height, width and position properties should be changed. This means they need to scale to the phone screen. The end result should look like this:

\begin{figure}[H]
   \centering
   \includegraphics[width=1.0\linewidth,height=0.5\linewidth]{fig110006.png}
   \caption{Game Design}
\label{fig110006}
\end{figure}

In addition to these elements, a place where the result will be written must be added. From the Layout element group, a HorizontalArrangement element must be added so that two Label elements can be placed inside it. One will be for the result message and the second will be for the result value. This means that the property value for the first label will be "Your score is: " and for the second - "0". It is a good practice to rename elements to make them easier to program. In a later stage of the game, the element "Notifier" will be needed. It is from the group of hidden elements.

\begin{figure}[H]
   \centering
   \includegraphics[width=1.0\linewidth,height=0.5\linewidth]{fig110007.png}
   \caption{All Game Elements}
\label{fig110007}
\end{figure}

\section{Programming}
Necessary instructions should also be added to the items that are added for the game. For this purpose, it is necessary to switch to the other view, which is for adding the instructions.

The first instructions will be for the goalkeeper. It will move left and right. Its purpose is to keep the goal from scoring. When the game starts, the goalkeeper must start moving. This means adding the when Screen1.Initialize statement found in the Screen1 element. Inside this event, two instructions should be added, which are located in the goalkeepr element. One instruction is to set a value to the Interval property, which specifies how many milliseconds this character's position will change. The second instruction sets a value to the Speed property, which is responsible for the movement of the character.

One more event should be added - when goalkeeper.EdgeReached. The instructions from this event will be executed when the goalkeeper touches the edge of the screen. When this happens, the instruction goalkeeper.Bounce edge is called, and the value that is set is get edge.

\begin{figure}[H]
   \centering
   \includegraphics[width=1.0\linewidth,height=0.5\linewidth]{fig110008.png}
   \caption{Goalkeeper Movement}
\label{fig110008}
\end{figure}

When the game starts, it is noticed that the goalkeeper turns with his head down. This can be changed by unchecking the Rotates property.

\begin{figure}[H]
   \centering
   \includegraphics[width=1.0\linewidth,height=0.5\linewidth]{fig110009.png}
   \caption{Stopping goalkeeper rotation}
\label{fig110009}
\end{figure}

Instructions should be added so that when the player drags the ball, it moves. First the when football.Flug event needs to be added. Similarly, Interval and Speed values should be set for the goalkeeper. The value of the Interval property should be 10, and that of the Speed 20. In addition to the movement, the angle at which the ball will move should also be set. For this purpose, a value must be set for the Heading property, which comes from the get heading event.

\begin{figure}[H]
   \centering
   \includegraphics[width=1.0\linewidth,height=0.5\linewidth]{fig110010.png}
   \caption{Ball Movement}
\label{fig110010}
\end{figure}

In the last phase of the game, what happens when the ball touches the net and when it touches the goalkeeper must be programmed. According to the rules of the game, when the ball touches the net, the score increases by 1, and when it touches the goalkeeper, then the game ends. This means that a variable must be created in which the result of the game will be stored. The initial value of this variable is 0.

The event that is needed is when football.CollidedWith. Two checks must be added inside this event - whether the ball touched the net and whether the ball touched the goalkeeper.

\begin{figure}[H]
   \centering
   \includegraphics[width=1.0\linewidth,height=0.5\linewidth]{fig110011.png}
   \caption{Adding the checks}
\label{fig110011}
\end{figure}

When the ball touches the goalkeeper the game must end. In the language of the instructions, some of the properties of the ball should be changed and a message should be displayed that the game is over. The first ball property to change is Enabled to false. That means it won't move. Its position should be changed. The values for the coordinates are x=120, y=320 and z=1. The Speed property should be changed to 0.

To display a dialog message, an instruction is added from the Notifier1 element, which is ShowMessageDialog. It should have the following fields - message="Game over", title, which consists of score.Text and scoreValue.Text and the last one is buttonText="Restart game". When the game is over, the initial values of the variable, which is 0, must be returned to the scoreValue text field, which is also 0. Don't forget to change the value of the ball Enabled to true, so that it can again be moves the ball.

\begin{figure}[H]
   \centering
   \includegraphics[width=1.0\linewidth,height=0.5\linewidth]{fig110012.png}
   \caption{When the ball touches the goalkeeper}
\label{fig110012}
\end{figure}

When the ball touches the net, the instructions are basically the same. First the position of the ball must be changed. The difference is that the variable must be incremented and the new result displayed.

\begin{figure}[H]
   \centering
   \includegraphics[width=1.0\linewidth,height=0.5\linewidth]{fig110013.png}
   \caption{When the ball hits the net}
\label{fig110013}
\end{figure}

The game is ready. You can make a competition with your friends to see who is better at scoring goals.
\newpage
%\chapter{Gather the Fruit}

The objective of this game is for the player to catch the falling fruits to prevent them from falling to the ground. For each fallen fruit, he loses a life. For each fruit caught, he earns one point. It's game over when you lose your three lives. The goal is to collect the maximum number of points.

\begin{figure}[H]
   \centering
   \includegraphics[width=1.0\linewidth,height=0.5\linewidth]{fig120001.png}
   \caption{Collect the fruits}
\label{fig120001}
\end{figure}

\section{Creating the Design}
Building the game starts with creating the design, what the game will look like. The first step is to add a Layout to be a VerticalArrangement. It is required for the splash screen when the game starts. The dimensions of this element should be as they are on the phone screen. For this, the height and width properties must be changed.

\begin{figure}[H]
   \centering
   \includegraphics[width=1.0\linewidth,height=0.5\linewidth]{fig120002.png}
   \caption{Home screen}
\label{fig120002}
\end{figure}

To make the beginning more attractive, an image can be added. All images shown in this example can be replaced. Any images distributed under a free license may be used.

The selected image should be added to the Image property to be displayed on the phone screen when the game starts.

\begin{figure}[H]
   \centering
   \includegraphics[width=1.0\linewidth,height=0.5\linewidth]{fig120003.png}
   \caption{Home Screen Background}
\label{fig120003}
\end{figure}

The game will start when the player presses the start button. A button element should also be added for this purpose. It is possible to use the built-in button element, but to make the game more attractive the Image element can be used. An image can be used for the button. It is important to highlight the Clickable property. The Height and Width properties are responsible for how tall and wide the image should be.

\begin{figure}[H]
   \centering
   \includegraphics[width=1.0\linewidth,height=0.5\linewidth]{fig120004.png}
   \caption{Start button}
\label{fig120004}
\end{figure}

When the game starts this view will appear. When the player presses the button another view should appear in which the character to be controlled and the falling fruits will appear. In order to distinguish different game views when programming, it is good practice to have descriptive names. The VerticalArrangement1 element will be named StartLayout and the Image1 element will be named StartButton.

\begin{figure}[H]
   \centering
   \includegraphics[width=1.0\linewidth,height=0.5\linewidth]{fig120005.png}
   \caption{Final version of splash screen}
\label{fig120005}
\end{figure}

To create the other view, a VerticalArrangement element called MainLayout must be added again. The height and width of this element must be the same as they are on the screen. In order to make the design of this view easier, the Visible property of the StartLayout element should be unchecked.

\begin{figure}[H]
   \centering
   \includegraphics[width=1.0\linewidth,height=0.5\linewidth]{fig120006.png}
   \caption{Creating main game screen}
\label{fig120006}
\end{figure}

The Canvas and HorizontalArrangement elements should be added to the MainLayout element.
The Canvas element is necessary because it allows the elements inside it to move. There are several items in the game that will move. On the one hand, it is the character who will catch the fruits, and on the other, the fruits themselves.
The height and width of this element must be the same as they are on the screen. Again, a background can also be added to this element.

\begin{figure}[H]
   \centering
   \includegraphics[width=1.0\linewidth,height=0.5\linewidth]{fig120007.png}
   \caption{Adding background to main game screen}
\label{fig120007}
\end{figure}

Several fruit images should be added to this element. Also an image to be the character and three images to be his lives. Changing the Height and Width properties can change their sizes, and changing the X, Y, and Z properties can change their position. It is also important to change the names of the elements so that they can be recognized when it comes to programming them.
The figure shows an example arrangement of the fruits, the character and his lives. For the purposes of the game, the number of fruits or their placement is irrelevant.

\begin{figure}[H]
   \centering
   \includegraphics[width=1.0\linewidth,height=0.5\linewidth]{fig120008.png}
   \caption{Adding the fruits, lives and character to the game}
\label{fig120008}
\end{figure}

To complete this view, a HorizontalArrangement element must be added, to which the two buttons - left arrow and right arrow - will be added. These buttons will control the player to move. A Label element can be placed between them, in which the number will be displayed how many points did the player earn?
The width of the HorizontalArrangement element should be the width of the screen. Its height should be small, for example 10 percent. In order for the elements inside to align, the AlignHorizontal and AlignVertical properties must be changed to Center.

\begin{figure}[H]
   \centering
   \includegraphics[width=1.0\linewidth,height=0.5\linewidth]{fig120009.png}
   \caption{Add control bar}
\label{fig120009}
\end{figure}

The buttons should be added by resizing them and adding images to look like left and right arrows.

\begin{figure}[H]
   \centering
   \includegraphics[width=1.0\linewidth,height=0.5\linewidth]{fig120010.png}
   \caption{Adding the controls}
\label{fig120010}
\end{figure}

For the Label element, where the points will be written, the value of the FontSize property must be changed so that the text can be larger and visible.

\begin{figure}[H]
   \centering
   \includegraphics[width=1.0\linewidth,height=0.5\linewidth]{fig120011.png}
   \caption{Change point size}
\label{fig120011}
\end{figure}

The last view to be designed is when the player loses their lives. Then a "Try again" button should be displayed. In order to make the design easier, the Visible property of the MainLayout element must be unchecked.

\begin{figure}[H]
   \centering
   \includegraphics[width=1.0\linewidth,height=0.5\linewidth]{fig120012.png}
   \caption{Add end game view}
\label{fig120012}
\end{figure}

The view that is going to end the game should be HorizontalArrangement again, with the height and width dimensions being the same as they are on the screen. The background color of this element can be changed or an endgame image can be added. The AlignHorizontal and AlignVertical property values must be set to Center in order for the game restart button to be centered.

\begin{figure}[H]
   \centering
   \includegraphics[width=1.0\linewidth,height=0.5\linewidth]{fig120013.png}
   \caption{Change background color}
\label{fig120013}
\end{figure}

Lastly, the game restart button should be added. The shape, text size and color properties of the button can be changed.

\begin{figure}[H]
   \centering
   \includegraphics[width=1.0\linewidth,height=0.5\linewidth]{fig120014.png}
   \caption{Add game restart button}
\label{fig120014}
\end{figure}

The Visible property of the GameOverLayout element should be unchecked. Only the game start screen should be visible. For this, you need to macro the StartLayout element.

\begin{figure}[H]
   \centering
   \includegraphics[width=1.0\linewidth,height=0.5\linewidth]{fig120015.png}
   \caption{Checking the Visible property for the home screen}
\label{fig120015}
\end{figure}

\section{Creating the Program}
Blocks called "procedures" will be used in the construction of the game code. A procedure is characterized by a name and a set of instructions. They are executed when it is called. The purpose of a procedure is to contain instructions that are repeated at some point in the code. The following steps explain where and why procedures need to be used.

The first procedure to create is to start the game. The instructions that will be in it are for the initial positions of the fruits and setting the speeds at which they will move. This sequence of steps must be performed when the game starts, ie. the player presses the start game button, but also when the player presses the restart game button. For this, a procedure is used to avoid constructing the same steps twice.

\begin{figure}[H]
   \centering
   \includegraphics[width=1.0\linewidth,height=0.5\linewidth]{fig120016.png}
   \caption{Create Game Start Procedure}
\label{fig120016}
\end{figure}

The first time this procedure will be called is when the StartButton is pressed. Besides calling the procedure, the other instructions to execute are to hide the start view and bring up the main view. This is implemented using the Visible property, which can be controlled not only through design, but also programmatically.

\begin{figure}[H]
   \centering
   \includegraphics[width=1.0\linewidth,height=0.5\linewidth]{fig120017.png}
   \caption{Programming the start game button}
\label{fig120017}
\end{figure}

The game character will move left and right when clicking left arrow or right arrow respectively. For this purpose, two events should be added - when a left arrow is clicked or when a right arrow is clicked. The instructions in these two events are similar - the character must change his position relative to the X coordinate. In one case it should increase, and in the other it should decrease.

\begin{figure}[H]
   \centering
\includegraphics[width=1.0\linewidth,height=0.5\linewidth]{fig120018.png}
   \caption{Programming the character to move}
\label{fig120018}
\end{figure}

In this game, the following variables must be created - 3 that will be responsible for the speed of the three fruits and one that will be for the hero's lives.

\begin{figure}[H]
   \centering
   \includegraphics[width=1.0\linewidth,height=0.5\linewidth]{fig120019.png}
   \caption{Adding variables to the game}
\label{fig120019}
\end{figure}

When the player touches a fruit, his points will increase. A procedure will again be constructed for this purpose, since the instruction that changes the points must be called for all three fruits.

\begin{figure}[H]
   \centering
   \includegraphics[width=1.0\linewidth,height=0.5\linewidth]{fig120020.png}
   \caption{Procedure for changing points}
\label{fig120020}
\end{figure}

In the next step, instructions will be added that will be executed when the player touches a fruit. The algorithm is as follows - the position of the fruit is changed, the points are increased by 1 (there is a procedure created) and the character's speed is increased.

\begin{figure}[H]
   \centering
   \includegraphics[width=1.0\linewidth,height=0.5\linewidth]{fig120021.png}
   \caption{Algorithm when player touches fruit}
\label{fig120021}
\end{figure}

This algorithm should also be constructed for the other fruit characters.

\begin{figure}[H]
   \centering
   \includegraphics[width=1.0\linewidth,height=0.5\linewidth]{fig120022.png}
   \caption{The All Fruit Algorithm}
\label{fig120022}
\end{figure}

The last procedure to create is to check if the game should end. If the number of lives is equal to 0, then the main screen should be hidden and the endgame screen should appear. Also the speed of all fruits should become equal to 0.

\begin{figure}[H]
   \centering
   \includegraphics[width=1.0\linewidth,height=0.5\linewidth]{fig120023.png}
   \caption{Checking if the character loses the game}
\label{fig120023}
\end{figure}

Checks should be made if the lives are 2 or 1. If there are two, it means the player has lost 1 and must hide one of the lives. Again the Visible property needs to be changed. If the number is 1, it means that the player has lost 2 lives and the picture of one more life should be hidden.

\begin{figure}[H]
   \centering
   \includegraphics[width=1.0\linewidth,height=0.5\linewidth]{fig120024.png}
   \caption{Checking if character has life}
\label{fig120024}
\end{figure}

When the fruit character touches the end of the screen, then the endgame check procedure should be called and change its position.

\begin{figure}[H]
   \centering
   \includegraphics[width=1.0\linewidth,height=0.5\linewidth]{fig120025.png}
   \caption{Checking if a fruit has touched the edge}
\label{fig120025}
\end{figure}

The last instructions to be constructed are when the game is over and the player presses the restart game button.

\begin{figure}[H]
   \centering
   \includegraphics[width=1.0\linewidth,height=0.5\linewidth]{fig120026.png}
   \caption{Game Over}
\label{fig120026}
\end{figure}

Game over. It should be tested on the phone.
\newpage
%\chapter{Bowling}

In this project, you will create a favorite game of children - bowling. The player will have to shoot a ball from the bottom of the screen to reach the pins at the top. The goal is for the player to knock down all the pins.

\begin{figure}[H]
   \centering
   \includegraphics[width=1.0\linewidth,height=0.5\linewidth]{fig130001.png}
   \caption{Bowling}
\label{fig130001}
\end{figure}

\section{Creating the Design}

In the first step, you will create the game's home screen. There will be a button on it that will start the game. From the Layout group, the VerticalArrangement element must be added. The element's dimensions must be the same as the screen's dimensions. For this, the height and width properties need to be changed. For the background of the game, you can create your own color in addition to the ready-made colors.

\begin{figure}[H]
   \centering
   \includegraphics[width=1.0\linewidth,height=0.5\linewidth]{fig130002.png}
   \caption{Home Screen}
\label{fig130002}
\end{figure}

You should also add a button to start the game. Change the design of the button and position it in the middle of the screen.

\begin{figure}[H]
   \centering
   \includegraphics[width=1.0\linewidth,height=0.5\linewidth]{fig130003.png}
   \caption{Start Game Button}
\label{fig130003}
\end{figure}

In the next step, you will create the endgame screen. There will be a button on it to restart the game. You must also add a caption that says the game is over. First, from Layout, add a VerticalArrangment element. The element's dimensions should again be as they are on the screen. Also, choose a suitable background color for this screen.

\begin{figure}[H]
   \centering
   \includegraphics[width=1.0\linewidth,height=0.5\linewidth]{fig130004.png}
   \caption{End screen}
\label{fig130004}
\end{figure}

Add the button to start the game again. Change its design and position it in the middle of the screen.

\begin{figure}[H]
   \centering
   \includegraphics[width=1.0\linewidth,height=0.5\linewidth]{fig130005.png}
   \caption{Start Game Again Button}
\label{fig130005}
\end{figure}

Also, add a Lable element that says Game Over.

\begin{figure}[H]
   \centering
   \includegraphics[width=1.0\linewidth,height=0.5\linewidth]{fig130006.png}
   \caption{End of Game Caption}
\label{fig130006}
\end{figure}

In the final step, you will create a game design. Add a VerticalArrangment element again. Resize the element so that the height and width are as they are on the screen. Add the Canvas element from the Drawing and Animation section inside this element. Change element size and background color.

\begin{figure}[H]
   \centering
   \includegraphics[width=1.0\linewidth,height=0.5\linewidth]{fig130007.png}
   \caption{Game screen}
\label{fig130007}
\end{figure}

Add the ball element and change the dimensions, position, and color of the element. The ball must be placed at the bottom of the screen.

\begin{figure}[H]
   \centering
   \includegraphics[width=1.0\linewidth,height=0.5\linewidth]{fig130008.png}
   \caption{Adding the ball to the game}
\label{fig130008}
\end{figure}

You can use an image distributed under a free license for the pins. Add three ImageSprite elements to the Canvas element. Add the image you downloaded to each of the items. Change the dimensions and position. The three pins should be at the top of the screen.

\begin{figure}[H]
   \centering
   \includegraphics[width=1.0\linewidth,height=0.5\linewidth]{fig130009.png}
   \caption{Add the pins}
\label{fig130009}
\end{figure}

In the last part of the design, you should also add the ball launch button. First, add a HorizontalArrangement element. Change the width of the element to fit on the screen.

\begin{figure}[H]
   \centering
   \includegraphics[width=1.0\linewidth,height=0.5\linewidth]{fig130010.png}
   \caption{Add space for shot button}
\label{fig130010}
\end{figure}

Add two more elements to this element - a button and an element that will display how many balls the character has. Position the elements and change the text of the elements. You can also change the color and shape.

\begin{figure}[H]
   \centering
   \includegraphics[width=1.0\linewidth,height=0.5\linewidth]{fig130011.png}
   \caption{Add button to shoot the ball}
\label{fig130011}
\end{figure}

\section{Creating the Program}

Before programming the game, leaving only the game start screen visible. Then switch to the Add Blocks view.

Start by programming the game start button. Add the Click statement, which means the instructions will be executed when the button is pressed. When this button is pressed, it will set the speed of the ball.

\begin{figure}[H]
   \centering
   \includegraphics[width=1.0\linewidth,height=0.5\linewidth]{fig130012.png}
   \caption{Home Button Instructions}
\label{fig130012}
\end{figure}

Add a variable that will be the number of pins. Let the initial value be 3.

\begin{figure}[H]
   \centering
   \includegraphics[width=1.0\linewidth,height=0.5\linewidth]{fig130013.png}
   \caption{Variable for the number of pins}
\label{fig130013}
\end{figure}

When the ball touches one of the edges, it must deflect. To execute the ball deflection instructions, an event must first be added. Inside this event, the following checks must be done:
- if the ball is at the leftmost edge of the screen, it must move to the right
- if the ball is at the far right of the screen, it must move to the left
- if the ball is at the top edge of the screen, it must move to its original position.

\begin{figure}[H]
   \centering
   \includegraphics[width=1.0\linewidth,height=0.5\linewidth]{fig130014.png}
   \caption{Movement of the ball when it touches the edge of the screen}
\label{fig130014}
\end{figure}

The other event that applies to the ball is when it touches something other than the screen. In this game, it can only be a pin. Then the number of pins must be changed by removing one. Another essential thing to do is check for more pins on the screen. If not, this screen should be hidden, and the last screen should be shown. It is essential to change the message to say, Winner.

If the number of pins is not equal to 0, then a check for which pin touched the ball must be added. This aims to make it invisible and not be part of the game.

\begin{figure}[H]
   \centering
   \includegraphics[width=1.0\linewidth,height=0.5\linewidth]{fig130015.png}
   \caption{Instructions when the ball hits the pin}
\label{fig130015}
\end{figure}

The following instructions will be when the ball launch button is pressed. The first check is if the number of balls shot field is 0, then the game is over, and the end of the game screen should be displayed. If it is not equal to 0, then the ball's direction should be set, and the count should be decremented by 1.

\begin{figure}[H]
   \centering
   \includegraphics[width=1.0\linewidth,height=0.5\linewidth]{fig130016.png}
   \caption{Ball Shoot Button Instructions}
\label{fig130016}
\end{figure}

The last instructions to add are for the button that starts the game from the beginning. These instructions include displaying the game screen and the pins and setting the number of pins to 3 and the number of balls to 10.

\begin{figure}[H]
   \centering
   \includegraphics[width=1.0\linewidth,height=0.5\linewidth]{fig130017.png}
   \caption{Instructions for restart button}
\label{fig130017}
\end{figure}

We wish you an enjoyable game!
\newpage
%\chapter{Jump the Obstacles}

With this project you will create a popular computer game. The player will control a character using the arrow keys. His goal will be to cross the course without touching the obstacles. In cases where it touches them, it will start from the beginning. If he manages to pass the entire route, he moves to the next level.

\begin{figure}[H]
   \centering
   \includegraphics[width=1.0\linewidth,height=0.5\linewidth]{fig140001.png}
   \caption{Jump the obstacles}
\label{fig140001}
\end{figure}

\section{Creating the Design}
The first step in creating the game will be adding the appropriate characters and background. For the game character, you can choose from ready-made characters in Scratch. Another option is to draw it to make your game more interesting. Using the drawing tools you can create your character (Fig. \ref{fig140002}). In this project I will demonstrate how to create an effect when the character moves. For this purpose, one more suit should be added to it.

\begin{figure}[H]
   \centering
   \includegraphics[width=1.0\linewidth,height=0.5\linewidth]{fig140002.png}
   \caption{Adding Main Character}
\label{fig140002}
\end{figure}

The game will need one character to be in the lower part of the track. This is a revenge hero that will serve as a reference point. The purpose of the purple character is to move on it. Create it using the drawing tools.

\begin{figure}[H]
   \centering
   \includegraphics[width=1.0\linewidth,height=0.5\linewidth]{fig140003.png}
   \caption{Adding the supporting character}
\label{fig140003}
\end{figure}

The last character to add is the one for the obstacles. It must be drawn again with the auxiliary tools. To have more levels in the game, add more costumes to your character. When your purple character crosses the track, ie. reach the rightmost point of the screen, then that character will change his costume, which means the player will go to the second level.

\begin{figure}[H]
   \centering
   \includegraphics[width=1.0\linewidth,height=0.5\linewidth]{fig140004.png}
   \caption{Add obstacles}
\label{fig140004}
\end{figure}

To make the game more attractive, add a suitable background. Also, like the characters, you can choose it from the ready-made ones in Scratch or draw it yourself using the tools.

\begin{figure}[H]
   \centering
   \includegraphics[width=1.0\linewidth,height=0.5\linewidth]{fig140005.png}
   \caption{Add background}
\label{fig140005}
\end{figure}

\section{Programming Character Movement}

The main code you need to add is located in the character you will be controlling, in this case the purple character. In this game you will need 3 variables - xVel, yVel and jump. From the Variables section, select the Make a Variable option and add the variables with the appropriate names. To prevent them from appearing during gameplay, you can remove their ticks.

\begin{figure}[H]
   \centering
   \includegraphics[width=1.0\linewidth,height=0.5\linewidth]{fig140006.png}
   \caption{Adding the variables}
\label{fig140006}
\end{figure}

At the beginning of the game, the value of these variables should be equal to 0. To make the game more interesting, the character's position will be in the middle part of the screen. When the game starts he will drop down to the platform.

\begin{figure}[H]
   \centering
   \includegraphics[width=1.0\linewidth,height=0.5\linewidth]{fig140007.png}
   \caption{Initialize variables}
\label{fig140007}
\end{figure}

The character must move until it reaches the right side of the screen or until the character touches an obstacle. For this purpose, the first statement you need to add is a repeat until block. The conditions you need to add are two that are separated by the OR operator. The first condition is that the character's x position is greater than 240, which is the end of the screen. The second condition is to touch the red character.

\begin{figure}[H]
   \centering
   \includegraphics[width=1.0\linewidth,height=0.5\linewidth]{fig140008.png}
   \caption{Loop with character movement condition}
\label{fig140008}
\end{figure}

To move the character left and right you will use the xVel variable. From the variable section, the change xVel by statement is needed. From the operators section, select the one to subtract. On one side put the instruction for key right arrow pressed, and on the right arrow key left arrow pressed. This statement will set a value to the variable. All that remains is to add the change x by instruction and place the variable. So if you start the game, the character will move very fast left and right. For this add the instruction set xVel to xVel * 0.9. Start the game to test how the character moves left and right.

\begin{figure}[H]
   \centering
   \includegraphics[width=1.0\linewidth,height=0.5\linewidth]{fig140009.png}
   \caption{Character movement left and right}
\label{fig140009}
\end{figure}

You must program the character to jump In the first part, you will program when the game starts or another level that the character from the initial position goes down to the platform. To do this, you need to change the variable yVel to -1 and use the statement change y by with the value variable. Next is the check if it touches the platform. If the condition is met, then the value of the variable yVel should be set to 0. So the only problem will be that the character will not stop moving when it touches the platform, because the main loop will spin one more time, because not ready condition is not met. For this, the y position must be changed by yVel multiplied by -1.

\begin{figure}[H]
   \centering
   \includegraphics[width=1.0\linewidth,height=0.5\linewidth]{fig140010.png}
   \caption{Moving Character Down}
\label{fig140010}
\end{figure}

The last part of creating the purple character's movement algorithm will be adding the jumps. If the player presses the up arrow, then the jump variable will be equal to 1. In any of the other cases, it should be equal to 0, which means that the character will not jump. First you need to add a check before making yVel equal to 0. This check should check if yVel is less than 0 then the character should not jump.

\begin{figure}[H]
   \centering
   \includegraphics[width=1.0\linewidth,height=0.5\linewidth]{fig140011.png}
   \caption{Checking when character doesn't jump}
\label{fig140011}
\end{figure}

There is also a check to see if the player has pressed the up arrow. If it is and the character is not jumping, then yVel should be changed to a positive number and the jump variable should be set to 1. Playtest to see how your character jumps and moves left and right.

\begin{figure}[H]
   \centering
   \includegraphics[width=1.0\linewidth,height=0.5\linewidth]{fig140012.png}
   \caption{Hero jump when pressing up arrow}
\label{fig140012}
\end{figure}

\section{Creating an effect when the character moves}

After you have finished moving the character, you can add additional instructions to create an effect when the character moves. After checking for the pressed up arrow, add the create clone of myself statement. The reason for this is that a clone of that character will be created, but with his second costume. For this add a new instruction which is When I start as a clone. Then the character's costume should be changed, it should be his second. Add a loop that repeats 10 times. Inside the loop, reduce the size of the clone character and add a transparent effect. After the cycle, delete the clone. Test the game to see what effect is added when you move.

\begin{figure}[H]
   \centering
   \includegraphics[width=1.0\linewidth,height=0.5\linewidth]{fig140013.png}
   \caption{Add motion effect}
\label{fig140013}
\end{figure}

\section{Go to next level}

In the last step, you need to add instructions for the character to be able to move to the next level. For this purpose, in the main character, add the check if he has reached the end of the screen. If it is, then let it send a message.

\begin{figure}[H]
   \centering
   \includegraphics[width=1.0\linewidth,height=0.5\linewidth]{fig140014.png}
   \caption{Send Next Level Message}
\label{fig140014}
\end{figure}

Choose the hero with the red obstacles. Add the event when the game starts, then the character should be in his original costume. Also add a second event when he gets the next level message, he then has to change his suit. Start the game and have fun.

\begin{figure}[H]
   \centering
   \includegraphics[width=1.0\linewidth,height=0.5\linewidth]{fig140015.png}
   \caption{Go to next level}
\label{fig140015}
\end{figure}
\newpage
%\chapter{Darts}

This project represents the popular darts game. The player shoots short arrows at a circular target. The closer it is to the center of the target, the more points it earns. The goal of the player is to collect the maximum number of points for the three shots available. If the collected points are more than 13 - win the game, otherwise lose the game.

\begin{figure}[H]
   \centering
   \includegraphics[width=1.0\linewidth,height=0.5\linewidth]{fig150001.png}
   \caption{Darts}
\label{fig150001}
\end{figure}

\section{Creating the Design}
Before you start programming the game, you must first create the design. First, add the two main characters of the game - the target (Fig. \ref{fig150002}) and the arrow (Fig. \ref{fig150003}). Use the tools in Scratch to draw the required characters.

\begin{figure}[H]
   \centering
   \includegraphics[width=1.0\linewidth,height=0.5\linewidth]{fig150002.png}
   \caption{Target}
\label{fig150002}
\end{figure}

\begin{figure}[H]
   \centering
   \includegraphics[width=1.0\linewidth,height=0.5\linewidth]{fig150003.png}
   \caption{Arrow}
\label{fig150003}
\end{figure}

Add a new character that you can also draw using the Scratch tools. This character will be responsible for showing how many shots the player has left. To show how many shots are left, add costumes to that character. For example, on the first suit you can write that there are 3 shots left, and on the second - 2 shots.

\begin{figure}[H]
   \centering
   \includegraphics[width=1.0\linewidth,height=0.5\linewidth]{fig150004.png}
   \caption{Number of Shots}
\label{fig150004}
\end{figure}

The next character to add is the one that shows how many points the player has earned. Again, add costumes to the character with the corresponding number of points he can earn. In this case, these are from 0 to 5.

\begin{figure}[H]
   \centering
   \includegraphics[width=1.0\linewidth,height=0.5\linewidth]{fig150005.png}
   \caption{Number of points}
\label{fig150005}
\end{figure}

The last Scratch character is the inscription whether the player wins or loses. Like the previous two characters, add two suits with different inscriptions - one for winning and the other for losing.

\begin{figure}[H]
   \centering
   \includegraphics[width=1.0\linewidth,height=0.5\linewidth]{fig150006.png}
   \caption{End of Game Caption}
\label{fig150006}
\end{figure}

\section{Programming the target and boom}
Before we move on to programming the respective characters, add a new variable points to the game. This variable will contain the score that the character will accumulate during the game. From the Variables section, select Make a Variable and name the variable.

\begin{figure}[H]
   \centering
   \includegraphics[width=1.0\linewidth,height=0.5\linewidth]{fig150007.png}
   \caption{Creating in-game variable}
\label{fig150007}
\end{figure}

Let's move on to programming the target. The only instructions you need to add to this character are that it stays in the same position throughout the game. Don't forget to also set an initial value to the variable you created. At the beginning of each game, the player must have 0 points.

\begin{figure}[H]
   \centering
   \includegraphics[width=1.0\linewidth,height=0.5\linewidth]{fig150008.png}
   \caption{Target instructions}
\label{fig150008}
\end{figure}

You should also add instructions for the arrow. If the arrow is too big, you can resize it. Create a new variable to hold what the status of the arrow is. There are two statuses - shoot or throw. At the start of the game, the status of the arrow will be shoot.

\begin{figure}[H]
   \centering
   \includegraphics[width=1.0\linewidth,height=0.5\linewidth]{fig150009.png}
   \caption{Set Arrow Status}
\label{fig150009}
\end{figure}

Add the following check - if the arrow's status is shoot, then the arrow must follow the player's mouse. Also add a new event which is when this sprite clicked. When the player clicks the arrow, you need to change the value of the status variable to throw. Also, the character must send a message to the other characters that the player has clicked the mouse, which means he is shooting.

\begin{figure}[H]
   \centering
   \includegraphics[width=1.0\linewidth,height=0.5\linewidth]{fig150010.png}
   \caption{Send shooting message}
\label{fig150010}
\end{figure}

After the player has fired the arrow you need to program it to change its direction so that you can imitate a shot. Also resize it to make it smaller, as if it is moving away from the player. You know that when the player clicks on the arrow, it sends a message. Use this message to change the direction and size of the arrow after firing.

\begin{figure}[H]
   \centering
   \includegraphics[width=1.0\linewidth,height=0.5\linewidth]{fig150011.png}
   \caption{Change arrow direction and size after shot}
\label{fig150011}
\end{figure}

When the player fires the arrow, it starts moving away and sends a Check Points message, which should check how many points the player has earned. To determine the number of points earned, it is necessary to check what color the arrow touched. In order for the game to be fairest, the check that will be made is the black color (this is the tip of the arrow) to which color it has touched. Add a new variable to hold the currently earned points. The last instruction to add is for the arrow to send a message that it gives points to the player.

\begin{figure}[H]
   \centering
   \includegraphics[width=1.0\linewidth,height=0.5\linewidth]{fig150012.png}
   \caption{Check Points Earned}
\label{fig150012}
\end{figure}

To make the game even more interesting, you can add a variable to be a randomly assigned number. This variable will deflect the arrow to the side depending on what random number has landed.

\begin{figure}[H]
   \centering
   \includegraphics[width=1.0\linewidth,height=0.5\linewidth]{fig150013.png}
   \caption{Arrow deflection to the side}
\label{fig150013}
\end{figure}

\section{Programming the character that counts the number of shots}

In this step, you will add instructions that will show the player how many more shots he is allowed to make. To do this, select the character you added for the number of shots. Place this character in a position of your choice. Create a new variable that will contain the number of shots fired. Then make the necessary checks - if the number of shots is 3, then one should switch to a suit that shows three shots, if the number of shots is two - then one should switch to the suit with two shots, and so on. At the last check - if the number of shots is 0, then a message should be sent that the game is over.

\begin{figure}[H]
   \centering
   \includegraphics[width=1.0\linewidth,height=0.5\linewidth]{fig150014.png}
   \caption{Programming the number of shots}
\label{fig150014}
\end{figure}

\section{Programming the character that displays the number of points}

Select the character that shows how many points the player has earned. This hero, in addition to showing how many points there are, will also be responsible for reducing the number of shots, setting a new wind direction, as well as changing the status of the arrow to ready to shoot again.

Since you don't know how many points the player will earn, and to avoid doing thousands of checks, you can cross out the suit names by the number of points. So you can easily switch from suit to suit, since you have a variable that holds the number of points currently earned.

\begin{figure}[H]
   \centering
   \includegraphics[width=1.0\linewidth,height=0.5\linewidth]{fig150015.png}
   \caption{Programming the number of points}
\label{fig150015}
\end{figure}

\section{Programming the end game}

The last step left to do is program the endgame. Depending on how many points the player has earned, he will either win or lose after the third shot. If the number of points is greater than 13 - he wins. Otherwise, he loses.

\begin{figure}[H]
   \centering
   \includegraphics[width=1.0\linewidth,height=0.5\linewidth]{fig150016.png}
   \caption{Programming the end game}
\label{fig150016}
\end{figure}

You are ready to test how accurate and good you are in the game you created.
\newpage
%\addcontentsline{toc}{chapter}{Conclusions}
\chapter*{Conclusions}
\thispagestyle{empty}

Block programming is an effective and affordable way to introduce children to coding concepts. By breaking programming down into visual blocks, kids can learn to put together small programs without worrying about syntax or typing errors. The Scratch and App Inventor programming environments are proven tools for teaching block programming to children. Scratch's intuitive interface and colorful blocks make it an ideal option for younger children, while App Inventor's ability to create real mobile apps may appeal to older ones. The book provides a comprehensive guide to learning block programming with Scratch and App Inventor. It covers game design and creation, mobile app creation, and more. The book also includes step-by-step instructions and many visual examples to help kids understand programming concepts. Children can develop basic skills such as problem-solving, logical thinking, and creativity through block programming. These skills can be applied in future endeavors, including computer science and other science, technology, engineering, and mathematics fields. Overall, the Block Programming for Kids with Scratch and App Inventor book is an excellent resource for parents and educators who want to introduce children to programming possibilities. Using visual blocks and easy-to-understand instructions, kids can learn to code in a fun and engaging way, setting them up for future personal and professional success.

\newpage

% Списък с използвана литература и източници на информация.
%\addcontentsline{toc}{chapter}{References}
%\input{chapters/references}\newpage

% Азбучен указател на използваните термини.
\printindex

% Задна корица.
\includepdf[pages=-]{covers-en/back}

\end{document}
