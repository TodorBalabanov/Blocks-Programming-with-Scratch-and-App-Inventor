\documentclass[14pt, a4paper, openany]{book}

% Използване на български език.
\usepackage[T2A, T1]{fontenc}
\usepackage[utf8]{inputenc}
\usepackage[english, bulgarian]{babel}

% Използва се за групиране на изображения.
\usepackage{subcaption}

% Използване на графика.
\usepackage[pdftex]{graphicx}

% Използване на по прецизни позиции за изображенията.
\usepackage{float}

% Използване на PDF-и за кориците.
\usepackage{pdfpages}

% Използване на хедър и футър.
\usepackage{fancyhdr}

% Използване на кавички при цитиране.
%\usepackage{dirtytalk}
\usepackage{csquotes}

% Използва се за създаване на азбучен указател.
\usepackage{imakeidx}

% Добавя възможност за сензитивни хипер-връзки в самия документ.
\usepackage[pdftex, bookmarks, linktocpage]{hyperref}

% Команда с множество опции за настройка на поведението на пакета hyperref, с най-полезната опция - кирилизация на заглавията от Bookmarks в Acrobat.
\hypersetup{unicode=true, colorlinks=true, linkcolor=black, citecolor=black, urlcolor=black}

% Използва се за листинги с програмен код.
\usepackage{listings}

% Използва се за многоредови коментари.
\usepackage{verbatim}

% Използвасе за междуредово разстояние.
\usepackage{lipsum}

% Използва се за таблици, които да са на повече от една страница.
\usepackage{longtable}

% Заглавие.
\title{Блоково програмиране със Scratch и App Inventor}

% Автори.
\author{Тодор Балабанов, Галя Петрова, Антония Гузгунова}

% Директория с изображения.
\graphicspath{{images/}}

% Избор на активен език.
\selectlanguage{bulgarian}

% Текстове за декорация на страницата в горната и долната част.
\pagestyle{fancy}
\fancyhf{}
\fancyhead[LE,RO]{\thepage}
\fancyhead[RE]{Блоково програмиране със Scratch и App Inventor}
\fancyhead[LO]{Тодор Балабанов, Галя Петрова, Антония Гузгунова}
\fancyfoot[LE,RO]{Издателство ''Образование и познание'', 2023}

% Дебелина на разделителните линии.
\renewcommand{\headrulewidth}{2pt}
\renewcommand{\footrulewidth}{1pt}

% Генериране на азбучен указател.
\onecolumn
\makeindex[columns=2, title=Азбучен указател, intoc]

% Подменя думата използван а за ноемрация на фрагментите програмен код.
\renewcommand{\lstlistingname}{Листинг}

% Смяна на названието за списъка от листингите.
\renewcommand{\lstlistlistingname}{Списък на листингите}

% Определя характеристиките на листигните за програмния код.
\lstset{backgroundcolor=\color{gray!30}, breaklines=true, language=r, frame=single}

% Разстояние от ред и половина.
\linespread{1.5}

% Начало на документа.
\begin{document}

% Предна корица.
\includepdf[pages={1}]{covers/front}
\thispagestyle{empty}

% Страница с авторски права.
~\vfill
\thispagestyle{empty}

\noindent Авторски права \copyright\ 2023 \\

\noindent Тодор Балабанов, Галя Петрова, Антония Гузгунова \\ 

\noindent \textsc{Издателство \textquote{Образование и Познание}} \\
\noindent \textsc{https://www.obrazovaniebg.net/} \\

\noindent Разпространява се под свободен лиценз: \\ 
Creative Commons Attribution-NonCommercial-NoDerivatives 4.0 \\
International Public License \\
\url{https://creativecommons.org/licenses/by-nc-nd/4.0} \\

\noindent {\footnotesize This book was supported by the Bulgarian National Science Fund by the project \textquote{Mathematical models, methods, and algorithms for solving hard optimization problems to achieve high security in communications and better economic sustainability, KP-06-N52/7/19-11-2021}}. \\

\noindent \textit{Първо издание, 2023}


% Номериране на страниците със служебна информация.
\pagenumbering{roman}
\setcounter{page}{1}

% Таблица на съдържанието.
\addcontentsline{toc}{chapter}{Съдържание}
\tableofcontents\newpage

% Списък с фигурите.
\addcontentsline{toc}{chapter}{Списък на фигурите}
\listoffigures\newpage

% Списък с таблиците.
\addcontentsline{toc}{chapter}{Списък на таблиците}
\listoftables\newpage

% Списък с листингите.
\addcontentsline{toc}{chapter}{Списък на листингите}
\lstlistoflistings\newpage

% Номериране на страниците с основното изложение.
\pagenumbering{arabic}
\setcounter{page}{1}

% Отделните глави са в отделни файлове.
\addcontentsline{toc}{chapter}{Предговор}
\chapter*{Предговор}
\thispagestyle{empty}
\newpage
\chapter{Работни среди}

Блоковите програмни езици са подразделение на визуалните програмни езици. Същината на блоковите езици е, че програмните инструкции се въвеждат под формата на цветни блокове, а не както е в класическите програмни езици, чрез изписване на текстови команди. Най-основната цел на блоковите езици е да направят областта програмиране значително по-достъпна за начинаещите. Тази цел се постига чрез три основни направления. От синтактична гледна точка, инструкциите в блоковите езици са под формата на цветни иконки. Това значително намалява възможността за изписването на грешна програмна инструкция. На второ място се подобрява семантиката, като всяка от възможните програмни инструкции е добре документирана. На трето място е прагматизма, който позволява изучаването на различните състояния в които може да изпадне програмата. Програмните среди за блоково програмиране набират все по-голяма популярност през последното десетилетие. Някои от най-популярните са: Scratch, Blockly, App Inventor for Android, Ardublock и други. В тази книга ще се спрем на две от програмните среди за блоково програмиране, създадени в Масачузетския технологичен институт, Scratch и App Inventor for Android. Причината за този избор е, че Scratch има насоченост към най-малките, а именно децата в началните училищни класове, което много добре се съчетава с възможностите блоковите програми да бъдат визуализирани и на мобилен телефон, чрез App Inventor for Android. И при двете програмни среди не се изисква инсталирането на специализиран софтуер. Достатъчно е наличието на съвременен компютър, свързан в Интернет и съвременна версия на уеб браузър. 

\section{Първи стъпки в Sratch}

Работата в средата на Sratch започва със зареждане на главната уеб страница (Фиг. \ref{fig0001}), която се намира на адрес: \\ \href{https://scratch.mit.edu/}{https://scratch.mit.edu/}

\begin{figure}[H]
  \centering
  \includegraphics[width=1.0\linewidth,height=0.5\linewidth]{fig0001.png}
  \caption{Начална уеб страница на Sratch}
\label{fig0001}
\end{figure}

Програмната среда на Sratch е организирана на принципа на облачните услуги. Поради тази причина, всеки желаещ да използва услугата трябва да си направи регистрация (Фиг. \ref{fig0002}). Регистрацията се състои от потребителско име и парола.

\begin{figure}[H]
  \centering
  \includegraphics[width=1.0\linewidth,height=0.5\linewidth]{fig0002.png}
  \caption{Регистрация на потребител в Sratch}
\label{fig0002}
\end{figure}

След избора на потребителско име и парола следва определяне на географския регион в който се намира потребителят (Фиг. \ref{fig0003}).

\begin{figure}[H]
  \centering
  \includegraphics[width=1.0\linewidth,height=0.5\linewidth]{fig0003.png}
  \caption{Географско местоположение}
\label{fig0003}
\end{figure}

Платформата е насочена предимно към деца, изразяващи интерес към програмирането, но също така към родители и учители. Поради тази причина, системата събира информация за възрастта на потребителя (Фиг. \ref{fig0004}).

\begin{figure}[H]
  \centering
  \includegraphics[width=1.0\linewidth,height=0.5\linewidth]{fig0004.png}
  \caption{Възраст на потребителя}
\label{fig0004}
\end{figure}

Освен класификация по възраст, системата събира информация и за класификация по полова принадлежност. Тази информация е незадължителна, основно за да не бъде дискриминираща (Фиг. \ref{fig0005}).

\begin{figure}[H]
  \centering
  \includegraphics[width=1.0\linewidth,height=0.5\linewidth]{fig0005.png}
  \caption{Пол на потребителя}
\label{fig0005}
\end{figure}

Потребителският профил, освен с потребителско име и парола, трябва да бъде аоцииран и с адрес на електронна пощенска кутия (Фиг. \ref{fig0006}).

\begin{figure}[H]
  \centering
  \includegraphics[width=1.0\linewidth,height=0.5\linewidth]{fig0006.png}
  \caption{Адрес на електронна поща на потребителя}
\label{fig0006}
\end{figure}

Процесът по регистрация на потребител в системата е почти завършен (Фиг. \ref{fig0007}). Остава само стъпката за потвърждаване на избрания адрес за електронна поща.

\begin{figure}[H]
  \centering
  \includegraphics[width=1.0\linewidth,height=0.5\linewidth]{fig0007.png}
  \caption{Приключване на процеса за въвеждане на информация за потребителя}
\label{fig0007}
\end{figure}

Електронното писмо, за потвърждаване на потребителската регистрация, съдържа електронна препратка до уеб сайта на Sratch (Фиг. \ref{fig0008}). Тази препратка трябва да бъде последвана за да се завърши процесът по регистрация на нов потребител. 

\begin{figure}[H]
  \centering
  \includegraphics[width=1.0\linewidth,height=0.5\linewidth]{fig0008.png}
  \caption{Електронно съобщение за потвърждаване на електронния адрес}
\label{fig0008}
\end{figure}

Регистрацията на новия потребител приключва със зареждането на начален работен екран (Фиг. \ref{fig0009}). Горе, в дясно се вижда изписано потребителското име, избрано на първата стъпка от процеса по регистрацията.

\begin{figure}[H]
  \centering
  \includegraphics[width=1.0\linewidth,height=0.5\linewidth]{fig0009.png}
  \caption{Начален работен екран}
\label{fig0009}
\end{figure}

Успешната регистрация може да се потвърди и чрез създаването на много малък проект, който да демонстрира функционирането на развойната среда. За тази цел се избира опцията „Create“ от менюто (Фиг. \ref{fig0010}).

\begin{figure}[H]
  \centering
  \includegraphics[width=1.0\linewidth,height=0.5\linewidth]{fig0010.png}
  \caption{Избор на опция от менюто за нов проект}
\label{fig0010}
\end{figure}

Създаването на нов проект преминава през серия стъпки, свързани със заделянето на първоначално нужните ресурси (Фиг. \ref{fig0011}).

\begin{figure}[H]
  \centering
  \includegraphics[width=1.0\linewidth,height=0.5\linewidth]{fig0011.png}
  \caption{Зареждане на ресурсите}
\label{fig0011}
\end{figure}

След зареждането на новия проект се визуализира работното пространство (Фиг. \ref{fig0012}). Най- в ляво е списъкът с възможни програмни инструкции, под формата на парченца от пъзел. В централната част е работното пространство, където инструкциите се подреждат. А най- в дясно е активната сцена, където се визуализират действията, заложени в серията от инструкции. 

\begin{figure}[H]
  \centering
  \includegraphics[width=1.0\linewidth,height=0.5\linewidth]{fig0012.png}
  \caption{Организация на работното пространство}
\label{fig0012}
\end{figure}

Всяка компютърна програма има своя начална точка и своя крайна точка. В Scratch, за началото на програмата има специално отделен блок (програмна инструкция), която е показна на Фиг. \ref{fig0013}. Изпълнението на програмите, написани в Scratch, започва с натискането на зеления флаг. Точно поради тази причина, блокчето за старт на програмата е свързано със събитието за натискане на зеления флаг. 

\begin{figure}[H]
  \centering
  \includegraphics[width=1.0\linewidth,height=0.5\linewidth]{fig0013.png}
  \caption{Начало на програмата}
\label{fig0013}
\end{figure}

Една от най-интуитивните и същевременно лесно разбираеми инструкции е преместване с определен брой стъпки (Фиг. \ref{fig0014}). Главен актьор в началната сцена на Scratch е оранжевият котарак. Ако сцената не бъде променяна, то инструкциите за извършване на различни действия се насочват точно към този котарак. 

\begin{figure}[H]
  \centering
  \includegraphics[width=1.0\linewidth,height=0.5\linewidth]{fig0014.png}
  \caption{Инструкция за придвижване на стъпки}
\label{fig0014}
\end{figure}

След преместването на котарака е от съществено значение да има една пауза на изчакване, така че визуално да се забележи преместването. За тази цел може да се приложи инструкция за изчакване, за определен брой секунди (Фиг. \ref{fig0015}).

\begin{figure}[H]
  \centering
  \includegraphics[width=1.0\linewidth,height=0.5\linewidth]{fig0015.png}
  \caption{Инструкция за изчакване}
\label{fig0015}
\end{figure}

След изчакването, котката може да се върне на първоначалната си позиция, като се изпълни инструкция за придвижване с отрицателен брой стъпки (Фиг. \ref{fig0016}).

\begin{figure}[H]
  \centering
  \includegraphics[width=1.0\linewidth,height=0.5\linewidth]{fig0016.png}
  \caption{Инструкция за преместване обратно}
\label{fig0016}
\end{figure}

След като всички предвидени инструкции са изпълнени е разумно да се сложи край на програмата, за което е предвидено отделно блокче в списъка с инструкции (Фиг. \ref{fig0017}).

\begin{figure}[H]
  \centering
  \includegraphics[width=1.0\linewidth,height=0.5\linewidth]{fig0017.png}
  \caption{Инструкция за край}
\label{fig0017}
\end{figure}

Така написаната програма се изпълнява, чрез натискане на зеления флаг (Фиг. \ref{fig0018}), а при нужда от аварийно спиране се натиска червеният кръг, от дясно на зеления флаг.

\begin{figure}[H]
  \centering
  \includegraphics[width=1.0\linewidth,height=0.5\linewidth]{fig0018.png}
  \caption{Изпълнение на програмата}
\label{fig0018}
\end{figure}

Всяка програма, която се пише в Scratch се помества в отделен проект. Достъп до всички проекти на потребителя може да се получи от менюто „My Stuff“, което е част от списъка с опции за боравене с регистрирания потребител (Фиг. \ref{fig0019}).

\begin{figure}[H]
  \centering
  \includegraphics[width=1.0\linewidth,height=0.5\linewidth]{fig0019.png}
  \caption{Меню за организация на проектите}
\label{fig0019}
\end{figure}

Първоначално, всеки проект има служебно име (Фиг. \ref{fig0020}), което в последствие може да бъде променено. Едно от най-атрактивните предимства на програмната среда е, че проектите на потребителите могат да се споделят (Sharing) с много широка аудитория. Това позволява бърз трансфер на знания и умения, както и оценка за положения труд. 

\begin{figure}[H]
  \centering
  \includegraphics[width=1.0\linewidth,height=0.5\linewidth]{fig0020.png}
  \caption{Списък с проекти}
\label{fig0020}
\end{figure}

Най-голямото очарование блоковите езици получават от факта, че писането на инструкциите и формирането на цялостна програма прилича на подреждането на пъзел. Почти всички деца обичат да редят пазели. Харесват ярките цветове и красивите картини. Когато чарът на класическите пъзели се пренесе в една толкова атрактивна област, каквато е програмирането, резултатите могат да бъдат смайващи. 

\section{Първи стъпки в App Inventor}

Работата в средата на App Inventor започва със зареждане на главната уеб страница (Фиг. \ref{fig0021}), която се намира на адрес: \\ \href{https://appinventor.mit.edu/}{https://appinventor.mit.edu/}

\begin{figure}[H]
  \centering
  \includegraphics[width=1.0\linewidth,height=0.5\linewidth]{fig0021.png}
  \caption{Начална уеб страница на App Inventor}
\label{fig0021}
\end{figure}

Въпреки че App Inventor също е продукт на Масачузетския технологичен институт в някои аспекти работата с него се различава от начина по който се работи в Scratch. App Inventor също се предлага под формата на облачна услуга в която е необходима регистрация. За разлика от  Scratch, в App Inventor може да се пропусне създаването на потребителски профил, а влизането в системата да се осъществи, чрез класическа регистрация в услугата GMail. Този процес започва след избирането на оранжевия бутон „Create Apps!“ (Фиг. \ref{fig0022}).

\begin{figure}[H]
  \centering
  \includegraphics[width=1.0\linewidth,height=0.5\linewidth]{fig0022.png}
  \caption{Избор на GMail потребител за включване в системата}
\label{fig0022}
\end{figure}

След избора на потребител, с който да се работи в средата на  App Inventor е нужно този потребител да се автентифицира, чрез въвеждане на парола (Фиг. \ref{fig0023}).

\begin{figure}[H]
  \centering
  \includegraphics[width=1.0\linewidth,height=0.5\linewidth]{fig0023.png}
  \caption{Автентификация на потребителя}
\label{fig0023}
\end{figure}

За да бъде позволена работа в програмната среда на App Inventor се изисква съгласие от потребителя с общите условия на платформата (Фиг. \ref{fig0024}).

\begin{figure}[H]
  \centering
  \includegraphics[width=1.0\linewidth,height=0.5\linewidth]{fig0024.png}
  \caption{Общи условия за ползване на програмната среда}
\label{fig0024}
\end{figure}

Процесът по вход в програмната среда завършва с поздравителна уеб страница (Фиг. \ref{fig0025}). На тази страница е представена по-подробна информация за програмната среда, видът на стартираната инстанция и версия. 

\begin{figure}[H]
  \centering
  \includegraphics[width=1.0\linewidth,height=0.5\linewidth]{fig0025.png}
  \caption{Поздравителна страница}
\label{fig0025}
\end{figure}

Първата възможност, която се предлага на потребителя е да избере от възможности за разглеждане на няколко учебни проекта, служещи за първоначално въвеждане в начина за работа с програмната среда (Фиг. \ref{fig0026}).

\begin{figure}[H]
  \centering
  \includegraphics[width=1.0\linewidth,height=0.5\linewidth]{fig0026.png}
  \caption{Възможност за избор на учебни проекти}
\label{fig0026}
\end{figure}

Ако не бъде избран учебен проект или опцията за създаване на празен проект, системата насочва потребителя към страницата със списък от собствени проекти (Фиг. \ref{fig0027}).

\begin{figure}[H]
  \centering
  \includegraphics[width=1.0\linewidth,height=0.5\linewidth]{fig0027.png}
  \caption{Страница със списък от собствени проекти}
\label{fig0027}
\end{figure}

\newpage
\chapter{Програмни конструкции}

Компютърните програми са съставени от стриктна последователност инструкции. Такива последователности се наричат алгоритъм. Ние хората всеки ден изпълняваме различни алгоритми, като част от нашето ежедневие. Да се приготвим и да отидем на училище е един алгоритъм. Събуждаме се, ставаме, обличаме се, правим си сутрешния тоалет, закусваме, излизаме от вкъщи, придвижваме се до училище. Много ярък пример за алгоритъм са рецептите за готвене. В една рецепта има начални продукти, след това точни инструкции как продуктите са се обработят и смесят, като има ясна представа какъв трябва да бъде крайният резултат. При компютърните програми има основен набор от инструкции, които съставляват изразните средства на съответния програмен език. Чарът на блоковите езици е, че този основен набор от инструкции е представен визуално, под формата на цветни блокчета. Подредбата на цветните блокчета в строго определена последователност води до създаването на малки компютърни програми. 

В случая на Scratch, програмата има ясно определена стартова точка и ясно определена финална точка. При App Inventor подходът е малко по-различен. Там последователността от инструкции, съставляващи писаната програма, се въвежда в малки фрагменти, наречени събития. Събитията възникват при различни действия от страна на потребителя или операционната система. При Scratch говорим за последователно програмиране, а при App Inventor говорим за събитийно програмиране. Основните програмни конструкции в двете програмни среди до голяма степен са идентични, но има и някои съществени разлики. За да можем да пишем ефективни и надеждни програми е важно добре да познаваме изразните средства на програмните среди с които работим. 

\section{Изразни средства в Scratch}

Базовите градивни блокчета в Scratch са организирани в цветни групи (Фиг. \ref{fig0051}). Тази организация помага за по-бързо ориентиране и по-ефективна употреба на различните блокчета. 

\begin{figure}[H]
  \centering
  \includegraphics[width=1.0\linewidth,height=0.5\linewidth]{fig0051.png}
  \caption{Групиране на инструкциите}
\label{fig0051}
\end{figure}

Най-важното блокче в програмата е блокчето, което дава старт за изпълнение на инструкциите, които са подредени под него. Това блокче има зелен флаг (Фиг. \ref{fig0052}) и определя какво ще последва след стартирането на програмата.

\begin{figure}[H]
  \centering
  \includegraphics[width=1.0\linewidth,height=0.5\linewidth]{fig0052.png}
  \caption{Начална точка на програмата}
\label{fig0052}
\end{figure}

Блокчето за старт на програмата се намира в светло оранжевата група, която е предназначена да реагира на събития от страна на потребителя. Точният момент в който потребителят иска програмата да започне своето изпълнение е неопределен във времето и поради тази причина Scratch трябва да улови събитие, предизвикано от самия потребител. 

Второто по важност блокче служи за край на програмата (Фиг. \ref{fig0053}). То се намира в тъмно оранжевата група и има за задача да спре всички процеси, извършващи се по време на изпълнението на самата програма.

\begin{figure}[H]
  \centering
  \includegraphics[width=1.0\linewidth,height=0.5\linewidth]{fig0053.png}
  \caption{Крайна точка на програмата}
\label{fig0053}
\end{figure}

Тъмно оранжевата група съдържа блокчета за контрол на изпълнението. Тези блокчета позволяват програмата да поема по различни пътища, както и група от действия да се повтарят многократно. 

В Scratch блокчетата инструкции основно контролират картинки, наречени спрайтове (sprites). За разлика от обикновеното компютърно изображение, спрайтът е графичен обект, който съдържа множество кадри, показващи изображението на героя в различни конфигурации. Всяка нова програма в Scratch започва с един спрайт, на оранжевата котка, разположена на координати (x=0,y=0). Работното пространство е двуизмерна координатна система с център (0,0). 

\begin{figure}[H]
  \centering
  \includegraphics[width=1.0\linewidth,height=0.5\linewidth]{fig0054.png}
  \caption{Завършване веднага след започване}
\label{fig0054}
\end{figure}

Ако бъдат съединени, блокчетата за начало и за край (Фиг. \ref{fig0054}), то програмата не изпълнява нищо. Практически, тази програма приключва веднага след като е започнала. Програма, която не прави нищо е напълно безсмислена. За да започне нещо да се случва се използват блокчетата в синята група. Първото блокче инструктира котето да се премести 10 стъпки, като броя стъпки може да бъде променени, чрез изписване на друго число във вътрешността на блокчето (Фиг. \ref{fig0055}).

\begin{figure}[H]
  \centering
  \includegraphics[width=1.0\linewidth,height=0.5\linewidth]{fig0055.png}
  \caption{Преместване на героя}
\label{fig0055}
\end{figure}

Следващият блок в групата инструктира героя да се завърти на определено число градуси, по часовниковата стрелка, спрямо собствения си център (Фиг. \ref{fig0056}).

\begin{figure}[H]
  \centering
  \includegraphics[width=1.0\linewidth,height=0.5\linewidth]{fig0056.png}
  \caption{Завъртане по часовниковата стрелка}
\label{fig0056}
\end{figure}

Аналогично, със следващото блокче в групата, завъртането може да се изпълни и в посока обратна на часовниковата стрелка (Фиг. \ref{fig0057}).

\begin{figure}[H]
  \centering
  \includegraphics[width=1.0\linewidth,height=0.5\linewidth]{fig0057.png}
  \caption{Завъртане обратно на часовниковата стрелка}
\label{fig0057}
\end{figure}

Следващия блок в групата дава възможност героят да се премести на случайни координати или на координати посочени с мишката (Фиг. \ref{fig0058}).

\begin{figure}[H]
  \centering
  \includegraphics[width=1.0\linewidth,height=0.5\linewidth]{fig0058.png}
  \caption{Преместване на случайна позиция}
\label{fig0058}
\end{figure}

Движението на героя може да бъде зададено и чрез абсолютни координати с блокче, позволяващо да се впишат числа за абцисната и ординатната ос (Фиг. \ref{fig0059}).

\begin{figure}[H]
  \centering
  \includegraphics[width=1.0\linewidth,height=0.5\linewidth]{fig0059.png}
  \caption{Преместване по абсолютни координати}
\label{fig0059}
\end{figure}

Плавно придвижване, по предварително зададен интервал от време, е възможно на случайни координати или координати посочени с мишката, благодарение на следващото блокче в групата (Фиг. \ref{fig0060}).

\begin{figure}[H]
  \centering
  \includegraphics[width=1.0\linewidth,height=0.5\linewidth]{fig0060.png}
  \caption{Плъзгане до случайна позиция}
\label{fig0060}
\end{figure}

Плавното плъзгане до предварително зададени координати, за предварително определен интервал от време, е възможно с блокчето предназначено за тази цел (Фиг. \ref{fig0061}).

\begin{figure}[H]
  \centering
  \includegraphics[width=1.0\linewidth,height=0.5\linewidth]{fig0061.png}
  \caption{Плъзгане до зададени координати}
\label{fig0061}
\end{figure}

Анимираният герой има характеристика за ориентация, под формата на ъгъл. При 90 градуса, оранжевата котка гледа на дясно. За да се промени ориентацията на героя се използва блокче с възможност за въвеждане на конкретен ъгъл (Фиг. \ref{fig0062}).

\begin{figure}[H]
  \centering
  \includegraphics[width=1.0\linewidth,height=0.5\linewidth]{fig0062.png}
  \caption{Ъглова ориентация}
\label{fig0062}
\end{figure}

При по-сложни сценарии за управление на героя, понякога е нужно героят да следи показалеца на мишката. За тази цел има определено блокче, което изпълнява тази инструкция (Фиг. \ref{fig0063}).

\begin{figure}[H]
  \centering
  \includegraphics[width=1.0\linewidth,height=0.5\linewidth]{fig0063.png}
  \caption{Ориентация по показалеца на мишката}
\label{fig0063}
\end{figure}

Блокчетата могат да се поставят едно след друго, като за последователна промяна на относителните x и y координатите (относителни, спрямо текущата позиция) на героя има специално определени блокчета (Фиг. \ref{fig0064}).

\begin{figure}[H]
  \centering
  \includegraphics[width=1.0\linewidth,height=0.5\linewidth]{fig0064.png}
  \caption{Последователна промяна на относителни координати}
\label{fig0064}
\end{figure}

Освен относителна промяна на координатите е възможна и абсолютна промяна на координатите, като абсолютната промяна е спрямо центъра на координатната система (Фиг. \ref{fig0065}).

\begin{figure}[H]
  \centering
  \includegraphics[width=1.0\linewidth,height=0.5\linewidth]{fig0065.png}
  \caption{Последователна промяна на абсолютни координати}
\label{fig0065}
\end{figure}

При своето движение, когато анимираният герой достигне границите на работното пространство, единият вариант е движението да продължи извън видимата зона. Другият вариант е да се вземат мерки и героят да отскача от ръбовете на работното пространство. За това отскачане има конкретно блокче (Фиг. \ref{fig0066}). За да се илюстрира работата му е нужна малко по-сложна последователност от инструкции. При всяко стартиране на програмата, първо се променят относителните координати, а след това се извършва отскачане от ръба, ако е необходимо. За да бъде малко по-интересен сценарият за проверка, вместо фиксирани стойности за относително отместване се използва вграждане на едно от зелените блокчета, което позволява генериране на случайно число в предварително определен диапазон. Съществено е да се забележи, че зеленото блокче има овална форма, което подсказва, че то е предназначено за вграждане в някой от другите блокове, които имат овален слот. 

\begin{figure}[H]
  \centering
  \includegraphics[width=1.0\linewidth,height=0.5\linewidth]{fig0066.png}
  \caption{Отскачане от ръбовете}
\label{fig0066}
\end{figure}

Следващо, много полезно блокче, от групата на тъмно оранжевите е блокчето за изчакване на период от време (Фиг. \ref{fig0067}). Когато това блокче бъде поставено между блокчетата за начало и край, програмата изчаква зададения брой секунди, преди да преустанови изпълнението си. По време на изпълнение, ясно може да се забележи, че около последователността от инструкции се появява жълта рамка, която символизира режима на изпълняващи се инструкции. 

\begin{figure}[H]
  \centering
  \includegraphics[width=1.0\linewidth,height=0.5\linewidth]{fig0067.png}
  \caption{Инструкция за изчакване}
\label{fig0067}
\end{figure}

Групата на лилавите блокчета съдържат инструкции за външното оформление на анимирания герой. Първите две блокчета са предназначени за реплики (Фиг. \ref{fig0068}), които героят казва (изписват се както в комикс). Първото блокче задава текст, който стои на екрана до следващата инструкция. Точно за това е нужно да има няколко секунди изчакване, така че текстът да остане видим за потребителя. Второто блокче има и параметър с който да се определи колко секунди текстът да бъде видим за потребителя. 

\begin{figure}[H]
  \centering
  \includegraphics[width=1.0\linewidth,height=0.5\linewidth]{fig0068.png}
  \caption{Изписване на реплики за изговаряне}
\label{fig0068}
\end{figure}

Вторите две блокчета са предвидени за реплики, които анимираният герой си мисли, но не изрича. Разликата се състои в начина по който се визуализира текстът (Фиг. \ref{fig0069}).

\begin{figure}[H]
  \centering
  \includegraphics[width=1.0\linewidth,height=0.5\linewidth]{fig0069.png}
  \caption{Изписване на реплики, като мисъл}
\label{fig0069}
\end{figure}

Анимираните герои в Scratch са под формата на спрайтове. Спрайтът е набор от различни изображения за героя в различни пози. За смяната на тези различни пози се използват две блокчета (Фиг. \ref{fig0070}), като първото задава конкретен кадър в спрайта, а второто задава следващия кадър в последователността.

\begin{figure}[H]
  \centering
  \includegraphics[width=1.0\linewidth,height=0.5\linewidth]{fig0070.png}
  \caption{Смяна на пози}
\label{fig0070}
\end{figure}

На работната сцена освен анимираните герои (под формата на спрайтове) има и фоново изображение. Това фоново изображение също подлежи на промяна, за което са предвидени две отделни блочета (Фиг. \ref{fig0071}). С първото може да се избират фонови изображения напред, назад, по случаен принцип или с конкретно название, а с второто блокче следващото изображение в последователността. 

\begin{figure}[H]
  \centering
  \includegraphics[width=1.0\linewidth,height=0.5\linewidth]{fig0071.png}
  \caption{Смяна на фона}
\label{fig0071}
\end{figure}

За промяната на размера на анимирания герой има две конкретни блокчета, като първото променя размера в абсолютни стойности, а второто променя размера в проценти, спрямо оригиналния размер (Фиг. \ref{fig0072}).

\begin{figure}[H]
  \centering
  \includegraphics[width=1.0\linewidth,height=0.5\linewidth]{fig0072.png}
  \caption{Промяна на размерите}
\label{fig0072}
\end{figure}

За промяна на визуалното оформление на анимирания герой са предвидени три блокчета (Фиг. \ref{fig0073}). Първите две задават промяна, като промяната може да бъде в цвета, различни изкривявания, пикселизация, мозайка, прозрачност или яркост, а третото блокче отменя всички направени декорации. Първото блокче предизвиква относителна промяна, спрямо текущото състояние на героя, а второто блокче задава абсолютна промяна. Отново е важно да се дадат няколко секунди, така че промените да бъдат ясно различими. 

\begin{figure}[H]
  \centering
  \includegraphics[width=1.0\linewidth,height=0.5\linewidth]{fig0073.png}
  \caption{Промяна на външния вид}
\label{fig0073}
\end{figure}

Работата със спрайтове е предимно за постигане на анимирани ефекти. Различните анимирани герой в сцената имат определени взаимодействия по между си. Сценарият на изработвания проект определя в кой момент всеки от героите се появява на сцената и в кой момент изчезва. За да се осъществи появата и изчезването са предвидени две блокчета, извършващи тези действия (Фиг. \ref{fig0074}).

\begin{figure}[H]
  \centering
  \includegraphics[width=1.0\linewidth,height=0.5\linewidth]{fig0074.png}
  \caption{Скриване и повява}
\label{fig0074}
\end{figure}

Множество програмни продукти, работещи с растерни графични изображения, организират различните изображения в слоеве. Пример за такива са Adobe Photoshop, GIMP, Microsoft Word, LibreOffice Draw и много други. Организацията в слоеве е логична, тъй като различните спрайтове в определени моменти от времето могат да се припокриват. В някои от софтуерните пакети за графична обработка, наличието на слоеве се възприема като Z буфер. В Scratch също е налична възможността за работа със слоеве, като две конкретни блокчета позволяват спрайтът да се придвижва напред и назад по слоевете (Фиг. \ref{fig0075}).

\begin{figure}[H]
  \centering
  \includegraphics[width=1.0\linewidth,height=0.5\linewidth]{fig0075.png}
  \caption{Придвижване по слоевете}
\label{fig0075}
\end{figure}

Групата блокчета в пурпурен цвят са предназначени за звуково оформление. Изпълнението на звуци се постига с първите две блокчета в групата (Фиг. \ref{fig0076}). Първото блокче изпълнява звука докато той бъде приключен, а второто блокче го стартира и предава изпълнението към следващото блокче. С третото блокче всички изпълняващи се звуци биват спрени. Програмната среда позволява звуци да бъдат записани и от компютъра на потребителя. 

\begin{figure}[H]
  \centering
  \includegraphics[width=1.0\linewidth,height=0.5\linewidth]{fig0076.png}
  \caption{Изпълнение на звуци}
\label{fig0076}
\end{figure}

Две от характеристиките на звуците могат да се променят с блокчетата за височина (честотна) и стерео озвучаване (ляво/дясно). И двете блокчета имат числени стойности за посочените характеристики (Фиг. \ref{fig0077}).

\begin{figure}[H]
  \centering
  \includegraphics[width=1.0\linewidth,height=0.5\linewidth]{fig0077.png}
  \caption{Характеристики на звука}
\label{fig0077}
\end{figure}

За постигането на една по-богата звукова картина, силата на различните звуци може да се управлява с две блокчета (Фиг. \ref{fig0078}). Първото контролира силата на звука по абсолютна стойност, а второто като проценти. 

\begin{figure}[H]
  \centering
  \includegraphics[width=1.0\linewidth,height=0.5\linewidth]{fig0078.png}
  \caption{Сила на звука}
\label{fig0078}
\end{figure}

Оранжевата група блокчета са предназначени за възникване на събития. Събитията са инструмент за изпълнение на инструкции, когато няма ясна престава за момента в който програмните инструкции трябва да се изпълнят. Такова събитие е натискане на бутон по клавиатурата от страна на потребителя (Фиг. \ref{fig0079}).

\begin{figure}[H]
  \centering
  \includegraphics[width=1.0\linewidth,height=0.5\linewidth]{fig0079.png}
  \caption{Събитие за натискане на клавиш}
\label{fig0079}
\end{figure}

Кликването с мишката върху определен спрайт също може да бъде обработено с помощта на подходящо блокче (Фиг. \ref{fig0080}).

\begin{figure}[H]
  \centering
  \includegraphics[width=1.0\linewidth,height=0.5\linewidth]{fig0080.png}
  \caption{Събитие за кликане с мишката}
\label{fig0080}
\end{figure}

Смяната на фона също може да предизвика обработване на събитие. За тази цел има предвидено блокче (Фиг. \ref{fig0081}).

\begin{figure}[H]
  \centering
  \includegraphics[width=1.0\linewidth,height=0.5\linewidth]{fig0081.png}
  \caption{Събитие за смяна на фона}
\label{fig0081}
\end{figure}

Събитие може да бъде прихванато след изтичане на определено време към таймер или достигане на определено ниво на звук (Фиг. \ref{fig0082}).

\begin{figure}[H]
  \centering
  \includegraphics[width=1.0\linewidth,height=0.5\linewidth]{fig0082.png}
  \caption{Събитие от таймер или звук}
\label{fig0082}
\end{figure}

Работата със събития е свързана и с механизъм за предаване/получаване на съобщения. Един блок инструкции може да разпространи предварително дефинирано съобщение, а друг блок инструкции може да се абонира за получаването на точно този вид съобщение (Фиг. \ref{fig0083}).

\begin{figure}[H]
  \centering
  \includegraphics[width=1.0\linewidth,height=0.5\linewidth]{fig0083.png}
  \caption{Разпространяване и получаване на съобщения}
\label{fig0083}
\end{figure}

Тъй като работата с механизма за съобщения може да изисква синхронизация, то има отделно блокче, което разпространява съобщението и изчаква извършването на действията от прихващането му (Фиг. \ref{fig0084}). Програмистът може да създава различни съобщения, които да бъдат изпращани в различни ситуации. 

\begin{figure}[H]
  \centering
  \includegraphics[width=1.0\linewidth,height=0.5\linewidth]{fig0084.png}
  \caption{Разпространяване на съобщение с изчакване}
\label{fig0084}
\end{figure}

Най-важните, а и най-полезните блокчета са организирани в групата на тъмно оранжевите. Това са блокчета, които определят по коя пътека на изпълнение ще се поеме, спрямо възможните избори за изпълнение на инструкции. Когато желанието е определено действие да се изпълни многократно, при зададен брой повторения, за тази цел има конкретно блокче (Фиг. \ref{fig0085}). В програмирането, многократните повторения се осъществяват с помощта на конструкции за цикъл, какъвто е случаят и с това блокче за повторения.

\begin{figure}[H]
  \centering
  \includegraphics[width=1.0\linewidth,height=0.5\linewidth]{fig0085.png}
  \caption{Фиксиран брой повторения}
\label{fig0085}
\end{figure}

Думичката repeat от английски означава повтори. Числото в блокчето определя колко на брой повторения да бъдат изпълнени, а в слота на блочето се поставят инструкциите, които да бъдат повтаряни. В този пример, котето се премества на случайно избрани координати, след което следва изчакване от предварително определен брой секунди. В много редки ситуации има нужда от безкрайно повтарящ се цикъл, за което е предвидено отделно блокче (Фиг. \ref{fig0086}).

\begin{figure}[H]
  \centering
  \includegraphics[width=1.0\linewidth,height=0.5\linewidth]{fig0086.png}
  \caption{Безкрайни повторения}
\label{fig0086}
\end{figure}

Следващото блокче е едно от най-важните блокчета в програмирането. То се нарича блокче за изпълнение при условие (Фиг. \ref{fig0087}) или условен преход. Съдържанието на блокчето се изпълнява само, ако условието в заглавната му част се изпълнява.

\begin{figure}[H]
  \centering
  \includegraphics[width=1.0\linewidth,height=0.5\linewidth]{fig0087.png}
  \caption{Изпълнение при условие}
\label{fig0087}
\end{figure}

Това тъмно оранжево блокче не може да се използва само. То винаги е в съчетание с поне едно зелено блокче, а понякога и с две, както е в настоящия пример. Част от зелените блокчета са неправилни шестоъгълници и са направени така, че да пасват в заглавната част на някои от тъмно оранжевите блокчета. Самото шестоъгълно блокче има овален слот в който се поместват някои от зелените овални блокчета. В примера е избрано зелено блокче, което изисква равенство към конкретно число, а за овалното блокче се ползва генератор на случайни числа, според предварително зададен интервал. Ако условието в заглавната част на блокчето за условен преход не бъде изпълнено, то тялото се пропуска и се преминава към следващите инструкции, след блочкето. Блокчето за условен преход има и вариант в който се предвиждат слотове за изпълнение и на двете възможности – вярно условие или невярно условие (Фиг. \ref{fig0088}). Ако условието е изпълнено, се изпълнява първият блок с инструкции. Ако условието не е изпълнено, се изпълнява вторият блок с инструкции.

\begin{figure}[H]
  \centering
  \includegraphics[width=1.0\linewidth,height=0.5\linewidth]{fig0088.png}
  \caption{Изпълнение при условие с алтернатива}
\label{fig0088}
\end{figure}

Следващото интересно блокче прави изчакване докато се случи определено събитие. В случая, събитието е спрайтът да бъде докоснат с мишката (Фиг. \ref{fig0089}). Случи ли се това докосване, изпълнението на програмата продължава към следващото блокче. Какво събитие се очаква е определено с допълнително блокче (светло синьо), което има формата на неправилен шестоъгълник.

\begin{figure}[H]
  \centering
  \includegraphics[width=1.0\linewidth,height=0.5\linewidth]{fig0089.png}
  \caption{Изчакване на условие}
\label{fig0089}
\end{figure}

Последните три блокчета в групата на тъмно оранжевите трябва да се демонстрират заедно (Фиг. \ref{fig0090}). Първото блокче задава нова верига от инструкции, когато определен спрай бъде клониран (копие на оригиналния спрайт). Второто блокче служи за клониране на текущия спрайт. А третото блокче служи за изтриване на текущия спрайт. 

\begin{figure}[H]
  \centering
  \includegraphics[width=1.0\linewidth,height=0.5\linewidth]{fig0090.png}
  \caption{Клониране на спрайтове}
\label{fig0090}
\end{figure}

Групата на светлосините блокчета е посветена на взаимодействия, отнасящите се до спрайта. Второто блокче в групата е предвидено за изпълняване на условие, когато спрайтът докосне конкретен цвят. Блокчето е с шестоъгълна форма, което подсказва, че е предназначено за вграждане. За да се демонстрира работата на това блокче, ще се завърти един цикъл, който ще премества котето на случайни координати и ще изчаква малък интервал от време, преди следващото преместване (Фиг. \ref{fig0091}). 

\begin{figure}[H]
  \centering
  \includegraphics[width=1.0\linewidth,height=0.5\linewidth]{fig0091.png}
  \caption{Циклично прескачане на случайни координати}
\label{fig0091}
\end{figure}

Така направен цикълът ще се върти безкрайно, тъй като не е зададено условие за край. Точно в условието за край мое да се помести блокчето, определящо докосването на цвят. Към сцената ще добавим нов спрайт (Фиг. \ref{fig0092}), на една червена ябълка (Фиг. \ref{fig0093}), която котето трябва да хване. Щом я хване, ще спре да подсказа и ще измяука (Фиг. \ref{fig0094}). 

\begin{figure}[H]
  \centering
  \includegraphics[width=1.0\linewidth,height=0.5\linewidth]{fig0092.png}
  \caption{Добавяне на спрайт}
\label{fig0092}
\end{figure}

\begin{figure}[H]
  \centering
  \includegraphics[width=1.0\linewidth,height=0.5\linewidth]{fig0093.png}
  \caption{Избор на спрайт от галерията}
\label{fig0093}
\end{figure}

\begin{figure}[H]
  \centering
  \includegraphics[width=1.0\linewidth,height=0.5\linewidth]{fig0094.png}
  \caption{Позициониране на ябълката}
\label{fig0094}
\end{figure}

При работата със спрайтове, една от най-често решаваните задачи е дали два спрайта се докосват или припокриват. Има различни техники за установяване на колизии между спрайтове, но една от най-ефективните е докосването на определен цвят. При съвременните компютри се работи с малко над 16 милиона различни цвята. Едно разумно подбиране на цветовете, които имат героите, може да даде безгранични възможности за откриване на колизии. Тъй като ябълката е червена, изборът за край на цикъла е когато котето докосне червения цвят (Фиг. \ref{fig0095}).

\begin{figure}[H]
  \centering
  \includegraphics[width=1.0\linewidth,height=0.5\linewidth]{fig0095.png}
  \caption{Докосване по цвят}
\label{fig0095}
\end{figure}

С предходното блокче, независимо коя част на котето докосне ябълката, цикълът спира да се върти и се чува мяукането. Много по-фино определяне на колизията между спрайтовете може да се получи, ако само черният контур на котето се проверява за докосване до червения цвят на ябълката, за което служи следващото блокче (Фиг. \ref{fig0096}).

\begin{figure}[H]
  \centering
  \includegraphics[width=1.0\linewidth,height=0.5\linewidth]{fig0096.png}
  \caption{Колизия при два предварително зададени цвята}
\label{fig0096}
\end{figure}

Следващото блокче е с овална форма и доставя на програмата разстоянието между спрайта и показалеца на мишката. Овалната форма подсказва, че това блокче трябва да бъде вградено в някое от блокчетата за аритметични изрази (Фиг. \ref{fig0097}).

\begin{figure}[H]
  \centering
  \includegraphics[width=1.0\linewidth,height=0.5\linewidth]{fig0097.png}
  \caption{Разстояние до показалеца на мишката}
\label{fig0097}
\end{figure}

Понякога се налага потребителят да напише нещо. За да се даде тази възможност е следващото блокче в групата на светло сините (Фиг. \ref{fig0098}). Анимираният герой подканя потребителя, като в конкретен текст подсказва какво се очаква да бъде написано. 

\begin{figure}[H]
  \centering
  \includegraphics[width=1.0\linewidth,height=0.5\linewidth]{fig0098.png}
  \caption{Въвеждане на текст}
\label{fig0098}
\end{figure}

Следващото блокче е от шестоъгълните и е предназначено за вграждане. Това блокче връща резултат „истина“, когато бъде натиснат определен клавиш (Фиг. \ref{fig0099}).

\begin{figure}[H]
  \centering
  \includegraphics[width=1.0\linewidth,height=0.5\linewidth]{fig0099.png}
  \caption{Определяне на натиснат клавиш}
\label{fig0099}
\end{figure}

Сходно поведение може да се постигне и със следващото блокче, но вместо натискане на клавиш от клавиатурата се очаква натискане на клавиша на мишката (Фиг. \ref{fig0100}).

\begin{figure}[H]
  \centering
  \includegraphics[width=1.0\linewidth,height=0.5\linewidth]{fig0100.png}
  \caption{Определяне на натиснат бутон от мишката}
\label{fig0100}
\end{figure}

Следващите две блокчета са овални и също са за вграждане. Първото дава координатите на анимирания герой по абцисната ос, а второто дава координатите на анимирания герой по ординатната ос (Фиг. \ref{fig0101}).

\begin{figure}[H]
  \centering
  \includegraphics[width=1.0\linewidth,height=0.5\linewidth]{fig0101.png}
  \caption{Координати на анимирания герой}
\label{fig0101}
\end{figure}

По време на работа на програмата има функциониращ таймер, който отмерва времето от началото на изпълнението. Със следващото блокче този таймер може да се нулира (Фиг. \ref{fig0102}).

\begin{figure}[H]
  \centering
  \includegraphics[width=1.0\linewidth,height=0.5\linewidth]{fig0102.png}
  \caption{Нулиране на таймера}
\label{fig0102}
\end{figure}

Следващото блокче е от овалните и служи за доставяне на информация за фона, променливи или нивото на звука (Фиг. \ref{fig0103}).

\begin{figure}[H]
  \centering
  \includegraphics[width=1.0\linewidth,height=0.5\linewidth]{fig0103.png}
  \caption{Информация за компоненти от сцената}
\label{fig0103}
\end{figure}

Последното блокче в групата е предвидено също за вграждане и връща броя дни от година 2000 (Фиг. \ref{fig0104}).

\begin{figure}[H]
  \centering
  \includegraphics[width=1.0\linewidth,height=0.5\linewidth]{fig0104.png}
  \caption{Брой дни от началото на века}
\label{fig0104}
\end{figure}

\newpage
\chapter{Игра кликни и победи}

В този проект двата героя се придвижват напред, когато играчът клика върху синия или червения бутон. Колкото по- бързо играчът клика върху бутона, толкова по- бързо неговият герой се придвижва напред. Печели този играч, който първи достигне до зелената финална линия.

\begin{figure}[H]
  \centering
  \includegraphics[width=1.0\linewidth,height=0.5\linewidth]{fig0200.png}
  \caption{Кликни и победи}
\label{fig0200}
\end{figure}

\section{Добавяне на фон и герои}
Конструирането на играта започва с избора на фон. От секция Backdrops->Choose a Backdrop може да се избере подходящ от наличните, които Scratch предоставя.

\begin{figure}[H]
  \centering
  \includegraphics[width=1.0\linewidth,height=0.5\linewidth]{fig0201.png}
  \caption{Избор на подходящ фон за играта}
\label{fig0201}
\end{figure}

В тази гира, за да се направи състезателното трасе заедно със зелената финална линия, трябва фонът да бъде подобрен. За тази цел първо се избира опция Bacdrops.

\begin{figure}[H]
  \centering
  \includegraphics[width=1.0\linewidth,height=0.5\linewidth]{fig0202.png}
  \caption{Рисуване на допълнителни елементи върху фона}
\label{fig0202}
\end{figure}

С помощта на инструмента линия се добавя състезателното трасе. Ако се смени дебелината и цвета на линията може да бъде добавена и финалната линия.

\begin{figure}[H]
  \centering
  \includegraphics[width=1.0\linewidth,height=0.5\linewidth]{fig0203.png}
  \caption{Финален фон на играта}
\label{fig0203}
\end{figure}

Ако играта няма нужда от основния герой в Scratch котката, той може да бъде изтрит.

\begin{figure}[H]
  \centering
  \includegraphics[width=1.0\linewidth,height=0.5\linewidth]{fig0204.png}
  \caption{Изтриване на основния герой}
\label{fig0204}
\end{figure}

Следва да бъдат добавени и героите към играта. В Scratch има налични много спрайтове. За целите на тази игра са необходими два - един, който е позициониран в ляво и друг - в дясно (Фиг. \ref{fig0205}). Чрез свойствата Size и Direction се променя размерът и посоката на героя.

\begin{figure}[H]
  \centering
  \includegraphics[width=1.0\linewidth,height=0.5\linewidth]{fig0205.png}
  \caption{Герои в играта}
\label{fig0205}
\end{figure}

Освен тези два спрайта са необходими и още два, които са бутоните, върху които играчите трябва да кликат. Те се намират отново в секция Sprite. В тази игра бутоните трябва да бъдат различен цвят, за да се различават. За да се промени цвета на бутон, то трябва да се промени неговия костюм .

\begin{figure}[H]
  \centering
  \includegraphics[width=1.0\linewidth,height=0.5\linewidth]{fig0206.png}
  \caption{Син бутон}
\label{fig0206}
\end{figure}

С помощта на инструмента Fill се сменя цвета на бутона (Фиг. \ref{fig0207}). Същото може да се направи и за червения бутон. Отново чрез свойството Size може да промени размерът на този спрайт.

\begin{figure}[H]
  \centering
  \includegraphics[width=1.0\linewidth,height=0.5\linewidth]{fig0207.png}
  \caption{Червен бутон}
\label{fig0207}
\end{figure}

\section{Програмиране на синия бутон}
Когато играчът кликне върху синия бутон, той трябва да изпрати съобщение "blue". Първото начално блокче, което трябва да се постави е когато този спрайт бъде кликнат.

\begin{figure}[H]
  \centering
  \includegraphics[width=1.0\linewidth,height=0.5\linewidth]{fig0208.png}
  \caption{Когато героят е кликнат}
\label{fig0208}
\end{figure}

Следва героят да изпрати съобщение "blue". От тъмно оранжевата група инструкцията за разпространение на съобщение. Съобщението, което трябва да разпространи е "blue".

\begin{figure}[H]
  \centering
  \includegraphics[width=1.0\linewidth,height=0.5\linewidth]{fig0209.png}
  \caption{Изпращане на съобщение}
\label{fig0209}
\end{figure}

Кодът на този герой изглежда по следния начин:

\begin{figure}[H]
  \centering
  \includegraphics[width=1.0\linewidth,height=0.5\linewidth]{fig0210.png}
  \caption{Целият код на синия бутон}
\label{fig0210}
\end{figure}

\section{Програмиране на червения бутон}
Когато играчът кликне върху червения бутон, аналогично на синия, той трябва да изпрати съобщение "red". Кодът на червения бутон е следния:

\begin{figure}[H]
  \centering
  \includegraphics[width=1.0\linewidth,height=0.5\linewidth]{fig0211.png}
  \caption{Целият код на червения бутон}
\label{fig0211}
\end{figure}

До този момент програмата се състои в това, когато играчът натисне синия или червения бутон, те да изпращат съответните съобщения.

\section{Програмиране героите да се движат}
Инструкциите на синия бутон са, че той изпраща съобщение. Левият герой трябва да се абонира да получи съобщението.

\begin{figure}[H]
  \centering
  \includegraphics[width=1.0\linewidth,height=0.5\linewidth]{fig0212.png}
  \caption{Абониране за съобщението от синия бутон}
\label{fig0212}
\end{figure}

Героят трябва да се премести надясно към зелената финална линия, което означава, че трябва да се промени x координатата, като се увеличи с 3 стъпки.

\begin{figure}[H]
  \centering
  \includegraphics[width=1.0\linewidth,height=0.5\linewidth]{fig0213.png}
  \caption{Движение на героя надясно}
\label{fig0213}
\end{figure}

Инструкциите за другият герой е аналогичен. Основните разлики са две:
- съобщението, за което този герой се абонира е изпратено от червения бутон
- героят трябва да се движи наляво към зелената финална линия, което означава, че трябва да се промени x координатата, като се намали с 3 стъпки

\begin{figure}[H]
  \centering
  \includegraphics[width=1.0\linewidth,height=0.5\linewidth]{fig0214.png}
  \caption{Движение на героя наляво}
\label{fig0214}
\end{figure}

\section{Програмиране победителя}
За да се завърши играта остава да се направи проверка кой от героите е достигнал зелената финална линия.

В десния герой първата инструкция, която трябва да се добави е началната за начало на играта. Във всеки един момент на играта трябва да се проверява дали героят е достигнал до финалната линия. Поради тази причина трябва да се добави инструкция за цикъл завинаги.

\begin{figure}[H]
  \centering
  \includegraphics[width=1.0\linewidth,height=0.5\linewidth]{fig0215.png}
  \caption{Цикъл завинаги}
\label{fig0215}
\end{figure}

Вътре в тялото на цикъла следва да се направи проверката, за това дали героят е докоснал финалната линия.

\begin{figure}[H]
  \centering
  \includegraphics[width=1.0\linewidth,height=0.5\linewidth]{fig0216.png}
  \caption{Проверка дали героят е достигнал финал}
\label{fig0216}
\end{figure}

За да се избере същия зелен цвят, какъвто е на финалната линия, трябва да се използва инструментът пипетка.

\begin{figure}[H]
  \centering
  \includegraphics[width=1.0\linewidth,height=0.5\linewidth]{fig0217.png}
  \caption{Избор на цвят}
\label{fig0217}
\end{figure}

Инструкциите, които се намират вътре в условието ще се изпълнят, когато героят победи. Тогава той трябва да уголеми размера си и да изпише съобщение "Победих!".

\begin{figure}[H]
  \centering
  \includegraphics[width=1.0\linewidth,height=0.5\linewidth]{fig0218.png}
  \caption{Инструкциите за победа}
\label{fig0218}
\end{figure}

Последното подобрение, което трябва да се направи, за да се завърши този герой е да бъде поставен в начална позиция всеки път, когато играта започне. В синята секция се намира инструкцията, която казва къде да бъде позициониран героят.

\begin{figure}[H]
  \centering
  \includegraphics[width=1.0\linewidth,height=0.5\linewidth]{fig0219.png}
  \caption{Инструкцията за позициониране на героя}
\label{fig0219}
\end{figure}

Финалния код на този герой е:

\begin{figure}[H]
  \centering
  \includegraphics[width=1.0\linewidth,height=0.5\linewidth]{fig0220.png}
  \caption{Финален код на левия герой}
\label{fig0220}
\end{figure}

След като този герой е готов, остава да се добавят и инструкциите за победа и на другия герой. Кодът е аналогичен. Единствената разлика е в началните координати.

\begin{figure}[H]
  \centering
  \includegraphics[width=1.0\linewidth,height=0.5\linewidth]{fig0221.png}
  \caption{Финален код на десния герой}
\label{fig0221}
\end{figure}

Време е за забавление заедно с приятели и съпоставяне на силите, за това кой герой по- бързо ще достигне до финала.\newpage
\chapter{Bouquet for Mom}

The ultimate goal of this project is for the children to create a bouquet for their mothers. When the player clicks on this spot, a beautiful flower will appear. To get the most beautiful bouquet, the player can change the type of flowers using the up arrow. Use the left and right arrows to adjust the size of the flower.

\begin{figure}[H]
   \centering
   \includegraphics[width=1.0\linewidth,height=0.5\linewidth]{fig040001.png}
   \caption{Bouquet for Mom}
\label{fig040001}
\end{figure}

\section{Adding Background and Characters}
The first step in creating the game is adding a suitable background. From the Backdrops->Choose a Backdrop section. A suitable one can be selected from the available ones that Scratch provides.

In this game, there will be no need for the main character in Scratch the cat. For this, he must be deleted.

\begin{figure}[H]
   \centering
   \includegraphics[width=1.0\linewidth,height=0.5\linewidth]{fig040002.png}
   \caption{Game Background}
\label{fig040002}
\end{figure}

The character representing a flower petal created when the player clicks on the screen should be added. This sprite must be drawn using the tools. More than one costume can be added for this character to make the game more interesting. Each of the suits will represent a different petal.

\begin{figure}[H]
   \centering
   \includegraphics[width=1.0\linewidth,height=0.5\linewidth]{fig040003.png}
   \caption{Drawing Petal Character}
\label{fig040003}
\end{figure}

\section{Drawing Flowers}

When the player clicks on the background, a flower will be drawn. This means that instructions should be placed on the background when clicked to send a "draw" message.

\begin{figure}[H]
   \centering
   \includegraphics[width=1.0\linewidth,height=0.5\linewidth]{fig040004.png}
   \caption{Background Instructions}
\label{fig040004}
\end{figure}

A new instruction section should be added to draw the flowers and stems. This is the Pen section.

\begin{figure}[H]
   \centering
   \includegraphics[width=1.0\linewidth,height=0.5\linewidth]{fig040005.png}
   \caption{Add Pen Section}
\label{fig040005}
\end{figure}

When the game starts, the petal character must be hidden, and everything drawn up to that point must be erased. The instruction that erases everything on the screen is located in the new Pen section and is "erase all".

\begin{figure}[H]
   \centering
   \includegraphics[width=1.0\linewidth,height=0.5\linewidth]{fig040006.png}
   \caption{Start of the game}
\label{fig040006}
\end{figure}

Before drawing the flower itself, its companion will be drawn first. Drawing the handle will start when the "draw" message is received. First, the thickness of the pencil that will be drawn and the color will be set. First, the pencil must be positioned. The initial x coordinate is the same as the mouse's. The instruction in the light blue group gives the mouse position for x. The initial y coordinate should be -150. Once the character is positioned, the pencil should be instructed to come down to start drawing. To complete the flower's stem, the character must go to where the mouse is and pick up the pencil.

The result of the following code (Fig. \ref{fig040007}) is that the flower will be drawn when the player clicks on different places on the screen.

\begin{figure}[H]
   \centering
   \includegraphics[width=1.0\linewidth,height=0.5\linewidth]{fig040007.png}
   \caption{Drawing the flower stems}
\label{fig040007}
\end{figure}

The drawn character must leave traces, like a stamp, to get the flower. Each time it leaves a trail, it must rotate and decrease its size by one. This algorithm must be repeated. In this way, the effect of the flowers is achieved. To obtain different flowers, the iterations of the drawing algorithm will be a random number between 40 and 60.

\begin{figure}[H]
   \centering
   \includegraphics[width=1.0\linewidth,height=0.5\linewidth]{fig040008.png}
   \caption{Drawing the flowers}
\label{fig040008}
\end{figure}

The result so far is that when the player clicks on different places on the screen, the flowers for mom will appear.

\section{Changing the type and size of flowers}

To make the game more interesting, using the up arrow will change the character's costumes. So the more suits there are, i.e., the more petals there are, the more diverse the bouquet will be.

\begin{figure}[H]
   \centering
   \includegraphics[width=1.0\linewidth,height=0.5\linewidth]{fig040009.png}
   \caption{Character Costume Changes}
\label{fig040009}
\end{figure}

Another improvement that can be made is to change the size of the flowers. For this purpose, the first thing to be done is a variable to hold the character's size. A characteristic of variables is that they have an initial value, and that value can be changed. In this case, the initial value of the variable will be 100. Pressing the left arrow will decrease the variable by 10, and pressing the right arrow will increase the variable by 10. This is what the final program code looks like:

\begin{figure}[H]
   \centering
   \includegraphics[width=1.0\linewidth,height=0.5\linewidth]{fig040010.png}
   \caption{Full program code}
\label{fig040010}
\end{figure}

The children must present the most beautiful bouquet to their mothers!
\newpage
\chapter{Унгарска осморка}

\begin{figure}[H]
  \centering
  \includegraphics[width=1.0\linewidth,height=0.5\linewidth]{fig050001.png}
  \caption{„Унгарска осморка“ \\ https://www.sfu.ca/~jtmulhol/math302/images/pic-puzzle-hr.png}
\label{fig050001}
\end{figure}

Играта „Унгарска осморка“ (Фиг. \ref{fig050001}) е от групата на логическите пъзели, каквото е и кубчето на Рубик. Състои се от две пресичащи се окръжности, улеи запълнени с цветни топчета. Окръжностите се пресичат в две точки, така че две от топчетата принадлежат и на двата улея. Топчетата в улеите могат да се въртят, така че пъзелът да се разбърка. След разбъркването, целта на играта е топчетата да се подредят в първоначалното състояние. 

Макар и играта да изглежда относително елементарна, подреждането й е свързано с относителна сложност и не малко математика. Самата организация на пъзела не е особено сложна, което го прави идеален кандидат за реализация в програмна среда, каквато е Scratch. Само с помощта на тридесет и осем пулчета и четири стрелки, може да се изгради целият интерфейс на играта. Работата започва със започването на нов проект (Фиг. \ref{fig050002}).

\begin{figure}[H]
  \centering
  \includegraphics[width=1.0\linewidth,height=0.5\linewidth]{fig050002.png}
  \caption{Създаване на проект за „Унгарска осморка“}
\label{fig050002}
\end{figure}

Първата стъпка е да се избере име на проекта и съответно да се почисти работното пространство от спрайта на котката (Фиг. \ref{fig050003}). 

\begin{figure}[H]
  \centering
  \includegraphics[width=1.0\linewidth,height=0.5\linewidth]{fig050003.png}
  \caption{Избор на име за проекта}
\label{fig050003}
\end{figure}

Разполагането на цветните топчета (в този случай пулчета) е относително трудоемка задача, тъй като трябва да се опишат две пресичащи се окръжности. Работата значително може да се улесни, ако се работи по предварително изготвена схема (Фиг. \ref{fig050004}). Схемата на пулчетата ще бъде видима само докато се подредят останалите спрайтове. В режим на игра схемата ще бъде направена невидима.

\begin{figure}[H]
  \centering
  \includegraphics[width=1.0\linewidth,height=0.5\linewidth]{fig050004.png}
  \caption{Схема на играта \\ https://www.sfu.ca/~jtmulhol/math302/images/hungarianrings-labeled-nocolor.png}
\label{fig050004}
\end{figure}

Графичният файл за схемата на играта се добавя в групата от спрайтове, чрез иконката за добавяне (Фиг. \ref{fig050005}).

\begin{figure}[H]
  \centering
  \includegraphics[width=1.0\linewidth,height=0.5\linewidth]{fig050005.png}
  \caption{Добавяне на изображението за схемата на играта}
\label{fig050005}
\end{figure}

Ново добавеният спрайт се центрира на координати x=0 и y=0 (Фиг. \ref{fig050006}), за да се постигне визуална симетрия и изображенията на предстоящите за добавяне пулчета да са в средата на визуалното пространство. 

\begin{figure}[H]
  \centering
  \includegraphics[width=1.0\linewidth,height=0.5\linewidth]{fig050006.png}
  \caption{Центриране на схемата}
\label{fig050006}
\end{figure}

От галерията с предварително налични спрайтове (Фиг. \ref{fig050007}) може да се избере подходящ спрайт за стрелките с които ще се предизвиква въртенето на двете окръжности.

\begin{figure}[H]
  \centering
  \includegraphics[width=1.0\linewidth,height=0.5\linewidth]{fig050007.png}
  \caption{Галерия с предварително налични спрайтове}
\label{fig050007}
\end{figure}

Спрайтът за стрелка има четири състояния (Фиг. \ref{fig050008}), които позволяват този спрайт да бъде използван за посоките на въртене.

\begin{figure}[H]
  \centering
  \includegraphics[width=1.0\linewidth,height=0.5\linewidth]{fig050008.png}
  \caption{Спрайт за стрелка}
\label{fig050008}
\end{figure}

Първата стрелка се разполага горе-дясно (Фиг. \ref{fig050009}), като тя ще служи за завъртане на десният ринг в посока по часовниковата стрелка.

\begin{figure}[H]
  \centering
  \includegraphics[width=1.0\linewidth,height=0.5\linewidth]{fig050009.png}
  \caption{Стрелка за завъртане на десния ринг по часовниковата стрелка}
\label{fig050009}
\end{figure}

Втората стрелка се разполага долу-дясно (Фиг. \ref{fig050010}), като тя ще служи за завъртане на десния ринг в посока обратна на часовниковата стрелка.

\begin{figure}[H]
  \centering
  \includegraphics[width=1.0\linewidth,height=0.5\linewidth]{fig050010.png}
  \caption{Стрелка за завъртане на десния ринг обратно на часовниковата стрелка}
\label{fig050010}
\end{figure}

Третата стрелка се разполага долу-ляво, като се избира за активен вторият кадър в спрайта, така че стрелката да сочи на ляво (Фиг. \ref{fig050011}).

\begin{figure}[H]
  \centering
  \includegraphics[width=1.0\linewidth,height=0.5\linewidth]{fig050011.png}
  \caption{Стрелка за завъртане на левия ринг по часовниковата стрелка}
\label{fig050011}
\end{figure}

Четвъртата стрелка се разполага горе-ляво (Фиг. \ref{fig050012}), като нейната задача е да води до завъртане на левия ринг по посока обратна на часовниковата стрелка.

\begin{figure}[H]
  \centering
  \includegraphics[width=1.0\linewidth,height=0.5\linewidth]{fig050012.png}
  \caption{Стрелка за завъртане на левия ринг обратно на часовниковата стрелка}
\label{fig050012}
\end{figure}

Една от възможностите за онагледяване на топчетата от оригиналната игра е чрез спрайта за топка (Фиг. \ref{fig050013}). Този спрайт позволява топката да бъде визуализирана в няколко различни цвята, което идеално пасва на необходимостта да се визуализират четири различни по цвят топчета.

\begin{figure}[H]
  \centering
  \includegraphics[width=1.0\linewidth,height=0.5\linewidth]{fig050013.png}
  \caption{Избор на спрайт за точетата}
\label{fig050013}
\end{figure}

Първото поставено пулче отива на позицията означена с номер едно на вече включената схема. За да се наслагват успешно различните спрайтове е необходимо спрайта на схемата да бъде изпратен най-отзад в Z-буфера, така че всички други спрайтове да се визуализират пред него. 

Пулчето се смалява (в случая до 55) и след това с помощта на мишката се нагласява да попадне точно върху пространството, маркирано с единица (Фиг. \ref{fig050014}). 

\begin{figure}[H]
  \centering
  \includegraphics[width=1.0\linewidth,height=0.5\linewidth]{fig050014.png}
  \caption{Оразмеряване и позициониране на първото пулче}
\label{fig050014}
\end{figure}

Когато игата започне (Фиг. \ref{fig050015}), състоянието на игралното поле ще се отразява в списъчна структура. Числата от едно до четири ще отразяват какъв цвят топче трябва да бъде визуализирано на позицията със съответен номер.

\begin{figure}[H]
  \centering
  \includegraphics[width=1.0\linewidth,height=0.5\linewidth]{fig050015.png}
  \caption{Начало на играта в програмното поле на сцената}
\label{fig050015}
\end{figure}

За тази цел се създава списък „state“ и в него ще бъдат записани числата, определящи цветовете на пулчетата. 

\begin{figure}[H]
  \centering
  \includegraphics[width=1.0\linewidth,height=0.5\linewidth]{fig050016.png}
  \caption{Списък за състоянието на игралното табло}
\label{fig050016}
\end{figure}

Съдържанието на списъка винаги първо изцяло се изтрива, за да не останат стойности от предишно стартиране. Първите пет позиции са с първия цвят (Фиг. \ref{fig050017}).

\begin{figure}[H]
  \centering
  \includegraphics[width=1.0\linewidth,height=0.5\linewidth]{fig050017.png}
  \caption{Цвят на първите пет позиции}
\label{fig050017}
\end{figure}

На шеста позиция се появява четвъртия цвят, след което от седма до шестнадесета позиця е втория цвят. Следват четири позиции от първия цвят, а от двадесет и едно до тридесет са позиции на третия цвят. Всички останали пулове са с четвъртия цвят (Фиг. \ref{fig050018}).

\begin{figure}[H]
  \centering
  \includegraphics[width=1.0\linewidth,height=0.5\linewidth]{fig050018.png}
  \caption{Общо състояние по цветове}
\label{fig050018}
\end{figure}

След промяна във вътрешното състояние на списъка е важно да се разпрати съобщение за обновяване на всички спрайтове, които визуализират пулчетата (Фиг. \ref{fig050019}). 

\begin{figure}[H]
  \centering
  \includegraphics[width=1.0\linewidth,height=0.5\linewidth]{fig050019.png}
  \caption{Съобщение за визуализация}
\label{fig050019}
\end{figure}

Съобщението се изпраща с команда, която изчаква изпълнението му (Фиг. \ref{fig050020}). Всяко от 38-те пулчета ще се абонира за получаване на това съобщение. 

\begin{figure}[H]
  \centering
  \includegraphics[width=1.0\linewidth,height=0.5\linewidth]{fig050020.png}
  \caption{Изпращане с изчакване}
\label{fig050020}
\end{figure}

Преди да започне копирането на първия пул, така че да се размножи още 37 пъти, следва да се състави кодът, който ще прослушва за съобщение за изрисуване. Първо се написва този код, така че той да се размножи 37 пъти при дублирането на спрайта. Пул с номер едно прави четири проверки в елемент от списъка с номер едно. Според числото в списъка се избира един от четирите възможни цвята (Фиг. \ref{fig050021}).

\begin{figure}[H]
  \centering
  \includegraphics[width=1.0\linewidth,height=0.5\linewidth]{fig050021.png}
  \caption{Инструкции за прерисуване на пула}
\label{fig050021}
\end{figure}

Така подготвеният първи пул може да се дублира и разпространи по цялата схема, като се съобразяват цветовете на отделните позиции (Фиг. \ref{fig050022}). 

\begin{figure}[H]
  \centering
  \includegraphics[width=1.0\linewidth,height=0.5\linewidth]{fig050022.png}
  \caption{Дублиране на пула}
\label{fig050022}
\end{figure}

При подреждането на всичките 38 пула, върху схемата на играта, ясно се оформят двата ринга (Фиг. \ref{fig050023}).

\begin{figure}[H]
  \centering
  \includegraphics[width=1.0\linewidth,height=0.5\linewidth]{fig050023.png}
  \caption{Дублиране на пула}
\label{fig050023}
\end{figure}

Схемата на играта, на този етап, е само помощна (Фиг. \ref{fig050024}), когато се премахне номерацията, същото изображение може да се ползва за фон на двата ринга от пулове.

\begin{figure}[H]
  \centering
  \includegraphics[width=1.0\linewidth,height=0.5\linewidth]{fig050024.png}
  \caption{Премахване на схемата}
\label{fig050024}
\end{figure}

При натискане на първата стрелка (горе-дясно) първо се разпространява съобщение за извършване на ротация в десния ринг, по часовниковата стрелка (Фиг. \ref{fig050025}). След това се разпространява съобщение за обновяване на цялото визуално пространство.

\begin{figure}[H]
  \centering
  \includegraphics[width=1.0\linewidth,height=0.5\linewidth]{fig050025.png}
  \caption{Съобщение за ротация и изчертаване}
\label{fig050025}
\end{figure}

По абсолютно аналогичен начин се подават съобщения и от другите три стрелки, като съобщенията указват ринга и посоката на въртене. След първоначалната инициализация на списъкът с цветовете, следва визуализация, след това се дава малък интервал, така че потребителят да види началното състояние и се изпраща съобщение за разбъркване на пъзела (Фиг. \ref{fig050026}).

\begin{figure}[H]
  \centering
  \includegraphics[width=1.0\linewidth,height=0.5\linewidth]{fig050026.png}
  \caption{Изпращане на съобщение за разбъркване}
\label{fig050026}
\end{figure}

Разбъркването на пъзела може да се случи по много начини, но най-удачният е чрез случайно извикване на четирите възможности за ротация. Този алгоритъм също ще бъде сглобен в пространството на основната сцена, а не като код за някои от спрайтовете. Алгоритъмът започва при получаване на съобщението за разбъркване. Тъй като пулчетата са 38 на брой, средно статистически може да се даде шанс на всяко да се мръдне 10 пъти. Това подсказва, че общият брой случайни движения може да се определи на 380, което е 38 по 10 (Фиг. \ref{fig050027}).

\begin{figure}[H]
  \centering
  \includegraphics[width=1.0\linewidth,height=0.5\linewidth]{fig050027.png}
  \caption{Алгоритъм за разбъркване}
\label{fig050027}
\end{figure}

Инструкциите за ротация на ринговете също ще бъдат поместени в пространството на сцената. Всяка от четирите стрелки изпраща подходящо съобщение. Прихванатото съобщение за ротация трябва да промени съдържанието на списъка по такъв начин, че то да отразява желаната ротация. За правилно извършване на разместванията, схемата на играта е от голяма полза, защото указва кой номер къде трябва да отиде. 

Ротацията на левия ринг, по часовниковата стрелка се извършва със следната последователност от действия (Фиг. \ref{fig050028}).

\begin{figure}[H]
  \centering
  \includegraphics[width=1.0\linewidth,height=0.5\linewidth]{fig050028.png}
  \caption{Инструкции за ротация на левия ринг по часовниковата стрелка}
\label{fig050028}
\end{figure}

Ротацията на левия ринг, обратно на часовниковата стрелка се извършва със следната последователност от действия (Фиг. \ref{fig050029}).

\begin{figure}[H]
  \centering
  \includegraphics[width=1.0\linewidth,height=0.5\linewidth]{fig050029.png}
  \caption{Инструкции за ротация на левия ринг обратно на часовниковата стрелка}
\label{fig050029}
\end{figure}

Ротацията на десния ринг, по часовниковата стрелка се извършва със следната последователност от действия (Фиг. \ref{fig050030}).

\begin{figure}[H]
  \centering
  \includegraphics[width=1.0\linewidth,height=0.5\linewidth]{fig050030.png}
  \caption{Инструкции за ротация на десния ринг по часовниковата стрелка}
\label{fig050030}
\end{figure}

Ротацията на десния ринг, обратно на часовниковата стрелка се извършва със следната последователност от действия (Фиг. \ref{fig050031}).

\begin{figure}[H]
  \centering
  \includegraphics[width=1.0\linewidth,height=0.5\linewidth]{fig050031.png}
  \caption{Инструкции за ротация на десния ринг обратно на часовниковата стрелка}
\label{fig050031}
\end{figure}

След изпълнението и на последните инструкции играта придобива своя завършен вид. Разбира се, възможно е да се продължи разработването в посока на алгоритми за нареждане, но тази задача далеч надхвърля възможностите в настоящото изложение. С помощта на бутона „SHARE“ играта бива публикувана за широката аудитория (Фиг. \ref{fig050032}).

\begin{figure}[H]
  \centering
  \includegraphics[width=1.0\linewidth,height=0.5\linewidth]{fig050032.png}
  \caption{Споделяне на завършения проект}
\label{fig050032}
\end{figure}

Играта има своя завършен вид, но все още не е оформена като готов, краен продукт. Хубаво би било да се добави функционалност за подреждане на пъзела. Липсва помощна информация. Възможно е да се добави отчитане на времето за подреждане. Всичко изброено носи допълнителна сложност, която е извън обхвата на настоящото изложение, но пък е възможност за допълнително упражнение от страна на читателите.

\newpage
\chapter{Играта 15}

\begin{figure}[H]
  \centering
  \includegraphics[width=1.0\linewidth,height=0.5\linewidth]{fig060001.png}
  \caption{„Играта 15“ \\ http://www.murderousmaths.co.uk/games/loyd/15 puzzle wood.gif}
\label{fig060001}
\end{figure}

„Играта 15“ (Фиг. \ref{fig060001}) е детски пъзел от групата игри „пътешествие по граф“. Игралното поле е оформено в 4x4 клетки, като в него са поместени 15 плочки. Плочките са номерирани и шестнадесетата позиция е празна. Празната позиция служи като буфер в който могат да се местят съседните плочки. С помощта на буферната клетка, играта се разбърква и целта е плочките да бъдат подредени според началната номерация. 

\section{Структуриране на графичния интерфейс}

Това е детска игра, подреждането на която не е особено трудно и сложност създава единствено последния ред в пъзела. Относително простичката организация на играта я прави идеална за реализация като App Inventor приложение. С помощта само на 16 бутона може да се изгради целият нужен интерфейс. Изработката започва със създаването на нов проект (Фиг. \ref{fig060002}).

\begin{figure}[H]
  \centering
  \includegraphics[width=1.0\linewidth,height=0.5\linewidth]{fig060002.png}
  \caption{Стартиране на нов проект за „Играта 15“}
\label{fig060002}
\end{figure}

Тъй като ще се използват 15 номерирани бутона и един без номерация, то най-удачно е те да бъдат организирани с мениджър на разположението от тип таблица, с 4 реда и 4 колони (Фиг. \ref{fig060003}).

\begin{figure}[H]
  \centering
  \includegraphics[width=1.0\linewidth,height=0.5\linewidth]{fig060003.png}
  \caption{Мениджър на съдържанието с 4x4 клетки}
\label{fig060003}
\end{figure}

Следва поставяне на 16 бутона (Фиг. \ref{fig060004}), като нечетните числа се оцветяват в червено, а четните числа в синьо. Оцветяването подсилва визуалния ефект на играта. Шестнадесетият бутон вместо надпис има два празни интервала, така че ширината му да съвпада с ширината на другите бутони. Без допълнителни настройки, ширината на бутоните се определя от броя символи в надписите им. 

\begin{figure}[H]
  \centering
  \includegraphics[width=1.0\linewidth,height=0.5\linewidth]{fig060004.png}
  \caption{Поставяне на 16 бутона}
\label{fig060004}
\end{figure}

Възможно е да се разменят самите бутони, когато се натисне бутон в съседство на празната клетка, но значително по-лесно е да се разменят надписите и цветовете на бутоните, а самите бутони да остават винаги на първоначалните си места. Натискането на бутона ще се прихваща в общо събитие за всички бутони, но след натискането трябва да се определи дали в съседство е празната клетка. Най-елегантно съседството може да бъде установено, ако структура от тип речник съдържа всички бутони като ключове (Фиг. \ref{fig060005}), а като стойности списъчни структури със съседите.

\begin{figure}[H]
  \centering
  \includegraphics[width=1.0\linewidth,height=0.5\linewidth]{fig060005.png}
  \caption{Бутоните като ключови стойности}
\label{fig060005}
\end{figure}

\section{Структури от данни}

Този речник на съответствията става достъпен като глобална променлива, така че да се ползва в различните събития от визуалния интерфейс. Бутон едно за съседи има бутон две и бутон пет  (Фиг. \ref{fig060006}). 

\begin{figure}[H]
  \centering
  \includegraphics[width=1.0\linewidth,height=0.5\linewidth]{fig060006.png}
  \caption{Бутони като списък от съседи}
\label{fig060006}
\end{figure}

Бутон две има за съседи бутон едно, шест и три. Съседи на бутон три са две, седем и четири. Съседи на бутон четири са три и осем. Съседи на бутон пет са едно, шест и девет. Съседи на бутон шест са две, пет, седем и десет. Съседи на бутон седем са три, шест, осем и единадесет. Съседи на бутон осем са четири, седем и дванадесет. Съседи на бутон девет са пет, десет и тринадесет. Съседи на бутон десет са шест, девет, единадесет и четиринадесет. Съседи на бутон единадесет са седем, десет, дванадесет и петнадесет. Съседи на бутон дванадесет са осем, единадесет и шестнадесет. Съседи на бутон тринадесет са девет и четиринадесет. Съседи на бутон четиринадесет са десет, тринадесет и петнадесет. Съседи на бутон петнадесет са единадесет, четиринадесет и шестнадесет. Съседи на бутон шестнадесет са дванадесет и петнадесет. Списъците за съседство се попълват по идентичен начин (Фиг. \ref{fig060007}).

\begin{figure}[H]
  \centering
  \includegraphics[width=1.0\linewidth,height=0.5\linewidth]{fig060007.png}
  \caption{Идентично попълване на съседните бутони}
\label{fig060007}
\end{figure}

Натискането на който и да е от бутоните предизвика събитие, което е общо за всички бутони (Фиг. \ref{fig060008}). Тъй като не е определено кой бутон ще натисне потребителя, то се прави проверка за натиснатия бутон (идва като параметър на събитието). 

\begin{figure}[H]
  \centering
  \includegraphics[width=1.0\linewidth,height=0.5\linewidth]{fig060008.png}
  \caption{Събитие за натиснат бутон}
\label{fig060008}
\end{figure}

\section{Алгоритми за обработка на състоянието на играта}

Проверката дали празната клетка е съседство и евентуалната размяна с празната клетка се поверява на допълнителна процедура (Фиг. \ref{fig060009}). Процедурата получава като входен параметър компонента, предизвикал събитието. 

\begin{figure}[H]
  \centering
  \includegraphics[width=1.0\linewidth,height=0.5\linewidth]{fig060009.png}
  \caption{Процедура за размяна}
\label{fig060009}
\end{figure}

За да се определи дали празната клетка е в съседство на натиснатия бутон се проверява целият списък, който се намира в речника на ключова позиция, посочена от компонента предизвикал събитието. Тази проверка е възможна с цикъл за обхождане на елементите в списъчна структура (Фиг. \ref{fig060010}). Ключовата стойност би трябвало винаги да връща списък със съседните бутони, но ако ключът не е намерен, за безопасност, се връща празен списък.

\begin{figure}[H]
  \centering
  \includegraphics[width=1.0\linewidth,height=0.5\linewidth]{fig060010.png}
  \caption{Цикъл за обхождане на съседите}
\label{fig060010}
\end{figure}

Размяна с прави само, ако в списъка бъде открита празната клетка. Направата на размяната спира и цикъла за търсене на празната клетка. Условието съседен бутон да обозначава праната клетка е неговият надпис да бъде два интервала (Фиг. \ref{fig060011}). 

\begin{figure}[H]
  \centering
  \includegraphics[width=1.0\linewidth,height=0.5\linewidth]{fig060011.png}
  \caption{Проверка за празната клетка}
\label{fig060011}
\end{figure}

За улесняване на размяната се залагат четири локални, помощни променливи. Две променливи за текстовете на двата бутона и две променливи за цветовете на текстовете (Фиг. \ref{fig060012}).

\begin{figure}[H]
  \centering
  \includegraphics[width=1.0\linewidth,height=0.5\linewidth]{fig060012.png}
  \caption{Локални помощни променливи}
\label{fig060012}
\end{figure}

Размяната се осъществява, чрез записване на помощните променливи, като нови стойности за двата бутона (Фиг. \ref{fig060013}).

\begin{figure}[H]
  \centering
  \includegraphics[width=1.0\linewidth,height=0.5\linewidth]{fig060013.png}
  \caption{Размяна на стойностите}
\label{fig060013}
\end{figure}

На този етап играта притежава абсолютно цялата функционалност, която притежава и механичната играчка (Фиг. \ref{fig060014}).

\begin{figure}[H]
  \centering
  \includegraphics[width=1.0\linewidth,height=0.5\linewidth]{fig060014.png}
  \caption{Завършен вид на играта}
\label{fig060014}
\end{figure}

Въпреки, че всичко необходимо е налично, използването на компютър позволява да се добави още една полезна функция, а именно – автоматично разбъркване на пъзела. Операционната система позволява да се прихване събитие за дълго натискане на бутон (Фиг. \ref{fig060015}).

\begin{figure}[H]
  \centering
  \includegraphics[width=1.0\linewidth,height=0.5\linewidth]{fig060015.png}
  \caption{Събитие за дълго натискане на бутон}
\label{fig060015}
\end{figure}

Бутонът за празната клетка не е натоварен с много действия и поради тази причина може да се използва точно като бутон за разбъркване, стига да бъде натиснат за продължително време (Фиг. \ref{fig060016}).

\begin{figure}[H]
  \centering
  \includegraphics[width=1.0\linewidth,height=0.5\linewidth]{fig060016.png}
  \caption{Активиране на разбъркването}
\label{fig060016}
\end{figure}

Тъй като при разбъркването празната клетка ще се мести, то е добър вариант позицията й да се съхранява в локална, помощна променлива (Фиг. \ref{fig060017}).

\begin{figure}[H]
  \centering
  \includegraphics[width=1.0\linewidth,height=0.5\linewidth]{fig060017.png}
  \caption{Помощна променлива за позицията на празната клетка}
\label{fig060017}
\end{figure}

Игралното табло се състои от 16 позиции. За да може всяка позиция да участва, средно-статистически, 10 пъти в разбъркването, случайно избрани размествания могат да се направят 160 пъти, което е 16 клетки по десет пъти. За целта на разбъркването, цикъл с единична стъпка е най-удачната конструкция (Фиг. \ref{fig060018}).

\begin{figure}[H]
  \centering
  \includegraphics[width=1.0\linewidth,height=0.5\linewidth]{fig060018.png}
  \caption{Цикъл за разбъркване}
\label{fig060018}
\end{figure}

Изборът на следваща празна клетка се прави на случаен принцип от съседите на текущата празна клетка  (Фиг. \ref{fig060019}). Следващата празна клетка се записва във временна променлива, докато се извърши размяната.

\begin{figure}[H]
  \centering
  \includegraphics[width=1.0\linewidth,height=0.5\linewidth]{fig060019.png}
  \caption{Избор на случаен съсед на празната клетка}
\label{fig060019}
\end{figure}

Самата размяна се извършва с помощната процедура (Фиг. \ref{fig060020}). След размяната, празната клетка за следващото завъртане на цикъла се сменя.

\begin{figure}[H]
  \centering
  \includegraphics[width=1.0\linewidth,height=0.5\linewidth]{fig060020.png}
  \caption{Размяна на празната клетка}
\label{fig060020}
\end{figure}

\section{Публикуване на проекта}

След като играта е завършена, проектът може да се публикува за широката аудитория с помощта на бутона „Publish to Gallery“ (Фиг. \ref{fig060021}).

\begin{figure}[H]
  \centering
  \includegraphics[width=1.0\linewidth,height=0.5\linewidth]{fig060021.png}
  \caption{Бутон за публикуване в галерия}
\label{fig060021}
\end{figure}

Дори семпло описание (Фиг. \ref{fig060022}) на приложението е важно за потребителите, тъй като това е второто нещо, което привлича вниманието им.

\begin{figure}[H]
  \centering
  \includegraphics[width=1.0\linewidth,height=0.5\linewidth]{fig060022.png}
  \caption{Описание на приложението}
\label{fig060022}
\end{figure}

Първото най-важно нещо в едно софтуерно приложение е картинка, която най-много може да привлече вниманието на потребителите (Фиг. \ref{fig060023}).

\begin{figure}[H]
  \centering
  \includegraphics[width=1.0\linewidth,height=0.5\linewidth]{fig060023.png}
  \caption{Картинка за представяне на приложението}
\label{fig060023}
\end{figure}

В публичната страница на проекта (Фиг. \ref{fig060024}), потребителите могат да стартират програмата или да я заредят в средата на App Inventor.

\begin{figure}[H]
  \centering
  \includegraphics[width=1.0\linewidth,height=0.5\linewidth]{fig060024.png}
  \caption{Публична страница на проекта}
\label{fig060024}
\end{figure}

Играта е напълно използваема, но все още липсват някои функционалности. Хубаво би било да се добави възможност за автоматично нареждане на пъзела. Липсата на помощна информация също е недостатък. Възможно е да се добави отчитане на времето за подреждане, така че в по-напреднал етап да има възможност за организиране на онлайн класация. Изброените функционалности носят определена сложност и излизат извън рамките на настоящото изложение, но пък са идеална възможност за допълнително упражнение за читателите. 

\newpage
\chapter{Leaping Rainbow}

The aim of the game is to get the ball into the bowl. The player controls the ball using the mouse. When the game starts, the bowl bounces from the left side of the screen to the right side, leaving behind a trail that is a rainbow. The player must control the ball so that it passes only along the arc. If it does not touch the rainbow - the game starts over.

\begin{figure}[H]
   \centering
   \includegraphics[width=1.0\linewidth,height=0.5\linewidth]{fig070001.png}
   \caption{Leaping Rainbow}
\label{fig070001}
\end{figure}

\section{Adding Background and Characters}
The first step of the game is to add a suitable background and characters. Characteristic of the background is that there should be a brown band at the bottom, which will serve as a reference. If the background to be selected is not there, then it can be added additionally using the drawing tools.

The main character is not needed in this game, for that he should be deleted. The characters bowl and ball are among the ready-made characters in Scratch. The rainbow character must be drawn (Fig. \ref{fig070002}).

\begin{figure}[H]
   \centering
   \includegraphics[width=1.0\linewidth,height=0.5\linewidth]{fig070002.png}
   \caption{Adding the character arc}
\label{fig070002}
\end{figure}

One more character needs to be drawn. This is the inscription that will appear when the ball touches the bowl, ie. the player successfully goes through the entire arc.

\begin{figure}[H]
   \centering
   \includegraphics[width=1.0\linewidth,height=0.5\linewidth]{fig070003.png}
   \caption{Adding the character to end the game}
\label{fig070003}
\end{figure}

\section{Programming the Bowl}
The first instructions to be constructed are those for the bowl. Her goal is to go from the left side of the screen to the right by bouncing. The algorithm to be constructed for the bounce effect requires the creation of a variable. This variable will contain the value by which the y-coordinate will be changed. Until the bowl touches the brown border the variable will decrement by 1. When it touches the border then it will take a value of 15.

When the game starts the cup position should be on the left side of the screen. This means that the value of the x-coordinate should be -199 and that of the Y-coordinate 148. The initial value of the variable "velocity" should be equal to 0. The size of the character should be changed to be smaller.

\begin{figure}[H]
   \centering
   \includegraphics[width=1.0\linewidth,height=0.5\linewidth]{fig070004.png}
   \caption{Position of Hero Cup}
\label{fig070004}
\end{figure}

The movement of the bowl must be programmed. This movement must be repeated, which means that a loop must be added with a goal that is "until the character touches an edge". This means that the character will move until it touches the right part of the screen. In addition to moving the character with the 3-step movement instruction, it must also change its y coordinate with the value of the "velocity" variable.

An if/else construct should be added to check if the character touches the brown border. If it is not touching it, then the variable must be decreased to move the character down. If it touches the border - the variable should have a value of 15. To make the check that the brown color is not touched, a negation instruction should be added, which is located in the green group. The color checker is in the light blue group. To find the exact color, use the eyedropper tool by placing it on the brown color. By instructions, the movement algorithm looks like this:

\begin{figure}[H]
   \centering
   \includegraphics[width=1.0\linewidth,height=0.5\linewidth]{fig070005.png}
   \caption{Movement of hero cup}
\label{fig070005}
\end{figure}

After the character reaches his goal, he must send a "start game" message. Only then can the game begin. Instructions for sending messages are located in the yellow instructions group.

\begin{figure}[H]
   \centering
   \includegraphics[width=1.0\linewidth,height=0.5\linewidth]{fig070006.png}
   \caption{The Hero Cup Code}
\label{fig070006}
\end{figure}

When the game starts, it is noticed that the character goes from one part of the screen to the other, jumping.

\section{Programming the Rainbow}
To program this character to leave traces, a new group of instructions must be added - Pen.

\begin{figure}[H]
   \centering
   \includegraphics[width=1.0\linewidth,height=0.5\linewidth]{fig070007.png}
   \caption{New group of Pen instructions}
\label{fig070007}
\end{figure}

Based on how the character is drawn, it should be reduced. This can be done without instructions, but by changing the Size property.

\begin{figure}[H]
   \centering
   \includegraphics[width=1.0\linewidth,height=0.5\linewidth]{fig070008.png}
   \caption{Change the Size property}
\label{fig070008}
\end{figure}

This character's instructions are only to follow the cup character and leave tracks. At the beginning of the game, everything drawn must be erased. This is done using the "erase all" instruction, which is peace in the newly added group of instructions. The rainbow, in addition to following the movement of the bowl character, must also follow its direction. This is done using the instruction from the blue "point in direction" group. To indicate exactly which direction to follow, an instruction from the light blue group, which is "backdrop of Stage", should be used. First, the second value of the "Stage" of the character name is changed. Then the type is also changed, which should be "direction".

\begin{figure}[H]
   \centering
   \includegraphics[width=1.0\linewidth,height=0.5\linewidth]{fig070009.png}
   \caption{Movement of the character arc}
\label{fig070009}
\end{figure}

If you start the program, you will notice that the rainbow character only follows the cup character. To leave a mark, the "stamp" instruction had to be used, which was added to the new group.

\begin{figure}[H]
   \centering
   \includegraphics[width=1.0\linewidth,height=0.5\linewidth]{fig070010.png}
   \caption{All character code arc}
\label{fig070010}
\end{figure}

\section{Programming the Ball}
The ball's edges must be such that it can pass through the entire arc touching the purple (or outermost) color. These can be changed by changing the value of the Size property.

\begin{figure}[H]
   \centering
   \includegraphics[width=1.0\linewidth,height=0.5\linewidth]{fig070011.png}
   \caption{Hero Ball Size}
\label{fig070011}
\end{figure}

The ball should appear when the bowl reaches the end of the screen. By instructions, this means when the character gets the "start game" message, then it appears. When the green flag is pressed, it should be hidden. When displayed, a slight animation can be made to move it to a position with coordinates for x=-231 and for y=114.

\begin{figure}[H]
   \centering
   \includegraphics[width=1.0\linewidth,height=0.5\linewidth]{fig070012.png}
   \caption{Starting position of the ball character}
\label{fig070012}
\end{figure}

The character should start moving when pressed. It moves as it touches the outermost color - in this case purple. Again, a loop with a goal should be used. The goal should be "until you stop touching the color purple". The statements outside the loop will be executed when the condition is false. In this case, it means when the character does not touch the arc, then move to the starting position.

The movement instruction should be - follow the mouse. This instruction is in the blue instruction group and is go to mouse-pointer. If the character needs to move slower, this is done by adding the wait instruction from the orange group.

\begin{figure}[H]
   \centering
   \includegraphics[width=1.0\linewidth,height=0.5\linewidth]{fig070013.png}
   \caption{Movement of character ball}
\label{fig070013}
\end{figure}

The last thing to do is check if the character has reached the end of the arc. If it is, then it will send a message to the character that is captioned to appear and the game will end. At the end of the arc is the arc hero. Then checking whether to end the game is very easy - has the ball touched the rainbow. If it is - sends a message and ends the game. If not, the game continues.

\begin{figure}[H]
   \centering
   \includegraphics[width=1.0\linewidth,height=0.5\linewidth]{fig070014.png}
   \caption{The Orb Hero Code}
\label{fig070014}
\end{figure}

\section{End Game}
The character that is the end of the game must be hidden at the beginning. It appears when the ball sends an end-of-game message.

\begin{figure}[H]
   \centering
   \includegraphics[width=1.0\linewidth,height=0.5\linewidth]{fig070015.png}
   \caption{Hero Code Caption}
\label{fig070015}
\end{figure}
\newpage
\chapter{Бикове и крави}

\begin{figure}[H]
  \centering
  \includegraphics[width=1.0\linewidth,height=0.5\linewidth]{fig080001.png}
  \caption{„Бикове и крави“ \\ https://i.ytimg.com/vi/r\_dw8iV\_52g/hqdefault.jpg}
\label{fig080001}
\end{figure}

Играта „Бикове и крави“ (Фиг. \ref{fig080001}) е от групата игри за разгадаване на шифър. Играе се от двама играчи, с лист и молив, като всеки играч си намисля четири цифрено число (тайна), което не може да започва с цифрата нула. Всеки от играчите се опитва да разгадае числото на противника. Процесът на разгадаване включва споменаването на четири цифрено число, като противникът отговаря с информация за това колко цифри от предположението са на точните си места и колко цифри са на други места. За всяка цифра, която съвпада по позиция с тайното число се докладва по един бик, а за всяка цифра, която не съвпада по позиция се докладва по една крава. Играта приключва, когато един от играчите успее да предположи число, което води до четири бика. 

Играта не е сложна, но е идеален вариант да се демонстрира компютърен опонент, който използва похвати от теорията за множествата. Разработката на играта започва със създаването на нов проект (Фиг. \ref{fig080002}).

\begin{figure}[H]
  \centering
  \includegraphics[width=1.0\linewidth,height=0.5\linewidth]{fig080002.png}
  \caption{Създаване на нов проект за „Бикове и крави“}
\label{fig080002}
\end{figure}

Възможни са различни начини за организиране на графичния потребителски интерфейс, но с демонстрационна цел е удачно да се използва най-опростеният вариант. В мениджър за управление на визуалните колони от тип таблица може да се разположат визуални контроли в матрица от 9 колони и 2 реда (Фиг. \ref{fig080003}).

\begin{figure}[H]
  \centering
  \includegraphics[width=1.0\linewidth,height=0.5\linewidth]{fig080003.png}
  \caption{Мениджър на визуалните компоненти 9x2}
\label{fig080003}
\end{figure}

Тъй като всеки играч прави запитвания за четири цифрени числа, то на първия ред в таблицата, първите четири позиции, се запълват с четири визуални компонента за избор от списък (Фиг. \ref{fig080004}). Тези четири списъчни контроли ще служат за визуализация на четири цифрените числа, предполагани от компютърния опонент. Всеки от компонентите първоначално визуализира символа звезда, а списъка от възможни стойности са цифрите от нула до девет. В първият визуален компонент не би трябвало да е възможно избирането на нулата, но за симетрия нулата е оставена в списъка. 

\begin{figure}[H]
  \centering
  \includegraphics[width=1.0\linewidth,height=0.5\linewidth]{fig080004.png}
  \caption{Визуални контроли за предположенията на компютъра}
\label{fig080004}
\end{figure}

По аналогичен начин се подреждат още четири списъчни контрола (Фиг. \ref{fig080005}), които ще се използват от играча, за да прави предположения кое е тайното число на компютърния опонент. Между двете групи контроли се оставя една празна клетка, която ще бъде използвана за бутон, който да стартира процедурите по разгадаване. 

\begin{figure}[H]
  \centering
  \includegraphics[width=1.0\linewidth,height=0.5\linewidth]{fig080005.png}
  \caption{Визуални контроли за предположенията на човека}
\label{fig080005}
\end{figure}

На втория ред се разполагат още четири списъчни контрола, които ще служат за отбелязване на броя бикове и броя крави (Фиг. \ref{fig080006}). Първите два са за броя бикове и крави, познати от компютърния опонент, а вторите два са за броя бикове и крави, познати от играча. В двойките, левият контрол ще показва биковете, а десният контрол кравите. Възможните опции са от нула до четири, тъй като възможните бикове или крави са от нула, до четири включително.

\begin{figure}[H]
  \centering
  \includegraphics[width=1.0\linewidth,height=0.5\linewidth]{fig080006.png}
  \caption{Визуални контроли за отчитане на биковете и кравите}
\label{fig080006}
\end{figure}

Компонентите за списъчен избор не визуализират в текста си избраната опция автоматично. Поради тази причина е нужно да се прихване събитието за направен избор и избраната опция да се впише в текстовото поле на визуалния компонент. За тази цел се прихваща общо събитие, генерирано от всички списъчни компоненти на екрана (Фиг. \ref{fig080007}).

\begin{figure}[H]
  \centering
  \includegraphics[width=1.0\linewidth,height=0.5\linewidth]{fig080007.png}
  \caption{Визуализация на избраната опция}
\label{fig080007}
\end{figure}

Пред клетките за отчитане на броя бикове и броя крави е разумно да се поставят етикети (Фиг. \ref{fig080008}). Латинската буква B се ползва за биковете, а латинската буква C за броя на кравите. 

\begin{figure}[H]
  \centering
  \includegraphics[width=1.0\linewidth,height=0.5\linewidth]{fig080008.png}
  \caption{Етикети пред контролите за отчитане на биковете и кравите}
\label{fig080008}
\end{figure}

Потребителският интерфейс завършва с два бутона (Фиг. \ref{fig080009}). Първият активизира компютърния опонент да познае числото на играча, а втория активизира компютърния опонент да съобщи броя уцелени от човека бикове и крави.

\begin{figure}[H]
  \centering
  \includegraphics[width=1.0\linewidth,height=0.5\linewidth]{fig080009.png}
  \caption{Бутони за активизиране на процеса по разгадаване}
\label{fig080009}
\end{figure}

За нуждите на компютърния опонент и според теорията на множествата, важно е да се създаде списъчна структура (Фиг. \ref{fig080010}), която да съдържа всички възможни комбинации на четири цифрени числа, цифрите на които не се повтарят и не започват с нула. При всеки отговор на човека, от списъка се изключват всички комбинации, които не отговарят на условията за установени бикове и крави.

\begin{figure}[H]
  \centering
  \includegraphics[width=1.0\linewidth,height=0.5\linewidth]{fig080010.png}
  \caption{Списък на комбинациите}
\label{fig080010}
\end{figure}

Числата са четири цифрени, така че списъкът с комбинациите може да се запълни с помощта на четири цикъла (Фиг. \ref{fig080011}). Първият се върти от едно до девет, тъй като нулата не може да участва, като първа цифра. Другите три цикъла се въртят от нула до девет. За броячи на циклите се избират латинските букви a, b, c и d. Всеки от броячите ще участва във формирането на четири цифрено число.

\begin{figure}[H]
  \centering
  \includegraphics[width=1.0\linewidth,height=0.5\linewidth]{fig080011.png}
  \caption{Цикли за генериране на комбинациите}
\label{fig080011}
\end{figure}

За да се избегне повторение на цифрите е нужно да се направят серия проверки (Фиг. \ref{fig080012}). За първия цикъл проверка няма да се прави, но вторият цикъл се завърта само, ако броячът b се различава от брояча a. Третият цикъл се завърта само, ако броячът c се различава от брояча a и броячът c се различава от брояча b. Четвъртият цикъл се завърта само, ако броячът d се различава от брояча a, различава се от брояча b и се различава от брояча c. 

\begin{figure}[H]
  \centering
  \includegraphics[width=1.0\linewidth,height=0.5\linewidth]{fig080012.png}
  \caption{Проверки за избягване на дублиращи се цифри}
\label{fig080012}
\end{figure}

Две глобални променливи спомагат за съхраняването на тайните числа за компютърния опонент и опонентът човек (Фиг. \ref{fig080013}). Още две глобални променливи пък помагат за обработка на предположението от компютърния опонент и предположението на човека. И четирите променливи, първоначално са служебно инициализирани с нули.

\begin{figure}[H]
  \centering
  \includegraphics[width=1.0\linewidth,height=0.5\linewidth]{fig080013.png}
  \caption{Помощни променливи за тайните числа на играчите}
\label{fig080013}
\end{figure}

Циклите за инициализация на списъка с комбинации и последващия избор на една от тях, като тайно число на компютърния опонент, трябва да се случат в събитие маркиращо инициализацията на работния екран (Фиг. \ref{fig080014}).

\begin{figure}[H]
  \centering
  \includegraphics[width=1.0\linewidth,height=0.5\linewidth]{fig080014.png}
  \caption{Начална инициализация на екрана}
\label{fig080014}
\end{figure}

Задачата на първия бутон е да вземе предположение от компютърния опонент (извикване на допълнителна процедура) и да нулира всички останали цифри по визуалните компоненти (Фиг. \ref{fig080015}).

\begin{figure}[H]
  \centering
  \includegraphics[width=1.0\linewidth,height=0.5\linewidth]{fig080015.png}
  \caption{Действия на първия бутон}
\label{fig080015}
\end{figure}

Компютърният опонент прави предположение за тайното число на играча, като избира един елемент от списъка с останали комбинации. Ако списъкът е празен (Фиг. \ref{fig080016}), то това означава, че човекът е подал неправилна информация на някой от предишните ходове и познаването на числото му е невъзможно. В компонент за нотификации се издава съобщение, че играта не може да продължи. 

\begin{figure}[H]
  \centering
  \includegraphics[width=1.0\linewidth,height=0.5\linewidth]{fig080016.png}
  \caption{Проверка за останали налични комбинации}
\label{fig080016}
\end{figure}

Ако в списъка с комбинации има елементи, то се избира един от елементите на случаен принцип (Фиг. \ref{fig080017}). Избраното число се визуализира в първите четири клетки от интерфейса, като за тази цел се вземат отделните цифри. 

\begin{figure}[H]
  \centering
  \includegraphics[width=1.0\linewidth,height=0.5\linewidth]{fig080017.png}
  \caption{Избор на число за предположение}
\label{fig080017}
\end{figure}

При натискането на втория бутон се изпълняват три действия, като за всяко от тях се извиква помощна процедура (Фиг. \ref{fig080018}). Първо се взема информацията от визуалните контроли, за предположението направено от човека. Второто действие е да се визуализират броя бикове и крави, уцелени от човека. Третото действие е да се изключат всички комбинации в списъка, които не отговарят на критериите от последното предположение, направено от компютърния опонент.

\begin{figure}[H]
  \centering
  \includegraphics[width=1.0\linewidth,height=0.5\linewidth]{fig080018.png}
  \caption{Действия на първия бутон}
\label{fig080018}
\end{figure}

Предположението на човека се взема от визуалните контроли, като всяка от цифрите се слепва с останалите и се записва в помощната глобална променлива (Фиг. \ref{fig080019}).

\begin{figure}[H]
  \centering
  \includegraphics[width=1.0\linewidth,height=0.5\linewidth]{fig080019.png}
  \caption{Вземане на предположението от човека}
\label{fig080019}
\end{figure}

Броят познати бикове и крави от човека се определят с помощна процедура, която връща списък с два елемента. Първият елемент съдържа броя на биковете, а втория елемент броя на кравите. За да се изчислят тези бройки се използва тайното число на компютърния опонент и предположението, направено от играча (Фиг. \ref{fig080020}). Резултатът се показва в последните два компонента от графичния потребителски интерфейс.

\begin{figure}[H]
  \centering
  \includegraphics[width=1.0\linewidth,height=0.5\linewidth]{fig080020.png}
  \caption{Визуализация на броя бикове и крави познати от човека}
\label{fig080020}
\end{figure}

За да се редуцира броят на комбинациите, трябва да се вземе резултатът от проверката на направеното предположение, тоест броя бикове и крави, докладвани от човека. Това се постига с помощна процедура, която също връща списък, на който първият елемент съдържа броя бикове, а вторият елемент броя крави (Фиг. \ref{fig080021}).

\begin{figure}[H]
  \centering
  \includegraphics[width=1.0\linewidth,height=0.5\linewidth]{fig080021.png}
  \caption{Вземане на броя бикове и крави обявени от човека}
\label{fig080021}
\end{figure}

Отсяването на ненужните комбинации става чрез вземане на отговора, даден от човекът опонент. След това се създава нов празен списък, в който списък ще влязат само числата, които отговарят на посочените в отговора критерии. Отсяването се постига с прехождане на списъка с комбинации (Фиг. \ref{fig080022}). Всяка комбинация се подава на помощната функция, която да определи колко бика и крави формира комбинацията, спрямо предположението, направено от компютърния опонент. Ако текущо проверяваната комбинация има същите характеристики (брой бикове и брой крави), както са характеристиките, подадени в отговор от човека опонент, то текущата комбинация влиза в новия списък. След цялостното прехождане на списъка от комбинации, старият списък се подменя с новия списък.

\begin{figure}[H]
  \centering
  \includegraphics[width=1.0\linewidth,height=0.5\linewidth]{fig080022.png}
  \caption{Отсяване на ненужните комбинации}
\label{fig080022}
\end{figure}

Изчисляването на броя бикове и броя крави, за две числа става, чрез завъртането на два цикъла. Единият цикъл обхожда първото число, цифра по цифра, а вторият цикъл второто число, пак цифра по цифра (Фиг. \ref{fig080023}).

\begin{figure}[H]
  \centering
  \includegraphics[width=1.0\linewidth,height=0.5\linewidth]{fig080023.png}
  \caption{Обхождане на цифрите на две числа}
\label{fig080023}
\end{figure}

Текущо разглежданите цифри на двете числа се зареждат в две помощни променливи (Фиг. \ref{fig080024}). С тези две променливи и с броячите на циклите се правят нужните сравнения за наличие на бик или крава. 

\begin{figure}[H]
  \centering
  \includegraphics[width=1.0\linewidth,height=0.5\linewidth]{fig080024.png}
  \caption{Помощни променливи за цифрите}
\label{fig080024}
\end{figure}

Ако двете цифри съвпадат, то това означава или бик или крава (Фиг. \ref{fig080025}). Дали съвпадението е бик или крава се определя от стойностите на броячите за двата цикъла.

\begin{figure}[H]
  \centering
  \includegraphics[width=1.0\linewidth,height=0.5\linewidth]{fig080025.png}
  \caption{Съвпадение на цифрите}
\label{fig080025}
\end{figure}

Ако двата брояча са с една и съща стойност, то е налице бик, в противен случай е на лице крава (Фиг. \ref{fig080026}).

\begin{figure}[H]
  \centering
  \includegraphics[width=1.0\linewidth,height=0.5\linewidth]{fig080026.png}
  \caption{Съвпадение на броячите}
\label{fig080026}
\end{figure}

При срещането на бик се увеличава с единица стойността на първия елемент в резултата, а при срещането на крава се увеличава с единица вторият елемент на списъка с резултата (Фиг. \ref{fig080027}).

\begin{figure}[H]
  \centering
  \includegraphics[width=1.0\linewidth,height=0.5\linewidth]{fig080027.png}
  \caption{Отброяване на биковете и кравите}
\label{fig080027}
\end{figure}

Играта се играе в последователност (Фиг. \ref{fig080028}), като първо се натисне първият бутон. Натискането на първия бутон дава възможност на компютърния опонент да направи предположение за тайното число на човека опонент. След това, човекът маркира колко бика и колко крави е успял да познае компютърният опонент. Третата стъпка е човекът опонент да направи своето предположение за тайното число на компютърния опонент. Следва натискане на втория бутон, при който компютърният опонент съобщава колко бика и колко крави е познал човекът. 

\begin{figure}[H]
  \centering
  \includegraphics[width=1.0\linewidth,height=0.5\linewidth]{fig080028.png}
  \caption{Начален екран на играта}
\label{fig080028}
\end{figure}

Чрез следване на последователността от стъпки за работа с графичния интерфейс (Фиг. \ref{fig080029}), играта продължава докато един от играчите уцели четири бика.

\begin{figure}[H]
  \centering
  \includegraphics[width=1.0\linewidth,height=0.5\linewidth]{fig080029.png}
  \caption{Междинен ход в играта}
\label{fig080029}
\end{figure}

След достигане на пълнофункционален вариант на играта, проектът може да бъде публикуван в галерията, така че да бъде достъпен за широката аудитория (Фиг. \ref{fig080030}).

\begin{figure}[H]
  \centering
  \includegraphics[width=1.0\linewidth,height=0.5\linewidth]{fig080030.png}
  \caption{Описание на проекта}
\label{fig080030}
\end{figure}

След публикуването, в страницата на приложението се появяват хипервръзки за стартиране на програмата или разглеждане на кода й в развойната среда (Фиг. \ref{fig080031}).

\begin{figure}[H]
  \centering
  \includegraphics[width=1.0\linewidth,height=0.5\linewidth]{fig080031.png}
  \caption{Страница на публикувания проект}
\label{fig080031}
\end{figure}

Макар и пълнофункционална, играта все още не е завършена до нивото на краен продукт. Липсват серия от проверки за това дали потребителят въвежда коректно числата (примерно повтарящи се цифри). Липсва функционалност за обявяване на победителя, въпреки че появата на четири бика ясно обозначават кой побеждава. Липсва екран с помощна информация, както и звуково оформление. Липсващата функционалност е извън обхвата на настоящото изложение и би могла да послужи, като допълнително упражнение за желаещите да надградят знанията си.

\newpage
\chapter{Pop the Balloons}

The goal of this game is for the player to pop as many balloons as possible in 30 seconds. The player will have a catapult with the help of which the player will pop the balloons.

\begin{figure}[H]
   \centering
   \includegraphics[width=1.0\linewidth,height=0.5\linewidth]{fig090001.png}
   \caption{Pop the balloons}
\label{fig090001}
\end{figure}

\section{Adding Background and Characters}
The first step of the game is to choose a suitable background and characters. The required characters in this game are an arrow representing the catapult, a balloon, and a character announcing the score when the 30 seconds are up.

\begin{figure}[H]
   \centering
   \includegraphics[width=1.0\linewidth,height=0.5\linewidth]{fig090002.png}
   \caption{Adding background and characters}
\label{fig090002}
\end{figure}

\section{Programming the Catapult}
In this game, two variables must be defined - the first will store the time, and the second the score. At the start of the game, the initial value of the time variable should be 30, and the value of the score variable should be 0.

\begin{figure}[H]
   \centering
   \includegraphics[width=1.0\linewidth,height=0.5\linewidth]{fig090003.png}
   \caption{Initialize variables}
\label{fig090003}
\end{figure}

The game continues until the time variable is equal to 0. It must decrease its value every 1 second.

\begin{figure}[H]
   \centering
   \includegraphics[width=1.0\linewidth,height=0.5\linewidth]{fig090004.png}
   \caption{Change time variable}
\label{fig090004}
\end{figure}

When the time becomes equal to 0, the game is over. Then that character should send a "stop game" message (Fig. \ref{fig090005}. When the game starts, the time variable will be decremented by 1 every second.

\begin{figure}[H]
   \centering
   \includegraphics[width=1.0\linewidth,height=0.5\linewidth]{fig090005.png}
   \caption{Send Game End Message}
\label{fig090005}
\end{figure}

During gameplay, the catapult must be positioned at the bottom of the screen and follow the direction of the mouse until the player clicks it. A nested loop must be used, meaning there will be a loop that contains another inside of itself. The outer loop will be infinite. It will end when the game is over. The inner loop will be a loop with a goal, with the goal being until the mouse is clicked. Inside the loop body, the statement should be "point towards mouse-pointer, " meaning "follow the mouse".
 
\begin{figure}[H]
   \centering
   \includegraphics[width=1.0\linewidth,height=0.5\linewidth]{fig090006.png}
   \caption{Mouse Tracking}
\label{fig090006}
\end{figure}

When the game is launched, it is noticed that the catapult follows the direction of the mouse. What remains to be done is when it is clicked to fire, and when it touches any of the edges of the screen, it returns to its original state. Created by instructions, this is done by adding one more inner loop. This time the condition of this loop should be - until the catapult touches some edge of the screen. And in the body of the cycle, the instruction should be placed - it moves by 20 steps. The instructions outside this loop are executed when the character touches the edge. For this, the last instruction is to position the character in a starting position.

\begin{figure}[H]
   \centering
   \includegraphics[width=1.0\linewidth,height=0.5\linewidth]{fig090007.png}
   \caption{Final catapult code}
\label{fig090007}
\end{figure}

\section{Programming the Balloon}
In the next step, the instructions for the balloon will also be constructed. Many balloons appear during the game, and the hero is only one. This is done by adding instructions to clone the balloon character. The instructions needed for cloning are in the orange group and are "create a clone of myself" and the instruction "delete this clone".

When the game starts, the original balloon character must hide. His sole purpose is to create clones. Have the bubble create 10 clones of itself, then wait 5 seconds and delete one clone. This algorithm should be repeated 5 times. Here again, the nested loop construct must be used.

\begin{figure}[H]
   \centering
   \includegraphics[width=1.0\linewidth,height=0.5\linewidth]{fig090008.png}
   \caption{Clone Creation}
\label{fig090008}
\end{figure}

The balloon clone should also be programmed. From the orange group, the instruction "When I start as a clone" should be used, which means "When I start a clone". The first thing to do is position the clone. To make the game more interesting, let its position be random but at the top of the screen. This means that the number for the x-coordinate should be randomly between -200 and 200 (that's the borders of the screen), and the y-coordinate should be between 50 and 150 (the top of the screen).

Once positioned, the clone should be displayed and resized to be smaller. The two instructions are located in the purple instruction group.

\begin{figure}[H]
   \centering
   \includegraphics[width=1.0\linewidth,height=0.5\linewidth]{fig090009.png}
   \caption{Clone Positioning}
\label{fig090009}
\end{figure}

Once the bubble appears, it must move one step. In addition to the balloon's movement, two checks must be made. One check is to see if the arrow has touched the clone. If the condition is true, then the value of the result variable should be incremented by 1, and the bubble should be placed at a random position again.

\begin{figure}[H]
   \centering
   \includegraphics[width=1.0\linewidth,height=0.5\linewidth]{fig090010.png}
   \caption{Increasing score}
\label{fig090010}
\end{figure}

The second check to be made is if the balloon is not hit by the catapult and reaches the end of the screen. It must be checked if the x coordinate is greater than 220. If the condition is true, the clone must be placed on the left side of the screen, and to make the game more attractive, it will also change its costume (color).

\begin{figure}[H]
   \centering
   \includegraphics[width=1.0\linewidth,height=0.5\linewidth]{fig090011.png}
   \caption{The Bubble Code}
\label{fig090011}
\end{figure}

The game is almost ready. When launched, clones of the balloon appear and can be burst using the catapult. Also, the time variable decreases, and the result variable increases.

The final step is to program the character to report the final result. When the game starts, this character must be hidden. He should show up when he gets the "stop game" message from the catapult. The instruction to print the result is from the purple group - "thing Hmm... for 2 seconds". The result should be displayed instead of the "Hmm..." message. The "join apple banana" instruction pastes the two words from the green group of instructions. In this game, again, two things need to stick together. One is the message "Your score is," and the second is the value of the score variable.

\begin{figure}[H]
   \centering
   \includegraphics[width=1.0\linewidth,height=0.5\linewidth]{fig090012.png}
   \caption{The character code that announces the result}
\label{fig090012}
\end{figure}

Game over. Play it with your friends to see who can pop the most balloons in 30 seconds.
\newpage
\chapter{Война}

\begin{figure}[H]
  \centering
  \includegraphics[width=1.0\linewidth,height=0.5\linewidth]{fig100001.png}
  \caption{„Война“ \\ https://images.squarespace-cdn.com/content/v1/59ea6080a803bb2f70ecbae5/1529350057743-92YH5Y0BN0JUMYB6X7NV/close-call-slide.jpg}
\label{fig100001}
\end{figure}

Играта „Война“ (Фиг. \ref{fig100001}) е детска игра с карти, като в основния си вариант се играе от двама играчи. Картите са стандартни, 52 карти за игра. Боите на картите са равноправни и няма сила по цвят. Всяка карта има сила с която участва в играта, като се започва от двойките (2 точки) и се стига до асата (14 точки). Тестето карти се разбърква и се раздава по равно на двамата играчи. Картите са с лицата на долу, като на всеки ход всеки от играчите показва най-горната карта. Играчът с по-силна карта взема и двете карти. Ако картите са с еднаква сила, то това е „война“ и играчите показват по три карти. Войната се печели от играча с по-силна трета карта. Ако и третите карти съвпадат, войната продължава, докато единият играч загуби войната. Спечелилият играч прибира всички карти, обърнати с лицето нагоре. Прибраните карти винаги застават най-отдолу в тестето на съответния играч. Играта се губи от този играч, който остане без карти. 

Играта е относително проста и не изисква специални умения, което я прави идеален вариант за малки деца. Работата по изработването на тази игра, под формата на мобилно приложение, започва със създаването на нов проект (Фиг. \ref{fig100002}).

\begin{figure}[H]
  \centering
  \includegraphics[width=1.0\linewidth,height=0.5\linewidth]{fig100002.png}
  \caption{Създаване на нов проект за играта „Война“}
\label{fig100002}
\end{figure}

Потребителският интерфейс ще бъде възможно най-опростен. Два бутона (Фиг. \ref{fig100003}), най-отгоре в работното пространство, ще служат за стартиране на нова игра и направата на ход.

\begin{figure}[H]
  \centering
  \includegraphics[width=1.0\linewidth,height=0.5\linewidth]{fig100003.png}
  \caption{Бутони за стартиране на нова игра и направа на ход}
\label{fig100003}
\end{figure}

Веднага под бутоните, в табличен вид се подреждат два реда с визуални компоненти за показване на графични изображения (Фиг. \ref{fig100004}). В първата колона ще се визуализира гръб на карта, което символизира тестетата на двамата играчи, а в другите три съседни компонента ще се показва една карта, когато няма война и три карти, когато има война. 

\begin{figure}[H]
  \centering
  \includegraphics[width=1.0\linewidth,height=0.5\linewidth]{fig100004.png}
  \caption{Компоненти за визуализация на картите}
\label{fig100004}
\end{figure}

За изображенията на самите карти може да се използва всеки комплект от карти, които се разпространяват със свободен лиценз за не търговска употреба (Фиг. \ref{fig100005}).

\begin{figure}[H]
  \centering
  \includegraphics[width=1.0\linewidth,height=0.5\linewidth]{fig100005.png}
  \caption{Изображения на карти за игра}
\label{fig100005}
\end{figure}

Ако комплектът карти е в общо изображение, то се нарязва на 52 отделни изображения и поне едно изображение за гръб на картите. Така подготвените 53 графични файла, се зареждат в проекта, като се качват файл по файл (Фиг. \ref{fig100006}).

\begin{figure}[H]
  \centering
  \includegraphics[width=1.0\linewidth,height=0.5\linewidth]{fig100006.png}
  \caption{Качване на графични файлове}
\label{fig100006}
\end{figure}

От така качените графични файлове, в първата колона от изображения се зарежда изображението за гръб на картите (Фиг. \ref{fig100007}).

\begin{figure}[H]
  \centering
  \includegraphics[width=1.0\linewidth,height=0.5\linewidth]{fig100007.png}
  \caption{Изображения за маркирането на тестетата}
\label{fig100007}
\end{figure}

Картите ще циркулират в модела на играта под формата на цели числа. За тази цел се обявяват пет помощни, глобални променливи - списък с основното тесте, два списъка за картите държани от играча и два списъка за картите поставени на масата от играчите (Фиг. \ref{fig100008}). 

\begin{figure}[H]
  \centering
  \includegraphics[width=1.0\linewidth,height=0.5\linewidth]{fig100008.png}
  \caption{Основни помощни променливи}
\label{fig100008}
\end{figure}

Два допълнителни списъка биха улеснили визуализацията на картите, които се поставят на масата. В тези списъци се поместват отпратки към визуалните компоненти за показване на изображения (Фиг. \ref{fig100009}).

\begin{figure}[H]
  \centering
  \includegraphics[width=1.0\linewidth,height=0.5\linewidth]{fig100009.png}
  \caption{Помощни списъци за визуализация}
\label{fig100009}
\end{figure}

Последната помощна променлива подрежа в списък изображенията на картите. Подредбата е важна, като двойките стоят на първите четири места, на вторите четири места стоят тройките и така нататък, докато се стигне до последните четири места на които стоят асата (Фиг. \ref{fig100010}). Благодарение на тази подредба, при пресмятането на победителите в отделните рундове ще се постига с просто аритметично пресмятане. 

\begin{figure}[H]
  \centering
  \includegraphics[width=1.0\linewidth,height=0.5\linewidth]{fig100010.png}
  \caption{Помощна променлива за реда на картите по сила}
\label{fig100010}
\end{figure}

\newpage
\chapter{Soccer}

One of the favorite games for children is soccer. The goal of this game is for the player to score as many goals as possible. If he misses, it's game over.

\begin{figure}[H]
   \centering
   \includegraphics[width=1.0\linewidth,height=0.5\linewidth]{fig110001.png}
   \caption{Football}
\label{fig110001}
\end{figure}

\section{Creating Game Design}
The first step of creating this game is adding all the components that will be programmed. The color of the field should be green, for this the value of the BackgroundColor property should be changed to green.

\begin{figure}[H]
   \centering
   \includegraphics[width=1.0\linewidth,height=0.5\linewidth]{fig110002.png}
   \caption{Change background color}
\label{fig110002}
\end{figure}

One of the most important tasks in mobile application programming is their interface. To make this game more attractive, you can change the value of the Theme property to Device Default and the name of the Title property to Footbal.

\begin{figure}[H]
   \centering
   \includegraphics[width=1.0\linewidth,height=0.5\linewidth]{fig110003.png}
   \caption{Change topic}
\label{fig110003}
\end{figure}

In order for the player to shoot the ball when he touches the screen of his phone and also for the goalkeeper to move on the screen, the Canvas element must be added. Its width and height should be the same as it is on the phone screen. To do this the Height property values to be Fill parent... The Width property value is the same. This element should blend in with the background color. For this purpose, the value of the BackgroundColor property must be changed.

\begin{figure}[H]
   \centering
   \includegraphics[width=1.0\linewidth,height=0.5\linewidth]{fig110004.png}
   \caption{The Canvas Element}
\label{fig110004}
\end{figure}

Three ImageSprite elements should be added for the net, goalkeeper and ball respectively. They can be renamed to make it clear which element is for what. Images of these elements can be used which are freely licensed for non-commercial use.

\begin{figure}[H]
   \centering
   \includegraphics[width=1.0\linewidth,height=0.5\linewidth]{fig110005.png}
   \caption{Image of a soccer ball}
\label{fig110005}
\end{figure}

Once the images are available they should be placed on the appropriate ImageSprite elements and the height, width and position properties should be changed. This means they need to scale to the phone screen. The end result should look like this:

\begin{figure}[H]
   \centering
   \includegraphics[width=1.0\linewidth,height=0.5\linewidth]{fig110006.png}
   \caption{Game Design}
\label{fig110006}
\end{figure}

In addition to these elements, a place where the result will be written must be added. From the Layout element group, a HorizontalArrangement element must be added so that two Label elements can be placed inside it. One will be for the result message and the second will be for the result value. This means that the property value for the first label will be "Your score is: " and for the second - "0". It is a good practice to rename elements to make them easier to program. In a later stage of the game, the element "Notifier" will be needed. It is from the group of hidden elements.

\begin{figure}[H]
   \centering
   \includegraphics[width=1.0\linewidth,height=0.5\linewidth]{fig110007.png}
   \caption{All Game Elements}
\label{fig110007}
\end{figure}

\section{Programming}
Necessary instructions should also be added to the items that are added for the game. For this purpose, it is necessary to switch to the other view, which is for adding the instructions.

The first instructions will be for the goalkeeper. It will move left and right. Its purpose is to keep the goal from scoring. When the game starts, the goalkeeper must start moving. This means adding the when Screen1.Initialize statement found in the Screen1 element. Inside this event, two instructions should be added, which are located in the goalkeepr element. One instruction is to set a value to the Interval property, which specifies how many milliseconds this character's position will change. The second instruction sets a value to the Speed property, which is responsible for the movement of the character.

One more event should be added - when goalkeeper.EdgeReached. The instructions from this event will be executed when the goalkeeper touches the edge of the screen. When this happens, the instruction goalkeeper.Bounce edge is called, and the value that is set is get edge.

\begin{figure}[H]
   \centering
   \includegraphics[width=1.0\linewidth,height=0.5\linewidth]{fig110008.png}
   \caption{Goalkeeper Movement}
\label{fig110008}
\end{figure}

When the game starts, it is noticed that the goalkeeper turns with his head down. This can be changed by unchecking the Rotates property.

\begin{figure}[H]
   \centering
   \includegraphics[width=1.0\linewidth,height=0.5\linewidth]{fig110009.png}
   \caption{Stopping goalkeeper rotation}
\label{fig110009}
\end{figure}

Instructions should be added so that when the player drags the ball, it moves. First the when football.Flug event needs to be added. Similarly, Interval and Speed values should be set for the goalkeeper. The value of the Interval property should be 10, and that of the Speed 20. In addition to the movement, the angle at which the ball will move should also be set. For this purpose, a value must be set for the Heading property, which comes from the get heading event.

\begin{figure}[H]
   \centering
   \includegraphics[width=1.0\linewidth,height=0.5\linewidth]{fig110010.png}
   \caption{Ball Movement}
\label{fig110010}
\end{figure}

In the last phase of the game, what happens when the ball touches the net and when it touches the goalkeeper must be programmed. According to the rules of the game, when the ball touches the net, the score increases by 1, and when it touches the goalkeeper, then the game ends. This means that a variable must be created in which the result of the game will be stored. The initial value of this variable is 0.

The event that is needed is when football.CollidedWith. Two checks must be added inside this event - whether the ball touched the net and whether the ball touched the goalkeeper.

\begin{figure}[H]
   \centering
   \includegraphics[width=1.0\linewidth,height=0.5\linewidth]{fig110011.png}
   \caption{Adding the checks}
\label{fig110011}
\end{figure}

When the ball touches the goalkeeper the game must end. In the language of the instructions, some of the properties of the ball should be changed and a message should be displayed that the game is over. The first ball property to change is Enabled to false. That means it won't move. Its position should be changed. The values for the coordinates are x=120, y=320 and z=1. The Speed property should be changed to 0.

To display a dialog message, an instruction is added from the Notifier1 element, which is ShowMessageDialog. It should have the following fields - message="Game over", title, which consists of score.Text and scoreValue.Text and the last one is buttonText="Restart game". When the game is over, the initial values of the variable, which is 0, must be returned to the scoreValue text field, which is also 0. Don't forget to change the value of the ball Enabled to true, so that it can again be moves the ball.

\begin{figure}[H]
   \centering
   \includegraphics[width=1.0\linewidth,height=0.5\linewidth]{fig110012.png}
   \caption{When the ball touches the goalkeeper}
\label{fig110012}
\end{figure}

When the ball touches the net, the instructions are basically the same. First the position of the ball must be changed. The difference is that the variable must be incremented and the new result displayed.

\begin{figure}[H]
   \centering
   \includegraphics[width=1.0\linewidth,height=0.5\linewidth]{fig110013.png}
   \caption{When the ball hits the net}
\label{fig110013}
\end{figure}

The game is ready. You can make a competition with your friends to see who is better at scoring goals.
\newpage
\chapter{Събери плодовете}

Целта на тази игра е играчът да улавя падащите плодове, за да не позволява да паднат на земята. За всеки паднал плод той губи един живот. За всеки уловен плод той печели по една точка. Играта приключва, когато загуби трите си живота. Целта е да събере максимален брой точки.

\begin{figure}[H]
  \centering
  \includegraphics[width=1.0\linewidth,height=0.5\linewidth]{fig120001.png}
  \caption{Събери плодовете}
\label{fig120001}
\end{figure}

\section{Създаване на дизайна}
Конструирането на играта започва със създаването на дизайна, как ще изглежда играта. Първата стъпка е добавяне на Layout, който да бъде VerticalArrangement. Той е необходим за началния екран, когато играта започне. Размерите на този елемент трябва да бъдат такива каквито са на екрана на телефона. За това свойствата височина и ширина трябва да се сменят.

\begin{figure}[H]
  \centering
  \includegraphics[width=1.0\linewidth,height=0.5\linewidth]{fig120002.png}
  \caption{Начален екран}
\label{fig120002}
\end{figure}

За да е по- атрактивно началото може да се добави изображение. Всички изображения, които са показани в този пример могат да бъдат заменени. Може да се използват всякакви изображения, които се разпространяват със свободен лиценз.

Към свойството Image трябва да се добави избраното изображение, за да се визуализира върху екрана на телефона, когато играта започне.

\begin{figure}[H]
  \centering
  \includegraphics[width=1.0\linewidth,height=0.5\linewidth]{fig120003.png}
  \caption{Фон на началния екран}
\label{fig120003}
\end{figure}

Играта ще започне, когато играчът натисне бутона за начало. За тази цел трябва да бъде добавен и елемент за бутона. Възможно е да се използва вградения елемент за бутон, но за да се направи играта по- атрактивна може да се използва елемента Image. Може да се използва изображение за бутона. Важно е да се маркира свойството Clickable. Свойствата Height и Width отговарят за това колко да бъде високо и широко изображението.

\begin{figure}[H]
  \centering
  \includegraphics[width=1.0\linewidth,height=0.5\linewidth]{fig120004.png}
  \caption{Бутон за начало}
\label{fig120004}
\end{figure}

Когато играта започне ще се появява този изглед. Когато играчът натисне бутонът трябва да се появи друг изглед, в който ще се появи героят, който ще бъде управляван и падащите плодове. С цел да се различават различните изгледи на играта, когато се програмират е добра практика да имат описателни имена. Елементът VerticalArrangement1 ще се казва StartLayout, а елементът Image1 ще се казва StartButton.

\begin{figure}[H]
  \centering
  \includegraphics[width=1.0\linewidth,height=0.5\linewidth]{fig120005.png}
  \caption{Финална версия на началния екран}
\label{fig120005}
\end{figure}

За създаването на другия изглед отново трябва да се добави елемент VerticalArrangement, който да се казва MainLayout. Височината и ширината на този елемент трябва да бъдат същите, каквито са на екрана. С цел да се направи по- лесно дизайна на този изглед трябва да се размаркира свойството Visible на елемента StartLayout.

\begin{figure}[H]
  \centering
  \includegraphics[width=1.0\linewidth,height=0.5\linewidth]{fig120006.png}
  \caption{Създаване на основен екран на играта}
\label{fig120006}
\end{figure}

Към елемента MainLayout следва да се добавят елементите Canvas и HorizontalArrangement.
Елементът Canvas е необходим, защото той позволява на елементите, които са вътре в него да се движат. В играта съществуват няколко елемента, които ще се движат. От една страна това е героят, който ще улавя плодовете, а от друга самите плодове.
Височината и ширината на този елемент трябва да бъдат същите, каквито са на екрана. Отново може да бъде добавен и фон на този елемент.

\begin{figure}[H]
  \centering
  \includegraphics[width=1.0\linewidth,height=0.5\linewidth]{fig120007.png}
  \caption{Добавяне на фон към основния екран на играта}
\label{fig120007}
\end{figure}

Към този елемент трябва да бъдат добавени няколко изображения на плодове. Също така изображение, което да е героят и три изображения, които да са неговите животи. С промяната на свойствата Height и Width може да се променят размерите им, а чрез промяна на свойствата X, Y и Z може да се промени и тяхното положение. Важно е да се променят и имената на елементите, за да може да се разпознават, когато се премине към програмирането им.
На фигурата е показано примерно разположение на плодовете, героя и неговите животи. За целите на играта не е от значение броя на плодовете или тяхното разположение.

\begin{figure}[H]
  \centering
  \includegraphics[width=1.0\linewidth,height=0.5\linewidth]{fig120008.png}
  \caption{Добавяне на плодовете, животите и героя към играта}
\label{fig120008}
\end{figure}

За да бъде завършен този изглед трябва да се добави елемент HorizontalArrangement, към който ще се добавят двата бутона - стрелка наляво и стрелка надясно. Тези бутони ще контролират играча да се движи. Между тях може да бъде поставен и елемент Label, в който ще се показва колко точки е спечелил играчът.
Ширината на елемента HorizontalArrangement трябва да бъде колкото е тази на екрана. Височината му трябва да бъде малка, например 10 процента. За да могат елементите вътре да се подредят трябва да се променят свойствата AlignHorizontal и AlignVertical да бъдат със стойност Center.

\begin{figure}[H]
  \centering
  \includegraphics[width=1.0\linewidth,height=0.5\linewidth]{fig120009.png}
  \caption{Добавяне на лента за контролите}
\label{fig120009}
\end{figure}

Следва да бъдат добавени бутоните, като се променят техните размери и се добавят изображения, така че да заприличат на стрелки лява и дясна. 

\begin{figure}[H]
  \centering
  \includegraphics[width=1.0\linewidth,height=0.5\linewidth]{fig120010.png}
  \caption{Добавяне на контролите}
\label{fig120010}
\end{figure}

За елемента Label, където ще се изписват точките трябва да се промени стойността на свойството FontSize, за да може текста да бъде по- голям и да се вижда.

\begin{figure}[H]
  \centering
  \includegraphics[width=1.0\linewidth,height=0.5\linewidth]{fig120011.png}
  \caption{Промяна на размера на точките}
\label{fig120011}
\end{figure}

Последният изглед, на който трябва да бъде направен дизайн е когато играчът загуби животите си. Тогава следва да се покаже бутон “Опитай пак”. За да може по- лесно да се направи дизайна трябва да се размаркира свойството Visible на елемента MainLayout.

\begin{figure}[H]
  \centering
  \includegraphics[width=1.0\linewidth,height=0.5\linewidth]{fig120012.png}
  \caption{Добавяне на изглед за край на играта}
\label{fig120012}
\end{figure}

Изгледът, който е ще е за край на играта трябва да бъде отново HorizontalArrangement, като размерите височина и ширина трябва да бъдат същите, каквито са на екрана. Може да се промени цвета на фона на този елемент или да се добави картинка за край на играта. Стойностите на свойствата AlignHorizontal и AlignVertical трябва да бъдат със стойност Center, за да може бутона за рестартиране на играта да бъде поставен в средата.

\begin{figure}[H]
  \centering
  \includegraphics[width=1.0\linewidth,height=0.5\linewidth]{fig120013.png}
  \caption{Промяна на цвета на фона}
\label{fig120013}
\end{figure}

Последно трябва да бъде добавен и бутона за рестартиране на играта. Свойствата за формата, големината на текста и цвета на бутона могат да бъдат променени.

\begin{figure}[H]
  \centering
  \includegraphics[width=1.0\linewidth,height=0.5\linewidth]{fig120014.png}
  \caption{Добавяне на бутон за рестарт на играта}
\label{fig120014}
\end{figure}

Свойството Visible на елемента GameOverLayout трябва да бъде размаркирано. Единствено екрана за начало на играта трябва да се вижда. За това трябва да се маркира за елемента StartLayout.

\begin{figure}[H]
  \centering
  \includegraphics[width=1.0\linewidth,height=0.5\linewidth]{fig120015.png}
  \caption{Маркиране на свойството Visible за началния екран}
\label{fig120015}
\end{figure}

\section{Създаване на програмата}
В конструирането на кода на играта ще бъдат използвани блокове, които се наричат “процедури”. Процедурата се характеризира с име и набор от инструкции. Те се изпълняват тогава, когато тя бъде извикана. Целта на процедурата е да съдържа инструкции, които се повтарят в даден момент в кода. В следващите стъпки е обяснено къде и защо има нужда да се използват процедури.

Първата процедура, която трябва да се създаде е за начало на играта. Инструкциите, които ще бъдат в нея са за първоначалните позиции на плодовете и задаване на скоростите, с които ще се движат. Тази последователност от стъпки трябва да се изпълни, когато играта започне, т.е. играчът натисне бутона за начало на играта, но също така и когато играчът натисне бутона за рестартиране на играта. За това се използва процедура с цел избягването на конструиране на едни и същи стъпки два пъти.

\begin{figure}[H]
  \centering
  \includegraphics[width=1.0\linewidth,height=0.5\linewidth]{fig120016.png}
  \caption{Създаване на процедура за начало на играта}
\label{fig120016}
\end{figure}

Първият път когато ще се извика тази процедура е когато се натисне бутона StartButton. Освен извикването на процедурата, другите инструкции, които трябва да се изпълнят са стартовия изгледа да се скрие и да се появи основния изглед. Това се реализира с помощта на свойството Visible, което може да се контролира освен през дизайна, а и програмно.

\begin{figure}[H]
  \centering
  \includegraphics[width=1.0\linewidth,height=0.5\linewidth]{fig120017.png}
  \caption{Програмиране на бутона за начало на играта}
\label{fig120017}
\end{figure}

Героят на играта ще се движи наляво и надясно, когато се клика върху стрелка наляво или стрелка надясно съответно. За тази цел трябва да бъдат добавени две събития - когато се кликне върху стрелка наляво или когато се кликне върху стрелка надясно. Инструкциите в тези две събития са подобни - героят трябва да си смени позицията спрямо X координатата. В единия случай трябва да се увеличи, а в другия да се намали.

\begin{figure}[H]
  \centering
  \includegraphics[width=1.0\linewidth,height=0.5\linewidth]{fig120018.png}
  \caption{Програмиране на героя да се движи}
\label{fig120018}
\end{figure}

В тази игра трябва да се създадат следните променливи - 3, които ще отговарят за скоростта на трите плода и една, която ще бъде за животите на героя.

\begin{figure}[H]
  \centering
  \includegraphics[width=1.0\linewidth,height=0.5\linewidth]{fig120019.png}
  \caption{Добавяне на променливи към играта}
\label{fig120019}
\end{figure}

Когато играчът докосне плод следва точките му да се увеличат. За тази цел отново ще се конструира процедура, тъй като и за трите плода трябва да се извика инструкцията, която променя точките.

\begin{figure}[H]
  \centering
  \includegraphics[width=1.0\linewidth,height=0.5\linewidth]{fig120020.png}
  \caption{Процедура за промяна на точките}
\label{fig120020}
\end{figure}

В следващата стъпка ще се добавят инструкции, които ще се изпълнят, когато играчът докосне плод. Алгоритъмът е следния - сменя се позицията на плода, увеличават се точките с 1 (има създадена процедура) и се увеличава скоростта на героя.

\begin{figure}[H]
  \centering
  \includegraphics[width=1.0\linewidth,height=0.5\linewidth]{fig120021.png}
  \caption{Алгоритъм когато играчът докосне плод}
\label{fig120021}
\end{figure}

Този алгоритъм трябва да се конструира и за останалите герои плодове.

\begin{figure}[H]
  \centering
  \includegraphics[width=1.0\linewidth,height=0.5\linewidth]{fig120022.png}
  \caption{Алгоритъмът за всички плодове}
\label{fig120022}
\end{figure}

Последната процедура, която трябва да се създаде е да се направи проверка дали играта трябва да приключи. Ако броя на животите е равен на 0, то тогава трябва да се скрие основния екран и да се появи екрана за край на играта. Също така скоростта на всички плодове трябва да стане равна на 0.

\begin{figure}[H]
  \centering
  \includegraphics[width=1.0\linewidth,height=0.5\linewidth]{fig120023.png}
  \caption{Проверка дали героят губи играта}
\label{fig120023}
\end{figure}

Следва да се направят проверки ако животите са 2 или 1. Ако са два, това означава, че играчът е загубил 1 и трябва един от животите да се скрие. Отново трябва да се промени свойството Visible. Ако броят е 1, то това означава, че играчът е загубил 2 живота и следва да се скрие картинката и на още един от животите.

\begin{figure}[H]
  \centering
  \includegraphics[width=1.0\linewidth,height=0.5\linewidth]{fig120024.png}
  \caption{Проверка дали героят има живот}
\label{fig120024}
\end{figure}

Когато героят плод докосне края на екрана, тогава трябва да се извика процедурата, за проверка за край на играта и да промени позицията си.

\begin{figure}[H]
  \centering
  \includegraphics[width=1.0\linewidth,height=0.5\linewidth]{fig120025.png}
  \caption{Проверка дали плод е докоснал ръба}
\label{fig120025}
\end{figure}

Последните инструкции, които трябва да се конструират са, когато играта е приключила и играчът натисне бутона за рестартиране на играта.

\begin{figure}[H]
  \centering
  \includegraphics[width=1.0\linewidth,height=0.5\linewidth]{fig120026.png}
  \caption{Завършване на играта}
\label{fig120026}
\end{figure}

Играта е завършена. Следва да се тества върху телефона.
\newpage
\chapter{Боулинг}

В този проект ще създадете една любима игра на детацата - боулинг. Играчът ще трябва да изстреля топка от долната част на екрана, която ще достигне до кеглите, които се намират в горната част. Целта е играчът да събори всички кегли.

\begin{figure}[H]
  \centering
  \includegraphics[width=1.0\linewidth,height=0.5\linewidth]{fig130001.png}
  \caption{Боулинг}
\label{fig130001}
\end{figure}

\section{Създаване на дизайна}

В първата стъпка ще създадете началният екран на играта. Върху него ще има един бутон, който ще бъде начало на играта. От групата с елементите Layout трябва да бъде добавен елемента VerticalArrangement. Размерите на елемента трябва да бъдат такива, каквито са размерите на екрана. За това свойствата за височина и ширина трябва да се сменят. За фон на играта освен готовите цветове, може да създадете и свой цвят.

\begin{figure}[H]
  \centering
  \includegraphics[width=1.0\linewidth,height=0.5\linewidth]{fig130002.png}
  \caption{Начален екран}
\label{fig130002}
\end{figure}

Следва да добавите и бутон за начало на играта. Променете дизайна на бутона и го позиционирайте в средата на екрана.

\begin{figure}[H]
  \centering
  \includegraphics[width=1.0\linewidth,height=0.5\linewidth]{fig130003.png}
  \caption{Бутон за начало на играта}
\label{fig130003}
\end{figure}

В следващата стъпка ще създадете екранът за край на играта. Върху него ще има един бутон за стартиране отново на играта. Също така ще трябва да добавите и надпис, че играта е приключила. Първо от Layout добавете елемент VerticalArrangment. Размерите на елемента трябва да бъдат отново, такива каквито са на екрана. Изберете и подходящ цвят за фон на този екран.

\begin{figure}[H]
  \centering
  \includegraphics[width=1.0\linewidth,height=0.5\linewidth]{fig130004.png}
  \caption{Краен екран}
\label{fig130004}
\end{figure}

Добавете бутона за започване на играта отново. Променете дизайна му и го позиционирайте в средата на екрана.

\begin{figure}[H]
  \centering
  \includegraphics[width=1.0\linewidth,height=0.5\linewidth]{fig130005.png}
  \caption{Бутон за начало на играта отново}
\label{fig130005}
\end{figure}

Добавете и елемент Lable, който да казва Game Over.

\begin{figure}[H]
  \centering
  \includegraphics[width=1.0\linewidth,height=0.5\linewidth]{fig130006.png}
  \caption{Надпис за край на играта}
\label{fig130006}
\end{figure}

В последната стъпка ще създадете дизайн на играта. Добавете отново елемент VerticalArrangment. Променете размерите на елемента, като височината и ширината трябва да бъдат толкова, колкото са на екрана. Вътре в този елемент добавете елементът Canvas от секция Drawing and Animation. Променете размера на елемента и цвета на фона.

\begin{figure}[H]
  \centering
  \includegraphics[width=1.0\linewidth,height=0.5\linewidth]{fig130007.png}
  \caption{Екран на играта}
\label{fig130007}
\end{figure}

Добавете елемента топка и променете размерите, позицията и цвета на елемента. Топката трябва да бъде поставена в долната част на екрана.

\begin{figure}[H]
  \centering
  \includegraphics[width=1.0\linewidth,height=0.5\linewidth]{fig130008.png}
  \caption{Добавяне на топката в играта}
\label{fig130008}
\end{figure}

За кеглите може да използвате изображение, което се разпространява със свободен лиценз. Добавете три елемента ImageSprite към елемента Canvas. Към всеки от елементите добавете изображението, което сте изтеглили. Променете размерите и позицията. Трите кегли трбява да се намират в горната част на екрана.

\begin{figure}[H]
  \centering
  \includegraphics[width=1.0\linewidth,height=0.5\linewidth]{fig130009.png}
  \caption{Добавяне на кеглите}
\label{fig130009}
\end{figure}

В последната част от дизайна трябва да добавите и бутона за изстрелване на топката. Първо добавете елемент HorizontalArrangement. Променете ширината на елемента да бъде толкова, колкото е на екрана.

\begin{figure}[H]
  \centering
  \includegraphics[width=1.0\linewidth,height=0.5\linewidth]{fig130010.png}
  \caption{Добавяне на място за бутона за изстрел}
\label{fig130010}
\end{figure}

Към този елемент добавете още два елемент - бутон и елемент, в който ще за изписва с колко топки разполага героят. Позиционирайте елементите и променете текста на елементите. Може също да промените цвета и формата.

\begin{figure}[H]
  \centering
  \includegraphics[width=1.0\linewidth,height=0.5\linewidth]{fig130011.png}
  \caption{Добавяне на бутон за изстрел на топката}
\label{fig130011}
\end{figure}

\section{Създаване на програмата}

Преди да започнете да програмирате играта оставете видим само екранът за начало на играта. След това пременете към изгледът за добавяне на блокове.

Започнете с програмирането на бутона за начало на играта. Добавете инструкцията Click, което означава, че когато бутонът е натиснат, тогава ще се изпълнят инструкциите. Когато този бутон бъде настиснат ще задава скорост на топката.

\begin{figure}[H]
  \centering
  \includegraphics[width=1.0\linewidth,height=0.5\linewidth]{fig130012.png}
  \caption{Инструкции за началния бутон}
\label{fig130012}
\end{figure}

Добавете променлива, която ще е броя на кеглите. Първоначалната стойност нека да бъде 3.

\begin{figure}[H]
  \centering
  \includegraphics[width=1.0\linewidth,height=0.5\linewidth]{fig130013.png}
  \caption{Променлива за броя кегли}
\label{fig130013}
\end{figure}

Когато топката доконсе някой от ръбовете тя трябва да се отклони. За да се изпълнят инструкциите за отклоняване на топката, първо трябва да се добави събитие. Вътре в това събитие трябва да се направят следните проверки:
- ако топката се намира в най- левия край на екран, тя трябва да се движи надясно
- ако топката се намира в най- десния край на екрана, тя трябва да се движи наляво
- ако топката се намира в най- горния край на екрана, тя трябва да се премести в първоначална позиция.

\begin{figure}[H]
  \centering
  \includegraphics[width=1.0\linewidth,height=0.5\linewidth]{fig130014.png}
  \caption{Движение на топката когато докосне ръб на екрана}
\label{fig130014}
\end{figure}

Другото събитие, което се отнася за топката е, когато тя докосне нещо различно от екрана. В тази игра това може да бъде единствено кегла. Тогава трябва да се промени броя на кеглите, като се извади едно. Друго важно нещо, което трябва да се направи е да се провери дали има още кегли на екрана. Ако няма, то тогава трябва да се скрие този екран и да се покаже последния екран. Важно е да се промени съобщението да изписва Winner.

Ако броя на кеглите не е равен на 0, то тогава трябва да се добави проверка коя кегла е докоснала топката. Целта на това е да може да се направи невидима и да не бъде част от играта.

\begin{figure}[H]
  \centering
  \includegraphics[width=1.0\linewidth,height=0.5\linewidth]{fig130015.png}
  \caption{Инструкции, когато топката докосне кегла}
\label{fig130015}
\end{figure}

Следващите инструкции ще бъдат когато се натисне бутона за изстрелване на топката. Първаа проверка, която трябва да се направи е, ако полето за броя на изстреляните топки е 0, то тогава играта приключва и трябва да се покаже екранът за край на играта. Ако не е равен на 0, то тогава трябва да се зададе посока на топката и да се намали броя с 1.

\begin{figure}[H]
  \centering
  \includegraphics[width=1.0\linewidth,height=0.5\linewidth]{fig130016.png}
  \caption{Инструкции за бутона за изстрел на топката}
\label{fig130016}
\end{figure}

Последните инструкции, които трябва да се добавят са за бутона, който стартира играта от начало. Тези инструкции включват показване на игровия екран, показване на кеглите, задаване на техния брой да бъде 3 и броя на топките да бъде 10.

\begin{figure}[H]
  \centering
  \includegraphics[width=1.0\linewidth,height=0.5\linewidth]{fig130017.png}
  \caption{Инструкции за бутона за започване на играта отначало}
\label{fig130017}
\end{figure}

Пожелаваме ви приятна игра!
\newpage
\chapter{Jump the Obstacles}

With this project you will create a popular computer game. The player will control a character using the arrow keys. His goal will be to cross the course without touching the obstacles. In cases where it touches them, it will start from the beginning. If he manages to pass the entire route, he moves to the next level.

\begin{figure}[H]
   \centering
   \includegraphics[width=1.0\linewidth,height=0.5\linewidth]{fig140001.png}
   \caption{Jump the obstacles}
\label{fig140001}
\end{figure}

\section{Creating the Design}
The first step in creating the game will be adding the appropriate characters and background. For the game character, you can choose from ready-made characters in Scratch. Another option is to draw it to make your game more interesting. Using the drawing tools you can create your character (Fig. \ref{fig140002}). In this project I will demonstrate how to create an effect when the character moves. For this purpose, one more suit should be added to it.

\begin{figure}[H]
   \centering
   \includegraphics[width=1.0\linewidth,height=0.5\linewidth]{fig140002.png}
   \caption{Adding Main Character}
\label{fig140002}
\end{figure}

The game will need one character to be in the lower part of the track. This is a revenge hero that will serve as a reference point. The purpose of the purple character is to move on it. Create it using the drawing tools.

\begin{figure}[H]
   \centering
   \includegraphics[width=1.0\linewidth,height=0.5\linewidth]{fig140003.png}
   \caption{Adding the supporting character}
\label{fig140003}
\end{figure}

The last character to add is the one for the obstacles. It must be drawn again with the auxiliary tools. To have more levels in the game, add more costumes to your character. When your purple character crosses the track, ie. reach the rightmost point of the screen, then that character will change his costume, which means the player will go to the second level.

\begin{figure}[H]
   \centering
   \includegraphics[width=1.0\linewidth,height=0.5\linewidth]{fig140004.png}
   \caption{Add obstacles}
\label{fig140004}
\end{figure}

To make the game more attractive, add a suitable background. Also, like the characters, you can choose it from the ready-made ones in Scratch or draw it yourself using the tools.

\begin{figure}[H]
   \centering
   \includegraphics[width=1.0\linewidth,height=0.5\linewidth]{fig140005.png}
   \caption{Add background}
\label{fig140005}
\end{figure}

\section{Programming Character Movement}

The main code you need to add is located in the character you will be controlling, in this case the purple character. In this game you will need 3 variables - xVel, yVel and jump. From the Variables section, select the Make a Variable option and add the variables with the appropriate names. To prevent them from appearing during gameplay, you can remove their ticks.

\begin{figure}[H]
   \centering
   \includegraphics[width=1.0\linewidth,height=0.5\linewidth]{fig140006.png}
   \caption{Adding the variables}
\label{fig140006}
\end{figure}

At the beginning of the game, the value of these variables should be equal to 0. To make the game more interesting, the character's position will be in the middle part of the screen. When the game starts he will drop down to the platform.

\begin{figure}[H]
   \centering
   \includegraphics[width=1.0\linewidth,height=0.5\linewidth]{fig140007.png}
   \caption{Initialize variables}
\label{fig140007}
\end{figure}

The character must move until it reaches the right side of the screen or until the character touches an obstacle. For this purpose, the first statement you need to add is a repeat until block. The conditions you need to add are two that are separated by the OR operator. The first condition is that the character's x position is greater than 240, which is the end of the screen. The second condition is to touch the red character.

\begin{figure}[H]
   \centering
   \includegraphics[width=1.0\linewidth,height=0.5\linewidth]{fig140008.png}
   \caption{Loop with character movement condition}
\label{fig140008}
\end{figure}

To move the character left and right you will use the xVel variable. From the variable section, the change xVel by statement is needed. From the operators section, select the one to subtract. On one side put the instruction for key right arrow pressed, and on the right arrow key left arrow pressed. This statement will set a value to the variable. All that remains is to add the change x by instruction and place the variable. So if you start the game, the character will move very fast left and right. For this add the instruction set xVel to xVel * 0.9. Start the game to test how the character moves left and right.

\begin{figure}[H]
   \centering
   \includegraphics[width=1.0\linewidth,height=0.5\linewidth]{fig140009.png}
   \caption{Character movement left and right}
\label{fig140009}
\end{figure}

You must program the character to jump In the first part, you will program when the game starts or another level that the character from the initial position goes down to the platform. To do this, you need to change the variable yVel to -1 and use the statement change y by with the value variable. Next is the check if it touches the platform. If the condition is met, then the value of the variable yVel should be set to 0. So the only problem will be that the character will not stop moving when it touches the platform, because the main loop will spin one more time, because not ready condition is not met. For this, the y position must be changed by yVel multiplied by -1.

\begin{figure}[H]
   \centering
   \includegraphics[width=1.0\linewidth,height=0.5\linewidth]{fig140010.png}
   \caption{Moving Character Down}
\label{fig140010}
\end{figure}

The last part of creating the purple character's movement algorithm will be adding the jumps. If the player presses the up arrow, then the jump variable will be equal to 1. In any of the other cases, it should be equal to 0, which means that the character will not jump. First you need to add a check before making yVel equal to 0. This check should check if yVel is less than 0 then the character should not jump.

\begin{figure}[H]
   \centering
   \includegraphics[width=1.0\linewidth,height=0.5\linewidth]{fig140011.png}
   \caption{Checking when character doesn't jump}
\label{fig140011}
\end{figure}

There is also a check to see if the player has pressed the up arrow. If it is and the character is not jumping, then yVel should be changed to a positive number and the jump variable should be set to 1. Playtest to see how your character jumps and moves left and right.

\begin{figure}[H]
   \centering
   \includegraphics[width=1.0\linewidth,height=0.5\linewidth]{fig140012.png}
   \caption{Hero jump when pressing up arrow}
\label{fig140012}
\end{figure}

\section{Creating an effect when the character moves}

After you have finished moving the character, you can add additional instructions to create an effect when the character moves. After checking for the pressed up arrow, add the create clone of myself statement. The reason for this is that a clone of that character will be created, but with his second costume. For this add a new instruction which is When I start as a clone. Then the character's costume should be changed, it should be his second. Add a loop that repeats 10 times. Inside the loop, reduce the size of the clone character and add a transparent effect. After the cycle, delete the clone. Test the game to see what effect is added when you move.

\begin{figure}[H]
   \centering
   \includegraphics[width=1.0\linewidth,height=0.5\linewidth]{fig140013.png}
   \caption{Add motion effect}
\label{fig140013}
\end{figure}

\section{Go to next level}

In the last step, you need to add instructions for the character to be able to move to the next level. For this purpose, in the main character, add the check if he has reached the end of the screen. If it is, then let it send a message.

\begin{figure}[H]
   \centering
   \includegraphics[width=1.0\linewidth,height=0.5\linewidth]{fig140014.png}
   \caption{Send Next Level Message}
\label{fig140014}
\end{figure}

Choose the hero with the red obstacles. Add the event when the game starts, then the character should be in his original costume. Also add a second event when he gets the next level message, he then has to change his suit. Start the game and have fun.

\begin{figure}[H]
   \centering
   \includegraphics[width=1.0\linewidth,height=0.5\linewidth]{fig140015.png}
   \caption{Go to next level}
\label{fig140015}
\end{figure}
\newpage
\chapter{Дартс}

Този проект представлява популярната дартс игра. Играчът хвърля къси стрели по кръгла мишена. Колкото по- близо е до центъра на мишената, толкова повече точки печели. Целта на играча е да събере максимален брой точки за трите удара, с които разполага. Ако събраните точки са над 13 - печели играта, в противен случай губи играта.

\begin{figure}[H]
  \centering
  \includegraphics[width=1.0\linewidth,height=0.5\linewidth]{fig150001.png}
  \caption{Дартс}
\label{fig150001}
\end{figure}

\section{Създаване на дизайна}
Преди да започнете с програмирането на играта, първо трябва да създадете дизайна. Първо добавете двата основни героя на играта - мишената (Фиг. \ref{fig150002}) и стрелата (Фиг. \ref{fig150003}). Използвайте инструментите в Scratch, за да нарисувате необходимите герои.

\begin{figure}[H]
  \centering
  \includegraphics[width=1.0\linewidth,height=0.5\linewidth]{fig150002.png}
  \caption{Мишена}
\label{fig150002}
\end{figure}

\begin{figure}[H]
  \centering
  \includegraphics[width=1.0\linewidth,height=0.5\linewidth]{fig150003.png}
  \caption{Стрела}
\label{fig150003}
\end{figure}

Добавете нов герой, който също може да нарисувате с помощта на Scratch инструментите. Този герой ще бъде отговорен за това да показва колко изстрела остават на играча. За да покажете колко изстрела остават, добавяте костюми на този герой. Например на първия костюм може да напишете, че остават 3 изстрела, а на втория - 2 изстрела.

\begin{figure}[H]
  \centering
  \includegraphics[width=1.0\linewidth,height=0.5\linewidth]{fig150004.png}
  \caption{Брой изстрели}
\label{fig150004}
\end{figure}

Следващият герой, който следва да добавите е, този, който показва колко точки е спечелил играчът. Отново добавете костюми на героя със съответния брой точки които той може да спечели. В случая това са от 0 до 5.

\begin{figure}[H]
  \centering
  \includegraphics[width=1.0\linewidth,height=0.5\linewidth]{fig150005.png}
  \caption{Брой точки}
\label{fig150005}
\end{figure}

Последният Scratch герой е надписа дали играчът печели или губи. Подобно на предишните два героя, добавете два костюма с различни надписа - единият да бъде за победа, а другият за загуба.

\begin{figure}[H]
  \centering
  \includegraphics[width=1.0\linewidth,height=0.5\linewidth]{fig150006.png}
  \caption{Надпис за край на играта}
\label{fig150006}
\end{figure}

\section{Програмиране на мишената и стрелата}
Преди да преминем към програмирането на съответните герои, добавете нова променлива points към играта. Тази променлива ще съдържа в себе си резултата, който героят ще трупа по време на играта. От секция Variables изберете Make a Variable и задайте име на променливата.

\begin{figure}[H]
  \centering
  \includegraphics[width=1.0\linewidth,height=0.5\linewidth]{fig150007.png}
  \caption{Създаване на променлива в играта}
\label{fig150007}
\end{figure}

Нека да преминем към програмирането на мишената. Единствените инструкции, които трябва да добавите към този герой са, той да бъде на една и съща позиция по време на цялата игра. Не забравяйте да зададете и първоначална стойност на променливата, която създадохте. В началото на всяка игра, играчът трябва да има 0 точки.

\begin{figure}[H]
  \centering
  \includegraphics[width=1.0\linewidth,height=0.5\linewidth]{fig150008.png}
  \caption{Инструкции на мишената}
\label{fig150008}
\end{figure}

Следва да добавите инструкции и за стрелата. Ако стрелата в твърде голяма, може да промените размерите ѝ. Създайте нова променлива, която да съдържа в себе си какъв е статуса на стрелата. Съществуват два статуса - shoot или throw. В началото на играта статуса на стрелата ще бъде shoot.

\begin{figure}[H]
  \centering
  \includegraphics[width=1.0\linewidth,height=0.5\linewidth]{fig150009.png}
  \caption{Задаване на статус на стрелата}
\label{fig150009}
\end{figure}

Добавете следната проверка - ако статуса на стрелата е shoot, то тогава стрелата трябва да следва мишката на играча. Също така добавете и ново събитие, което е when this sprite clicked. Когато играчът кликне върху стрелата, трябва да промените стойността на променливата status да бъде throw. Също така героят трябва да изпрати съобщение на другите герои, че играчът е натиснал мишката, което означава, че стреля.

\begin{figure}[H]
  \centering
  \includegraphics[width=1.0\linewidth,height=0.5\linewidth]{fig150010.png}
  \caption{Изпращане на съобщение за стрелба}
\label{fig150010}
\end{figure}

След като играчът е изстрелял стрелата трябва да я програмирате да променя посоката си, за да може да имитирате изстрел. Също така променете и размерите ѝ да става по- малка, все едно се отдалечава от играчът. Знаете, че когато играчът кликне върху стрелата, то се изпраща съобщение. Използвайте това съобщение за да промените посоката и размера на стрелата след изстрел.

\begin{figure}[H]
  \centering
  \includegraphics[width=1.0\linewidth,height=0.5\linewidth]{fig150011.png}
  \caption{Промяна на посоката и размера на стрелата след изстрел}
\label{fig150011}
\end{figure}

Когато играчът изстреля стрелата, тя започва да се отдалечава и изпраща съпбщение Check Points, което трябва да провери колко точки е спечелил играчът. За да се определи броя на спечелените точки, трябва да се провери до какъв цвят се е докоснала стрелата. За да бъде най- честна играта проверката, която ще се направи е черният цвят (това е върхът на стрелата) до кой цвят се е докоснал. Добавете нова променлива, която да съдържа текущо спечелените точки. Последната инснтрукция, която трябва да добавите е, стрелата да изпраща съобщение, че дава точки на играча.

\begin{figure}[H]
  \centering
  \includegraphics[width=1.0\linewidth,height=0.5\linewidth]{fig150012.png}
  \caption{Проверка на спечелените точки}
\label{fig150012}
\end{figure}

За да направите играта още по- интересна може да добавите променлива, която да бъде случайно зададено число. Тази променлива ще отклонява стрелата настрани, спрямо това какво случайно число се е паднало.

\begin{figure}[H]
  \centering
  \includegraphics[width=1.0\linewidth,height=0.5\linewidth]{fig150013.png}
  \caption{Отклоняване на стрелата встрани}
\label{fig150013}
\end{figure}

\section{Програмиране на героя, който отброява броя изстрели}

В тази стъпка ще добавите инструкции, които ще показват на играча колко изстрела още има право да направи. За целта изберете героя, който добавихте за брой изстрели. Поставете този герой на избрана от вас позиция. Създайте нова променлива, която ще съдържа в себе си броя на изстрелите. След това направете нужните проверки - ако броя на изстрелите е 3, то тогава трябва да се премине към костюм, който показва три изстрела, ако броя на изстрелите е две - то тогава трябва да се премине към костюма с два изстрела и т.н. При последната проверка - ако броя на изстрелите е 0, то тогава трябва да се изпрати съобщение, че играта е приключила.

\begin{figure}[H]
  \centering
  \includegraphics[width=1.0\linewidth,height=0.5\linewidth]{fig150014.png}
  \caption{Програмиране на брой изстрели}
\label{fig150014}
\end{figure}

\section{Програмиране на героя, който показва броя на точките}

Изберете героя, който показва колко точки е спечелил играчът. Този герой освен за това да покаже колко са точките, той ще отговаря и за това да намали броя на изстрелите, да зададе нова посока на вятъра, както и да промени статуса на стрелката да бъде готова отново за изстрел.

Тъй като не знаете колко точки ще спечели играчът и за да не правите хиляди проверки, може да си кръсисте имената на костюмите на броя точки. Така лесно ще може да преминавате от костюм на костюм, тъй като имате променлива, която държи в себе си броя на текущо спечелените точки.

\begin{figure}[H]
  \centering
  \includegraphics[width=1.0\linewidth,height=0.5\linewidth]{fig150015.png}
  \caption{Програмиране броя на точките}
\label{fig150015}
\end{figure}

\section{Програмиране края на играта}

Последната стъпка, която остана да направите е да програмирате края на играта. Спрямо това колко точки е спечелил играчът той или ще спечели или ще загуби след третия изстрел. Ако броя на точките е по- голям от 13 - той печели. В противен случай той губи.

\begin{figure}[H]
  \centering
  \includegraphics[width=1.0\linewidth,height=0.5\linewidth]{fig150016.png}
  \caption{Програмиране края на играта}
\label{fig150016}
\end{figure}

Готови сте да проверите колко сте точни и добри в играта, която създадохте.\newpage
\include{chapters/chapter16}\newpage
\addcontentsline{toc}{chapter}{Conclusions}
\chapter*{Conclusions}
\thispagestyle{empty}

Block programming is an effective and affordable way to introduce children to coding concepts. By breaking programming down into visual blocks, kids can learn to put together small programs without worrying about syntax or typing errors. The Scratch and App Inventor programming environments are proven tools for teaching block programming to children. Scratch's intuitive interface and colorful blocks make it an ideal option for younger children, while App Inventor's ability to create real mobile apps may appeal to older ones. The book provides a comprehensive guide to learning block programming with Scratch and App Inventor. It covers game design and creation, mobile app creation, and more. The book also includes step-by-step instructions and many visual examples to help kids understand programming concepts. Children can develop basic skills such as problem-solving, logical thinking, and creativity through block programming. These skills can be applied in future endeavors, including computer science and other science, technology, engineering, and mathematics fields. Overall, the Block Programming for Kids with Scratch and App Inventor book is an excellent resource for parents and educators who want to introduce children to programming possibilities. Using visual blocks and easy-to-understand instructions, kids can learn to code in a fun and engaging way, setting them up for future personal and professional success.

\newpage

% Списък с използвана литература и източници на информация.
\addcontentsline{toc}{chapter}{Библиография}
\input{chapters/references}\newpage

% Азбучен указател на използваните термини.
\printindex

% Задна корица.
\includepdf[pages=-]{covers/back}

\end{document}
