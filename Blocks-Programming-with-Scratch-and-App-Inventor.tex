\documentclass[14pt,a4paper]{book}

% Използване на български език.
\usepackage[english,bulgarian]{babel}
\usepackage[utf8]{inputenc}

% Използване на графика.
\usepackage[pdftex]{graphicx}

% Използване на PDF-и за кориците.
\usepackage{pdfpages}

% Използване на хедър и футър.
\usepackage{fancyhdr}

% Използване на кавички при цитиране.
\usepackage{dirtytalk}

% Използва се за създаване на азбучен указател.
\usepackage{imakeidx}

% Използва се за сензитивни хипер-връзки в самия документ.
\usepackage[pdftex, bookmarks, linktocpage]{hyperref}

% Използва се за листинги с програмен код.
\usepackage{listings}

% Заглавие.
\title{Блоково програмиране със Scratch и App Inventor}

% Автори.
\author{Тодор Балабанов, Галя Петрова}

% Директория с изображения.
\graphicspath{{images/}}

% Избор на активен език.
\selectlanguage{bulgarian}

% Текстове за декорация на страницата в горната и долната част.
\pagestyle{fancy}
\fancyhf{}
\fancyhead[LE,RO]{\thepage}
\fancyhead[RE]{Блоково програмиране със Scratch и App Inventor}
\fancyhead[LO]{Тодор Балабанов, Галя Петрова}
\fancyfoot[LE,RO]{Издателство \say{Образование и Познание}, 2020}

% Дебелина на разделителните линии.
\renewcommand{\headrulewidth}{2pt}
\renewcommand{\footrulewidth}{1pt}

% Генериране на азбучен указател.
\onecolumn
\makeindex[columns=2, title=Азбучен указател, intoc]

% Подменя думата използван а за ноемрация на фрагментите програмен код.
\renewcommand{\lstlistingname}{Листинг}

% Смяна на названието за списъка от листингите.
\renewcommand{\lstlistlistingname}{Списък на листингите}

% Определя характеристиките на листигните за програмния код.
\lstset{backgroundcolor=\color{gray!30}, breaklines=true, language=r, frame=single}

% Начало на документа.
\begin{document}

% Предна корица.
\includepdf[pages={1}]{covers/front}
\thispagestyle{empty}

% Страница с авторски права.
~\vfill
\thispagestyle{empty}

\noindent Авторски права \copyright\ 2023 \\

\noindent Тодор Балабанов, Галя Петрова, Антония Гузгунова \\ 

\noindent \textsc{Издателство \textquote{Образование и Познание}} \\
\noindent \textsc{https://www.obrazovaniebg.net/} \\

\noindent Разпространява се под свободен лиценз: \\ 
Creative Commons Attribution-NonCommercial-NoDerivatives 4.0 \\
International Public License \\
\url{https://creativecommons.org/licenses/by-nc-nd/4.0} \\

\noindent {\footnotesize This book was supported by the Bulgarian National Science Fund by the project \textquote{Mathematical models, methods, and algorithms for solving hard optimization problems to achieve high security in communications and better economic sustainability, KP-06-N52/7/19-11-2021}}. \\

\noindent \textit{Първо издание, 2023}


% Номериране на страниците със служебна информация.
\pagenumbering{roman}
\setcounter{page}{1}

% Таблица на съдържанието.
\addcontentsline{toc}{chapter}{Съдържание}
\tableofcontents\newpage

% Списък с фигурите.
\addcontentsline{toc}{chapter}{Списък на фигурите}
\listoffigures\newpage

% Списък с таблиците.
\addcontentsline{toc}{chapter}{Списък на таблиците}
\listoftables\newpage

% Списък с листингите.
\addcontentsline{toc}{chapter}{Списък на листингите}
\lstlistoflistings\newpage

% Номериране на страниците с основното изложение.
\pagenumbering{arabic}
\setcounter{page}{1}

% Отделните глави са в отделни файлове.
\addcontentsline{toc}{chapter}{Предговор}
\chapter*{Предговор}
\thispagestyle{empty}
\newpage
%\chapter{Работни среди}

Блоковите програмни езици са подразделение на визуалните програмни езици. Същината на блоковите езици е, че програмните инструкции се въвеждат под формата на цветни блокове, а не както е в класическите програмни езици, чрез изписване на текстови команди. Най-основната цел на блоковите езици е да направят областта програмиране значително по-достъпна за начинаещите. Тази цел се постига чрез три основни направления. От синтактична гледна точка, инструкциите в блоковите езици са под формата на цветни иконки. Това значително намалява възможността за изписването на грешна програмна инструкция. На второ място се подобрява семантиката, като всяка от възможните програмни инструкции е добре документирана. На трето място е прагматизма, който позволява изучаването на различните състояния в които може да изпадне програмата. Програмните среди за блоково програмиране набират все по-голяма популярност през последното десетилетие. Някои от най-популярните са: Scratch, Blockly, App Inventor for Android, Ardublock и други. В тази книга ще се спрем на две от програмните среди за блоково програмиране, създадени в Масачузетския технологичен институт, Scratch и App Inventor for Android. Причината за този избор е, че Scratch има насоченост към най-малките, а именно децата в началните училищни класове, което много добре се съчетава с възможностите блоковите програми да бъдат визуализирани и на мобилен телефон, чрез App Inventor for Android. И при двете програмни среди не се изисква инсталирането на специализиран софтуер. Достатъчно е наличието на съвременен компютър, свързан в Интернет и съвременна версия на уеб браузър. 

\section{Първи стъпки в Sratch}

Работата в средата на Sratch започва със зареждане на главната уеб страница (Фиг. \ref{fig0001}), която се намира на адрес: \\ \href{https://scratch.mit.edu/}{https://scratch.mit.edu/}

\begin{figure}[H]
  \centering
  \includegraphics[width=1.0\linewidth,height=0.5\linewidth]{fig0001.png}
  \caption{Начална уеб страница на Sratch}
\label{fig0001}
\end{figure}

Програмната среда на Sratch е организирана на принципа на облачните услуги. Поради тази причина, всеки желаещ да използва услугата трябва да си направи регистрация (Фиг. \ref{fig0002}). Регистрацията се състои от потребителско име и парола.

\begin{figure}[H]
  \centering
  \includegraphics[width=1.0\linewidth,height=0.5\linewidth]{fig0002.png}
  \caption{Регистрация на потребител в Sratch}
\label{fig0002}
\end{figure}

След избора на потребителско име и парола следва определяне на географския регион в който се намира потребителят (Фиг. \ref{fig0003}).

\begin{figure}[H]
  \centering
  \includegraphics[width=1.0\linewidth,height=0.5\linewidth]{fig0003.png}
  \caption{Географско местоположение}
\label{fig0003}
\end{figure}

Платформата е насочена предимно към деца, изразяващи интерес към програмирането, но също така към родители и учители. Поради тази причина, системата събира информация за възрастта на потребителя (Фиг. \ref{fig0004}).

\begin{figure}[H]
  \centering
  \includegraphics[width=1.0\linewidth,height=0.5\linewidth]{fig0004.png}
  \caption{Възраст на потребителя}
\label{fig0004}
\end{figure}

Освен класификация по възраст, системата събира информация и за класификация по полова принадлежност. Тази информация е незадължителна, основно за да не бъде дискриминираща (Фиг. \ref{fig0005}).

\begin{figure}[H]
  \centering
  \includegraphics[width=1.0\linewidth,height=0.5\linewidth]{fig0005.png}
  \caption{Пол на потребителя}
\label{fig0005}
\end{figure}

Потребителският профил, освен с потребителско име и парола, трябва да бъде аоцииран и с адрес на електронна пощенска кутия (Фиг. \ref{fig0006}).

\begin{figure}[H]
  \centering
  \includegraphics[width=1.0\linewidth,height=0.5\linewidth]{fig0006.png}
  \caption{Адрес на електронна поща на потребителя}
\label{fig0006}
\end{figure}

Процесът по регистрация на потребител в системата е почти завършен (Фиг. \ref{fig0007}). Остава само стъпката за потвърждаване на избрания адрес за електронна поща.

\begin{figure}[H]
  \centering
  \includegraphics[width=1.0\linewidth,height=0.5\linewidth]{fig0007.png}
  \caption{Приключване на процеса за въвеждане на информация за потребителя}
\label{fig0007}
\end{figure}

Електронното писмо, за потвърждаване на потребителската регистрация, съдържа електронна препратка до уеб сайта на Sratch (Фиг. \ref{fig0008}). Тази препратка трябва да бъде последвана за да се завърши процесът по регистрация на нов потребител. 

\begin{figure}[H]
  \centering
  \includegraphics[width=1.0\linewidth,height=0.5\linewidth]{fig0008.png}
  \caption{Електронно съобщение за потвърждаване на електронния адрес}
\label{fig0008}
\end{figure}

Регистрацията на новия потребител приключва със зареждането на начален работен екран (Фиг. \ref{fig0009}). Горе, в дясно се вижда изписано потребителското име, избрано на първата стъпка от процеса по регистрацията.

\begin{figure}[H]
  \centering
  \includegraphics[width=1.0\linewidth,height=0.5\linewidth]{fig0009.png}
  \caption{Начален работен екран}
\label{fig0009}
\end{figure}

Успешната регистрация може да се потвърди и чрез създаването на много малък проект, който да демонстрира функционирането на развойната среда. За тази цел се избира опцията „Create“ от менюто (Фиг. \ref{fig0010}).

\begin{figure}[H]
  \centering
  \includegraphics[width=1.0\linewidth,height=0.5\linewidth]{fig0010.png}
  \caption{Избор на опция от менюто за нов проект}
\label{fig0010}
\end{figure}

Създаването на нов проект преминава през серия стъпки, свързани със заделянето на първоначално нужните ресурси (Фиг. \ref{fig0011}).

\begin{figure}[H]
  \centering
  \includegraphics[width=1.0\linewidth,height=0.5\linewidth]{fig0011.png}
  \caption{Зареждане на ресурсите}
\label{fig0011}
\end{figure}

След зареждането на новия проект се визуализира работното пространство (Фиг. \ref{fig0012}). Най- в ляво е списъкът с възможни програмни инструкции, под формата на парченца от пъзел. В централната част е работното пространство, където инструкциите се подреждат. А най- в дясно е активната сцена, където се визуализират действията, заложени в серията от инструкции. 

\begin{figure}[H]
  \centering
  \includegraphics[width=1.0\linewidth,height=0.5\linewidth]{fig0012.png}
  \caption{Организация на работното пространство}
\label{fig0012}
\end{figure}

Всяка компютърна програма има своя начална точка и своя крайна точка. В Scratch, за началото на програмата има специално отделен блок (програмна инструкция), която е показна на Фиг. \ref{fig0013}. Изпълнението на програмите, написани в Scratch, започва с натискането на зеления флаг. Точно поради тази причина, блокчето за старт на програмата е свързано със събитието за натискане на зеления флаг. 

\begin{figure}[H]
  \centering
  \includegraphics[width=1.0\linewidth,height=0.5\linewidth]{fig0013.png}
  \caption{Начало на програмата}
\label{fig0013}
\end{figure}

Една от най-интуитивните и същевременно лесно разбираеми инструкции е преместване с определен брой стъпки (Фиг. \ref{fig0014}). Главен актьор в началната сцена на Scratch е оранжевият котарак. Ако сцената не бъде променяна, то инструкциите за извършване на различни действия се насочват точно към този котарак. 

\begin{figure}[H]
  \centering
  \includegraphics[width=1.0\linewidth,height=0.5\linewidth]{fig0014.png}
  \caption{Инструкция за придвижване на стъпки}
\label{fig0014}
\end{figure}

След преместването на котарака е от съществено значение да има една пауза на изчакване, така че визуално да се забележи преместването. За тази цел може да се приложи инструкция за изчакване, за определен брой секунди (Фиг. \ref{fig0015}).

\begin{figure}[H]
  \centering
  \includegraphics[width=1.0\linewidth,height=0.5\linewidth]{fig0015.png}
  \caption{Инструкция за изчакване}
\label{fig0015}
\end{figure}

След изчакването, котката може да се върне на първоначалната си позиция, като се изпълни инструкция за придвижване с отрицателен брой стъпки (Фиг. \ref{fig0016}).

\begin{figure}[H]
  \centering
  \includegraphics[width=1.0\linewidth,height=0.5\linewidth]{fig0016.png}
  \caption{Инструкция за преместване обратно}
\label{fig0016}
\end{figure}

След като всички предвидени инструкции са изпълнени е разумно да се сложи край на програмата, за което е предвидено отделно блокче в списъка с инструкции (Фиг. \ref{fig0017}).

\begin{figure}[H]
  \centering
  \includegraphics[width=1.0\linewidth,height=0.5\linewidth]{fig0017.png}
  \caption{Инструкция за край}
\label{fig0017}
\end{figure}

Така написаната програма се изпълнява, чрез натискане на зеления флаг (Фиг. \ref{fig0018}), а при нужда от аварийно спиране се натиска червеният кръг, от дясно на зеления флаг.

\begin{figure}[H]
  \centering
  \includegraphics[width=1.0\linewidth,height=0.5\linewidth]{fig0018.png}
  \caption{Изпълнение на програмата}
\label{fig0018}
\end{figure}

Всяка програма, която се пише в Scratch се помества в отделен проект. Достъп до всички проекти на потребителя може да се получи от менюто „My Stuff“, което е част от списъка с опции за боравене с регистрирания потребител (Фиг. \ref{fig0019}).

\begin{figure}[H]
  \centering
  \includegraphics[width=1.0\linewidth,height=0.5\linewidth]{fig0019.png}
  \caption{Меню за организация на проектите}
\label{fig0019}
\end{figure}

Първоначално, всеки проект има служебно име (Фиг. \ref{fig0020}), което в последствие може да бъде променено. Едно от най-атрактивните предимства на програмната среда е, че проектите на потребителите могат да се споделят (Sharing) с много широка аудитория. Това позволява бърз трансфер на знания и умения, както и оценка за положения труд. 

\begin{figure}[H]
  \centering
  \includegraphics[width=1.0\linewidth,height=0.5\linewidth]{fig0020.png}
  \caption{Списък с проекти}
\label{fig0020}
\end{figure}

Най-голямото очарование блоковите езици получават от факта, че писането на инструкциите и формирането на цялостна програма прилича на подреждането на пъзел. Почти всички деца обичат да редят пазели. Харесват ярките цветове и красивите картини. Когато чарът на класическите пъзели се пренесе в една толкова атрактивна област, каквато е програмирането, резултатите могат да бъдат смайващи. 

\section{Първи стъпки в App Inventor}

Работата в средата на App Inventor започва със зареждане на главната уеб страница (Фиг. \ref{fig0021}), която се намира на адрес: \\ \href{https://appinventor.mit.edu/}{https://appinventor.mit.edu/}

\begin{figure}[H]
  \centering
  \includegraphics[width=1.0\linewidth,height=0.5\linewidth]{fig0021.png}
  \caption{Начална уеб страница на App Inventor}
\label{fig0021}
\end{figure}

Въпреки че App Inventor също е продукт на Масачузетския технологичен институт в някои аспекти работата с него се различава от начина по който се работи в Scratch. App Inventor също се предлага под формата на облачна услуга в която е необходима регистрация. За разлика от  Scratch, в App Inventor може да се пропусне създаването на потребителски профил, а влизането в системата да се осъществи, чрез класическа регистрация в услугата GMail. Този процес започва след избирането на оранжевия бутон „Create Apps!“ (Фиг. \ref{fig0022}).

\begin{figure}[H]
  \centering
  \includegraphics[width=1.0\linewidth,height=0.5\linewidth]{fig0022.png}
  \caption{Избор на GMail потребител за включване в системата}
\label{fig0022}
\end{figure}

След избора на потребител, с който да се работи в средата на  App Inventor е нужно този потребител да се автентифицира, чрез въвеждане на парола (Фиг. \ref{fig0023}).

\begin{figure}[H]
  \centering
  \includegraphics[width=1.0\linewidth,height=0.5\linewidth]{fig0023.png}
  \caption{Автентификация на потребителя}
\label{fig0023}
\end{figure}

За да бъде позволена работа в програмната среда на App Inventor се изисква съгласие от потребителя с общите условия на платформата (Фиг. \ref{fig0024}).

\begin{figure}[H]
  \centering
  \includegraphics[width=1.0\linewidth,height=0.5\linewidth]{fig0024.png}
  \caption{Общи условия за ползване на програмната среда}
\label{fig0024}
\end{figure}

Процесът по вход в програмната среда завършва с поздравителна уеб страница (Фиг. \ref{fig0025}). На тази страница е представена по-подробна информация за програмната среда, видът на стартираната инстанция и версия. 

\begin{figure}[H]
  \centering
  \includegraphics[width=1.0\linewidth,height=0.5\linewidth]{fig0025.png}
  \caption{Поздравителна страница}
\label{fig0025}
\end{figure}

Първата възможност, която се предлага на потребителя е да избере от възможности за разглеждане на няколко учебни проекта, служещи за първоначално въвеждане в начина за работа с програмната среда (Фиг. \ref{fig0026}).

\begin{figure}[H]
  \centering
  \includegraphics[width=1.0\linewidth,height=0.5\linewidth]{fig0026.png}
  \caption{Възможност за избор на учебни проекти}
\label{fig0026}
\end{figure}

Ако не бъде избран учебен проект или опцията за създаване на празен проект, системата насочва потребителя към страницата със списък от собствени проекти (Фиг. \ref{fig0027}).

\begin{figure}[H]
  \centering
  \includegraphics[width=1.0\linewidth,height=0.5\linewidth]{fig0027.png}
  \caption{Страница със списък от собствени проекти}
\label{fig0027}
\end{figure}

\newpage
\addcontentsline{toc}{chapter}{Conclusions}
\chapter*{Conclusions}
\thispagestyle{empty}

Block programming is an effective and affordable way to introduce children to coding concepts. By breaking programming down into visual blocks, kids can learn to put together small programs without worrying about syntax or typing errors. The Scratch and App Inventor programming environments are proven tools for teaching block programming to children. Scratch's intuitive interface and colorful blocks make it an ideal option for younger children, while App Inventor's ability to create real mobile apps may appeal to older ones. The book provides a comprehensive guide to learning block programming with Scratch and App Inventor. It covers game design and creation, mobile app creation, and more. The book also includes step-by-step instructions and many visual examples to help kids understand programming concepts. Children can develop basic skills such as problem-solving, logical thinking, and creativity through block programming. These skills can be applied in future endeavors, including computer science and other science, technology, engineering, and mathematics fields. Overall, the Block Programming for Kids with Scratch and App Inventor book is an excellent resource for parents and educators who want to introduce children to programming possibilities. Using visual blocks and easy-to-understand instructions, kids can learn to code in a fun and engaging way, setting them up for future personal and professional success.

\newpage

% Списък с използвана литература и източници на информация.
\addcontentsline{toc}{chapter}{Библиография}
\input{chapters/references}\newpage

% Азбучен указател на използваните термини.
\printindex

% Задна корица.
\includepdf[pages=-]{covers/back}

\end{document}
