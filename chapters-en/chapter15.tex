\chapter{Darts}

This project represents the popular darts game. The player shoots short arrows at a circular target. The closer it is to the center of the target, the more points it earns. The player aims to collect the maximum points for the three shots available. If the accumulated points are more than 13 - win the game. Otherwise, lose the game.

\begin{figure}[H]
   \centering
   \includegraphics[width=1.0\linewidth,height=0.5\linewidth]{fig150001.png}
   \caption{Darts}
\label{fig150001}
\end{figure}

\section{Creating the Design}
Before you start programming the game, you must first create the design. First, add the two main characters of the game - the target (Fig. \ref{fig150002}) and the arrow (Fig. \ref{fig150003}). Use the tools in Scratch to draw the required characters.

\begin{figure}[H]
   \centering
   \includegraphics[width=1.0\linewidth,height=0.5\linewidth]{fig150002.png}
   \caption{Target}
\label{fig150002}
\end{figure}

\begin{figure}[H]
   \centering
   \includegraphics[width=1.0\linewidth,height=0.5\linewidth]{fig150003.png}
   \caption{Arrow}
\label{fig150003}
\end{figure}

Add a new character that you can also draw using the Scratch tools. This character will show how many shots the player has left. To show how many shots are left, you add costumes to that character. For example, on the first suit, you can write that there are 3 shots left and on the second - 2 shots.

\begin{figure}[H]
   \centering
   \includegraphics[width=1.0\linewidth,height=0.5\linewidth]{fig150004.png}
   \caption{Number of Shots}
\label{fig150004}
\end{figure}

The next character to add is the one that shows how many points the player has earned. Again, add costumes to the character with the corresponding points he can achieve. In this case, these are from 0 to 5.

\begin{figure}[H]
   \centering
   \includegraphics[width=1.0\linewidth,height=0.5\linewidth]{fig150005.png}
   \caption{Number of points}
\label{fig150005}
\end{figure}

The last Scratch character is the inscription of whether the player wins or loses. Like the previous two characters, add two suits with different inscriptions - one for winning and the other for losing.

\begin{figure}[H]
   \centering
   \includegraphics[width=1.0\linewidth,height=0.5\linewidth]{fig150006.png}
   \caption{End of Game Caption}
\label{fig150006}
\end{figure}

\section{Programming the target and boom}
Before we move on to programming the respective characters, add new variable points to the game. This variable will contain the score that the character will accumulate during the game. From the Variables section, select Make a Variable and name the variable.

\begin{figure}[H]
   \centering
   \includegraphics[width=1.0\linewidth,height=0.5\linewidth]{fig150007.png}
   \caption{Creating in-game variable}
\label{fig150007}
\end{figure}

Let's move on to programming the target. The only instructions you need to add to this character are that it stays in the same position throughout the game. Remember to also set an initial value to the variable you created. At the beginning of each game, the player must have 0 points.

\begin{figure}[H]
   \centering
   \includegraphics[width=1.0\linewidth,height=0.5\linewidth]{fig150008.png}
   \caption{Target instructions}
\label{fig150008}
\end{figure}

You should also add instructions for the arrow. If the arrow is too big, you can resize it. Create a new variable to hold the status of the arrow. There are two statuses - shoot or throw. At the game's start, the arrow's status will be shot.

\begin{figure}[H]
   \centering
   \includegraphics[width=1.0\linewidth,height=0.5\linewidth]{fig150009.png}
   \caption{Set Arrow Status}
\label{fig150009}
\end{figure}

Add the following check - if the arrow's status is shoot, then the arrow must follow the player's mouse. Also, add a new event which is when this sprite clicked. When the player clicks the arrow, you need to change the value of the status variable to throw. Also, the character must message the other characters that the player has clicked the mouse, which means he is shooting.

\begin{figure}[H]
   \centering
   \includegraphics[width=1.0\linewidth,height=0.5\linewidth]{fig150010.png}
   \caption{Send shooting message}
\label{fig150010}
\end{figure}

After the player has fired the arrow, you need to program it to change its direction so that you can imitate a shot. Also, resize it to make it smaller, as if it is moving away from the player. You know that when the player clicks on the arrow, it sends a message. Use this message to change the direction and size of the arrow after firing.

\begin{figure}[H]
   \centering
   \includegraphics[width=1.0\linewidth,height=0.5\linewidth]{fig150011.png}
   \caption{Change arrow direction and size after shot}
\label{fig150011}
\end{figure}

When the player shoots the arrow, it starts moving away and sends a Check Points message, which should check how many points the player has earned. To determine the number of points earned, it is necessary to check what color the arrow touched. To make the game fairest, the check that will be made is the black color (the tip of the arrow) to which color it has touched. Add a new variable to hold the currently earned points. The last instruction to add is for the arrow to send a message that it gives points to the player.

\begin{figure}[H]
   \centering
   \includegraphics[width=1.0\linewidth,height=0.5\linewidth]{fig150012.png}
   \caption{Check Points Earned}
\label{fig150012}
\end{figure}

To make the game even more interesting, you can add a variable to be a randomly assigned number. This variable will deflect the arrow to the side depending on the random number that has landed.

\begin{figure}[H]
   \centering
   \includegraphics[width=1.0\linewidth,height=0.5\linewidth]{fig150013.png}
   \caption{Arrow deflection to the side}
\label{fig150013}
\end{figure}

\section{Programming the character that counts the number of shots}

In this step, you will add instructions showing the player how many more shots he can make. Select the character you added for the number of shots to do this. Place this character in a position of your choice. Create a new variable that will hold the number of shots. Then make the necessary checks - if the number of shots is 3, one should switch to a suit that shows three shots. If the number of shots is two - then one should change to the suit with two shots, and so on. At the last check - if the number of shots is 0, a message should be sent that the game is over.

\begin{figure}[H]
   \centering
   \includegraphics[width=1.0\linewidth,height=0.5\linewidth]{fig150014.png}
   \caption{Programming the number of shots}
\label{fig150014}
\end{figure}

\section{Programming the character that displays the number of points}

Select the character that shows how many points the player has earned. This hero, in addition to showing how many points there are, will also be responsible for reducing the number of shots, setting a new wind direction, and changing the arrow's status to ready to shoot again.

Since you don't know how many points the player will earn, you can cross the suit names on the number of points to avoid doing thousands of checks. This way, you can easily switch from suit to suit as you have a variable that holds the number of points currently earned.

\begin{figure}[H]
   \centering
   \includegraphics[width=1.0\linewidth,height=0.5\linewidth]{fig150015.png}
   \caption{Programming the number of points}
\label{fig150015}
\end{figure}

\section{Programming the end game}

The last step left to do is to program the end game. He will win or lose after the third shot, depending on how many points the player has earned. If the number of points is greater than 13 - he wins. Otherwise, he loses.

\begin{figure}[H]
   \centering
   \includegraphics[width=1.0\linewidth,height=0.5\linewidth]{fig150016.png}
   \caption{Programming the end game}
\label{fig150016}
\end{figure}

You are ready to test how accurate and good you are in the game you created.
