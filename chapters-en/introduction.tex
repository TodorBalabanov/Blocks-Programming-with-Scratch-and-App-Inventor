\addcontentsline{toc}{chapter}{Preface}
\chapter*{Preface}
\thispagestyle{empty}

This book caters to individuals with a passion for programming and an interest in teaching children in this field. We aim to provide valuable insights and information to anyone intrigued by this study area. Drawing from our experience in academia and children's education, we have crafted the content to illuminate the fundamental principles of experiential learning. This book serves as a comprehensive guide, enabling readers with limited computer knowledge to grasp programming concepts effectively.

From our perspective, employing visualizations resembling classic puzzles to demonstrate program constructions offers extensive opportunities to acquire valuable knowledge and skills. This visualization approach proves particularly effective in reducing the age at which programming can be learned. While traditional programming languages are suitable for high school classrooms, block languages are more aptly utilized in the curriculum of junior high school students.

Much of the material presented delves into fundamental programming concepts, including instruction sequencing, conditional and unconditional branching, iterative actions, events, modular organization, and more. Additionally, another section of the book focuses on practical implementation, showcasing ideas that hold the potential to evolve into standalone software solutions. Throughout the book, we adopt a principle-driven approach, illustrating concepts through step-by-step examples. This approach allows readers to engage in hands-on learning and reinforce their understanding of programming principles.

The book is designed to be accessible to readers without advanced computer literacy skills. However, it does assume a basic understanding of computer systems, operating systems, internet browsing, and web page loading. The block programming systems discussed in the book are web-based, and the practical exercises are conducted in a cloud environment.

While a deep knowledge of mathematics is optional, having a basic understanding of algebra and geometry can aid in comprehending specific examples. It's important to note that artistic skills, such as musicality or visual arts, are not required for the book's content, but their presence can add an extra layer of creativity to the outcomes achieved.

The material in the book is structured into interconnected chapters, and to ensure a comprehensive understanding, it is recommended to read them in the provided sequence. The content draws upon topics covered in the junior high school curriculum, enhancing the relevance and applicability of the material for readers within that educational context. Following the suggested reading order, readers can effectively build upon foundational knowledge and progress through more advanced concepts logically and coherently.

\vspace{0.5cm}

\large{\textbf{Acknowledgments}}

\vspace{0.5cm}

The authors would like to thank their families for their patience and understanding while writing this book. They would also like to thank their colleagues and friends who helped to achieve higher quality.

