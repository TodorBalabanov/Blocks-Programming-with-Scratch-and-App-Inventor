\addcontentsline{toc}{chapter}{Preface}
\chapter*{Preface}
\thispagestyle{empty}

This book is intended for all people who are excited about topics in programming and especially about teaching children in this field. We hope anyone interested in the area finds something valuable in the material presented. Our experience primarily concerns the academic world and pedagogy dedicated to children's education. The material is presented in such a way as to reveal the basic mechanisms of learning by doing. Generally speaking, the book guides the reader with minimal computer knowledge to a good understanding of programming concepts.

From our point of view, the presentation of program constructions with the help of visualization, as close as possible to classic puzzles, gives ample opportunities for learning valuable knowledge and skills. Such a visualization approach allows for an effective lowering of the programming learning age. While classical programming languages are appropriate in high school classrooms, block languages effectively find their application in the junior high school course of study.

Part of the presented material explains fundamental concepts in programming, such as - sequence of instructions, conditional and unconditional transitions, cyclicity of actions, events, modular organization, and others. Another part emphasizes practical implementation and ideas that have the potential to become independent software solutions. The chosen approach to presenting the information is through examples of the principle - do, step by step.

The book assumes no prerequisites for an advanced level of computer literacy but relies on the reader having a basic knowledge of what a computer system is, what an operating system is, what the Internet is, and how it handles loading and browsing web pages. The considered block programming systems are web-based, and work with them takes place in a cloud space. A deep knowledge of mathematics is not required, but a basic knowledge of algebra and geometry helps understand some examples. Artistic skills, such as musicality or visual arts, are unnecessary, but their presence would give additional flavor to the results achieved.

The material is organized into chapters related to each other, and for complete absorption, it is desirable to read them in the given sequence. Part of the exposition in the book is based on subjects taught in the junior high school course of study.

\vspace{0.5cm}

\large{\textbf{Acknowledgments}}

\vspace{0.5cm}

The authors would like to thank their families for their patience and understanding while writing this book. They would also like to thank their colleagues and friends who helped to achieve higher quality.

