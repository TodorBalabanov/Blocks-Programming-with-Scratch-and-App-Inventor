\chapter{Програмни конструкции}

Компютърните програми са съставени от стриктна последователност инструкции. Такива последователности се наричат алгоритъм. Ние хората всеки ден изпълняваме различни алгоритми, като част от нашето ежедневие. Да се приготвим и да отидем на училище е един алгоритъм. Събуждаме се, ставаме, обличаме се, правим си сутрешния тоалет, закусваме, излизаме от вкъщи, придвижваме се до училище. Много ярък пример за алгоритъм са рецептите за готвене. В една рецепта има начални продукти, след това точни инструкции как продуктите са се обработят и смесят, като има ясна представа какъв трябва да бъде крайният резултат. При компютърните програми има основен набор от инструкции, които съставляват изразните средства на съответния програмен език. Чарът на блоковите езици е, че този основен набор от инструкции е представен визуално, под формата на цветни блокчета. Подредбата на цветните блокчета в строго определена последователност води до създаването на малки компютърни програми. 

В случая на Scratch, програмата има ясно определена стартова точка и ясно определена финална точка. При App Inventor подходът е малко по-различен. Там последователността от инструкции, съставляващи писаната програма, се въвежда в малки фрагменти, наречени събития. Събитията възникват при различни действия от страна на потребителя или операционната система. При Scratch говорим за последователно програмиране, а при App Inventor говорим за събитийно програмиране. Основните програмни конструкции в двете програмни среди до голяма степен са идентични, но има и някои съществени разлики. За да можем да пишем ефективни и надеждни програми е важно добре да познаваме изразните средства на програмните среди с които работим. 

\section{Изразни средства в Scratch}

Базовите градивни блокчета в Scratch са организирани в цветни групи (Фиг. \ref{fig0051}). Тази организация помага за по-бързо ориентиране и по-ефективна употреба на различните блокчета. 

\begin{figure}[H]
  \centering
  \includegraphics[width=1.0\linewidth,height=0.5\linewidth]{fig0051.png}
  \caption{Групиране на инструкциите}
\label{fig0051}
\end{figure}

Най-важното блокче в програмата е блокчето, което дава старт за изпълнение на инструкциите, които са подредени под него. Това блокче има зелен флаг (Фиг. \ref{fig0052}) и определя какво ще последва след стартирането на програмата.

\begin{figure}[H]
  \centering
  \includegraphics[width=1.0\linewidth,height=0.5\linewidth]{fig0052.png}
  \caption{Начална точка на програмата}
\label{fig0052}
\end{figure}

Блокчето за старт на програмата се намира в светло оранжевата група, която е предназначена да реагира на събития от страна на потребителя. Точният момент в който потребителят иска програмата да започне своето изпълнение е неопределен във времето и поради тази причина Scratch трябва да улови събитие, предизвикано от самия потребител. 

Второто по важност блокче служи за край на програмата (Фиг. \ref{fig0053}). То се намира в тъмно оранжевата група и има за задача да спре всички процеси, извършващи се по време на изпълнението на самата програма.

\begin{figure}[H]
  \centering
  \includegraphics[width=1.0\linewidth,height=0.5\linewidth]{fig0053.png}
  \caption{Крайна точка на програмата}
\label{fig0053}
\end{figure}

Тъмно оранжевата група съдържа блокчета за контрол на изпълнението. Тези блокчета позволяват програмата да поема по различни пътища, както и група от действия да се повтарят многократно. 

В Scratch блокчетата инструкции основно контролират картинки, наречени спрайтове (sprites). За разлика от обикновеното компютърно изображение, спрайтът е графичен обект, който съдържа множество кадри, показващи изображението на героя в различни конфигурации. Всяка нова програма в Scratch започва с един спрайт, на оранжевата котка, разположена на координати (x=0,y=0). Работното пространство е двуизмерна координатна система с център (0,0). 

\begin{figure}[H]
  \centering
  \includegraphics[width=1.0\linewidth,height=0.5\linewidth]{fig0054.png}
  \caption{Завършване веднага след започване}
\label{fig0054}
\end{figure}

Ако бъдат съединени, блокчетата за начало и за край (Фиг. \ref{fig0054}), то програмата не изпълнява нищо. Практически, тази програма приключва веднага след като е започнала. Програма, която не прави нищо е напълно безсмислена. За да започне нещо да се случва се използват блокчетата в синята група. Първото блокче инструктира котето да се премести 10 стъпки, като броя стъпки може да бъде променени, чрез изписване на друго число във вътрешността на блокчето (Фиг. \ref{fig0055}).

\begin{figure}[H]
  \centering
  \includegraphics[width=1.0\linewidth,height=0.5\linewidth]{fig0055.png}
  \caption{Преместване на героя}
\label{fig0055}
\end{figure}

Следващият блок в групата инструктира героя да се завърти на определено число градуси, по часовниковата стрелка, спрямо собствения си център (Фиг. \ref{fig0056}).

\begin{figure}[H]
  \centering
  \includegraphics[width=1.0\linewidth,height=0.5\linewidth]{fig0056.png}
  \caption{Завъртане по часовниковата стрелка}
\label{fig0056}
\end{figure}

Аналогично, със следващото блокче в групата, завъртането може да се изпълни и в посока обратна на часовниковата стрелка (Фиг. \ref{fig0057}).

\begin{figure}[H]
  \centering
  \includegraphics[width=1.0\linewidth,height=0.5\linewidth]{fig0057.png}
  \caption{Завъртане обратно на часовниковата стрелка}
\label{fig0057}
\end{figure}

Следващия блок в групата дава възможност героят да се премести на случайни координати или на координати посочени с мишката (Фиг. \ref{fig0058}).

\begin{figure}[H]
  \centering
  \includegraphics[width=1.0\linewidth,height=0.5\linewidth]{fig0058.png}
  \caption{Преместване на случайна позиция}
\label{fig0058}
\end{figure}

Движението на героя може да бъде зададено и чрез абсолютни координати с блокче, позволяващо да се впишат числа за абцисната и ординатната ос (Фиг. \ref{fig0059}).

\begin{figure}[H]
  \centering
  \includegraphics[width=1.0\linewidth,height=0.5\linewidth]{fig0059.png}
  \caption{Преместване по абсолютни координати}
\label{fig0059}
\end{figure}

Плавно придвижване, по предварително зададен интервал от време, е възможно на случайни координати или координати посочени с мишката, благодарение на следващото блокче в групата (Фиг. \ref{fig0060}).

\begin{figure}[H]
  \centering
  \includegraphics[width=1.0\linewidth,height=0.5\linewidth]{fig0060.png}
  \caption{Плъзгане до случайна позиция}
\label{fig0060}
\end{figure}

Плавното плъзгане до предварително зададени координати, за предварително определен интервал от време, е възможно с блокчето предназначено за тази цел (Фиг. \ref{fig0061}).

\begin{figure}[H]
  \centering
  \includegraphics[width=1.0\linewidth,height=0.5\linewidth]{fig0061.png}
  \caption{Плъзгане до зададени координати}
\label{fig0061}
\end{figure}

Анимираният герой има характеристика за ориентация, под формата на ъгъл. При 90 градуса, оранжевата котка гледа на дясно. За да се промени ориентацията на героя се използва блокче с възможност за въвеждане на конкретен ъгъл (Фиг. \ref{fig0062}).

\begin{figure}[H]
  \centering
  \includegraphics[width=1.0\linewidth,height=0.5\linewidth]{fig0062.png}
  \caption{Ъглова ориентация}
\label{fig0062}
\end{figure}

При по-сложни сценарии за управление на героя, понякога е нужно героят да следи показалеца на мишката. За тази цел има определено блокче, което изпълнява тази инструкция (Фиг. \ref{fig0063}).

\begin{figure}[H]
  \centering
  \includegraphics[width=1.0\linewidth,height=0.5\linewidth]{fig0063.png}
  \caption{Ориентация по показалеца на мишката}
\label{fig0063}
\end{figure}

Блокчетата могат да се поставят едно след друго, като за последователна промяна на относителните x и y координатите (относителни, спрямо текущата позиция) на героя има специално определени блокчета (Фиг. \ref{fig0064}).

\begin{figure}[H]
  \centering
  \includegraphics[width=1.0\linewidth,height=0.5\linewidth]{fig0064.png}
  \caption{Последователна промяна на относителни координати}
\label{fig0064}
\end{figure}

Освен относителна промяна на координатите е възможна и абсолютна промяна на координатите, като абсолютната промяна е спрямо центъра на координатната система (Фиг. \ref{fig0065}).

\begin{figure}[H]
  \centering
  \includegraphics[width=1.0\linewidth,height=0.5\linewidth]{fig0065.png}
  \caption{Последователна промяна на абсолютни координати}
\label{fig0065}
\end{figure}

При своето движение, когато анимираният герой достигне границите на работното пространство, единият вариант е движението да продължи извън видимата зона. Другият вариант е да се вземат мерки и героят да отскача от ръбовете на работното пространство. За това отскачане има конкретно блокче (Фиг. \ref{fig0066}). За да се илюстрира работата му е нужна малко по-сложна последователност от инструкции. При всяко стартиране на програмата, първо се променят относителните координати, а след това се извършва отскачане от ръба, ако е необходимо. За да бъде малко по-интересен сценарият за проверка, вместо фиксирани стойности за относително отместване се използва вграждане на едно от зелените блокчета, което позволява генериране на случайно число в предварително определен диапазон. Съществено е да се забележи, че зеленото блокче има овална форма, което подсказва, че то е предназначено за вграждане в някой от другите блокове, които имат овален слот. 

\begin{figure}[H]
  \centering
  \includegraphics[width=1.0\linewidth,height=0.5\linewidth]{fig0066.png}
  \caption{Отскачане от ръбовете}
\label{fig0066}
\end{figure}

Следващо, много полезно блокче, от групата на тъмно оранжевите е блокчето за изчакване на период от време (Фиг. \ref{fig0067}). Когато това блокче бъде поставено между блокчетата за начало и край, програмата изчаква зададения брой секунди, преди да преустанови изпълнението си. По време на изпълнение, ясно може да се забележи, че около последователността от инструкции се появява жълта рамка, която символизира режима на изпълняващи се инструкции. 

\begin{figure}[H]
  \centering
  \includegraphics[width=1.0\linewidth,height=0.5\linewidth]{fig0067.png}
  \caption{Инструкция за изчакване}
\label{fig0067}
\end{figure}

Групата на лилавите блокчета съдържат инструкции за външното оформление на анимирания герой. Първите две блокчета са предназначени за реплики (Фиг. \ref{fig0068}), които героят казва (изписват се както в комикс). Първото блокче задава текст, който стои на екрана до следващата инструкция. Точно за това е нужно да има няколко секунди изчакване, така че текстът да остане видим за потребителя. Второто блокче има и параметър с който да се определи колко секунди текстът да бъде видим за потребителя. 

\begin{figure}[H]
  \centering
  \includegraphics[width=1.0\linewidth,height=0.5\linewidth]{fig0068.png}
  \caption{Изписване на реплики за изговаряне}
\label{fig0068}
\end{figure}

Вторите две блокчета са предвидени за реплики, които анимираният герой си мисли, но не изрича. Разликата се състои в начина по който се визуализира текстът (Фиг. \ref{fig0069}).

\begin{figure}[H]
  \centering
  \includegraphics[width=1.0\linewidth,height=0.5\linewidth]{fig0069.png}
  \caption{Изписване на реплики, като мисъл}
\label{fig0069}
\end{figure}



