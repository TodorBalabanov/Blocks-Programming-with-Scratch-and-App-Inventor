\chapter{Работни среди}

Блоковите програмни езици са подразделение на визуалните програмни езици. Същината на блоковите езици е, че програмните инструкции се въвеждат под формата на цветни блокове, а не както е в класическите програмни езици, чрез изписване на текстови команди. Най-основната цел на блоковите езици е да направят областта програмиране значително по-достъпна за начинаещите. Тази цел се постига чрез три основни направления. От синтактична гледна точка, инструкциите в блоковите езици са под формата на цветни иконки. Това значително намалява възможността за изписването на грешна програмна инструкция. На второ място се подобрява семантиката, като всяка от възможните програмни инструкции е добре документирана. На трето място е прагматизма, който позволява изучаването на различните състояния в които може да изпадне програмата. Програмните среди за блоково програмиране набират все по-голяма популярност през последното десетилетие. Някои от най-популярните са: Scratch, Blockly, App Inventor for Android, Ardublock и други. В тази книга ще се спрем на две от програмните среди за блоково програмиране, създадени в Масачузетския технологичен институт, Scratch и App Inventor for Android. Причината за този избор е, че Scratch има насоченост към най-малките, а именно децата в началните училищни класове, което много добре се съчетава с възможностите блоковите програми да бъдат визуализирани и на мобилен телефон, чрез App Inventor for Android. И при двете програмни среди не се изисква инсталирането на специализиран софтуер. Достатъчно е наличието на съвременен компютър, свързан в Интернет и съвременна версия на уеб браузър. 

\section{Първи стъпки в Sratch}


\section{Първи стъпки в App Inventor}

