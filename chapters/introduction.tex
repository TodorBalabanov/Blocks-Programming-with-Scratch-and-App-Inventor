\addcontentsline{toc}{chapter}{Предговор}
\chapter*{Предговор}
\thispagestyle{empty}

Тази книга е предназначена за всички хора, които се вълнуват от теми в програмирането и най-вече за обучението на деца в тази област. Надеждата ни е, че всеки с интерес в областта би намерил нещо ценно в изложения материал. Нашият опит е предимно свързан с академичния свят и педагогиката, посветена на обучението на деца. Материалът е изложен по такъв начин, че да разкрива основните механизми за учене чрез правене. Най-общо казано, книгата съпровожда читателя с минимални компютърни познания до едно задоволително ниво на разбиране за концепциите в програмирането. 

От наша гледна точка, представянето на програмните конструкции с помощта на визуализация, максимално доближаваща класическите пъзели, дава широки възможности за усвояване на ценни знания и умения. Подобен подход за онагледяване позволява ефективно снижаване на възрастта за обучение по програмиране. Докато класическите програмни езици са подходящи в класовете на гимназията, блоковите езици ефективно намират своето приложение в прогимназиалния курс на обучение. 

Част от изложения материал разяснява фундаментални концепции в програмирането, като - последователност от инструкции, условни и безусловни преходи, цикличност на действията, събития, модулна организация и други. Друга част набляга на практическа реализация и то на идеи, които имат потенциал да се превърнат в самостоятелни софтуерни решения. Избраният подход за представяне на информацията е чрез примери на принципа – направа, стъпка по стъпка. 

Книгата не предполага предварителни изисквания за напреднало ниво на компютърна грамотност, но разчита читателят да има базови познания за това какво е компютърна система, какво е операционна система, какво е Интернет, как се борави със зареждането и преглеждането на уеб страници. Разгледаните системи за блоково програмиране са уеб базирани и работата с тях се осъществява в облачно пространство. Не се изискват задълбочени познания по математика, но за разбирането на някои от примерите много помагат базови познания по алгебра и геометрия. Артистични умения, като музикалност или изобразителни изкуства не са нужни, но наличието им би дало допълнителен колорит на постигнатите резултати. 

Материалът е организиран в глави, които са свързани една с друга и за по-пълноценно усвояване е желателно прочитането им да се извърши в зададената последователност. Част от изложението в книгата се базира на учебни предмети, преподавани в прогимназиалния курс на обучение. 

\vspace{0.5cm}

\large{\textbf{Благодарности}}

\vspace{0.5cm}

Авторите биха искали да благодарят на своите семейства за търпението и разбирането, проявено в дългият период за написването на тази книга. Също така биха искали да благодарят на своите колеги и приятели, помогнали в достигането на едно по-високо качество. 

